\documentclass[../analisi-dei-requisiti.tex]{subfiles}

\begin{document}

\subsection{Struttura}%
\label{subs:struttura}
I casi d'uso vengono classificati e identificati secondo uno schema univoco, al fine di facilitarne la lettura e la comprensione. La classificazione segue la seguente codifica:
\begin{center}
  \centering
  \textbf{UC[P].[F].[FF].[FFF]}
\end{center} dove:
\begin{itemize}
  \item \textbf{UC}: Indica che il codice è un caso d'uso;
  \item \textbf{P}: Consistente in un numero progressivo, identifica il caso;
  \item \textbf{F}, \textbf{FF} e \textbf{FFF}: Consistenti ognuno di un numero progressivo, identificano il sottocaso.
\end{itemize}
Ogni caso d'uso è formato, non necessariamente integralmente, dai seguenti campi:
\begin{itemize}
  \item \textbf{\glossario{Diagrammi UML}}: Diagrammi esplicativi realizzati in linguaggio UML2.0;
  \item \textbf{Attori primari}: Attori primari del caso d'uso;
  \item \textbf{Attori secondari}: Attori secondari del caso d'uso;
  \item \textbf{Descrizione}: Descrizione del caso d'uso;
  \item \textbf{Precondizione}: Condizioni che devono essere soddisfatte perché gli eventi del caso d'uso si possano verificare;
  \item \textbf{Scenario principale}: Flusso degli eventi del caso d'uso;
  \item \textbf{Postcondizione}: Condizioni che devono essere soddisfatte dopo il verificarsi degli eventi;
  \item \textbf{Estensioni}: Estensioni coinvolte;
  \item \textbf{Generalizzazioni}: Generalizzazioni coinvolte.
\end{itemize}

\subsection{Attori}%
\label{subs:attori}

\subsubsection{Attori primari}%
\label{sssec:attori_primari}
\begin{itemize}
  \item \textbf{Utente}: Si fa riferimento all'utente generico che ha intenzione di associare i predittori ad un flusso dati per monitorare un'applicazione.
\end{itemize}

\subsubsection{Attori secondari}
\label{sssec:attori_secondari}
\begin{itemize}
  \item \textbf{Grafana}: Piattaforma che permette di essere estesa con dei plug-in, in essa infatti risiederà il nostro prodotto che permetterà all'utente di visualizzare le previsioni ottenute dal plug-in sviluppato dal team.
\end{itemize}

\newpage

\subsection{Elenco dei casi d'uso}
\label{subs:elenco_dei_casi_duso}

\subsubsection{UC1 - Addestramento del sistema}
\label{sssec:uc1}
\begin{description}
  \item[Attore primario:] Utente amministratore.
  \item[Descrizione:] Per creare il file JSON contenente il predittore da dare al sistema di monitoraggio Grafana l'amministratore deve creare un file contenente i dati, scegliere il modello di machine learning da utilizzare e infine deve avviare l'addestramento.
  \item[Precondizione:] L'amministratore deve avere dei dati e sapere che modello utilizzare.
  \item[Scenario principale:]
  \begin{enumerate}
    \item L'amministratore inserisce i dati (UC1.1);
    \item L'amministratore utilizza un predittore già addestrato, se ne è in possesso (UC1.2);
    \item L'amministratore sceglie il modello di machine learning (UC1.3);
    \item L'amministratore avvia l'addestramento (UC1.4);
    \item L'amministratore crea un file contenente il predittore (UC1.5).
  \end{enumerate}
  \item[Postcondizione:] Si è ottenuto un file contente il predittore da dare a Grafana.
\end{description}

\subsubsection{UC1.1 - Inserimento dati di allenamento}
\label{sssec:uc1.1}
\begin{description}
  \item[Attore primario:] Utente amministratore.
  \item[Descrizione:] L'amministratore inserisce i dati raccolti da un database in un file CSV.
  \item[Precondizione:] L'amministratore è in possesso di dati.
  \item[Scenario principale:]
  \begin{enumerate}
    \item L'amministratore caricati i dati all'interno del file CSV;
    \item I dati vengono caricati nel sistema.
  \end{enumerate}
  \item[Postcondizione:] Il file CSV contiene i dati raccolti dall'amministratore.
  \item[Estensioni:] UC1.1 viene esteso nel caso d'uso UC8 con la visualizzazione del messaggio di errore quando viene fornito un file di addestramento non valido.
\end{description}

\subsubsection{UC1.2 - Utilizzo del predittore già addestrato}
\label{sssec:uc1.2}
\begin{description}
  \item[Attore primario:] Utente amministratore.
  \item[Descrizione:] L'amministratore utilizza il file JSON contenente il predittore addestrato in un'applicazione esterna.
  \item[Precondizione:] L'amministratore ha a disposizione il predittore già allenato.
  \item[Scenario principale:] L'amministratore, qualora possieda un predittore precedentemente allenato, lo carica nel sistema.
  \item[Postcondizione:] L'amministratore ha a disposizione un predittore già addestrato.
  \item[Estensioni:] UC1.2 viene esteso nel caso d'uso UC16 con la visualizzazione del messaggio di errore quando si tenta di caricare un predittore in un formato non valido.
\end{description}

\subsubsection{UC1.3 - Scelta del modello di machine learning}
\label{sssec:uc1.3}
\begin{description}
  \item[Attore primario:] Utente amministratore.
  \item[Descrizione:] L'amministratore decide che tipo di machine learning utilizzare: SVM, RL o la Rete Neurale.
  \item[Precondizione:]
  \begin{enumerate}
    \item L'inserimento dei dati è avvenuto correttamente (UC1);
    \item L'amministratore ha a disposizione i tre modelli di machine learning.
  \end{enumerate}
  \item[Scenario principale:] L'amministratore, avente un file contenente i dati, decide che modello di machine learning utilizzare tra SVM, RL o la Rete Neurale.
  \item[Postcondizione:] L'amministratore ha scelto un modello di machine learning.
\end{description}

\subsubsection{UC1.3.1 - Scelto modello Support Vector Machine}
\label{sssec:uc1.3.1}
\begin{description}
  \item[Attore primario]: Utente amministratore.
  \item[Descrizione]: L'amministratore ha scelto come modello la Support Vector Machine.
  \item[Precondizione:]
  \begin{enumerate}
    \item L'inserimento dei dati è avvenuto correttamente (UC1);
    \item L'amministratore ha a disposizione i tre modelli di machine learning.
  \end{enumerate}
  \item[Scenario principale:] L'amministratore ha selezionato SVM come algoritmo per svolgere l'addestramento.
  \item[Postcondizione]: L'amministratore ha scelto come modello la SVM.
\end{description}

\subsubsection{UC1.3.2 - Scelto modello Regressione Lineare}
\label{sssec:uc1.3.2}
\begin{description}
  \item[Attore primario]: Utente amministratore.
  \item[Descrizione]: L'amministratore ha scelto come modello la Regressione Lineare.
  \item[Precondizione:]
  \begin{enumerate}
    \item L'inserimento dei dati è avvenuto correttamente (UC1);
    \item L'amministratore ha a disposizione i tre modelli di machine learning.
  \end{enumerate}
  \item[Scenario principale:] L'amministratore ha selezionato la RL come algoritmo per svolgere l'addestramento.
  \item[Postcondizione]: L'amministratore ha scelto come dello la RL.
\end{description}

\subsubsection{UC1.3.2.1 - Scelto modello Regressione Esponenziale}
\label{sssec:uc1.3.2.1}
\begin{description}
  \item[Attore primario]: Utente amministratore.
  \item[Descrizione]: L'amministratore ha scelto come modello la Regressione Esponenziale.
  \item[Precondizione:]
  \begin{enumerate}
    \item L'inserimento dei dati è avvenuto correttamente (UC1);
    \item L'amministratore ha a disposizione i tre modelli di machine learning.
  \end{enumerate}
  \item[Scenario principale:] L'amministratore ha selezionato la Regressione Esponenziale come algoritmo per svolgere l'addestramento.
  \item[Postcondizione]: L'amministratore ha scelto come modello la Regressione Esponenziale.
\end{description}

\subsubsection{UC1.3.2.2 - Scelto modello Regressione Logaritmica}
\label{sssec:uc1.3.2.2}
\begin{description}
  \item[Attore primario]: Utente amministratore.
  \item[Descrizione]: L'amministratore ha scelto come modello la Regressione Logaritmica.
  \item[Precondizione:]
  \begin{enumerate}
    \item L'inserimento dei dati è avvenuto correttamente (UC1);
    \item L'amministratore ha a disposizione i tre modelli di machine learning.
  \end{enumerate}
  \item[Scenario principale:] L'amministratore ha selezionato la Regressione Logaritmica come algoritmo per svolgere l'addestramento.
  \item[Postcondizione]: L'amministratore ha scelto come modello la Regressione Logaritmica.
\end{description}

\subsubsection{UC1.3.3 - Scelto modello Rete Neurale}
\label{sssec:uc1.3.3}
\begin{description}
  \item[Attore primario]: Utente amministratore.
  \item[Descrizione]: L'amministratore ha scelto come modello la Rete Neurale;
  \item[Precondizione:]
  \begin{enumerate}
    \item L'inserimento dei dati è avvenuto correttamente (UC1);
    \item L'amministratore ha a disposizione i tre modelli di machine learning.
  \end{enumerate}
  \item[Scenario principale:] L'amministratore ha selezionato la Rete Neurale come algoritmo per svolgere l'addestramento.
  \item[Postcondizione]: L'amministratore ha scelto come modello la Rete Neurale.
\end{description}


\subsubsection{UC1.4 - Avvio dell'addestramento}
\label{sssec:uc1.4}
\begin{description}
  \item[Attore primario:] Utente amministratore.
  \item[Descrizione:] L'amministratore utilizzando la libreria e i dati ottiene il predittore.
  \item[Precondizione:]
  \begin{enumerate}
    \item L'amministratore ha a disposizione il file contenente i dati (UC1);
    \item L'amministratore ha scelto un modello di machine learning.
  \end{enumerate}
  \item[Scenario principale:] Avendo a disposizione il file con i dati e avendo scelto un modello avviene l'addestramento.
  \item[Postcondizione:] Si è ottenuto il predittore.
\end{description}

\subsubsection{UC1.5 - Salvataggio del risultati ottenuti}
\label{sssec:uc1.5}
\begin{description}
  \item[Attore primario:] Utente amministratore.
  \item[Descrizione:] I risultati ottenuti vengono salvati nel file JSON, riportando anche la tipologia di modello di machine learning utilizzato.
  \item[Precondizione:]
  \begin{enumerate}
    \item L'amministratore ha i dati ottenuti in precedenza;
    \item L'amministratore ha deciso che modello di machine learnig  utilizzare.
  \end{enumerate}
  \item[Scenario principale:] Una volta ottenuti i risultati dall'addestramento, l'amministratore salva i dati e il modello utilizzato nel file JSON.
  \item[Postcondizione:] Il file contiene sia il predittore che il modello utilizzato.
\end{description}


\newpage
\subsubsection{UC2 - Addestramento del predittore}
\label{sssec:uc2}

\begin{figure}[h!]
  \begin{center}
    \includegraphics[width=19cm]{uc2.png}\\
    \caption{UC2 - Addestramento del predittore}%
    \label{fig:uc2}
  \end{center}
\end{figure}

\begin{itemize}
  \item \textbf{Attore primario}:  Utente;
  \item \textbf{Attore secondario}: Grafana;
  \item \textbf{Descrizione}: L'utente seleziona un flusso di dati presente in Grafana e il plug-in, grazie allo storico dei dati di quel flusso, addestara il predittore utilizzando un modello;
  \item \textbf{Precondizione}: L'utente ha creato un pannello del plug-in e sa che tipo di modello di machine learning va utilizzato;
  \item \textbf{Scenario principale}:
  \begin{enumerate}
    \item L'utente sceglie su che dati, tra quelli disponibili in Grafana, compiere l'addestramento (UC2.1);
    \item L'utente sceglie un modello da utlizzare per compiere l'addestramento(UC2.2).
  \end{enumerate}
  \item \textbf{Postcondizione}: Il plug-in ha generato il predittore ed ora è pronto a fare previsioni sui dati.
\end{itemize}

\paragraph{UC2.1 - Scelta del flusso dei dati}
\label{para:uc2.1}
\begin{itemize}
  \item \textbf{Attore primario}: Utente;
  \item \textbf{Attore secondario}: Grafana;
  \item \textbf{Descrizione}: L'utente sceglie su che flusso presente in Grafana compiere l'addestramento;
  \item \textbf{Precondizione}: L'utente ha a disposizione dei dati;
  \item \textbf{Scenario principale}: L'attore sceglie il flusso di dati per effettuare l'addestramento;
  \item \textbf{Postcondizione}: L'utente ha scelto un flusso di dati e con questo farà l'addestramento.
\end{itemize}

\paragraph{UC2.2 - Scelta del modello}
\label{para:uc2.2}
\begin{itemize}
  \item \textbf{Attore primario}: Utente;
  \item \textbf{Descrizione}: L'utente sceglie un modello da applicare ai dati tra SVM, RL, regressione esponenziale, regressione logaritmica, SVM adattata alla regressione o rete neurale;
  \item \textbf{Precondizione}:
  \item \begin{enumerate}
    \item L'utente ha scelto su che dati, tra quelli disponibili in Grafana, compiere l'addestramento (UC2.1);
    \item L'utente deve scegliere che modello di machine laerning utilizzare.
  \end{enumerate}
  \item \textbf{Scenario principale}: L'utente sceglie il modello da utilizzare per l'addestramento;
  \item \textbf{Postcondizione}: L'utente ha scelto il modello per l'addestramento;
  \item \textbf{Generalizzazioni}: UC2.2 viene generalizzato dai casi d'uso UC6, UC7, UC8, UC9, UC10 e UC11.
\end{itemize}



\subsubsection{UC3 - Attivazione apprendimento di flusso costante}
\label{sssec:UC3}
\begin{itemize}
  \item \textbf{Attore primario}: Utente;
  \item \textbf{Attore secondario}: Grafana;
  \item \textbf{Descrizione}: L'utente decide di attivare l'apprendimento in costante adattamento ai dati rilevati sul campo. Questo porta ad avere un aggiornamento automatico del predittore in base ai nuovi dati ricevuti;
  \item \textbf{Precondizione}:
  \begin{enumerate}
    \item L'utente ha abilitato correttamente il plug-in dalle impostazioni di Grafana;
    \item L'utente ha scelto di attivare l'apprendimento continuo accedendo al pannello dedicato a questo.
  \end{enumerate}
  \item \textbf{Scenario principale}:
  \begin{enumerate}
    \item L'utente seleziona quale algoritmo utilizzare per attivare l'apprendimento costante (UC3.1);
    \item L'utente seleziona su quale sorgente dati effettuare l'apprendimento costante (UC3.2).
  \end{enumerate}
  \item \textbf{Postcondizione}: E' stata attivita la modalità di apprendimento continuo.
\end{itemize}

\paragraph{UC3.1 - Scelta dell'algoritmo da utilizzare}
\label{para:uc3.1}
\begin{itemize}
  \item \textbf{Attore primario}: Utente;
  \item \textbf{Descrizione}: L'utente seleziona l'algoritmo da utilizzare per l'addestramento costante, scegliendo tra RL e SVM;
  \item \textbf{Precondizione}:
  \begin{enumerate}
    \item L'utente ha selezionato il pannello dedicato all'apprendimento di flusso costante;
    \item L'utente ha a disposizione due algoritmi per effettuare l'apprendimento costante.
  \end{enumerate}
  \item \textbf{Scenario principale}: L'utente seleziona l'algoritmo da utilizzare per l'apprendimento costante;
  \item \textbf{Postcondizione}: L'utente ha selezionato l'algoritmo per effettuare l'apprendimento costante, scegliendo tra RL e SVM;
  \item \textbf{Estensione}: UC3.1 viene generalizzato dai casi d'uso UC6 e UC7.
\end{itemize}

\paragraph{UC3.2 - Scelta della sorgente dati da utilizzare}
\label{para:uc3.2}
\begin{itemize}
  \item \textbf{Attore primario}: Utente;
  \item \textbf{Attore secondario}: Grafana;
  \item \textbf{Descrizione}: L'utente seleziona quale sorgente dati utilizzare;
  \item \textbf{Precondizione}: L'utente ha a disposizione un'insieme di dati;
  \item \textbf{Scenario principale}: L'utente sceglie quale sorgente di dati utilizzare per l'apprendimento di flusso costante;
  \item \textbf{Postcondizione}: L'utente ha selezionato quale sorgente di dati utilizzare per l'apprendimento di flusso costante.
\end{itemize}



\subsubsection{UC4 - Utilizzo del plug-in}
\label{sssec:uc4}
\begin{description}
  \item[Attore primario:] Utente amministratore.
  \item[Attore secondario:] Grafana.
  \item[Descrizione:] L'amministratore sceglie se utilizzare il plug-in per la predizione oppure interromperlo.
  \item[Precondizione:] Il plug-in è stato configurato correttamente (UC3).
  \item[Scenario principale:]
  \begin{enumerate}
    \item L'amministratore avvia la predizione (UC4.1);
    \item L'amministratore interrompe la predizione (UC4.2).
  \end{enumerate}
  \item[Postcondizione:] L'amministratore ha deciso se avviare o interrompere la predizione sui nodi.
\end{description}

\subsubsection{UC4.1 - Avvio plug-in}
\label{sssec:uc4.1}
\begin{description}
  \item[Attore primario:] Utente amministratore.
  \item[Descrizione:] L'amministratore ha deciso di avviare il plug-in. La predizione verrà eseguita sui nodi selezionati precedentemente.
  \item[Precondizione:] L'amministratore ha configurato correttamente il plug-in (UC3).
  \item[Scenario principale:]
  \begin{enumerate}
    \item L'amministratore ha deciso di avviare la predizione;
    \item La conferma del corretto avvio della predizione viene visualizzato per mezzo di un messaggio.
  \end{enumerate}
  \item[Postcondizione:] È stata svolta la predizioni sui nodi scelti in precedenza.
\end{description}

\subsubsection{UC4.2 - Interruzione del plug-in}
\label{sssec:uc4.2}
\begin{description}
  \item[Attore primario:] Utente amministratore.
  \item[Descrizione:] L'amministratore ha deciso di interrompere la predizione.
  \item[Precondizione:] L'amministratore ha configurato correttamente il plug-in (UC3).
  \item[Scenario principale:]
  \begin{enumerate}
    \item L'amministratore ha deciso di interrompe la predizione;
    \item La conferma dell'interruzione della predizione viene visualizzato per mezzo di un messaggio.
  \end{enumerate}
  \item[Postcondizione:] È stata interrotta la predizione.
\end{description}




\subsubsection{UC5 - Visualizzazione attraverso un indicatore}
\label{sssec:uc5}
\begin{itemize}
  \item \textbf{Attore primario}: Utente;
  \item \textbf{Descrizione}: L'attore sceglie di visualizzare la previsione attraverso un indicatore;
  \item \textbf{Precondizione}: L'utente deve scegliere il modo per visualizzare la previsione;
  \item \textbf{Scenario principale}: L'utente decide di visualizzare la previsione ottenuta attraverso un indicatore;
  \item \textbf{Postcondizione}: L'utente ha selezionato l'indicatore come modalità per visualizzare la previsione.
\end{itemize}


\subsubsection{UC6- Rimozione del pannello selezionato}
\label{sssec:uc6}
\begin{description}
	\item[Attore primario:] Utente amministratore.
	\item[Attore primario:] Grafana.
	\item[Precondizione:]
	\begin{enumerate}
		\item L'amministratore deve aver impostato correttamente il pannello di monitoraggio(UC3.1);
		\item L'amministratore deve aver avviato correttamente il plug-in(UC4.1).
	\end{enumerate}		
	\item[Scenario Principale:] L'amministratore seleziona il pannello di monitoraggio da rimuovere ed usando le impostazioni di Grafana lo rimuove.
	\item[Postcondizione:] L'amministratore rimuove il pannello di monitoraggio selezionato.
	\item[Estensioni:] UC6 viene esteso nel caso d'uso UC15 con la visualizzazione del messaggio di errore quando si tenta di rimuovere un pannello senza aver interrotta la predizione.
\end{description}

\subsubsection{UC7 - Scelto modello regressione lineare}%
\label{sssec:uc7}
\begin{itemize}
  \item \textbf{Attore primario}: Utente;
  \item \textbf{Descrizione}: L'utente ha scelto come modello la regressione lineare;
  \item \textbf{Precondizione}:
  \begin{enumerate}
    \item L'inserimento dei dati è avvenuto correttamente (UC1);
    \item L'utente deve scegliere un modello di machine learning.
  \end{enumerate}
  \item \textbf{Scenario principale}: L'utente ha selezionato la RL come algoritmo per svolgere l'addestramento;
  \item \textbf{Postcondizione}: L'utente ha scelto come modello la RL.
\end{itemize}


\subsubsection{UC8 - Visualizzazione messaggio di errore invalidazione dati file di addestramento}
\label{sssec:uc8}
\begin{description}
	\item[Attore primario:] Utente amministratore.
	\item[Precondizione:]
	\begin{enumerate}
		\item L'amministratore ha caricato i dati necessari per l'addestramento in un formato non valido(UC1.1);
		\item L'amministratore ha avviato l'addestramento del predittore in Grafana(UC1).
	\end{enumerate}
	\item[Scenario Principale:] L'amministratore visualizza il messaggio di errore "file non valido per l'addestramento" impedendo l'entrata a regime dell'addestramento.
	\item[Postcondizione:]
	\begin{enumerate}
		\item L'amministratore visualizza il messaggio di errore "file non valido per l'addestramento";
		\item L'addestramento non entra in funzione.
	\end{enumerate}
\end{description}

\subsubsection{UC9 - Visualizzazione messaggio di errore file di addestramento invalido}
\label{sssec:uc9}
\begin{description}
	\begin{enumerate}
		\item[Attore primario:] Utente amministratore.
		\item[Precondizione:] L'amministratore ha caricato il plug-in(UC3) senza aver dato in ingresso un file di addestramento valido.
		\item[Scenario Principale:] L'utente visualizza il messaggio di errore "nessun file di addestramento valido" impedendo il funzionamento del plug-in (UC3.1).
		\item[Postcondizione:]
		\begin{enumerate}
			\item L'utente visualizza il messaggio di errore "nessun file di addestramento valido" .
			\item Il plug-in non entra in funzione.
		\end{enumerate}
	\end{enumerate}
\end{description}

\subsubsection{UC10 - Scelto modello rete neurale}
\label{sssec:uc10}
\begin{itemize}
  \item \textbf{Attore primario}: Utente;
  \item \textbf{Descrizione}: L'utente ha scelto come modello la rete neurale;
  \item \textbf{Precondizione}:
  \begin{enumerate}
    \item L'inserimento dei dati è avvenuto correttamente (UC1);
    \item L'utente deve scegliere un modello di machine learning.
  \end{enumerate}
  \item \textbf{Scenario principale}: L' utente ha selezionato la rete neurale come algoritmo per svolgere l'addestramento;
  \item \textbf{Postcondizione}: L'utente ha scelto come modello la rete neurale.
\end{itemize}


\subsubsection{UC11 - Scelto modello regressione esponenziale}
\label{sssec:uc11}
\begin{itemize}
  \item \textbf{Attore primario}: Utente;
  \item \textbf{Descrizione}: L'utente ha scelto come modello la regressione esponenziale;
  \item \textbf{Precondizione}:
  \begin{enumerate}
    \item L'inserimento dei dati è avvenuto correttamente (UC1.1);
    \item L'utente deve scegliere un modello di machine learning.
  \end{enumerate}
  \item \textbf{Scenario principale}: L'utente seleziona la regressione esponenziale come algoritmo per svolgere l'addestramento;
  \item \textbf{Postcondizione}: L'utente ha scelto come modello la regressione esponenziale.
\end{itemize}


\subsubsection{UC12 - Visualizzazione messaggio di errore nessun nodo collegato}
\label{sssec:uc12}
\begin{description}
	\item[Attore primario:] Utente amministratore.
	\item[Precondizione:] L'amministratore ha configurato il plug-in(UC3) senza aver collegato nessun nodo.
	\item[Scenario Principale:] L'amministratore visualizza il messaggio di errore "nessun nodo collegato" impedendo il funzionamento del plug-in(UC3.2) .
	\item[Postcondizione:]
	\begin{enumerate}
		\item L'amministratore visualizza il messaggio di errore "nessun nodo collegato";
		\item Il plug-in non entra in funzione.
	\end{enumerate}
\end{description}

\subsubsection{UC13 - Visualizzazione messaggio di errore di tipo di visualizzazione non definita}
\label{sssec:uc13}
\begin{description}
	\begin{enumerate}
		\item[Attore primario:] Utente amministratore.
		\item[Precondizione:] L'amministratore ha caricato il plug-in(UC3) senza aver segnato il tipo di visualizzazione da usare.
		\item[Scenario Principale:] L'utente visualizza il messaggio di errore "visualizzazione non definita" impedendo il funzionamento del plug-in (UC7).
		\item[Postcondizione:]
		\begin{enumerate}
			\item L'utente visualizza il messaggio di errore "visualizzazione non definita" .
			\item Il plug-in non entra in funzione.
		\end{enumerate}
	\end{enumerate}
\end{description}

\subsubsection{UC14 - Configurazione alert}
\label{sssec:uc14}

\begin{figure}[h!]
  \begin{center}
    \includegraphics[width=10cm]{uc14.png}\\
    \caption{UC14 - Configurazione alert}%
    \label{fig:uc14}
  \end{center}
  \end{figure}

\begin{itemize}
  \item \textbf{Attore primario}: Utente;
  \item \textbf{Attore secondario}: Grafana;
  \item \textbf{Descrizione}: L'utente configura l'alert e imposta la soglia massimale;
  \item \textbf{Precondizione}:
  \begin{enumerate}
		\item L'utente ha configurato il plug-in correttamente(UC3);
		\item L'utente avvia il plug-in (UC12).
	\end {enumerate}
  \item \textbf{Scenario principale}:
  \begin{enumerate}
    \item L'utente definisce l'alert usando l'opzione "Crea alert"(UC14.1);
    \item L'utente definisce una soglia per l'alert appena creato, utilizzando i meccanismi offerti da Grafana(UC14.2).
  \end{enumerate}
  \item \textbf{Postcondizione}: L'utente ha impostato correttamente la soglia dell' alert.
\end{itemize}


\paragraph{UC14.1 - Creazione alert}
\label{para:uc14.1}
\begin{itemize}
  \item \textbf{Attore primario}: Utente;
  \item \textbf{Attore secondario}: Grafana;
  \item \textbf{Precondizione}: L'utente si trova sul pannello di predizione e il sistema permette di inserire un alert;
  \item \textbf{Scenario principale}: L'utente aggiunge un alert tramite l'opzione "Crea alert";
  \item \textbf{Postcondizione}: L'alert è stato creato.
\end{itemize}


\paragraph{UC14.2 - Definizione soglia massimale}
\label{para:uc14.2}
\begin{itemize}
  \item \textbf{Attore primario}: Utente;
  \item \textbf{Attore secondario}: Grafana;
  \item \textbf{Precondizione}: L'utente ha a disposizione un alert(UC14.1);
  \item \textbf{Scenario principale}: L'utente tramite Grafana definisce la soglia per l'alert e invia la conferma della creazione del suddetto;
  \item \textbf{Postcondizione}: L'utente ha impostato la soglia in modo che parta l'alert tramite Grafana.
\end{itemize}


\subsubsection{UC15 - Rimozione del pannello selezionato}
\label{sssec:uc15}
\begin{itemize}
  \item \textbf{Attore primario}: Utente;
  \item \textbf{Attore secondario}: Grafana;
  \item \textbf{Descrizione}: L'utente seleziona il pannello che vuole rimuovere dalla dashboard;
  \item \textbf{Precondizione}:
  \begin{enumerate}
		\item L'utente deve aver impostato correttamente il pannello di monitoraggio(UC3.1);
		\item L'utente deve aver avviato correttamente il plug-in(UC12).
	\end{enumerate}
  \item \textbf{Scenario principale}: L'utente seleziona il pannello di monitoraggio da eliminare ed usando le impostazioni di Grafana lo rimuove;
  \item \textbf{Postcondizione}: La dashboard ha subito la rimozione del pannello, come desiderato dall'utente;
  \item \textbf{Estensioni}: UC15 viene esteso nel caso d'uso UC24 con la visualizzazione del messaggio di errore quando si tenta di rimuovere un pannello senza che venga interrotta la predizione.
\end{itemize}


\subsubsection{UC16 - Dati di bontà dei modelli di precisione}
\label{sssec:uc16}
\begin{itemize}
  \item \textbf{Attore primario}: Utente;
  \item \textbf{Attore secondario}: Grafana;
  \item \textbf{Descrizione}: L'utente visualizza un pannello contenente la bontà della previsione effettuata dal plug-in. La misura della bontà viene identificata con F-Measure per quanto riguarda la SVM mentre con "R\textsuperscript{2}" per la RL;
  \item \textbf{Precondizione}: L'utente ha effettuato la previsione con il plug-in;
  \item \textbf{Scenario principale}: L'utente decide di visualizza la bontà della previsione ottenuta dal plug-in utilizzando un pannello attraverso Grafana;
  \item \textbf{Postcondizione}: L'utente visualizza la bontà della previsone ottenuta dal plug-in;
  \item \textbf{Generalizzazione}: UC16 viene generalizzato dai casi d'uso UC17 e UC18.
\end{itemize}


\subsubsection{UC17 - Visualizzazione messaggio di errore dati di flusso non conformi al predittore dato}
\label{sssec:uc17}
\begin{description}
	\begin{enumerate}
		\item[Attore primario:] Utente.
		\item[Precondizione:] L'utente ha caricato un predittore(UC3.1) non conforme ai dati di flusso (UC2.1).	
		\item[Scenario Principale:] l'utente visualizza il messaggio di errore "predittore non conforme al flusso di dati" impedendo il funzionamento del plug-in(UC3).
		\item[Postcondizione:]
		\begin{enumerate}		
			\item L'utente visualizza il messaggio di errore "predittore non conforme al flusso di dati".
			\item Il plug-in non entra in funzione.
		\end{enumerate}	
	\end{enumerate}
\end{description}

\subsubsection{UC18 - Bontà della SVM: F-Measure}
\label{sssec:uc18}
\begin{itemize}
  \item \textbf{Attore primario}: Utente;
  \item \textbf{Attore secondario}: Grafana;
  \item \textbf{Descrizione}: L'utente visualizza la bontà della F-Measure;
  \item \textbf{Precondizione}: L'utente ha effettuato la previsione utilizzando SVM;
  \item \textbf{Scenario principale}: L'utente visualizza il pannello contenente la bontà della previsione identificata con F-Measure;
  \item \textbf{Postcondizione}: L'utente visualizza la bontà della previsone ottenuta dal plug-in.
\end{itemize}


\subsubsection{UC19 - Bontà della RL: R\textsuperscript{2}}
\label{sssec:uc19}
\begin{itemize}
  \item \textbf{Attore primario}: Utente;
  \item \textbf{Attore secondario}: Grafana;
  \item \textbf{Descrizione}: L'utente visualizza la bontà della R\textsuperscript{2};
  \item \textbf{Precondizione}: L'utente ha effettuato la previsione utilizzando RL;
  \item \textbf{Scenario principale}: L'utente visualizza il pannello contenente la bontà della previsione identificata con R\textsuperscript{2};
  \item \textbf{Postcondizione}: L'utente visualizza la bontà della previsone ottenuta dal plug-in.
\end{itemize}


\subsubsection{UC20 - Configurazione alert}
\label{sssec:uc20}

\begin{figure}[h!]
  \begin{center}
    \includegraphics[width=10cm]{uc20.png}\\
    \caption{UC20 - Configurazione alert}%
    \label{fig:uc20}
  \end{center}
  \end{figure}

\begin{itemize}
  \item \textbf{Attore primario}: Utente;
  \item \textbf{Attore secondario}: Grafana;
  \item \textbf{Descrizione}: L'utente configura l'alert e imposta la soglia massimale;
  \item \textbf{Precondizione}:
  \begin{enumerate}
		\item L'utente ha configurato il plug-in correttamente(UC5);
		\item L'utente avvia il plug-in (UC15).
	\end {enumerate}
  \item \textbf{Scenario principale}:
  \begin{enumerate}
    \item L'utente definisce l'alert usando l'opzione "Crea alert"(UC20.1);
    \item L'utente definisce una soglia per l'alert appena creato, utilizzando i meccanismi offerti da Grafana(UC20.2).
  \end{enumerate}
  \item \textbf{Postcondizione}: L'utente ha impostato correttamente la soglia dell' alert.
\end{itemize}


\paragraph{UC20.1 - Creazione alert}
\label{para:uc20.1}
\begin{itemize}
  \item \textbf{Attore primario}: Utente;
  \item \textbf{Attore secondario}: Grafana;
  \item \textbf{Descrizione}: L'utente crea un alert usando l'opzione di creazione contenuta nel pannello di predizione;
  \item \textbf{Precondizione}: L'utente si trova sul pannello di predizione e il sistema permette di inserire un alert;
  \item \textbf{Scenario principale}: L'utente aggiunge un alert tramite l'opzione "Crea alert";
  \item \textbf{Postcondizione}: L'alert è stato creato.
\end{itemize}


\paragraph{UC20.2 - Definizione soglia massimale}
\label{para:uc20.2}
\begin{itemize}
  \item \textbf{Attore primario}: Utente;
  \item \textbf{Descrizione}: L'utente sceglie la soglia massimale per l'alert in creazione;
  \item \textbf{Precondizione}: L'utente ha a disposizione un alert(UC20.1);
  \item \textbf{Scenario principale}: L'utente tramite Grafana definisce la soglia per l'alert e invia la conferma della creazione del suddetto;
  \item \textbf{Postcondizione}: L'utente ha impostato la soglia in modo che parta l'alert tramite Grafana.
\end{itemize}


\subsubsection{UC21 - Rimozione del pannello selezionato}
\label{sssec:uc21}
\begin{itemize}
  \item \textbf{Attore primario}: Utente;
  \item \textbf{Attore secondario}: Grafana;
  \item \textbf{Descrizione}: L'utente seleziona il pannello che vuole rimuovere dalla dashboard;
  \item \textbf{Precondizione}:
  \begin{enumerate}
		\item L'utente deve aver impostato correttamente il pannello di monitoraggio(UC5.1);
		\item L'utente deve aver avviato correttamente il plug-in(UC14).
	\end{enumerate}
  \item \textbf{Scenario principale}: L'utente seleziona il pannello di monitoraggio da eliminare ed usando le impostazioni di Grafana lo rimuove;
  \item \textbf{Postcondizione}: La dashboard ha subito la rimozione del pannello, come desiderato dall'utente;
  \item \textbf{Estensioni}: UC21 viene esteso nel caso d'uso UC30 con la visualizzazione del messaggio di errore quando si tenta di rimuovere un pannello senza che venga interrotta la predizione.
\end{itemize}


\subsubsection{UC22 - Visualizzazione dei pannelli di previsione attivati}
\label{sssec:uc22}
\begin{itemize}
  \item \textbf{Attore primario}: Utente;
  \item \textbf{Descrizione}: L'utente visualizza le specifiche del pannello attivo selezionato;
  \item \textbf{Precondizione}:
  \begin{enumerate}
		\item L'utente deve aver impostato correttamente il plug-in(UC5);
		\item L'utente deve aver avviato correttamente il plug-in(UC15);
		\item L'utente esegue l'accesso alla sezione di visualizzazione dei pannelli di previsione attivati;
		\item L'utente seleziona il pannello di previsione di cui vuole visualizzare i dati.
	\end{enumerate}
  \item \textbf{Scenario principale}: L'utente, dopo aver eseguito l'accesso alla visualizzazione dei pannelli, visualizza le specifiche del pannello attivo selezionato;
  \item \textbf{Postcondizione}: L'utente visualizza le specifiche del pannello: indicatore, grafico di previsione e affidabilità della previsione.
\end{itemize}


\subsubsection{UC23 - Visualizzazione alert superamento soglia}
\label{sssec:uc23}
\begin{itemize}
  \item \textbf{Attore primario}: Utente;
  \item \textbf{Precondizione}:
  \begin{enumerate}
		\item L'utente ha impostato una soglia valida(UC14.2);
		\item L'algoritmo di previsione prevede di superare la soglia.
	\end{enumerate}
  \item \textbf{Scenario principale}:  L'utente visualizza l'alert nel pannello indicando il superamento della soglia impostata;
  \item \textbf{Postcondizione}:
  \begin{enumerate}
		\item L'utente visualizza un messaggio di superamento della soglia impostata;
		\item Nel pannello viene indicato il messaggio di errore tramite un alert.
	\end{enumerate}
\end{itemize}


\subsubsection{UC24 - Visualizzazione messaggio di errore del plug-in di predizione non interrotta}
\label{sssec:uc24}
\begin{itemize}
  \item \textbf{Attore primario}: Utente;
  \item \textbf{Precondizione}:
  \begin{enumerate}
		\item L'utente ha selezionato la rimozione di un pannello(UC15);
		\item Il plug-in di predizione non si interrompe.
	\end{enumerate}
  \item \textbf{Scenario principale}: L'utente visualizza il messaggio di errore "previsione non interrotta" impedendo l'eliminazione del pannello scelto;
  \item \textbf{Postcondizione}:
  \begin{enumerate}
		\item L'utente visualizza il messaggio di errore "previsione non interrotta";
		\item Il pannello selezionato non viene eliminato.
	\end{enumerate}
\end{itemize}


\subsubsection{UC25 - Visualizzazione messaggio di errore collegamento nodo}
\label{sssec:uc25}
\begin{itemize}
  \item \textbf{Attore primario}: Utente;
  \item \textbf{Descrizione}: L'utente visualizza il messaggio di errore causato dalla mancanza di nodi selezionati validi;
  \item \textbf{Precondizione}: L'utente ha configurato il plug-in(UC5) senza aver collegato nessun nodo valido;
  \item \textbf{Scenario Principale}: L'attore visualizza il messaggio di errore "nessun nodo valido collegato" impedendo il funzionamento del plug-in(UC5.2);
  \item \textbf{Postcondizione}:
  \begin{enumerate}
		\item L'utente visualizza il messaggio di errore "nessun nodo valido collegato";
		\item Il plug-in non può essere avviato.
	\end{enumerate}
\end{itemize}


\end{document}
