\subsubsection{UC13 - Interruzione del plug-in}
\label{sssec:uc13}
\begin{itemize}
  \item \textbf{Attore primario}: Utente;
  \item \textbf{Attore secondario}: Grafana;
  \item \textbf{Descrizione}: L'utente ha deciso di interrompere il plug-in;
  \item \textbf{Precondizione}: L'utente ha configurato correttamente il plug-in(UC3);
  \item \textbf{Scenario principale}: L'utente ha deciso di interrompere la predizione;
  \item \textbf{Postcondizione}: La predizione sui nodi selezionati precedentemente è stata interrotta.
\end{itemize}
