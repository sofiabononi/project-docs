\subsection{Requisiti di funzionalità}
\label{sub:requisiti_di_funzionalita}

\rowcolors{2}{white!80!lightgray!90}{white}
\renewcommand{\arraystretch}{2} % allarga le righe con dello spazio sotto e sopra
\begin{longtable}[H]{>{\centering\bfseries}m{2cm} >{\centering}m{9cm} >{\centering}m{2.5cm} >{\centering\arraybackslash}m{2.5cm}}
  \caption{Requisiti funzionali}%
  \label{tab:requisiti_funzionali}                                                    \\
  \rowcolor{lightgray}
  {\textbf{Requisito}} & {\textbf{Descrizione}} & {\textbf{Priorità}} & {\textbf{Fonte}}  \\
  \endfirsthead%
  \rowcolor{lightgray}
  {\textbf{Requisito}} & {\textbf{Descrizione}} & {\textbf{Priorità}} & {\textbf{Fonte}}  \\
  \endhead%
  \rowcolor{white}
  \multicolumn{4}{c}{\textit{Continua alla pagina successiva}}
  \endfoot%
  \endlastfoot%
  \textbf{RFO1} & Il sistema deve essere addestrato dall'utente. & Obbligatorio & UC1 \\
  \textbf{RFO1.1} & Il sistema deve mostrare un pulsante che permetta all'utente di caricare il file CSV contenente i dati di addestramento.  & Obbligatorio & UC1.1 \\
  \textbf{RFD1.1.1} & Il sistema potrebbe notificare un messaggio di errore nel caso in cui venga fornito un file di addestramento non valido. & Desiderabile & UC22 \\
  \textbf{RFO1.2} & Il sistema deve mostrare un pulsante che permetta all'utente di caricare il predittore allenato in precedenza, se ne è in possesso. & Obbligatorio & UC1.2 \\
  \textbf{RFD1.2.1} & Il sistema potrebbe notificare all'utente un messaggio d'errore nel caso in cui si tenti di caricare un predittore in formato non valido. & Desiderabile & UC30 \\
  \textbf{RFO1.3} & Il sistema deve permettere all'utente di scegliere un modello di machine learning da utilizzare per l'allenamento. & Obbligatorio & UC1.3 \\
  \textbf{RFO1.4} & Il sistema deve permettere all'utente di avviare l'addestramento. & Obbligatorio & UC1.4 \\
  \textbf{RFO1.5} & Il sistema deve mettere a disposizione all’utente un pulsante per il caricamento del file JSON contenente il predittore allenato in precedenza. & Obbligatorio & UC1.5 \\
  \textbf{RFF2} & Il sistema può mettere a disposizione all'utente un metodo per l’addestramento direttamente in Grafana. & Facoltativo & UC2 \\
  \textbf{RFF2.1} & Il sistema può permettere all’utente di avviare l’addestramento in Grafana. & Facoltativo & UC2 \\
  \textbf{RFF2.2} & Il sistema può permettere all'utente di selezionare un flusso di dati a cui applicare la predizione. & Facoltativo & UC2.1 \\
  \textbf{RFF2.3} & Il sistema può permettere all’utente di selezionare il modello di previsione desiderato. & Facoltativo & UC2.2 \\
  \textbf{RFF3} & Il sistema può permettere all'utente di avviare l'apprendimento del flusso di dati costante per mezzo di un pannello. & Facoltativo & UC3 \\
  \textbf{RFF4} & Il sistema può permettere all'utente di disattivare l'apprendimento del flusso di dati costante per mezzo di un pannello. & Facoltativo & UC4 \\
  \textbf{RFO5} & Il sistema deve permettere all’utente di configurare il plug-in. & Obbligatorio & UC5 \\
  \textbf{RFO5.1} & Il sistema deve mettere a disposizione all’utente un metodo per caricare il modello addestrato. & Obbligatorio & UC5.1 \\
  \textbf{RFD5.1.1} & Il sistema potrebbe notificare un messaggio d'errore nel caso in cui venga caricato un file contenente il modello addestrato non valido. & Desiderabile & UC23 \\
  \textbf{RFD5.1.2} & Il sistema potrebbe notificare un messaggio d'errore nel caso in cui non venga caricato nessun file. & Desiderabile & UC24 \\
  \textbf{RFO5.2} & Il sistema deve consentire all’utente di selezionare i nodi su cui desidera effettuare la predizione. & Obbligatorio & UC5.2 \\
  \textbf{RFD5.2.1} & Il sistema potrebbe notificare un messaggio d'errore nel caso in cui venga selezionato un nodo non valido. & Desiderabile & UC25 \\
  \textbf{RFD5.2.2} & Il sistema potrebbe notificare un messaggio d'errore nel caso in cui non venga selezionato alcun nodo. & Desiderabile & UC26 \\
  \textbf{RFO5.3} & Il sistema deve permettere all’utente di selezionare il tipo di visualizzazione della predizione. & Obbligatorio & UC5.3 \\
  \textbf{RFD5.3.1} & Il sistema potrebbe notificare un messaggio d'errore nel caso in cui non venga selezionato alcun tipo di visualizzazione. & Desiderabile & UC27 \\
  \textbf{RFO6} & Il sistema deve mettere a disposizione la possibilità di visualizzare la previsione utilizzando un grafico. & Obbligatorio & UC6 \\
  \textbf{RFO7} & Il sistema deve mettere a disposizione la possibilità di visualizzare la previsione utilizzando un indicatore. & Obbligatorio & UC7 \\
  \textbf{RFO8} & Il sistema deve permettere all'utente di selezionare il modello SVM per l'allenamento. & Obbligatorio & UC8 \\
  \textbf{RFO9} & Il sistema deve permettere all'utente di selezionare il modello RL per l'allenamento. & Obbligatorio & UC9 \\
  \textbf{RFF10} & Il sistema può permettere all'utente di selezionare la regressione esponenziale per l'allenamento. & Facoltativo & UC10 \\
  \textbf{RFF11} & Il sistema può permettere all'utente di selezionare la regressione logaritmica per l'allenamento. & Facoltativo & UC11 \\
  \textbf{RFF12} & Il sistema può permettere all'utente di selezionare il modello rete neurale per l'allenamento. & Facoltativo & UC12 \\
  \textbf{RFF13} & Il sistema può permettere all'utente di selezionare il modello SVM adattata alla regressione per l'allenamento. & Facoltativo & UC13 \\
  \textbf{RFO14} & Il sistema deve mettere a disposizione dell’utente un pulsante per l’avvio della predizione. & Obbligatorio & UC14 \\
  \textbf{RFO15} & Il sistema deve mettere a disposizione dell’utente un pulsante per interrompere la predizione. & Obbligatorio & UC15 \\
  \textbf{RFF16} & Il sistema può mostrare all'utente il pannello contenente le informazioni riguardanti la bontà della previsione. & Facoltativo & UC16 \\
  \textbf{RFF17} & Il sistema può mostrare all'utente il pannello contenente la bontà di previsione dell'algoritmo SVM descritta dal parametro F-Measure. & Facoltativo & UC17 \\
  \textbf{RFF18} & Il sistema può mostrare all'utente il pannello contenente la bontà di previsione dell'algoritmo RL descritta dal parametro R\textsuperscript{2}. & Facoltativo & UC18 \\
  \textbf{RFF19} & Il sistema può mettere a disposizione dell’utente un metodo per l’impostazione degli alert. & Facoltativo & UC19 \\
  \textbf{RFF19.1} & Il sistema può mostrare un pulsante che permetta all'utente di creare un alert. & Facoltativo & UC19.1 \\
  \textbf{RFF19.2} & Il sistema può mettere a disposizione un sistema per inserire una soglia di alert. & Facoltativo & 19.2 \\
  \textbf{RFD19.2.1} & Il sistema, utilizzando un alert, potrebbe notificare un messaggio di errore nel caso in cui venga superata la soglia impostata. & Desiderabile & UC28 \\
  \textbf{RFD20} & Il sistema potrebbe permettere all'utente di eliminare il pannello di monitoraggio. & Desiderabile & UC20 \\
  \textbf{RFD20.1} & Il sistema potrebbe visualizzare un messaggio di errore nel caso in cui si cerchi di eliminare il pannello senza interrompere la previsione. & Desiderabile & UC29 \\
  \textbf{RFO21} & Il sistema deve mettere a disposizione all'utente un metodo per la visualizzazione delle previsioni. & Obbligatorio & UC21 \\
\end{longtable}
