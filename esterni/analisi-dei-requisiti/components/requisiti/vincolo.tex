\subsection{Requisiti di vincolo}
\label{sub:requisiti_di_vincolo}

\rowcolors{2}{white!80!lightgray!90}{white}
\renewcommand{\arraystretch}{2} % allarga le righe con dello spazio sotto e sopra
\begin{longtable}[H]{>{\centering\bfseries}m{2cm} >{\centering}m{9cm} >{\centering}m{2.5cm} >{\centering\arraybackslash}m{2.5cm}}
  \caption{Requisiti di vincolo}%
  \label{tab:requisiti_di_vincolo}                                                    \\
  \rowcolor{lightgray}
  {\textbf{Requisito}} & {\textbf{Descrizione}} & {\textbf{Priorità}} & {\textbf{Fonte}}  \\
  \endfirsthead%
  \rowcolor{lightgray}
  {\textbf{Requisito}} & {\textbf{Descrizione}} & {\textbf{Priorità}} & {\textbf{Fonte}}  \\
  \endhead%
  \rowcolor{white}
  \multicolumn{4}{c}{\textit{Continua alla pagina successiva}}
  \endfoot%
  \endlastfoot%
  \textbf{RVO1} & Il codice del plug-in deve essere scritto in linguaggio ES6. & Obbligatorio & Grafana Developer Guide \\
  \textbf{RVO2} & Va usato un sistema di build che supporti systemjs. & Obbligatorio & Grafana Developer Guide \\
  \textbf{RVO3} & Il file da usare nel plug-in che contiene l'entry point del codice dovrà essere chiamato module.js. & Obbligatorio & Grafana Developer Guide \\
  \textbf{RVO4} & Il codice prodotto dal plug-in e la pagina d'addestramento deve essere open source & Obbligatorio. & Capitolato \\
  \textbf{RVO5} & La parte del software per l'allenamento del file da inserire su Grafana deve essere una pagina web esterna. & Obbligatorio & Capitolato \\
  \textbf{RVO5.1} & La pagina web d'addestramento deve essere scritta usando HTML5 e JavaScript. & Obbligatorio & Proponente \\
  \textbf{RVO5.2} &	L'allenamento del predittore per il modello RL deve utilizzare la libreria fornita dal proponente. & Obbligatorio & Proponente \\
  \textbf{RVO5.3} &	L'allenamento del predittore per il modello SVM deve utilizzare la libreria fornita dal proponente. & Obbligatorio & Proponente \\

\end{longtable}
