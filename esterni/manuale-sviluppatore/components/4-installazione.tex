\documentclass[../manuale-sviluppatore.tex]{subfiles}

\begin{document}
I file del progetto sono divisi in tre repository per maggior chiarezza:
\begin{itemize}
  \item \texttt{project-docs}: repository contenente la documentazione;
  \item \texttt{swe-training-app}: repository contenente i sorgenti e i file di configurazione necessari allo sviluppo del programma di addestramento;
  \item \texttt{swe-grafana-plugin}: repository contenente i sorgenti e i file di configurazione necessari allo sviluppo del plug-in di Grafana.
\end{itemize}
Le tre repository sono unite in un'unica repository pubblica contenitrice, chiamata \texttt{swe-predire-in-grafana} e disponibile al seguente indirizzo: \href{https://github.com/CoffeeCodeSWE/swe-predire-in-grafana}{https://github.com/CoffeeCodeSWE/swe-predire-in-grafana}. \\
Tale repository fa uso del meccanismo dei \glossario{git submodules} per combinare più repository in una unica.

\subsection{Installazione applicazione di addestramento}%
\label{subs:installazione_applicazione_di_addestramento}
\begin{enumerate}
    \item Scaricare o clonare la repository dell'applicazione di addestramento dalla seguente risorsa web: \\ \href{https://github.com/CoffeeCodeSWE/swe-training-app}{https://github.com/CoffeeCodeSWE/swe-training-app};
    \item Spostarsi nella cartella \texttt{training-app} contenuta nella cartella della repository;
    \item Installare le dipendenze con il seguente comando da terminale :
    \begin{verbatim}
      yarn install
    \end{verbatim}
    In alternativa, è possibile utilizzare il software \glossario{npm}. Per l'installazione delle dipendenze tramite npm, eseguire il seguente comando da terminale:
    \begin{verbatim}
      npm install
    \end{verbatim}
\end{enumerate}

\subsection{Installazione plug-in}
\label{subs:installazione_plug_in}
\begin{enumerate}
    \item Scaricare o clonare la repository del plug-in dalla seguente risorsa web: \href{https://github.com/CoffeeCodeSWE/swe-grafana-plugin}{https://github.com/CoffeeCodeSWE/swe-grafana-plugin};
    \item Spostare la cartella del plug-in all'interno dell'apposita cartella di Grafana. Questa varia a seconda del tipo di installazione della piattaforma e del sistema operativo. Nello specifico:
    \begin{itemize}
      \item Se la piattaforma viene utilizzata tramite binari \textit{standalone}, la cartella si trova alla posizione \texttt{grafana/data/plugins};
      \item Se la piattaforma è globalmente installata su un sistema Windows, la cartella si trova alla posizione \texttt{FS:\textbackslash Program Files\textbackslash GrafanaLabs\textbackslash grafana\textbackslash data\textbackslash plugins}, dove con \texttt{FS} si intende la lettera del drive contenente il sistema operativo;
      \item Se la piattaforma è globalmente installata su un sistema GNU/Linux, la cartella si trova alla posizione \texttt{/var/lib/grafana/plugins} o \texttt{/usr/local/var/lib/grafana/plugins};
      \item Se la piattaforma è globalmente installata su un sistema macOS, la cartella si trova alla posizione \texttt{/usr/local/var/lib/grafana/plugins}.
    \end{itemize}
    \item Installare le dipendenze con il seguente comando da terminale :
    \begin{verbatim}
      yarn install
    \end{verbatim}
    In alternativa, è possibile utilizzare il software npm. Per l'installazione delle dipendenze tramite npm, eseguire il seguente comando da terminale:
    \begin{verbatim}
      npm install
    \end{verbatim}
    \item Compilare il plug-in con il seguente comando da terminale:
    \begin{verbatim}
      yarn build
    \end{verbatim}
    In alternativa, è possibile utilizzare il software npm. Per la compilazione tramite npm, eseguire il seguente comando da terminale:
    \begin{verbatim}
      npm build
    \end{verbatim}
    È inoltre possibile compilare il plug-in in modalità sviluppatore tramite il comando
    \begin{verbatim}
      yarn dev
    \end{verbatim}
    oppure, alternativamente
    \begin{verbatim}
      npm dev
    \end{verbatim}
    Questo compila il componente software senza ottimizzazioni e senza eseguire i test di unità automatici;
    \item Avviare Grafana;
    \item Recarsi in Configurazione $>$ Plugins $>$ \emph{CoffeeCode Prediction Panel} e abilitare il plug-in;
    \item Selezionare \emph{CoffeCode Prediction Panel} durante l'aggiunta di un pannello alla dashboard.
\end{enumerate}

\end{document}
