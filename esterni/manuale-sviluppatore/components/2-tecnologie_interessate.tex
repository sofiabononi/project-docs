\documentclass[../manuale-sviluppatore.tex]{subfiles}

\begin{document}

Il prodotto si suddivide in due parti, un applicazione web per addestrare i \glossario{predittori} e un plug-in per Grafana, le suddette parti utilizzano tecnologie che verranno elencate nei successivi paragrafi.

\subsection{Addestramento}%
\label{subs:addestramento}
Le seguenti tecnologie sono utilizzate dall'applicazione web per l'addestramento:
\begin{itemize}
  \item \textbf{Javascript}: il linguaggio con cui la parte logica della pagina è stata scritta;
  \item \textbf{HTML/CSS}: usati per gestire la visualizzazione della pagina web;
  \item \textbf{Bootstrap}: libreria usata da HTML e CSS per gestire la visualizzazione della pagina web;
  \item \textbf{NodeJS}: tecnologia runtime di Javascript costruita sul motore V8 di Google Chrome;
  \item \textbf{Yarn}: usato per gestire i pacchetti di NodeJS;
  \item \textbf{ESlint}: strumento di analisi statica del codice per identificare eventuali problemi riscontrati;
  \item \textbf{Jest}:pacchetto NodeJS usato per effettuare i test;
  \item \textbf{SVM/RL}: librerie usate per generare i predittori allenati per il plug-in;
  \item \textbf{JQuery}: libreria Javascript usata per semplificare la procedura di scrittura del codice;
  \item \textbf{ChartJS}: libreria Javascript usata per la visualizzazione dei dati in grafici;
  \item \textbf{Electron}: usato per lo sviluppo di ambienti web in locale.
\end{itemize}

\subsection{Plug-in}%
\label{subsc:plug_in}
Le seguenti tecnologie sono utilizzate nel plug-in:
\begin{itemize}
  \item \textbf{Typescript}: il linguaggio con cui la parte logica della pagina è stata scritta;
  \item \textbf{Yarn}: usato per gestire i pacchetti di NodeJS;
  \item \textbf{HTML/CSS}: usati per gestire la visualizzazione della pagina web;
  \item \textbf{NodeJS}: tecnologia runtime di Javascript costruita sul motore V8 di Google Chrome;
  \item \textbf{Grafana}: software open-source per il monitoraggio di dati mediante grafici e indicatori;
  \item \textbf{Grafana/toolkit}: pacchetto di NodeJs usato per effettuare le build e test del plug-in;
  \item \textbf{React}: il framework usato da Grafana per gestire la visualizzazione e l'interazione tra le varie pagine del plug-in.
\end{itemize}

\end{document}
