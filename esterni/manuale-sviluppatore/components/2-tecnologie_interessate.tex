\documentclass[../manuale-sviluppatore.tex]{subfiles}

\begin{document}

Il prodotto si suddivide in due parti: un'applicazione web per addestrare i \glossario{predittori} e un plug-in per la piattaforma Grafana, che
utilizza i risultati ottenuti dal programma di addestramento per effettuare la predizione dei valori reali;
le suddette parti utilizzano tecnologie che verranno elencate nei successivi paragrafi.

\subsection{Addestramento}%
\label{subs:addestramento}
Per lo sviluppo del programma di addestramento vengono utilizzate le tecnologie riportate di seguito.
\subsubsection{Linguaggi di programmazione}
\paragraph{Javascript}
JavaScript è un linguaggio di programmazione \glossario{object-oriented} ed \glossario{event-driven} utilizzato comunemente nella programmazione Web lato client e sempre più anche lato server. Esso è il linguaggio con cui la parte logica della pagina è stata scritta.
\paragraph{HTML e CSS}
HTML, acronimo di \textit{HyperText Markup Language}, è un linguaggio di markup per la definizione di pagine e documenti web; CSS, acronimo di \textit{Cascade StyleSheet}, è un linguaggio usato per definire la formattazione di documenti HTML e XML. Questi due linguaggi sono usati per gestire la visualizzazione della pagina web.

\subsubsection{Estensioni del linguaggio}
\paragraph{Node.JS}
\glossario{Node.js} è una \glossario{runtime} open-source per l'esecuzione di codice JavaScript, costruita sul motore \glossario{JavaScript V8} di \glossario{Google Chrome}. Essa viene utilizzato per la gestione dell'intero programma di addestramento.

\paragraph{Bootstrap}
Bootstrap è una raccolta di strumenti per la creazione di siti e applicazioni per il web; essa contiene modelli di progettazione basati su HTML e CSS, sia per la tipografia che per le componenti dell'interfaccia. Essa viene usata per gestire la visualizzazione della pagina web.

\paragraph{Electron}
Electron è un framework open-source che combina il \glossario{motore di rendering} \glossario{Chromium} e la runtime Node.js, che permette lo sviluppo della \glossario{GUI} di applicazioni desktop utilizzando tecnologie web. Esso è usato per lo sviluppo dell'ambiente web locale.

\paragraph{JQuery}
JQuery è una libreria JavaScript per applicazioni web, nata con l'obiettivo di semplificare la selezione, la manipolazione, la gestione degli eventi e l'animazione di elementi \glossario{DOM} in pagine web. Essa viene usata per la gestione dell'input e dell'output all'interno dell'applicazione.

\paragraph{Chart.JS}
Chart.JS è una libreria JavaScript open-source per la visualizzazione di dati. Essa viene utilizzata per la visualizzazione dei dati inseriti tramite file CSV, per semplificare la scelta delle giuste variabili da predire.

\subsubsection{Algoritmi di predizione}
\paragraph{Regressione lineare}
La regressione lineare è quella tecnica statistica utilizzata per studiare le relazioni che intercorrono tra due o più caratteri (variabili) statistici.

\paragraph{Support Vector Machines}
Le Support Vector Machines (SVM) sono dei modelli di apprendimento supervisionato associati ad algoritmi di apprendimento per la regressione e la classificazione.

\subsubsection{Strumenti di supporto}
\paragraph{Yarn}
\glossario{Yarn} è un gestore di pacchetti per il linguaggio di programmazione JavaScript. Esso viene utilizzato dal gruppo per l'installazione dei pacchetti necessari allo sviluppo e all'esecuzione del codice.

\paragraph{ESLint}
\glossario{ESLint} è uno strumento di analisi statica del codice che identifica all'interno di questo eventuali problemi o discordanze da quanto definito per la codifica.

\subsubsection{Suite di testing}
\paragraph{Jest}
Jest è un framework che permette di eseguire automaticamente i test di unità per il codice JavaScript e TypeScript. Esso viene integrato nel progetto tramite il gestore di dipendenze Yarn, e permette di essere configurato nell'apposita sezione all'interno del file \texttt{package.json}. \\
Questo strumento viene utilizzato per l'esecuzione automatica dei test di unità del codice JavaScript.

\subsection{Plug-in}%
\label{subsc:plug_in}
Per lo sviluppo del plug-in vengono utilizzate le tecnologie riportate di seguito.
\subsubsection{Linguaggi di programmazione}
\paragraph{TypeScript}
TypeScript è un linguaggio di programmazione sviluppato da \glossario{Microsoft}; esso è un \glossario{super-set} di JavaScript, basato sulle caratteristiche ECMAScript 6 di quest'utilimo. Questo linguaggio estende la sintassi di JavaScript in modo che qualunque programma scritto in JavaScript sia in grado di funzionare anche con TypeScript senza bisogno di alcuna modifica. \\
Esso viene utilizzato dal gruppo per lo sviluppo del plug-in di Grafana, poiché permette la definizione dei tipi delle variabili e delle costanti, rendendo più rigorosa e facilitando quindi la programmazione.

\paragraph{HTML e CSS}
Vengono usati HTML e CSS per gestire la visualizzazione delle componenti web del plug-in.

\subsubsection{Estensioni del linguaggio}
\paragraph{React.JS}
\glossario{React} è una libreria Javascript per lo sviluppo di applicazioni web e interfacce utente. Esso viene usato per lo sviluppo del plug-in di Grafana poiché il framework \glossario{AngularJS} è ormai deprecato per la piattaforma in questione.

\subsubsection{Piattaforme}
\paragraph{Grafana}
Grafana è un software open-source per l'analisi dati e l'interazione con questi. Questo strumento viene utilizzato come sistema sul quale sviluppare il plug-in.

\paragraph{InfluxDB}
InfluxDB è un \glossario{database} di tipo \textit{time-series}. Esso viene utilizzato per fornire i dati in input al plug-in di Grafana e per salvare i dati processati da quest'ultimo.

\subsubsection{Strumenti di supporto}
\paragraph{Yarn}
Anche per lo sviluppo del plug-in viene utilizzato lo strumento Yarn per la gestione delle dipendenze.

\paragraph{ESLint}
Anche per lo sviluppo del plug-in viene utilizzato lo strumento ESLint per il \textit{linting} automatizzato del codice.

\subsubsection{Suite di testing}
\paragraph{Jest}
Anche per lo sviluppo del plug-in viene utilizzata la suite di test Jest per l'esecuzione automatica dei test di unità del codice TypeScript.

\end{document}
