\documentclass[../manuale-sviluppatore.tex]{subfiles}

\begin{document}

\subsection{Plugin}

\subsubsection{Aggiunta di nuovi modelli di predizione}
All'interno del plugin, per gestire i diversi algoritmi di predizione dei dati, è stato utilizzato il design pattern Strategy come riportato nella seguente figura.

\begin{figure}[H]
  \centering
  \includegraphics[width=15cm]{img/plugin/strategy.png}
  \caption{Strategy per la gestione degli algoritmi di predizione.}
\end{figure}

Dalla figura si possono notare le classi dei due algoritmi di predizione, ossia \texttt{PredictionRL} e \texttt{PredictionSVM}, con la stessa interfaccia comune \texttt{Strategy}.
Tali classi contengono al loro interno un metodo \texttt{predict()} che, ricevuti in input i valori sui quali effettuare la predizione e il parametro del predittore, restituiscono una matrice contenente i risultati della predizione. \\
Nel caso quindi si volesse includere un nuovo algoritmo di predizione dei dati i passaggi da svolgere per attuare tale cambiamento sono:
\begin{enumerate}
  \item Aggiungere una nuova Strategy, ossia la classe del nuovo algoritmo di predizione. Questa dovrà includere un metodo \texttt{predict()} con la stessa segnatura degli omonimi metodi degli algoritmi di predizione già implementati;
  \begin{figure}[H]
    \centering
    \includegraphics[width=10cm]{img/plugin/newstrategy.png}
    \label{fig:scice_documenti}
    \caption{Aggiunta di un nuovo algoritmo NewPredictionClass.}
  \end{figure}
  \item Aggiornare l'array associativo contenente l'identificativo degli algoritmi, aggiungendo l'id del nuovo algoritmo;
  \item Aggiungere il file di configurazione del nuovo algoritmo. Questo file conterrà i metodi propri di configurazione del nuovo algoritmo;
  \begin{figure}[H]
    \centering
    \includegraphics[width=10cm]{img/plugin/newconfig.png}
    \label{fig:scice_documenti}
    \caption{Aggiunta di una nuova configurazione ConfigNewAlgorithm.}
  \end{figure}
  \item Aggiornare l'array associativo contenente l'identificativo delle configurazioni degli algoritmi, aggiungendo l'id della configurazione del nuovo algoritmo.
\end{enumerate}


\end{document}
