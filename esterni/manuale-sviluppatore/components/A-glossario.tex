\documentclass[../manuale-sviluppatore.tex]{subfiles}

\begin{document}

\subsection*{A}
\begin{description}
    \item[Addestramento:] In informatica, si intende l'atto di fornire dati a un'intelligenza artificiale con lo scopo di costruire un modello probabilistico generale dello spazio delle occorrenze.
    \item[Alert:] Genericamente un segnale o un messaggio (finestra) di avviso, allarme, avvertimento o conferma.
    \item[Angular:] Angular è un framework open source per lo sviluppo di applicazioni web con licenza MIT, evoluzione di AngularJS.
\end{description}

\subsection*{C}
\begin{description}
  \item[Capitolato:] Il capitolato è un documento tecnico, in genere allegato ad un contratto di appalto, al quale si fa riferimento per definire in quella sede le specifiche tecniche delle opere che andranno ad eseguirsi per effetto del contratto stesso.
  \item[Chromium:] Chromium è un web browser libero da cui deriva Google Chrome.
Uno dei principali obiettivi del progetto è di rendere Chrome un window manager a schede, o shell per il web, in contrapposizione ai tradizionali browser. L'applicazione è progettata per avere un'interfaccia utente minimalista. Gli sviluppatori affermano che "deve essere leggero (cognitivamente e fisicamente) e veloce".
\end{description}

\subsection*{D}
\begin{description}
  \item[Dashboard:] Una dashboard è una schermata che permette di monitorare in tempo reale informazioni riguardanti uno o più processi utilizzando grafici, riepiloghi o liste.
  \item[Database:] Un database in informatica è un archivio di dati strutturato in modo da razionalizzare la gestione e l'aggiornamento delle informazioni e da permettere lo svolgimento di ricerche complesse.
\end{description}

\subsection*{E}
\begin{description}
  \item[Electron:] Electron è un framework open source gestito e ospitato da GitHub.
  \item[Event-driven:] Un'architettura event-driven, detta anche programmazione a eventi, è un paradigma di programmazione dell'informatica. Nei programmi scritti usando la tecnica ad eventi il flusso del programma è determinato in gran parte dal verificarsi di eventi esterni.
\end{description}

\subsection*{F}
\begin{description}
  \item[Framework:] Un framework, in informatica e specificamente nello sviluppo software, è un'architettura logica di supporto sul quale un software può essere progettato e realizzato, facilitandone lo sviluppo da parte del programmatore.
\end{description}

\subsection*{G}
\begin{description}
  \item[Git submodules:] Un submodule, come dice la documentazione ufficiale di Git, è un repository git inserito nel working tree di un altro repository git, come dipendenza.
  \item[Google Chrome:] Google Chrome, detto anche semplicemente Chrome, è un browser web sviluppato da Google, basato sul motore di rendering Blink (a partire dalla versione 28, precedentemente sfruttava WebKit).
Basato sul browser Chromium, Chrome, nel corso degli anni, è cresciuto a tal punto da diventare il browser più usato al mondo nell'aprile 2016 con una percentuale del 41,81\% secondo il sito Netmarketshare.
  \item[Grafana:] Grafana è un software multipiattaforma open source che provvede diagrammi, grafici e allarmi per poter analizzare i dati. È espandibile tramite un sistema di plug-in.
  \item[GUI:] L'interfaccia grafica, nota anche come GUI (dall'inglese Graphical User Interface), in informatica è un tipo di interfaccia utente che consente l'interazione uomo-macchina in modo visuale utilizzando rappresentazioni grafiche (es. widget) piuttosto che utilizzando i comandi tipici di un'interfaccia a riga di comando.
\end{description}

\subsection*{I}
\begin{description}
  \item [InfluxDB:] Un database di tipo time-series, utilizzato dal gruppo per fornire i dati in input al plug-in di predizione sviluppato per la piattaforma Grafana.
\end{description}

\subsection*{J}
\begin{description}
  \item[JavaScript:] In informatica JavaScript è un linguaggio di programmazione orientato agli oggetti e agli eventi, comunemente utilizzato nella programmazione Web lato client per la creazione, in siti web e applicazioni web, di effetti dinamici interattivi tramite funzioni di script invocate da eventi innescati in vari modi dall'utente sulla pagina web.
  \item[JavaScript V8:] V8 è un motore JavaScript open source sviluppato da Google, attualmente incluso in Google Chrome.
\end{description}

\subsection*{M}
\begin{description}
  \item[Machine Learning:] L’apprendimento automatico (noto anche come machine learning) è una branca dell'intelligenza artificiale che raccoglie e utilizza un insieme di metodi statistici per migliorare progressivamente la performance di un algoritmo nell'identificare pattern nei dati.
  \item[Microsoft:] Microsoft è un'azienda di informatica con sede a Redmond, Washington (USA), creata da Bill Gates e Paul Allen nel 1975.
  \item[Motore di rendering:] Un motore di rendering, in informatica ed in particolare nella computer grafica, è un componente hardware o software che interpreta delle informazioni in ingresso codificate secondo uno specifico formato e le elabora creandone una rappresentazione grafica.
\end{description}

\subsection*{N}
\begin{description}
  \item[Node.JS:] Node.js è una runtime Open Source di JavaScript multipiattagorma e orientato agli eventi per l'esecuzione di codice JavaScript, costruita sul motore JavaScript V8 di Google Chrome.
  \item[NPM:] Node Package Manager è un gestore di pacchetti per il linguaggio di programmazione JavaScript; nello specifico, è il gestore di pacchetti predefinito per l'ambiente Node.js.
\end{description}

\subsection*{O}
\begin{description}
  \item[Object-oriented:] La programmazione orientata agli oggetti è un paradigma di programmazione, che prevede di raggruppare in un'unica entità (la classe) sia le strutture dati che le procedure che operano su di esse, creando quindi un oggetto.
  \item[Open Source:] In informatica, il termine Open Source viene utilizzato per riferirsi ad un tipo di software o al suo modello di sviluppo o distribuzione. Un software open source è reso tale per mezzo di una licenza attraverso cui i detentori dei diritti favoriscono la modifica, lo studio, l'utilizzo e la redistribuzione del codice sorgente.s
\end{description}

\subsection*{P}
\begin{description}
  \item[Plug-in:] Il plug-in in campo informatico è un programma non autonomo che interagisce con un altro programma per ampliarne o estenderne le funzionalità originarie.
  \item[Predittori:] Una funzione dei dati, definita allo scopo di effettuare previsioni su una o più variabili.
\end{description}

\subsection*{R}
\begin{description}
  \item[Regressione lineare:] La regressione lineare è quella tecnica statistica utilizzata per studiare le relazioni che intercorrono tra due o più caratteri (variabili) statistici.
  \item[Runtime:] Runtime o run-time (tempo di esecuzione) indica il momento in cui un programma per computer viene eseguito, in contrapposizione ad altre fasi del ciclo di vita del software. 
\end{description}

\subsection*{S}
\begin{description}
  \item[Support Vector Machines:] Le Support Vector Machines (SVM) sono dei modelli di apprendimento supervisionato associati ad algoritmi di apprendimento per la regressione e la classificazione.
\end{description}

\subsection*{T}
\begin{description}
  \item [Time-series database:] Un database di tipo time-series è un sistema software ottimizzato per la memorizzazione e la manipolazione di time-series, ossia dati ordinati temporalmente.
\end{description}


\end{document}
