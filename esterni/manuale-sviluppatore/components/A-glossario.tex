\documentclass[../manuale-sviluppatore.tex]{subfiles}

\begin{document}

\subsection*{A}
\begin{description}
    \item[Addestramento:] In informatica, si intende l'atto di fornire dati a un'intelligenza artificiale con lo scopo di costruire un modello probabilistico generale dello spazio delle occorrenze.
    \item[Alert:] Genericamente un segnale o un messaggio (finestra) di avviso, allarme, avvertimento o conferma.
\end{description}

\subsection*{C}
\begin{description}
  \item[Capitolato:] Il capitolato è un documento tecnico, in genere allegato ad un contratto di appalto, al quale si fa riferimento per definire in quella sede le specifiche tecniche delle opere che andranno ad eseguirsi per effetto del contratto stesso.
\end{description}

\subsection*{D}
\begin{description}
  \item[Dashboard:] Una dashboard è una schermata che permette di monitorare in tempo reale informazioni riguardanti uno o più processi utilizzando grafici, riepiloghi o liste.
\end{description}

\subsection*{E}
\begin{description}
  \item[Electron:] Electron è un framework open source gestito e ospitato da GitHub.
\end{description}

\subsection*{F}
\begin{description}
  \item[Framework:] Un framework, in informatica e specificamente nello sviluppo software, è un'architettura logica di supporto sul quale un software può essere progettato e realizzato, facilitandone lo sviluppo da parte del programmatore.
\end{description}

\subsection*{G}
\begin{description}
  \item[Grafana:] Grafana è un software multipiattaforma open source che provvede diagrammi, grafici e allarmi per poter analizzare i dati. È espandibile tramite un sistema di plug-in.
\end{description}

\subsection*{J}
\begin{description}
  \item[JavaScript:] In informatica JavaScript è un linguaggio di programmazione orientato agli oggetti e agli eventi, comunemente utilizzato nella programmazione Web lato client per la creazione, in siti web e applicazioni web, di effetti dinamici interattivi tramite funzioni di script invocate da eventi innescati in vari modi dall'utente sulla pagina web.
\end{description}

\subsection*{M}
\begin{description}
  \item[Machine Learning:] L’apprendimento automatico (noto anche come machine learning) è una branca dell'intelligenza artificiale che raccoglie e utilizza un insieme di metodi statistici per migliorare progressivamente la performance di un algoritmo nell'identificare pattern nei dati.
\end{description}

\subsection*{O}
\begin{description}
  \item[Open Source:] In informatica, il termine Open Source viene utilizzato per riferirsi ad un tipo di software o al suo modello di sviluppo o distribuzione. Un software open source è reso tale per mezzo di una licenza attraverso cui i detentori dei diritti favoriscono la modifica, lo studio, l'utilizzo e la redistribuzione del codice sorgente.s
\end{description}

\subsection*{P}
\begin{description}
  \item[Plug-in:] Il plug-in in campo informatico è un programma non autonomo che interagisce con un altro programma per ampliarne o estenderne le funzionalità originarie.
\end{description}

\subsection*{R}
\begin{description}
  \item[Regressione lineare:] La regressione lineare è quella tecnica statistica utilizzata per studiare le relazioni che intercorrono tra due o più caratteri (variabili) statistici.  
\end{description}

\subsection*{S}
\begin{description}
  \item[Support Vector Machines:] Le Support Vector Machines (SVM) sono dei modelli di apprendimento supervisionato associati ad algoritmi di apprendimento per la regressione e la classificazione.
\end{description}

\end{document}
