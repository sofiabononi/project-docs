\documentclass[../manuale-sviluppatore.tex]{subfiles}

\begin{document}
\subsection{Addestramento}%
\label{subs:addestramento}

\subsubsection{NodeJS}
\label{sssec:nodejs}
     NodeJS è necessario per avviare il server web che servirà poi per la pagina web all'utente. La versione di NodeJS utilizzata è la 12.6.3 

\subsubsection{Yarn}
\label{sssec:yarn}
    Yarn è necessario per scaricare le dipendenze necessarie all'esecuzione del software specificate all'interno del file package.json. Per installare Yarn bisogna usare il seguente comando: \emph{yarn install} .
    
\subsubsection{Electron}
\label{sssec:electron}
    Electron è il framework di Yarn utilizzato per avviare in locale la pagina web senza l'utilizzo di un Broswer.
     
\subsection{Plug-in}%
\label{subs:plug_in}

\subsubsection{Yarn}
\label{sssec:yarn}
    Yarn è necessario per scaricare le dipendenze necessarie all'esecuzione del software specificate all'interno del file package.json. Per installare Yarn bisogna usare il seguente comando: \emph{yarn install} .
    
\subsubsection{Grafana}
\label{sssec:grafana}
    Per utilizzare il plug-in necessario avere il software Grafana installato alla versione 6.7.3, con possibili errori di GUI con versioni superiori.
    
\subsubsection{InfluxDB}
\label{sssec:influxdb}
    InfluxDB è il database che fornisce i dati per la predizione al plug-in.
    
\subsubsection{Telegraf}
\label{sssec:telegraf}
    Telegraf è lo strumento di monitoraggio che si interfaccia a InfluxDB per l'inserimento di dati.
\end{document}
