\begin{document}
\subsection*{ECMAScript 6 (o ES6)}
\addcontentsline{toc}{subsection}{ECMAScript 6 (o ES6)}
ECMAScript (o ES) è la specifica tecnica di un linguaggio di scripting, standardizzata e mantenuta da ECMA International nell'ECMA-262 ed ISO/IEC 16262.
Le implementazioni più conosciute di questo linguaggio (spesso definite come dialetti) sono JavaScript, JScript e ActionScript che sono entrati largamente in uso, inizialmente, come linguaggi lato client nello sviluppo web.

\subsection*{Efficacia}
\addcontentsline{toc}{subsection}{Efficacia}
Capacità di produrre l'effetto e i risultati voluti o sperati.

\subsection*{Efficienza}
\addcontentsline{toc}{subsection}{Efficienza}
In informatica, capacità di un software di utilizzare meno risorse possibili durante la sua esecuzione.

\subsection*{Electron}
\addcontentsline{toc}{subsection}{Electron}
Electron è un framework open source gestito e ospitato da GitHub.

\subsection*{ESLint}
\addcontentsline{toc}{subsection}{ESLint}
ESLint è uno strumento di analisi statica del codice per identificare i \textit{pattern} problematici in un codice JavaScript.

\subsection*{Ethereum}
\addcontentsline{toc}{subsection}{Ethereum}
Ethereum è una piattaforma decentralizzata per la creazione e la pubblicazione \textit{peer to peer} di contratti intelligenti.

\subsection*{Event-driven}
\addcontentsline{toc}{subsection}{Event-driven}
Un'architettura event-driven, detta anche programmazione a eventi, è un paradigma di programmazione dell'informatica. Nei programmi scritti usando la tecnica ad eventi il flusso del programma è determinato in gran parte dal verificarsi di eventi esterni.

\end{document}
