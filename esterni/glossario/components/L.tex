\begin{document}
\subsection*{LDAP}
\addcontentsline{toc}{subsection}{LDAP}
LDAP è l'acronimo di Lightweight Directory Access Protocol; esso è un protocollo standard per l'interrogazione e la modifica dei servizi di directory, come ad esempio un elenco di email.

\subsection*{Libreria}
\addcontentsline{toc}{subsection}{Libreria}
Una libreria, in informatica, è un insieme di funzioni o strutture dati predefinite e predisposte per essere collegate ad un programma software attraverso un opportuno collegamento.

\subsection*{Linting}
\addcontentsline{toc}{subsection}{Linting}
Con linting si intende l'esecuzione di un linter, ossia uno strumento che analizza il codice sorgente per contrassegnare errori di programmazione, bug o errori di formattazione del codice.

\subsection*{Liveness}
\addcontentsline{toc}{subsection}{Liveliness}
Liveness si riferisce a un insieme di proprietà di sistemi concorrenti, che richiedono un sistema per progredire nonostante il fatto che i suoi componenti in esecuzione simultanea ("processi") possano dover "alternare" in sezioni critiche, parti del programma che non possono essere simultaneamente gestito da più processi.
\end{document}
