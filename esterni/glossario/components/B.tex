\begin{document}
\subsection*{Baseline}
\addcontentsline{toc}{subsection}{Baseline}
In generale è una linea di base per un progetto. Questa può comprendere i requisiti, le motivazioni che stanno dietro alla scelta di tecnologie, framework e librerie per lo sviluppo del progetto.

\subsection*{Behaviour-Driven Developement}
\addcontentsline{toc}{subsection}{Behaviour-Driven Developement}
Il Behaviour-Driven Developement è una metodologia di sviluppo del software basata sul Test-Driven Developement (TDD).

\subsection*{Best Practice}
\addcontentsline{toc}{subsection}{Best Practice}
Una best practice, in italiano buona pratica, rappresenta le esperienze, le procedure e/o le azioni più significative e/o più efficaci relativamente a svariati contesti e oviettivi preposti. Nell'ingegneria del software, con buona pratica si intende un insieme di approcci empiricamente dimostrati allo sviluppo del software.

\subsection*{Blockchain}
\addcontentsline{toc}{subsection}{Blockchain}
La Blockchain è una struttura dati condivisa e immutabile. È definita come un registro digitale le cui voci sono raggruppate in blocchi, concatenati in ordine cronologico, e la cui integrità è garantita dall'uso della crittografia.

\subsection*{Bootstrap}
\addcontentsline{toc}{subsection}{Bootstrap}
Bootstrap è una raccolta di strumenti liberi per la creazione di siti e applicazioni per il web. Essa contiene modelli di progettazione basati su HTML e CSS.

\subsection*{Branch}
\addcontentsline{toc}{subsection}{Branch}
Un branch indica lo sviluppo di un nuovo progetto software, o di una nuova componente di esso, che parte da del codice sorgente già esistente. Nello specifico del progetto, un branch rappresenta una diramazione specifica propria di un componente del progetto, sia esso di natura software o documentativa.

\subsection*{Branching}
\addcontentsline{toc}{subsection}{Branching}
Indica l'attività di creazione di diversi branch all'interno di un progetto.

\subsection*{Briefing}
\addcontentsline{toc}{subsection}{Briefing}
Un briefing è una breve riunione di lavoro, all'interno della quale vengono impartiti ordini operativi o vengono scambiate informazioni riguardanti lo stato del compito a cui ognuno sta lavorando.

\subsection*{Browser web}
\addcontentsline{toc}{subsection}{Browser web}
Un browser è un'applicazione per l'acquisizione, la presentazione e la navigazione di risorse sul web; tali risorse sono messe a disposizione sul World Wide Web o, come nel caso del prodotto a cui questo glossario fa riferimento, su una rete locale o sullo stesso computer dove il browser è in esecuzione.

\subsection*{Business application language}
\addcontentsline{toc}{subsection}{Business application language}
Un business application language, abbreviato spesso come BAL, si riferisce a un qualsiasi linguaggio nato dal linguaggio di programmazione BASIC.

\end{document}
