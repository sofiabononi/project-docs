\begin{document}
\subsection*{Java}
\addcontentsline{toc}{subsection}{Java}
Java è un linguaggio di programmazione ad alto livello, orientato agli oggetti e a tipizzazione statica, che si appoggia sull'omonima piattaforma software di esecuzione.

\subsection*{Java JMX}
\addcontentsline{toc}{subsection}{Java JMX}
Java JMX è l'acronimo di Java Management Extension; esso è un insieme di specifiche e pattern che permettono di inserire all'interno di un'applicazione Java dei componenti per il monitoraggio di questa.

\subsection*{JavaScript}
\addcontentsline{toc}{subsection}{JavaScript}
In informatica JavaScript è un linguaggio di programmazione orientato agli oggetti e agli eventi, comunemente utilizzato nella programmazione Web lato client per la creazione, in siti web e applicazioni web, di effetti dinamici interattivi tramite funzioni di script invocate da eventi innescati in vari modi dall'utente sulla pagina web.

\subsection*{JavaScript V8}
\addcontentsline{toc}{subsection}{JavaScript V8}
V8 è un motore JavaScript open source sviluppato da Google, attualmente incluso in Google Chrome.

\subsection*{JMeter}
\addcontentsline{toc}{subsection}{JMeter}
Apache JMeter è un progetto Apache che può essere utilizzato come tool di test di carico per analizzare e misurare le performance di diversi servizi, con un focus sulle applicazioni web.

\subsection*{JSON}
\addcontentsline{toc}{subsection}{JSON}
JSON, acronimo di JavaScript Object Notation, è un formato adatto all'interscambio di dati fra applicazioni.
\end{document}
