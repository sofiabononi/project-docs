\begin{document}
\subsection*{Plug-in}
\addcontentsline{toc}{subsection}{Plug-in}
Il plug-in in campo informatico è un programma non autonomo che interagisce con un altro programma per ampliarne o estenderne le funzionalità originarie.

\subsection*{Predittore}
\addcontentsline{toc}{subsection}{Predittore}
Una funzione dei dati, definita allo scopo di effettuare previsioni su una o più variabili.

\subsection*{Preventivo}
\addcontentsline{toc}{subsection}{Preventivo}
Il preventivo è il calcolo dei costi per un determinato lavoro o servizio, proposto da un professionista ad un'azienda o ad un privato.

\subsection*{Previsione}
\addcontentsline{toc}{subsection}{Previsione}
Elaborazione relativa a fatti futuri, sulla base di congetture, indizi, dati, modelli.

\subsection*{Processi}
\addcontentsline{toc}{subsection}{Processi}
I processi sono un insieme di attività correlate tra loro, che vengono eseguite con una certa continuità per raggiungere un obiettivo preciso.

\subsection*{Product Baseline}
\addcontentsline{toc}{subsection}{Product Baseline}
La Product Baseline è la documentazione tecnica approvata che descrive la configurazione di una Continuous Integration durante la fasi di produzione e di rilascio del suo ciclo di vita.

\subsection*{Progetto}
\addcontentsline{toc}{subsection}{Progetto}
Un progetto consiste, in senso generale, nell'organizzazione di azioni nel tempo per il perseguimento di uno scopo predefinito, attraverso le varie fasi di progettazione da parte di uno o più progettisti.

\subsection*{Programmatore}
\addcontentsline{toc}{subsection}{Programmatore}
Un programmatore, in informatica, è un tecnico che, attraverso la relativa fase di programmazione, traduce o codifica l'algoritmo risolutivo di un problema dato nel codice sorgente del software da far eseguire ad un elaboratore, utilizzando un determinato linguaggio di programmazione. Vedere sezione 4.1.4.5 del documento \textsc{Norme di Progetto}.

\subsection*{Proof of concept}
\addcontentsline{toc}{subsection}{Proof of concept}
Con la locuzione inglese Proof of Concept, in acronimo PoC, si intende una realizzazione incompleta o abbozzata (sinopsi) di un determinato progetto o metodo, allo scopo di provarne la fattibilità o dimostrare la fondatezza di alcuni principi o concetti costituenti. Un esempio tipico è quello di un prototipo.

\subsection*{Pull request}
\addcontentsline{toc}{subsection}{Pull request}
Una Pull request è una richiesta, fatta all’autore originale di un software o di un documento, di includere le nostre personali modifiche al suo progetto.

\subsection*{Python}
\addcontentsline{toc}{subsection}{Python}
Python è un linguaggio di programmazione ad alto livello orientato agli oggetti, ideato all'inizio degli anni 90.

\end{document}
