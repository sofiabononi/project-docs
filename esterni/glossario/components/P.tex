\begin{document}
\subsection*{Plug-in}
\addcontentsline{toc}{subsection}{Plug-in}

\subsection*{Predittore}
\addcontentsline{toc}{subsection}{Predittore}

\subsection*{Preventivo}
\addcontentsline{toc}{subsection}{Preventivo}

\subsection*{Previsione}
\addcontentsline{toc}{subsection}{Previsione}

\subsection*{Processi}
\addcontentsline{toc}{subsection}{Processi}

\subsection*{Progetto}
\addcontentsline{toc}{subsection}{Progetto}

\subsection*{Programmatore}
\addcontentsline{toc}{subsection}{Programmatore}

\subsection*{Pull request}
\addcontentsline{toc}{subsection}{Pull request}
Una Pull request è una richiesta, fatta all’autore originale di un software o di un documento, di includere le nostre personali modifiche al suo progetto.

\end{document}
