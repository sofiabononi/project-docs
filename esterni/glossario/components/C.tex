\begin{document}
\subsection*{CamelCase}
\addcontentsline{toc}{subsection}{CamelCase}
CamelCase, detta anche notazione a cammello, è la pratica di scrivere parole composte o frasi unendo tutte le parole tra loro, ma lasciando le loro iniziali maiuscole.
\subsection*{Capitolato}
\addcontentsline{toc}{subsection}{Capitolato}
Il capitolato è un documento tecnico, in genere allegato ad un contratto di appalto, al quale si fa riferimento per definire in quella sede le specifiche tecniche delle opere che andranno ad eseguirsi per effetto del contratto stesso.

\subsection*{Ciclo di Deming}
\addcontentsline{toc}{subsection}{Ciclo di Deming}
Il ciclo di Deming è un metodo di gestione iterativo in quattro fasi utilizzato per il controllo e il miglioramento continuo dei processi e dei prodotti.

\subsection*{Ciclo di vita}
\addcontentsline{toc}{subsection}{Ciclo di vita}
Il ciclo di vita del software, in informatica, e in particolare nell'ingegneria del software, si riferisce al modo in cui una metodologia di sviluppo scompone l'attività di realizzazione di prodotti software in sottoattività fra loro coordinate, il cui risultato finale è la realizzazione del prodotto stesso e tutta la documentazione ad esso associata: fasi tipiche includono lo studio o analisi, la progettazione, la realizzazione, il collaudo, la messa a punto, l'installazione, la manutenzione e l'estensione, il tutto ad opera di uno o più sviluppatori software.

\subsection*{Cloud}
\addcontentsline{toc}{subsection}{Cloud}
Il cloud è uno spazio di archiviazione personale, chiamato talvolta anche cloud storage, che risulta essere accessibile in qualsiasi momento ed in ogni luogo utilizzando semplicemente una qualunque connessione ad Internet.

\subsection*{Command Line}
\addcontentsline{toc}{subsection}{Command Line}
La Command Line, in italiano riga di comando, è un tipo di interfaccia utente caratterizzata da un'interazione testuale tra utente ed elaboratore. L'utente impartisce comandi testuali in input mediante tastiera alfanumerica e riceve risposte testuali in output dall'elaboratore mediante display.

\subsection*{Commit}
\addcontentsline{toc}{subsection}{Commit}
Il comando commit è l'azione che nel software Git permette di aggiugere, rimuovere o modificare i file del repository, creando ogni volta un ID unico che permette tener traccia di ogni cambiamento e poter ripristinare il codice a qualunque versione.

\subsection*{Criticità}
\addcontentsline{toc}{subsection}{Criticità}
Indica il livello di una situazione critica o problematica.

\subsection*{CVS}
\addcontentsline{toc}{subsection}{CSV}
Il comma-separated values, abbreviato in CSV, è un formato di file basato su file di testo utilizzato per l'importazione ed esportazione di una tabella di dati.
\end{document}
