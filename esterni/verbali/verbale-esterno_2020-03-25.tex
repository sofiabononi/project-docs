\documentclass{article}

\usepackage[italian]{babel}
\usepackage[margin=20mm, footskip = 20pt]{geometry}
\usepackage{graphicx}
\usepackage{subfiles}
\usepackage{hyperref}
\usepackage{nameref}
\usepackage{titlesec}
\usepackage{longtable}
\usepackage[table]{xcolor}
\usepackage{titling}
\usepackage{lastpage}
\usepackage{ifthen}
\usepackage{calc}
\usepackage{soulutf8}
\usepackage{contour}
\usepackage{float}
\usepackage{fancyhdr}
\usepackage{multirow}
\usepackage{pgfgantt}

% definizione dei percorsi in cui cercare immagini
\graphicspath{ {./}
    {./img/}
}

% setup della sottolineatura
\setuldepth{Flat}
\contourlength{0.8pt}

\newcommand{\uline}[1]{%
  \ul{{\phantom{#1}}}%
  \llap{\contour{white}{#1}}%
}

% setup dei link
\hypersetup{
  % set true if you want colored links (instead of boxes)
  colorlinks=true,
  % set to all if you want both sections and subsections linked
  linktoc=all,
  % set color for file links
  filecolor=blue,
  % set color for internal links
  linkcolor=black,
  % set url color
  urlcolor=blue,
  % set characters encoding in the bookmarks tab
  pdfencoding=unicode,
}

% setup forma \paragraph e \subparagraph
\titleformat{\paragraph}[hang]{\normalfont\normalsize\bfseries}{\theparagraph}{1em}{}
\titleformat{\subparagraph}[hang]{\normalfont\normalsize\bfseries}{\thesubparagraph}{1em}{}

% setup profondità indice di default
\setcounter{secnumdepth}{5}
\setcounter{tocdepth}{5}

\makeatletter %% non togliere, i comandi che definiscono i placeholder vanno qui
% esempio di utilizzo: \appendToGraphicspath{./img/} (un comando diverso per ogni path da includere)
% N.B.: ci DEVE essere un forward slash alla fine del path, a indicare che è una cartella.
\newcommand\appendToGraphicspath[1]{%
  \g@addto@macro\Ginput@path{{#1}}%
}

\newcommand{\setTitle}[1]{%
  \newcommand{\@placeholderTitle}{#1}%
}
\newcommand{\placeholderTitle}{\@placeholderTitle}

\newcommand{\setUso}[1]{%
  \newcommand{\@uso}{#1}%
}
\newcommand{\uso}{\@uso}

\newcommand{\setVersione}[1]{%
  \newcommand{\@versione}{#1}%
}
\newcommand{\versione}{\@versione}

\newcommand{\disabilitaVersione}{%
  \renewcommand{\setVersione}[1]{}%
  \renewcommand{\versione}{DISABILITATA}
}

\newcommand{\setResponsabile}[1]{%
  \newcommand{\@responsabile}{#1}%
}
\newcommand{\responsabile}{\@responsabile}

\newcommand{\setRedattori}[1]{%
  \newcommand{\@redattori}{#1}%
}
\newcommand{\redattori}{\@redattori}

\newcommand{\setVerificatori}[1]{%
  \newcommand{\@verificatori}{#1}%
}
\newcommand{\verificatori}{\@verificatori}

\newcommand{\setDescrizione}[1]{%
  \newcommand{\@descrizione}{#1}%
}
\newcommand{\descrizione}{\@descrizione}

\newcommand{\setModifiche}[1]{%
  \newcommand{\@modifiche}{#1}%
}

\newcommand{\modifiche}{\@modifiche}
\makeatother %% non togliere, i comandi che definiscono i placeholder vanno qui

% hook per lo script che genera il glossario
\newcommand{\glossario}[1]{\underline{#1}\textsubscript{g}}

% comandi per rendere opzionali gli elenchi di figure
\newcommand{\elencoFigure}{%
  \renewcommand{\listfigurename}{Elenco delle figure}%
  \listoffigures%
}

\newcommand{\disabilitaElencoFigure}{%
  \renewcommand{\elencoFigure}{}%
}

% comandi per rendere opzionali le tabelle
\newcommand{\elencoTabelle}{%
  \renewcommand{\listtablename}{Elenco delle tabelle}%
  \listoftables%
}

\newcommand{\disabilitaElencoTabelle}{%
  \renewcommand{\elencoTabelle}{}%
}

%Qui ci andrà il percorso delle immagini da includere in analisi dei requisiti
\appendToGraphicspath{../../commons/img/}

\setTitle{Verbale esterno --- 25/03/2020}

\setResponsabile{}

\setRedattori{Enrico Buratto}

\setVerificatori{}

\setUso{Esterno}

\setDescrizione{Verbale della videoconferenza con Zucchetti del 25/03/2020.}

\setModifiche{}

\disabilitaVersione{}

%ADDMELATER
\disabilitaElencoFigure{}
\disabilitaElencoTabelle{}

\begin{document}

\pagenumbering{gobble}

\begin{titlepage}% per non stampare il numero della pagina

  \raggedleft% allinea a destra la pagina
  \rule{1pt}{\textheight}% linea verticale
  \hspace{0.05\textwidth}% spazio tra linea e testo
  % lasciare questa riga per il corretto funziomento di \parbox
  \parbox[b]{0.75\textwidth}{% paragrafo che tiene il testo a destra della riga cambiando la larghezza il testo si muove a destra o a sinistra
  {\hspace{0.15\textwidth}\includegraphics[width=3cm,height=3cm]{logo.jpg}}\\[2\baselineskip] % logo
  {\Huge\bfseries CoffeeCode \\[0.5\baselineskip] Predire in Grafana}\\[5\baselineskip] % titolo
  {\Large\textsc{\placeholderTitle{}}}\\[6\baselineskip] % nome del documento
  {\begin{tabular}{r l}
    % testo in grassetto
    \textbf{Versione}     & \versione{}               \\
    \textbf{Approvazione} & \responsabile{}           \\
    \textbf{Redazione}    & \redattori{}              \\
    \textbf{Verifica}     & \verificatori{}           \\
    \textbf{Uso}          & \uso{}                    \\
    \textbf{Destinato a}  & CoffeeCode                \\
                          & prof.\ Vardanega Tullio   \\
                          & prof.\ Cardin Riccardo    \\
    \ifthenelse{\equal{\uso}{Esterno}}{
                          & Zucchetti Group SPA       \\
    }{}
  \end{tabular}}\\[5\baselineskip]

  {\bfseries Descrizione}\\
  {\descrizione{}}\\[2\baselineskip]
  {\texttt{coffeecodeswe@gmail.com}}\\[\baselineskip] % email
  }

\end{titlepage}


\section{Informazioni generali}%
\label{sec:informazioni_generali}

\subsection{Informazioni incontro}%
\label{sub:informazioni_incontro}

\begin{description}
  \item[Modalità] Videoconferenza con programma Skype
  \item[Data] 25/03/2020
  \item[Ora inizio] 14:30
  \item[Ora fine] 15:15
\end{description}

\subsection{Partecipanti}%
\label{sub:partecipanti}

\begin{enumerate}
  \item Sofia Bononi
  \item Enrico Buratto
  \item Ian Nicholas Di Menna
  \item Alessandro Franchin
  \item Enrico Galdeman
  \item Nicholas Miazzo
  \item Marco Nardelotto
\end{enumerate}

\subsection{Partecipanti esterni}%
\label{sub:partecipanti esterni}

\begin{enumerate}
    \item Gregorio Piccoli (Referente, Zucchetti SPA)
\end{enumerate}


\section{Ordine del Giorno}%
\label{ordine_del_giorno}
Di seguito sono riportati i punti dell'ordine del giorno che sono stati discussi durante la riunione.
\begin{enumerate}
  \item Richiesta di informazioni su eventuale formazione su Machine Learning e librerie
  \item Richiesta di chiarimenti sull'architettura generale del progetto
  \item Richiesta su dove reperire i dati per l'addestramento
  \item Richiesta di conferma sui linguaggi da utilizzare
  \item Fissare un secondo appuntamento con l'azienda
\end{enumerate}

\section{Resoconto}%
\label{resoconto}
\begin{enumerate}
  \item \textbf{Richiesta di informazioni su eventuale formazione su Machine Learning e librerie}: l'azienda, nella figura
  del Dr. Gregorio Piccoli, si è resa disponibile a fornire al gruppo la formazione
  sugli algoritmi di Machine Learning che si andranno ad utilizzare (Regressione Lineare, SVM e accenno sulle reti neurali).
  La modalità sarà più videoconferenze con il Dr. Piccoli, che nell'ordine illustrerà
  al gruppo la Regressione Lineare, le Support Vector Machines e le reti neurali, con relative librerie.
  Il Dr. Piccoli ha inoltre suggerito di provare in prima persona la regressione lineare con lo strumento
  integrato in Microsoft Excel, e di studiare preventivamente la documentazione delle librerie necessarie. La data
  della prima conferenza non è stata decisa, ma verrà fissata e a noi comunicata dal Dott. Piccoli, in quanto la situazione
  attuale rende più difficoltoso l'incontro.
  \item \textbf{Richiesta di chiarimenti sull'architettura generale del progetto}: solo pochi componenti del gruppo hanno
  partecipato all'incontro di approfondimento tenuto in aula dal Dr. Piccoli; come conseguenza di ciò,
  alla maggior parte dei componenti non era chiara l'architettura che il progetto deve avere per incontrare le
  esigenze dell'azienda. Il Dr. Piccoli ha perciò chiarito tutti i dubbi in merito dei componenti del gruppo.
  \item \textbf{Richiesta su dove reperire i dati per l'addestramento}: il Dr. Piccoli ha suggerito al gruppo di utilizzare
  i dati presenti nei \glossario{database} dei nostri progetti sviluppati durante il corso di Tecnologie Web.
  \item \textbf{Richiesta di conferma sui linguaggi da utilizzare}: l'azienda ha confermato l'utilizzo di Javascript sia per
  quanto riguarda l'applicazione client-side che per l'elaborazione dati server-side. Ha inoltre suggerito
  l'utilizzo del framework Electron per lo sviluppo dell'applicazione web client-side.
  \item \textbf{Fissare un secondo appuntamento con l'azienda}: come riportato al primo punto, il prossimo appuntamento è a
  data da destinarsi, auspicabilmente  nel tempo di una settimana.
\end{enumerate}


\end{document}
