\documentclass[../specifica-tecnica.tex]{subfiles}
\begin{document}
\subsection{Scopo del documento}
Lo scopo del presente documento è presentare i diagrammi dei package, delle classi e di sequenza delle due componenti software del progetto
 Predire in Grafana del gruppo \emph{CoffeeCode}. Tali diagrammi illustrano e contestualizzano i design pattern adottati durante la progettazione
 e lo sviluppo del sistema.

\subsection{Struttura del documento}
Il documento è diviso in tre capitoli: la presente introduzione e un capitolo per ogni componente software appartenente al progetto. \\
Per ogni componente software vengono riportati i seguenti diagrammi:
\begin{itemize}
  \item Diagrammi dei package;
  \item Diagrammi delle classi;
  \item Diagrammi di sequenza.
\end{itemize}
Segue la contestualizzazione dei design pattern adottati.

\subsection{Design Pattern}
\subsubsection{Programma di addestramento}
\paragraph{MVC push model}
Il programma di addestramento consiste in un'applicazione sviluppata basandosi sulla runtime Node.js e sul framework Electron. Vengono quindi utilizzati il 
linguaggio di markup HTML e il linguaggio di formattazione CSS per la realizzazione della vista con la quale l'utente interagisce, e il linguaggio JavaScript 
per la realizzazione della vista e della componente back-and dell'applicativo. \\
Tale componente software fa uso dell'architettura MVC (\textit{model-view-controller}) push model: una classe \texttt{Controller} si occupa di fare da tramite tra una classe \texttt{View} e 
una classe \texttt{Model}; le classi \texttt{View} e \texttt{Model} si occupano invece rispettivamente della gestione dell'interfaccia utente e delle chiamate agli opportuni metodi, implementati nelle 
altre classi.

\paragraph{Object Adapter}
Viene utilizzato il design pattern Object Adapter per la gestione delle librerie esterne, le quali si occupano dell'implementazione degli algoritmi di Regressione Lineare e 
Support Vector Machines. \\
È quindi presente un'interfaccia \texttt{ModelTrain} che si occupa di eseguire il training richiamando le classi \texttt{RLAdapter} o \texttt{SVMAdapter}, a seconda 
del metodo di predizione scelto dall'utente. \\
Il vantaggio principale di tale design pattern è la possibilità di estendere facilmente l'applicativo con ulteriori algoritmi di predizione.

\subsubsection{Plug-in}
\paragraph{MVC push model}
Viene utilizzata l'architettura MVC anche per lo sviluppo del plug-in per la piattaforma Grafana; nello specifico, questo design pattern viene utilizzato 
per la gestione dell'input dei dati dalla componente back-end di Grafana, per la loro elaborazione tramite algoritmi di Regressione Lineare e Support Vector Machines e 
per il conseguente output a schermo tramite grafico. \\
È stata implementata una classe \texttt{PanelCtrl} che funge da controller per il pannello di visualizzazione dei dati in output; essa istanzia e richiama i metodi di una classe \texttt{Model}, i quali eseguono 
i metodi di input dei dati e di predizione. I risultati di tali metodi vengono ritornati alla classe \texttt{PanelCtrl}, la quale si occupa infine di richiamare la view, consistente 
nella classe \texttt{MainPanel}. Quest'ultima classe si occupa quindi della visualizzazione a schermo dei risultati.

\paragraph{Strategy}
Viene utilizzato il design pattern Strategy per fornire un ulteriore livello di astrazione ai metodi di predizione tramite Regressione Lineare e Support Vector Machines. 
A tale scopo è presente un'interfaccia Strategy che si occupa di eseguire i calcoli interfacciandosi alle classi \texttt{ConfigRL} e \texttt{PredictionRL} per la Regressione Lineare, 
o i metodi \texttt{ConfigSVM} e \texttt{PredictionSVM} per le Support Vector Machines. \\
Analogamente a object adapter per il programma di addestramento, la definizione di un'interfaccia con il design pattern Strategy permette 
la futura espansione delle funzionalità tramite aggiunta di ulteriori algoritmi di predizione.

\end{document}