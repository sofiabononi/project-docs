\documentclass[../piano-di-progetto.tex]{subfiles}

\begin{document}

\subsection{Progettazione della technology baseline}

Di seguito vengono riportate le ore di lavoro effettive durante il periodo di progettazione della technology baseline:
\begin{table}[H]
    \centering
    \begin{tabular}{lccccccc}
      \rowcolor{lightgray}
      \textbf{Nominativo}       & \textbf{Re}      & \textbf{Am} & \textbf{An}      & \textbf{Pt} & \textbf{Pr} & \textbf{Ve} & \textbf{Ore totali} \\
        Sofia Bononi              & -          & -               & -           & 3           & -          & 6           & 9                \\
Enrico Buratto            & 3          & -               & 6           & -           & -          & -           & 9                \\
Ian Nicolas Di Menna      & -          & -               & 3           & 5           & -          & -           & 8                \\
Alessandro Franchin       & -          & -               & 3           & -           & -          & 6           & 9                \\
Enrico Galdeman           & 2          & -               & -           & 5           & -          & -           & 7                \\
Nicholas Miazzo           & -          & 7 (-1)          & -           & 2           & -          & -           & 9 (-1)           \\
Marco Nardelotto          & -          & -               & -           & -           & -          & 6           & 6                \\
\textbf{Ore totali ruolo} & \textbf{5} & \textbf{7 (-1)} & \textbf{12} & \textbf{15} & \textbf{-} & \textbf{18} & \textbf{57 (-1)}
    \end{tabular}
    \caption{Resoconto orario effettivo del periodo di progettazione della technology baseline}
  \end{table}

  \begin{table}[H]
    \centering
    \begin{tabular}{lcccccc}
      \rowcolor{lightgray}
      \textbf{Ruolo}  & \textbf{Ore previste} & \textbf{Ore effettive} & \textbf{Costo previsto} & \textbf{Costo effettivo} & \textbf{Differenza} \\
Responsabile    & 5           & 5           & € 150,00            & € 150,00            & € 0,00            \\
Amministratore  & 8           & 7           & € 160,00            & € 140,00            & -€ 20,00          \\
Analista        & 12          & 12          & € 300,00            & € 300,00            & € 0,00            \\
Progettista     & 15          & 15          & € 330,00            & € 330,00            & € 0,00            \\
Programmatore   & 0           & 0           & € 0,00              & € 0,00              & € 0,00            \\
Verificatore    & 18          & 18          & € 270,00            & € 270,00            & € 0,00            \\
\textbf{Totale} & \textbf{58} & \textbf{57} & \textbf{€ 1.210,00} & \textbf{€ 1.190,00} & \textbf{-€ 20,00}

    \end{tabular}
    \caption{Resoconto economico effettivo del periodo di progettazione della technology baseline}
  \end{table}


\subsubsection{Conclusioni}
Durante questo periodo il gruppo ha impiegato un'ora in meno rispetto al preventivo. In particolare le motivazioni sono le seguenti:
\begin{itemize}
    \item \textbf{Amministratore}: il lavoro dell'amministratore è stato portato a temine in meno tempo del previsto permettendo di risparmiare un'ora di lavoro.
\end{itemize}
In conclusione, in questo periodo, sono stati risparmiati € 20,00.

\subsubsection{Preventivo a finire}
Il bilancio economico viene chiuso in negativo di € 20,00. Ciò ci permetterà di poter investire questa cifra nei successivi periodi, qualora ce ne fosse bisogno oppure per l'implementazione di requisiti non obbligatori.


\end{document}
