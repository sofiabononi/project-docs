\documentclass[../piano-di-progetto.tex]{subfiles}

\begin{document}

\subsection{X incremento}

Di seguito vengono riportate le ore di lavoro effettive durante il periodo del decimo incremento:
\begin{table}[H]
    \centering
    \begin{tabular}{lccccccc}
      \rowcolor{lightgray}
      \textbf{Nominativo}       & \textbf{Re}      & \textbf{Am} & \textbf{An}      & \textbf{Pt} & \textbf{Pr} & \textbf{Ve} & \textbf{Ore totali} \\
      Sofia Bonomi              & 2          & -           & -          & -          & 7                & 4           & 13               \\
Enrico Buratto            & -          & 5           & -          & -          & -                & 5           & 10               \\
Ian Nicolas Di Menna      & -          & -           & -          & -          & 7                & 4           & 11               \\
Alessandro Franchin       & 4          & -           & -          & -          & -                & 8           & 12               \\
Enrico Galdeman           & -          & -           & -          & 6          & -                & 6           & 12               \\
Nicholas Miazzo           & -          & -           & -          & -          & 7                & 5           & 12               \\
Marco Nardelotto          & -          & 5           & -          & -          & 5 (-2)           & 0           & 10 (-2)          \\
\textbf{Ore totali ruolo} & \textbf{6} & \textbf{10} & \textbf{0} & \textbf{6} & \textbf{26 (-2)} & \textbf{32} & \textbf{80 (-2)}

    \end{tabular}
    \caption{Resoconto orario effettivo del periodo del decimo incremento}
  \end{table}

  \begin{table}[H]
    \centering
    \begin{tabular}{lcccccc}
      \rowcolor{lightgray}
      \textbf{Ruolo}  & \textbf{Ore previste} & \textbf{Ore effettive} & \textbf{Costo previsto} & \textbf{Costo effettivo} & \textbf{Differenza} \\
      Responsabile    & 6           & 6           & € 180,00            & € 180,00            & € 0,00            \\
Amministratore  & 10          & 10          & € 200,00            & € 200,00            & € 0,00            \\
Analista        & -           & -           & € 0,00              & € 0,00              & € 0,00            \\
Progettista     & 6           & 6           & € 132,00            & € 132,00            & € 0,00            \\
Programmatore   & 28          & 26          & € 420,00            & € 390,00            & -€ 30,00          \\
Verificatore    & 32          & 32          & € 480,00            & € 480,00            & € 0,00            \\
\textbf{Totale} & \textbf{82} & \textbf{80} & \textbf{€ 1.412,00} & \textbf{€ 1.382,00} & \textbf{-€ 30,00}

    \end{tabular}
    \caption{Resoconto economico effettivo del periodo del decimo incremento}
  \end{table}

\subsubsection{Conclusioni}
In questo periodo sono state utilizzate due ore in meno rispetto al preventivo, in particolare:
\begin{itemize}
    \item \textbf{Programmatore}: il ruolo è stato svolto in meno tempo del previsto.
\end{itemize}
In conclusione, in questo periodo, sono stati risparmiati € 30,00 rispetto al preventivo.

\subsubsection{Preventivo a finire}
Il bilancio viene chiuso in positivo di € 32,00, considerando i € 2,00 in eccesso degli scorsi periodi. Questo risparmio ci consentirà di investire più ore nell'ultimo incremento così da garantire la qualità del prodotto. 
\end{document}
