\documentclass[../piano-di-progetto.tex]{subfiles}

\begin{document}

\subsection{III incremento}

Di seguito vengono riportate le ore di lavoro effettive durante il periodo del terzo incremento:
\begin{table}[H]
    \centering
    \begin{tabular}{lccccccc}
      \rowcolor{lightgray}
      \textbf{Nominativo}       & \textbf{Re}      & \textbf{Am} & \textbf{An}      & \textbf{Pt} & \textbf{Pr} & \textbf{Ve} & \textbf{Ore totali} \\
Sofia Bononi              & -          & -          & -          & -          & 4          & 1           & 5           \\
Enrico Buratto            & -          & -          & -          & 4          & -          & -           & 4           \\
Ian Nicolas Di Menna      & -          & -          & 1          & -          & 4          & -           & 5           \\
Alessandro Franchin       & -          & 2          & -          & -          & -          & 1           & 3           \\
Enrico Galdeman           & 2          & -          & -          & -          & -          & 2           & 4           \\
Nicholas Miazzo           & -          & -          & -          & -          & -          & 4           & 4           \\
Marco Nardelotto          & -          & -          & 2          & -          & -          & 2           & 4           \\
\textbf{Ore totali ruolo} & \textbf{2} & \textbf{2} & \textbf{3} & \textbf{4} & \textbf{8} & \textbf{10} & \textbf{29}

    \end{tabular}
    \caption{Resoconto orario effettivo del periodo del terzo incremento}
  \end{table}

  \begin{table}[H]
    \centering
    \begin{tabular}{lcccccc}
      \rowcolor{lightgray}
      \textbf{Ruolo}  & \textbf{Ore previste} & \textbf{Ore effettive} & \textbf{Costo previsto} & \textbf{Costo effettivo} & \textbf{Differenza} \\
Responsabile    & 2           & 2           & € 60,00           & € 60,00           & € 0,00          \\
Amministratore  & 2           & 2           & € 40,00           & € 40,00           & € 0,00          \\
Analista        & 3           & 3           & € 75,00           & € 75,00           & € 0,00          \\
Progettista     & 4           & 4           & € 88,00           & € 88,00           & € 0,00          \\
Programmatore   & 8           & 8           & € 120,00          & € 120,00          & € 0,00          \\
Verificatore    & 10          & 10          & € 150,00          & € 150,00          & € 0,00          \\
\textbf{Totale} & \textbf{29} & \textbf{29} & \textbf{€ 533,00} & \textbf{€ 533,00} & \textbf{€ 0,00}

    \end{tabular}
    \caption{Resoconto economico effettivo del periodo del terzo incremento}
  \end{table}


\subsubsection{Conclusioni}
Il periodo è stato svolto rispettando sia il monte ore totale sia le ore per la suddivisione dei ruoli. In conclusione, per questo periodo, è stato rispettato il preventivo economico.

\subsubsection{Preventivo a finire}
Il bilancio economico viene chiuso in positivo di € 35,00. Come preventivato nello scorso consuntivo, si stima di sanare questo eccesso entro la fine del prossimo incremento.

\end{document}
