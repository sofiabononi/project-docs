\documentclass[../piano-di-progetto.tex]{subfiles}

\begin{document}

\subsection{VI incremento}

Di seguito vengono riportate le ore di lavoro effettive durante il periodo del sesto incremento:
\begin{table}[H]
    \centering
    \begin{tabular}{lccccccc}
      \rowcolor{lightgray}
      \textbf{Nominativo}       & \textbf{Re}      & \textbf{Am} & \textbf{An}      & \textbf{Pt} & \textbf{Pr} & \textbf{Ve} & \textbf{Ore totali} \\
Sofia Bononi              & 3          & -               & -          & 5           & -           & 4           & 12               \\
Enrico Buratto            & -          & -               & -          & 4           & -           & 8           & 12               \\
Ian Nicolas Di Menna      & -          & 4 (+1)          & -          & 5           & -           & 4           & 13 (+1)          \\
Alessandro Franchin       & -          & 2               & -          & -           & 7           & 2           & 11               \\
Enrico Galdeman           & -          & -               & 2          & -           & 6           & -           & 8                \\
Nicholas Miazzo           & -          & -               & 1          & 7           & -           & 4           & 12               \\
Marco Nardelotto          & 2          & -               & -          & -           & 8           & 2           & 12               \\
\textbf{Ore totali ruolo} & \textbf{5} & \textbf{6 (+1)} & \textbf{3} & \textbf{21} & \textbf{21} & \textbf{24} & \textbf{80 (+1)}

    \end{tabular}
    \caption{Resoconto orario effettivo del periodo del sesto incremento}
  \end{table}

  \begin{table}[H]
    \centering
    \begin{tabular}{lcccccc}
      \rowcolor{lightgray}
      \textbf{Ruolo}  & \textbf{Ore previste} & \textbf{Ore effettive} & \textbf{Costo previsto} & \textbf{Costo effettivo} & \textbf{Differenza} \\
Responsabile    & 5           & 5           & € 150,00            & € 150,00            & € 0,00           \\
Amministratore  & 5           & 6           & € 100,00            & € 120,00            & € 20,00          \\
Analista        & 3           & 3           & € 75,00             & € 75,00             & € 0,00           \\
Progettista     & 21          & 21          & € 462,00            & € 462,00            & € 0,00           \\
Programmatore   & 21          & 21          & € 315,00            & € 315,00            & € 0,00           \\
Verificatore    & 24          & 24          & € 360,00            & € 360,00            & € 0,00           \\
\textbf{Totale} & \textbf{79} & \textbf{80} & \textbf{€ 1.462,00} & \textbf{€ 1.482,00} & \textbf{€ 20,00}
    \end{tabular}
    \caption{Resoconto economico effettivo del periodo del sesto incremento}
  \end{table}


\subsubsection{Conclusioni}
Questo periodo ha richiesto un'ora in più rispetto al preventivo, in particolare:
\begin{itemize}
    \item \textbf{Amministratore}: a causa della poca documentazione di Grafana, è stato richiesto maggior lavoro all'Amministratore per la configurazione.
\end{itemize}
In conclusione, in questo periodo, sono stati spesi € 20,00 in più rispetto al preventivo.

\subsubsection{Preventivo a finire}
Il bilancio viene chiuso in negativo di € 18,00, considerando i € 2,00 risparmiati degli scorsi periodi. Si ritiene che il lavoro svolto sarà molto utile anche agli incrementi successivi, il che potrebbe comportare un risparmio di tempo.
\end{document}
