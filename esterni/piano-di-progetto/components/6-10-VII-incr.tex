\documentclass[../piano-di-progetto.tex]{subfiles}

\begin{document}

\subsection{VII incremento}

Di seguito vengono riportate le ore di lavoro effettive durante il periodo del settimo incremento:
\begin{table}[H]
    \centering
    \begin{tabular}{lccccccc}
      \rowcolor{lightgray}
      \textbf{Nominativo}       & \textbf{Re}      & \textbf{Am} & \textbf{An}      & \textbf{Pt} & \textbf{Pr} & \textbf{Ve} & \textbf{Ore totali} \\
Sofia Bononi              & -          & -               & -          & 4           & 4 (+1)           & 5           & 13 (+1)          \\
Enrico Buratto            & -          & -               & -          & 6           & 4 (+2)           & 2           & 12 (+2)          \\
Ian Nicolas Di Menna      & -          & 2 (+1)          & -          & 5           & -                & 5           & 12 (+1)          \\
Alessandro Franchin       & -          & 2               & -          & -           & 7                & -           & 9                \\
Enrico Galdeman           & -          & -               & 2          & -           & 5                & 3           & 10               \\
Nicholas Miazzo           & 3          & -               & -          & -           & 5                & 2           & 10               \\
Marco Nardelotto          & -          & -               & -          & 6           & -                & 5           & 11               \\
\textbf{Ore totali ruolo} & \textbf{3} & \textbf{4 (+1)} & \textbf{2} & \textbf{21} & \textbf{25 (+3)} & \textbf{22} & \textbf{77 (+4)}

    \end{tabular}
    \caption{Resoconto orario effettivo del periodo del settimo incremento}
  \end{table}

  \begin{table}[H]
    \centering
    \begin{tabular}{lcccccc}
      \rowcolor{lightgray}
      \textbf{Ruolo}  & \textbf{Ore previste} & \textbf{Ore effettive} & \textbf{Costo previsto} & \textbf{Costo effettivo} & \textbf{Differenza} \\
Responsabile    & 3           & 3           & € 90,00             & € 90,00             & € 0,00           \\
Amministratore  & 3           & 4           & € 60,00             & € 80,00             & € 20,00          \\
Analista        & 2           & 2           & € 50,00             & € 50,00             & € 0,00           \\
Progettista     & 21          & 21          & € 462,00            & € 462,00            & € 0,00           \\
Programmatore   & 22          & 25          & € 330,00            & € 375,00            & € 45,00          \\
Verificatore    & 22          & 22          & € 330,00            & € 330,00            & € 0,00           \\
\textbf{Totale} & \textbf{73} & \textbf{77} & \textbf{€ 1.322,00} & \textbf{€ 1.387,00} & \textbf{€ 65,00}
    \end{tabular}
    \caption{Resoconto economico effettivo del periodo del settimo incremento}
  \end{table}


\subsubsection{Conclusioni}
In questo periodo sono state utilizzate quattro ore in eccesso rispetto al preventivo, in particolare:
\begin{itemize}
    \item \textbf{Amministratore}: le modifiche la configurazione della CI per Grafana ha richiesto più tempo del previsto;
    \item \textbf{Programmatore}: la scarsa documentazione di Grafana su determinati argomenti, ha richiesto più ore di sviluppo.
\end{itemize}
In conclusione, in questo periodo, sono stati utilizzati € 65,00 in più rispetto al preventivo.

\subsubsection{Preventivo a finire}
Il bilancio viene chiuso in negativo di € 83,00, considerando i € 18,00 in eccesso degli scorsi periodi. Tuttavia, il gruppo ritiene che il lavoro svolto finora sarà molto utile agli successivi e permetterà un notevole risparmio di ore.
\end{document}
