\documentclass[../piano-di-progetto.tex]{subfiles}

\begin{document}

\section{Introduzione}
\subsection{Scopo del documento}
Questo documento ha come obiettivo la specifica delle modalità attraverso le quali il gruppo \emph{CoffeeCode} svilupperà il progetto \emph{Predire in Grafana}. In particolare, il documento tratta i seguenti argomenti:
\begin{itemize}
  \item Analisi dei rischi;
  \item Suddivisione dei compiti tra i vari membri del gruppo;
  \item Stima dei costi, dei tempi e delle risorse necessarie per lo sviluppo del progetto.
\end{itemize}
%TODO mettere a posto la lista

\subsection{Scopo del prodotto}
L'obiettivo del prodotto è quello di creare un \glossario{plug-in} per lo strumento di monitoraggio \glossario{Grafana}, prodotto utilizzato dal proponente per sorvegliare i propri servizi. Lo scopo del plug-in è quello di effettuare delle \glossario{previsioni} al flusso dei dati raccolti da Grafana utilizzando due modelli di \glossario{machine learning}: \glossario{SVM} e \glossario{RL}, i quali verranno opportunamente \glossario{addestrati} attraverso un'applicazione. I valori prodotti dal plug-in verranno aggiunti al flusso del monitoraggio e resi disponibili al sistema di creazione di grafici e \glossario{dashboard} per la loro visualizzazione. Con questo plug-in si cerca di monitorare la \glossario{"liveliness"} del sistema e consigliare gli interventi o le zone di intervento alla linea di produzione del software.


\subsection{Ambiguità}
Con lo scopo di evitare eventuali ambiguità relative al linguaggio utilizzato nei documenti, viene fornito il \textsc{Glossario v1.0.0}. In questo documento vengono definiti e descritti tutti i termini con un significato particolare o ambigui. I termini la cui definizione può essere trovata nel glossario saranno sottolineati e con una 'G' al pedice.

\subsection{Riferimenti}
\subsubsection{Normativi}
\begin{itemize}
  \item \textbf{Norme di progetto}: \textsc{Norme di progetto v1.0.0};
  \item \textbf{Regole di organigramma e dettagli economici}: \href{https://www.math.unipd.it/~tullio/IS-1/2019/Progetto/RO.html}{https://www.math.unipd.it/~tullio/IS-1/2019/Progetto/RO.html};
  \item \textbf{Regolamento del progetto didattico}: \href{https://www.math.unipd.it/~tullio/IS-1/2019/Dispense/PD01.pdf}{https://www.math.unipd.it/~tullio/IS-1/2019/Dispense/PD01.pdf}.
\end{itemize}

\subsubsection{Informativi}
\begin{itemize}
  \item \textbf{\glossario{Capitolato} d’appalto C4 - Predire in Grafana}: \href{https://www.math.unipd.it/~tullio/IS-1/2019/Progetto/C4.pdf}{https://www.math.unipd.it/~tullio/IS-1/2019/Progetto/C4.pdf};
  \item \textbf{Analisi dei requisiti}: \textsc{Analisi dei requisiti v1.0.0};
  \item Slide \emph{Gestione di Progetto}: \href{https://www.math.unipd.it/~tullio/IS-1/2018/Dispense/L06.pdf}{https://www.math.unipd.it/~tullio/IS-1/2018/Dispense/L06.pdf};
  \item Libro di testo: \textsc{Software Engineering di Ian Sommerville}.
\end{itemize}

\subsection{Scadenze}
\label{scadenze}
Il gruppo \emph{CoffeeCose} si impegna a rispettare le seguenti scadenze per lo sviluppo del progetto \emph{Predire in Grafana}:
\begin{itemize}
    \item \textbf{Revisione dei Requisiti}: 2020-04-20;
    \item \textbf{Revisione di Progettazione}: 2020-05-18;
    \item \textbf{Revisione di Qualifica}: 2020-06-18;
    \item \textbf{Revisione di Accettazione}: 2020-07-13.
\end{itemize}

\subsection{Fonti}
Vengono riportate le fonti delle immagini non create dal gruppo.
\begin{itemize}
    \item Figura \ref{fig:modello-incrementale}: \textsc{Software Engineering di Ian Sommerville}
\end{itemize}

\end{document}
