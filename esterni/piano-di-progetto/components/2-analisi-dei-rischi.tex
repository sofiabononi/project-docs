\documentclass[../piano-di-progetto.tex]{subfiles}

\begin{document}

\section{Analisi dei rischi}
Durante lo sviluppo di un progetto complesso, è possibile incorrere in problemi che potrebbero essere evitati tramite un processo preliminare di analisi dei rischi. Con l'obiettivo di prevenire ed evitare queste evenienze, è stata effettuata un'attenta analisi dei principali fattori di rischio svolgendo i seguenti passi:

\begin{itemize}
    \item \textbf{Identificazione dei rischi}: individuazione dei possibili fattori di rischio che potrebbero rallentare o bloccare il normale svolgersi del progetto;
    \item \textbf{Analisi dei rischi}: valutazione delle probabilità di occorrenza del rischio, valutazione delle ripercussioni sul progetto e, infine, assegnazione di un indice di grado del rischio;
    \item \textbf{Pianificazione di controllo}: definizione della metodologia per evitare di incorrere nei rischi individuati e come procedere nell'eventualità in cui non sia possibile evitarne alcuni;
    \item \textbf{Monitoraggio dei rischi}: monitoraggio continuo al fine di rilevare quanto prima l'incombere di un rischio con l'obiettivo di poterlo evitare oppure limitare le ripercussioni sullo svolgersi del progetto.
\end{itemize}

\subsection{Valutazione}
Viene utilizzato il sistema a matrice di valutazione dei rischi. Ad ogni rischio viene associato un indice di gravità, ottenuto considerando la probabilità che tale rischio possa accadere e l'impatto che esso avrà sullo svolgersi del progetto. Questo indice può assumere i seguenti valori:
\begin{itemize}
    \item Basso;
    \item Medio;
    \item Alto;
    \item Critico.
\end{itemize}
\begin{figure}[H]
	\centering
	\includegraphics[width=10cm]{img/matrice-rischio.png}
	\caption{Matrice dell'indice di gravità del rischio}
	\label{fig:matrice-rischio}
  \end{figure}

\subsection{Classificazione}
Ad ogni rischio viene assegnato un codice univoco affinché possa essere identificato e facilmente riconoscibile. La struttura del codice identificativo è la seguente: \\
\texttt{RK-[Categoria]-[ID]-[Indice di gravità]} \\
I tre campi assumono i seguenti valori:
\begin{itemize}
    \item \textbf{Categoria}:
        \begin{itemize}
            \item \textbf{O}: Organizzativo;
            \item \textbf{P}: Personale;
            \item \textbf{T}: Tecnologico.
        \end{itemize}
    \item \textbf{ID}: rappresenta un numero intero incrementale all'interno della categoria;
    \item \textbf{Indice di gravità}: valore ottenuto dalla matrice in figura \ref{fig:matrice-rischio} con la seguente associazione:
        \begin{itemize}
            \item \textbf{1}: Basso;
            \item \textbf{2}: Medio;
            \item \textbf{3}: Alto;
            \item \textbf{4}: Critico.
        \end{itemize}
\end{itemize}

\subsection{Possibili rischi}
    \begin{longtable}[H]{ccc}
        \caption{Tabella dei rischi} \\
        \rowcolor{lightgray} 
        \textbf{Codice} & \textbf{Descrizione} & \textbf{Identificazione} \\

        \endfirsthead
        \multicolumn{3}{c}%
        {\tablename\ \thetable\ -- \textit{Continuazione della pagina precedente}} \\
        \rowcolor{lightgray}

        \textbf{Codice} & \textbf{Descrizione} & \textbf{Identificazione} \\

        \endhead
         \multicolumn{3}{r}{\textit{Continua alla prossima pagina}} \\
        \endfoot

        \endlastfoot
    \begin{tabular}[r]{@{}l@{}}RK-P2-3\\ \\ Inesperienza \\ gestionale \end{tabular}                   & \begin{tabular}[r]{@{}l@{}}Nessun membro del gruppo si è mai trovato nelle \\ condizioni di lavorare ad un progetto di dimensioni \\ così elevate \end{tabular}                                             & \begin{tabular}[r]{@{}l@{}}Il Responsabile avrà il compito\\ di assicurarsi che ogni membro \\ del gruppo abbia compreso \\ pienamente il ruolo e i compiti a \\ lui assegnati \end{tabular}  \\
    \textbf{Piano di contingenza}                                                                               & \multicolumn{2}{l}{\begin{tabular}[r]{@{}l@{}}Ogni componente si impegna a studiare in modo autonomo i compiti e doveri associati \\ ad ogni ruolo e informare il Responsabile in caso di perplessità \end{tabular}}                                                                                                                                                                                      \\
   \hline \begin{tabular}[r]{@{}l@{}}RK-P3-3\\ \\ Impegni \\ accademici \end{tabular}                        & \begin{tabular}[r]{@{}l@{}}Alcuni membri del gruppo potrebbero non essere \\ sempre disponibili per continuare lo sviluppo \\ del progetto a causa di impegni accademici \end{tabular}                      & \begin{tabular}[r]{@{}l@{}}Il Responsabile verificherà \\ periodicamente lo stato degli \\ impegni accademici dei singoli \\ componenti \end{tabular}                                       \\
    \textbf{Piano di contingenza}                                                                               & \multicolumn{2}{l}{\begin{tabular}[r]{@{}l@{}}Sarà compito dei singoli individui programmare gli impegni accademici in modo tale\\ che non interferiscano con lo svolgersi del progetto. Inoltre, il Responsabile a \\ conoscenza degli impegni accademici dei singoli, assegnerà compiti compatibili \\ con essi \end{tabular}}                                                                          \\
    \hline  \begin{tabular}[r]{@{}l@{}}RK-P4-2\\ \\ Impegni \\ personali \end{tabular}                         & \begin{tabular}[r]{@{}l@{}}Alcuni membri del gruppo potrebbero essere \\ indisponibili a causa di impegni personali \end{tabular}                                                                           & \begin{tabular}[r]{@{}l@{}}I membri del gruppo dovranno \\ segnalare al Responsabile \\ eventuali indisponibilità \end{tabular}                                                             \\
    \textbf{Piano di contingenza}                                                                               & \multicolumn{2}{l}{\begin{tabular}[r]{@{}l@{}}In caso di ritardi, sarà compito del Responsabile valutare una riassegnazione dei \\ compiti in modo da poter rispettare i tempi prestabiliti \end{tabular}}                                                                                                                                                                                                \\
    \hline  \begin{tabular}[r]{@{}l@{}}RK-P5-2\\ \\ Abbandono \\ del gruppo \end{tabular}                      & \begin{tabular}[r]{@{}l@{}}Per motivi personali, un membro del gruppo \\ potrebbe decidere di abbandonare lo sviluppo \end{tabular}                                                                         & \begin{tabular}[r]{@{}l@{}}Il componente che intende \\ lasciare lo sviluppo deve \\ informare il prima possibile il\\ resto del gruppo \end{tabular}                                       \\
    \textbf{Piano di contingenza}                                                                               & \multicolumn{2}{l}{\begin{tabular}[r]{@{}l@{}}Si valuterà se l'abbandono del gruppo da parte del componente potrà diventare \\ solo un temporaneo congedo \end{tabular}}                                                                                                                                                                                                                                  \\
    \hline  \begin{tabular}[r]{@{}l@{}}RK-O1-3\\ \\ Calcolo dei \\ costi e \\ tempistiche \end{tabular}        & \begin{tabular}[r]{@{}l@{}}A causa delle dimensioni molto elevate del progetto \\ e dalla inesperienza dei membri del gruppo, le \\ stime dei tempi e dei costi potrebbe risultare incoerenti \end{tabular} & \begin{tabular}[r]{@{}l@{}}Ogni membro ha il compito di \\ informare il Responsabile nel \\ caso in cui il proprio compito \\ stia superando le ore previste \end{tabular}                  \\
    \textbf{Piano di contingenza}                                                                               & \multicolumn{2}{l}{\begin{tabular}[r]{@{}l@{}}Se le ore effettive richieste da un certo compito stanno per superare le ore \\ preventivate, verrà considerata una ripartizione di quest'ultimo \end{tabular}}                                                                                                                                                                                             \\
    \hline \begin{tabular}[r]{@{}l@{}}RK-O2-2\\ \\ Conflitti interni \end{tabular}                            & \begin{tabular}[r]{@{}l@{}}Potrebbero verificarsi situazioni di tensione \\ tra due o più membri del gruppo \end{tabular}                                                                                   & \begin{tabular}[r]{@{}l@{}}Appena questa situazione si \\ verifica, è compito dei membri\\ coinvolti mettere al corrente il\\ resto del gruppo \end{tabular}                                \\
    \textbf{Piano di contingenza}                                                                               & \multicolumn{2}{l}{\begin{tabular}[r]{@{}l@{}}Il Responsabile, o un componente non coinvolto, assumerà l'incarico di \\ mediatore per risolvere la situazione \end{tabular}}                                                                                                                                                                                                                              \\
    \hline \begin{tabular}[r]{@{}l@{}}RK-O4-2\\ \\ Comunicazione \\ con il \\ proponente \end{tabular}        & \begin{tabular}[r]{@{}l@{}}Il proponente potrebbe non essere disponibile a causa \\ di impegni lavorativi \end{tabular}                                                                                     & \begin{tabular}[r]{@{}l@{}}Sin dal primo colloquio si \\ cercherà di capire le \\ disponibilità del proponente \end{tabular}                                                                \\
    \textbf{Piano di contingenza}                                                                               & \multicolumn{2}{l}{\begin{tabular}[r]{@{}l@{}}Il gruppo cercherà di chiarire quanti più dubbi possibili durante ogni colloquio \\ preparando in anticipo una lista di domande \end{tabular}}                                                                                                                                                                                                              \\
    \hline  \multicolumn{1}{c}{\begin{tabular}[r]{@{}l@{}}RK-O5-2\\\\Impossibilità di \\incontro\end{tabular}} & \multicolumn{1}{l}{\begin{tabular}[r]{@{}l@{}}Per ragioni logistiche e personali, potrebbe essere\\difficile incontrarsi fisicamente con tutto il gruppo\end{tabular}}                                      & \multicolumn{1}{l}{\begin{tabular}[r]{@{}l@{}}Gli incontri verranno organizzati\\in anticipo cercando data e \\luogo adatti\end{tabular}}                                                   \\
    \multicolumn{1}{l}{\textbf{Piano di contingenza}}                                                           & \multicolumn{2}{l}{Nel caso in cui non si riesca ad arrivare a un compromesso, verrà organizzato un incontro telematico}                                                                                                                                                                                                                                                                                              \\
\hline \begin{tabular}[r]{@{}l@{}}RK-T1-2\\ \\ Problemi software \\ e hardware \end{tabular}              & \begin{tabular}[r]{@{}l@{}}Qualcuno potrebbe riscontrare problemi software \\ o hardware all'interno del computer, non potendo \\ proseguire nell'immediato con lo sviluppo del progetto \end{tabular}      & \begin{tabular}[r]{@{}l@{}}Ogni componente dovrà \\ segnalare tempestivamente \\ il problema riscontrato \\ tramite un dispositivo non \\ affetto da guasti \end{tabular}                   \\
    \textbf{Piano di contingenza}                                                                               & \multicolumn{2}{l}{\begin{tabular}[r]{@{}l@{}}Nel caso la riparazione non potesse essere portata a termine in tempi molto brevi, il singolo \\ dovrà utilizzare un computer sostitutivo per continuare lo sviluppo. Nell'evenienza in cui il \\ componente non avesse a disposizione nessun dispositivo allora il Responsabile dovrà \\ procedere alla riassegnazione dei compiti \end{tabular}}  \end{longtable}
\end{document}
