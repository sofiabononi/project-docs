\documentclass[../piano-di-progetto.tex]{subfiles}

\begin{document}

\section{Consuntivi di periodo}
Di seguito verranno riportate le spese effettivamente sostenute raggruppate sia per ruolo, sia per persona. Il bilancio potrà risultare:

\begin{itemize}
    \item \textbf{Positivo}: se il preventivo supera il consuntivo;
    \item \textbf{Pari}: se il consuntivo e il preventivo sono pari;
    \item \textbf{Negativo}: se il consuntivo supera il preventivo.
\end{itemize}

\subsection{Analisi}
Le ore di lavoro sostenute in questo periodo non verranno rendicontate nel preventivo finale in quanto vengono considerate come ore di investimento per l'approfondimento personale.

Di seguito vengono riportate le ore di lavoro effettive durante il periodo di analisi:
\begin{table}[H]
    \centering
    \begin{tabular}{lccccccc}
      \rowcolor{lightgray}
      \textbf{Nominativo}       & \textbf{Re}      & \textbf{Am} & \textbf{An}      & \textbf{Pt} & \textbf{Pr} & \textbf{Ve} & \textbf{Ore totali} \\
      Sofia Bononi              & -                & 8           & 14          & -           & -           & 9           & 31 \\
      Enrico Buratto            & 1 (+1)                & 14          & 8            & -           & -           & 8           & 31 (+1)             \\
      Ian Nicolas Di Menna      & 11 (+2)          & -           & 9           & -           & -           & 12          & 32  (+2)            \\
      Alessandro Franchin       & 9                & -           & 9            & -           & -           & 13          & 31              \\
      Enrico Galdeman           & -                & 7           & 14             & -           & -           & 10       & 31                   \\
      Nicholas Miazzo           & 11 (+1)          & -           & 10               & -           & -           & 10          & 31 (+1)             \\
      Marco Nardelotto          & -                & 9           & 9                & -           & -           & 13          & 31                  \\
      \textbf{Ore totali ruolo} & \textbf{32 (+4)} & \textbf{38} & \textbf{73} & \textbf{-}  & \textbf{-}  & \textbf{75} & \textbf{218 (+4)}       
      
    \end{tabular}
    \caption{Resoconto orario effettivo del periodo di analisi}
  \end{table}

  \begin{table}[H]
    \centering
    \begin{tabular}{lcccccc}
      \rowcolor{lightgray}
      \textbf{Ruolo}  & \textbf{Ore previste} & \textbf{Ore effettive} & \textbf{Costo previsto} & \textbf{Costo effettivo} & \textbf{Differenza} \\
      Responsabile    & 28                    & 32                     & € 840,00                & € 960,00                 & € 120,00    \\
      Amministratore  & 38                    & 38                     & € 760,00                & € 760,00                 & € 0,00     \\
      Analista        & 73                    & 72                     & € 1.825,00              & € 1.800,00               & € -25,00   \\
      Progettista     & -                     & -                      & € 0,00                  & € 0,00                   & € 0,00     \\
      Programmatore   & -                     & -                      & € 0,00                  & € 0,00                   & € 0,00     \\
      Verificatore    & 75                    & 75                     & € 1.125,00              & € 1.125,00               & € 0,00     \\
      \textbf{Totale} & \textbf{214}          & \textbf{218}           & \textbf{€ 4.550,00}     & \textbf{€ 4.670,00}      & \textbf{€ 120,00}   
      
    \end{tabular}
    \caption{Resoconto economico effettivo del periodo di analisi}
  \end{table}


\subsubsection{Conclusioni}
In questo periodo il gruppo ha utilizzato più ore di quelle preventivate, aumentando il totale di € 120,00. In particolare le motivazioni sono le seguenti:
\begin{itemize}
    \item \textbf{Responsabile}: questo ruolo ha richiesto più ore del previsto a causa della complessità della pianificazione delle attività future.
    \end{itemize}

\subsubsection{Preventivo a finire}
Trattandosi di un periodo che non verrà rendicontato, non è necessario adottare contromisure per la pianificazione del totale delle ore e il prospetto economico.

\subfile{6-2-cons.tex}
\subfile{6-3-proj.tex}
\subfile{6-4-I-incr.tex}
\subfile{6-5-II-incr.tex}
\subfile{6-6-III-incr.tex}
\subfile{6-7-IV-incr.tex}
\subfile{6-8-V-incr.tex}
\subfile{6-9-VI-incr.tex}
\subfile{6-10-VII-incr.tex}
\subfile{6-11-VIII-incr.tex}
\subfile{6-12-IX.tex}
\subfile{6-14-X.tex}
\subfile{6-15-XI.tex}

\newpage
\section{Consuntivo finale}

\subsection{Totale ore}
Di seguito viene riportata la distribuzione oraria effettiva di tutti i periodi, inclusi i periodi non rendicontati:


\begin{table}[H]
    \centering
    \begin{tabular}{lccccccc}
      \rowcolor{lightgray}
      \textbf{Nominativo}       & \textbf{Re}      & \textbf{Am} & \textbf{An}      & \textbf{Pt} & \textbf{Pr} & \textbf{Ve} & \textbf{Ore totali} \\
      Sofia Bononi              & 12               & 13                & 18                & 21               & 31 (+0)           & 48               & 143 (+0)           \\
Enrico Buratto            & 8 (+1)           & 22 (-1)           & 17                & 22               & 30 (+1)           & 44               & 143 (+1)           \\
Ian Nicolas Di Menna      & 15 (+2)          & 14 (+1)           & 17                & 20               & 28 (-1)           & 49               & 143 (+2)           \\
Alessandro Franchin       & 13               & 17 (+1)           & 18                & 19 (-1)          & 29                & 47               & 143 (+0)           \\
Enrico Galdeman           & 7                & 12                & 22                & 24               & 29 (+1)           & 49 (-1)          & 143 (+0)           \\
Nicholas Miazzo           & 17 (+1)          & 14                & 18                & 20               & 25                & 49               & 143 (+1)           \\
Marco Nardelotto          & 10               & 14                & 14                & 26               & 25 (-2)           & 54 (+2)          & 143 (+0)           \\
\textbf{Ore totali ruolo} & \textbf{82 (+4)} & \textbf{106 (+1)} & \textbf{124 (+0)} & \textbf{152(-1)} & \textbf{197 (-1)} & \textbf{340(+1)} & \textbf{1001 (+4)}
    \end{tabular}
    \caption{Resoconto orario effettivo in tutti i periodi}
  \end{table}
  
  Di seguito viene riassunto il prospetto economico dell'intero progetto, inclusi i periodi non rendicontati:
    \begin{table}[H]
    \centering
    \begin{tabular}{lcccccc}
      \rowcolor{lightgray}
      \textbf{Ruolo}  & \textbf{Ore previste} & \textbf{Ore effettive} & \textbf{Costo previsto} & \textbf{Costo effettivo} & \textbf{Differenza} \\
Responsabile    & 78           & 82            & € 2.340,00           & € 2.460,00           & € 120,00         \\
Amministratore  & 105          & 106           & € 2.100,00           & € 2.120,00           & € 20,00          \\
Analista        & 124          & 123           & € 3.100,00           & € 3.075,00           & -€ 25,00         \\
Progettista     & 153          & 152           & € 3.366,00           & € 3.344,00           & -€ 22,00         \\
Programmatore   & 198          & 197           & € 2.970,00           & € 2.955,00           & -€ 15,00         \\
Verificatore    & 339          & 340           & € 5.085,00           & € 5.100,00           & € 15,00          \\
\textbf{Totale} & \textbf{997} & \textbf{1000} & \textbf{€ 18.961,00} & \textbf{€ 19.054,00} & \textbf{€ 93,00}
      
    \end{tabular}
    \caption{Resoconto economico effettivo di tutti i periodi}
  \end{table}
  
  \subsection{Totale ore rendicontate}
  
  Di seguito viene riassunta la distribuzione delle ore effettivamente impiegate e rendiconate nell'arco dell'intero progetto:
  \begin{table}[H]
    \centering
    \begin{tabular}{lccccccc}
      \rowcolor{lightgray}
      \textbf{Nominativo}       & \textbf{Re}      & \textbf{Am} & \textbf{An}      & \textbf{Pt} & \textbf{Pr} & \textbf{Ve} & \textbf{Ore totali} \\
Sofia Bonomi              & 9                & 5                & 4               & 21                & 31 (+0)           & 35                & 105 (+0)          \\
Enrico Buratto            & 7                & 8 (-1)           & 6               & 22                & 30 (+1)           & 32                & 105 (+0)          \\
Ian Nicolas Di Menna      & 4                & 8 (+1)           & 8               & 20                & 28 (-1)           & 37                & 105 (+0)          \\
Alessandro Franchin       & 4                & 14 (+1)          & 5               & 19 (-1)           & 29                & 34                & 105 (+0)          \\
Enrico Galdeman           & 7                & 5                & 4               & 24                & 29 (+1)           & 36 (-1)           & 105 (+0)          \\
Nicholas Miazzo           & 6                & 14 (+0)          & 4               & 20                & 25                & 36                & 105 (+0)          \\
Marco Nardelotto          & 7                & 5                & 5               & 26                & 25 (-2)           & 37 (+2)           & 105 (+0)          \\
\textbf{Ore totali ruolo} & \textbf{44 (+0)} & \textbf{59 (+1)} & \textbf{36(+0)} & \textbf{152 (-1)} & \textbf{197 (-1)} & \textbf{247 (+1)} & \textbf{735 (+0)}
    \end{tabular}
    \caption{Resoconto orario effettivo in tutti i periodi rendicontati}
  \end{table}
  
  Di seguito viene riportato il prospetto economico finale comprensivo delle sole ore rendicontate a carico del committente:
    \begin{table}[H]
    \centering
    \begin{tabular}{lcccccc}
      \rowcolor{lightgray}
      \textbf{Ruolo}  & \textbf{Ore previste} & \textbf{Ore effettive} & \textbf{Costo previsto} & \textbf{Costo effettivo} & \textbf{Differenza} \\
Responsabile    & 44           & 44           & € 1.320,00           & € 1.320,00           & € 0,00           \\
Amministratore  & 58           & 59           & € 1.160,00           & € 1.180,00           & € 20,00          \\
Analista        & 36           & 36           & € 900,00             & € 900,00             & € 0,00           \\
Progettista     & 153          & 152          & € 3.366,00           & € 3.344,00           & -€ 22,00         \\
Programmatore   & 198          & 197          & € 2.970,00           & € 2.955,00           & -€ 15,00         \\
Verificatore    & 246          & 247          & € 3.690,00           & € 3.705,00           & € 15,00          \\
\textbf{Totale} & \textbf{735} & \textbf{735} & \textbf{€ 13.406,00} & \textbf{€ 13.404,00} & \textbf{-€ 2,00}
      
    \end{tabular}
    \caption{Resoconto economico effettivo di tutti i periodi rendicontati}
  \end{table}
  
  \subsubsection{Conclusione}
  Il progetto si è concluso rispettando i tempi e i costi preventivati. L'intero gruppo si ritiene soddisfatto del prodotto finale in quanto è riuscito a raggiungere gli obiettivi prefissati e, allo stesso tempo, la qualità desiderata.
  
  Durante lo svolgimento del progetto non è mai stato possibile organizzare incontri di persona a causa delle regole contro il Covid-19, di conseguenza il gruppo ha deciso di adottare le nuove modalità di incontri per via telematica. Ciò ha permesso di garantire la collaborazione tra i componenti nonostante nessuno di essi avesse mai utilizzato tale modalità. Inoltre, grazie all'impegno di tutti i componenti e alla disponibilità del proponente, l'impossibilità di incontrarsi non ha avuto un impatto sulla qualità del prodotto finale.
  
  Lo svolgimento del progetto ha permesso ad ogni componente di comprendere l'efficacia di una buona pianificazione del lavoro, al fine di rispettare i tempi e costi preventivati, e di quanto i rischi preventivati e non, possano avere un impatto importante sulla realizzazione del progetto.
  
  In conclusione, il costo finale rendicontato per l'intera realizzazione del prodotto ammonta a € 13.404,00, discordando dal preventivo per € 2,00 risparmiati.
  
\end{document}
