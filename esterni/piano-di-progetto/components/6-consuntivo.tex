\documentclass[../piano-di-progetto.tex]{subfiles}

\begin{document}

\section{Consuntivi di periodo}
Di seguito verranno riportate le spese effettivamente sostenute raggruppate sia per ruolo, sia per persona. Il bilancio potrà risultare:

\begin{itemize}
    \item \textbf{Positivo}: se il preventivo supera il consuntivo;
    \item \textbf{Pari}: se il consuntivo e il preventivo sono pari;
    \item \textbf{Negativo}: se il consuntivo supera il preventivo.
\end{itemize}

\subsection{Analisi}
Le ore di lavoro sostenute in questo periodo non verranno rendicontate nel preventivo finale in quanto vengono considerate come ore di investimento per l'approfondimento personale.

Di seguito vengono riportate le ore di lavoro effettive durante il periodo di analisi:
\begin{table}[H]
    \centering
    \begin{tabular}{lccccccc}
      \rowcolor{lightgray}
      \textbf{Nominativo}       & \textbf{Re}      & \textbf{Am} & \textbf{An}      & \textbf{Pt} & \textbf{Pr} & \textbf{Ve} & \textbf{Ore totali} \\
      Sofia Bononi              & -                & 8           & 16 (+2)          & -           & -           & 9           & 32 (+2)             \\
      Enrico Buratto            & -                & 14          & 6 (-2)           & -           & -           & 8           & 28 (-2)             \\
      Ian Nicolas Di Menna      & 10 (+1)          & -           & 7 (-2)           & -           & -           & 12          & 29 (-1)             \\
      Alessandro Franchin       & 9                & -           & 8 (-1)           & -           & -           & 13          & 31 (-1)             \\
      Enrico Galdeman           & -                & 7           & 15 (+1)             & -           & -           & 10       & 32 (+1)                  \\
      Nicholas Miazzo           & 12 (+2)          & -           & 10               & -           & -           & 10          & 32 (+2)             \\
      Marco Nardelotto          & -                & 9           & 9                & -           & -           & 13          & 31                  \\
      \textbf{Ore totali ruolo} & \textbf{31 (+3)} & \textbf{38} & \textbf{71 (-2)} & \textbf{-}  & \textbf{-}  & \textbf{75} & \textbf{215 (+1)}       
      
    \end{tabular}
    \caption{Resoconto orario effettivo del periodo di analisi}
  \end{table}

  \begin{table}[H]
    \centering
    \begin{tabular}{lcccccc}
      \rowcolor{lightgray}
      \textbf{Ruolo}  & \textbf{Ore previste} & \textbf{Ore effettive} & \textbf{Costo previsto} & \textbf{Costo effettivo} & \textbf{Differenza} \\
      Responsabile    & 28                    & 31                     & € 840,00                & € 930,00                 & € 90,00    \\
      Amministratore  & 38                    & 38                     & € 760,00                & € 760,00                 & € 0,00     \\
      Analista        & 73                    & 71                     & € 1.825,00              & € 1.775,00               & € -50,00   \\
      Progettista     & -                     & -                      & € 0,00                  & € 0,00                   & € 0,00     \\
      Programmatore   & -                     & -                      & € 0,00                  & € 0,00                   & € 0,00     \\
      Verificatore    & 75                    & 75                     & € 1.125,00              & € 1.125,00               & € 0,00     \\
      \textbf{Totale} & \textbf{214}          & \textbf{215}           & \textbf{€ 4.550,00}     & \textbf{€ 4.590,00}      & \textbf{€ 40,00}   
      
    \end{tabular}
    \caption{Resoconto economico effettivo del periodo di analisi}
  \end{table}


\subsubsection{Conclusioni}
In questo periodo il gruppo ha utilizzato più ore di quelle preventivate, aumentando il totale di € 40,00. In particolare le motivazioni sono le seguenti:
\begin{itemize}
    \item \textbf{Responsabile}: questo ruolo ha richiesto più ore del previsto a causa della complessità della pianificazione delle attività future;
    
 \item   \textbf{Analista}: grazie ai colloqui con il proponente, la comprensione dei requisiti si è rilevata più immediata.
\end{itemize}

\subsubsection{Preventivo a finire}
Trattandosi di un periodo che non verrà rendicontato, non è necessario adottare contromisure per la pianificazione del totale delle ore e il prospetto economico.

\subfile{6-2-cons.tex}
\subfile{6-3-proj.tex}
\subfile{6-4-I-incr.tex}
\subfile{6-5-II-incr.tex}
\subfile{6-6-III-incr.tex}
\subfile{6-7-IV-incr.tex}

\end{document}
