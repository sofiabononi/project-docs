\documentclass[../piano-di-progetto.tex]{subfiles}

\begin{document}

\subsection{XI incremento}

Di seguito vengono riportate le ore di lavoro effettive durante il periodo dell'undicesimo incremento:
\begin{table}[H]
    \centering
    \begin{tabular}{lccccccc}
      \rowcolor{lightgray}
      \textbf{Nominativo}       & \textbf{Re}      & \textbf{Am} & \textbf{An}      & \textbf{Pt} & \textbf{Pr} & \textbf{Ve} & \textbf{Ore totali} \\
Sofia Bonomi              & -          & 5           & -          & -          & -           & 6                & 11               \\
Enrico Buratto            & -          & -           & -          & 2          & 9           & 2                & 13               \\
Ian Nicolas Di Menna      & 4          & -           & -          & -          & -           & 6                & 10               \\
Alessandro Franchin       & -          & -           & -          & -          & 9           & 2                & 11               \\
Enrico Galdeman           & -          & -           & -          & -          & 9           & 4                & 13               \\
Nicholas Miazzo           & -          & 5           & -          & -          & -           & 5                & 10               \\
Marco Nardelotto          & -          & -           & -          & 4          & -           & 11 (+2)          & 15 (+2)          \\
\textbf{Ore totali ruolo} & \textbf{4} & \textbf{10} & \textbf{0} & \textbf{6} & \textbf{27} & \textbf{36 (+2)} & \textbf{83 (+2)}

    \end{tabular}
    \caption{Resoconto orario effettivo del periodo dell'undicesimo incremento}
  \end{table}

  \begin{table}[H]
    \centering
    \begin{tabular}{lcccccc}
      \rowcolor{lightgray}
      \textbf{Ruolo}  & \textbf{Ore previste} & \textbf{Ore effettive} & \textbf{Costo previsto} & \textbf{Costo effettivo} & \textbf{Differenza} \\

Responsabile    & 4           & 4           & € 120,00            & € 120,00            & € 0,00           \\
Amministratore  & 10          & 10          & € 200,00            & € 200,00            & € 0,00           \\
Analista        & -           & -           & € 0,00              & € 0,00              & € 0,00           \\
Progettista     & 6           & 6           & € 132,00            & € 132,00            & € 0,00           \\
Programmatore   & 27          & 27          & € 405,00            & € 405,00            & € 0,00           \\
Verificatore    & 34          & 36          & € 510,00            & € 540,00            & € 30,00          \\
\textbf{Totale} & \textbf{81} & \textbf{83} & \textbf{€ 1.367,00} & \textbf{€ 1.397,00} & \textbf{€ 30,00}


    \end{tabular}
    \caption{Resoconto economico effettivo del periodo dell'undicesimo incremento}
  \end{table}

\subsubsection{Conclusioni}
In questo periodo sono state utilizzate due ore in più rispetto al preventivo, in particolare:
\begin{itemize}
    \item \textbf{Verificatore}: al fine di garantire un prodotto di qualità, sono state investite due ore extra.
\end{itemize}
In conclusione, in questo periodo, sono stati risparmiati € 20,00 rispetto al preventivo.

\subsubsection{Preventivo a finire}
Il bilancio viene chiuso in positivo di € 2,00, considerando i € 32,00 in eccesso degli scorsi periodi. Il totale avanzato dallo scorso incremento è stato impiegato per il ruolo di verificatore al fine di garantire un risultato di qualità.
\end{document}
