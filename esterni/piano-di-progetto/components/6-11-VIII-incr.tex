\documentclass[../piano-di-progetto.tex]{subfiles}

\begin{document}

\subsection{VIII incremento}

Di seguito vengono riportate le ore di lavoro effettive durante il periodo del ottavo incremento:
\begin{table}[H]
    \centering
    \begin{tabular}{lccccccc}
      \rowcolor{lightgray}
      \textbf{Nominativo}       & \textbf{Re}      & \textbf{Am} & \textbf{An}      & \textbf{Pt} & \textbf{Pr} & \textbf{Ve} & \textbf{Ore totali} \\
Sofia Bononi              & -          & -               & 2          & -           & 7 (-1)           & -           & 9 (-1)           \\
Enrico Buratto            & -          & -               & -          & -           & 9 (-1)           & -           & 9 (-1)           \\
Ian Nicolas Di Menna      & -          & 2 (-1)          & -          & -           & 3 (-1)           & 4           & 9 (-2)           \\
Alessandro Franchin       & -          & -               & -          & -           & -                & 8           & 8                \\
Enrico Galdeman           & -          & -               & -          & 5           & -                & 6           & 11               \\
Nicholas Miazzo           & 3          & -               & -          & 8           & -                & -           & 11               \\
Marco Nardelotto          & -          & -               & -          & 8           & -                & 4           & 12               \\
\textbf{Ore totali ruolo} & \textbf{3} & \textbf{2 (-1)} & \textbf{2} & \textbf{21} & \textbf{19 (-3)} & \textbf{22} & \textbf{69 (-4)}
    \end{tabular}
    \caption{Resoconto orario effettivo del periodo del ottavo incremento}
  \end{table}

  \begin{table}[H]
    \centering
    \begin{tabular}{lcccccc}
      \rowcolor{lightgray}
      \textbf{Ruolo}  & \textbf{Ore previste} & \textbf{Ore effettive} & \textbf{Costo previsto} & \textbf{Costo effettivo} & \textbf{Differenza} \\
Responsabile    & 3           & 3           & € 90,00             & € 90,00             & € 0,00            \\
Amministratore  & 3           & 2           & € 60,00             & € 40,00             & -€ 20,00          \\
Analista        & 2           & 2           & € 50,00             & € 50,00             & € 0,00            \\
Progettista     & 21          & 21          & € 462,00            & € 462,00            & € 0,00            \\
Programmatore   & 22          & 19          & € 330,00            & € 285,00            & -€ 45,00          \\
Verificatore    & 22          & 22          & € 330,00            & € 330,00            & € 0,00            \\
\textbf{Totale} & \textbf{73} & \textbf{69} & \textbf{€ 1.322,00} & \textbf{€ 1.257,00} & \textbf{-€ 65,00}

    \end{tabular}
    \caption{Resoconto economico effettivo del periodo dell'ottavo incremento}
  \end{table}
\subsubsection{Conclusioni}
In questo periodo sono state utilizzate tre ore in meno rispetto al preventivo, in particolare:


\begin{itemize}
    \item \textbf{Amministratore}: il ruolo è stato svolto in meno tempo del previsto;
    \item \textbf{Programmatore}: l'aggiunta di nuove funzioni si è rivelata più semplice del previsto grazie alla qualità del lavoro svolto in precedenza.
\end{itemize}
In conclusione, in questo periodo, sono stati risparmiati € 65,00 rispetto al preventivo.

\subsubsection{Preventivo a finire}
Il bilancio viene chiuso in negativo di € 18,00, considerando i € 65,00 in eccesso degli scorsi periodi. Tuttavia, il gruppo ritiene che il lavoro svolto finora sarà molto utile agli successivi e permetterà un notevole risparmio di ore.
\end{document}
