\documentclass[../piano-di-progetto.tex]{subfiles}

\begin{document}

\subsection{IV incremento}

Di seguito vengono riportate le ore di lavoro effettive durante il periodo del quarto incremento:
\begin{table}[H]
    \centering
    \begin{tabular}{lccccccc}
      \rowcolor{lightgray}
      \textbf{Nominativo}       & \textbf{Re}      & \textbf{Am} & \textbf{An}      & \textbf{Pt} & \textbf{Pr} & \textbf{Ve} & \textbf{Ore totali} \\
Sofia Bononi              & -          & -          & -          & -               & -                & 4           & 4                \\
Enrico Buratto            & -          & -          & -          & -               & -                & 6           & 6                \\
Ian Nicolas Di Menna      & -          & -          & 1          & -               & -                & 5           & 6                \\
Alessandro Franchin       & -          & 2          & 2          & 5 (-1)          & -                & -           & 9 (-1)           \\
Enrico Galdeman           & 3          & -          & -          & -               & 0 (-1)           & 4           & 9 (-1)           \\
Nicholas Miazzo           & -          & -          & -          & -               & 5                & -           & 5                \\
Marco Nardelotto          & -          & -          & -          & 3               & 6                & 1           & 10               \\
\textbf{Ore totali ruolo} & \textbf{3} & \textbf{2} & \textbf{3} & \textbf{8 (-1)} & \textbf{11 (-1)} & \textbf{20} & \textbf{47 (-2)}

    \end{tabular}
    \caption{Resoconto orario effettivo del periodo del quarto incremento}
  \end{table}

  \begin{table}[H]
    \centering
    \begin{tabular}{lcccccc}
      \rowcolor{lightgray}
      \textbf{Ruolo}  & \textbf{Ore previste} & \textbf{Ore effettive} & \textbf{Costo previsto} & \textbf{Costo effettivo} & \textbf{Differenza} \\
Responsabile    & 3           & 3           & € 90,00           & € 0,00            & € 0,00            \\
Amministratore  & 2           & 2           & € 40,00           & € 0,00            & € 0,00            \\
Analista        & 3           & 3           & € 75,00           & € 0,00            & € 0,00            \\
Progettista     & 9           & 8           & € 198,00          & € 176,00          & -€ 22,00          \\
Programmatore   & 12          & 11          & € 180,00          & € 165,00          & -€ 15,00          \\
Verificatore    & 20          & 20          & € 300,00          & € 0,00            & € 0,00            \\
\textbf{Totale} & \textbf{49} & \textbf{47} & \textbf{€ 883,00} & \textbf{€ 846,00} & \textbf{-€ 37,00}

    \end{tabular}
    \caption{Resoconto economico effettivo del periodo del quarto incremento}
  \end{table}


\subsubsection{Conclusioni}
In questo periodo è stato rispettato il monte ore complessivo ma è stato necessario modificare la suddivisione oraria tra i ruoli, in particolare:
\begin{itemize}
    \item \textbf{Progettazione}: il ruolo è stato svolto pi+ velocemente del previsto;
    \item \textbf{Programmatore}: grazie alla buona quantità di materiale online sulla documentazione di Grafana, è stato possibile ridurre il lavoro di un'ora;
\end{itemize}
In conclusione, in questo periodo, sono stati risparmiati € 37,00 rispetto al preventivo.

\subsubsection{Preventivo a finire}
Il bilancio viene chiuso in negativo di€ 2,00, considerando i € 35,00 in eccesso degli scorsi periodi. Come preventivato, l'attenta progettazione effettuata nel terzo incremento ha reso l'integrazione delle funzionalità molto semplice e ci ha permesso di risparmiare tempo e rientrare nei costi. 
\end{document}
