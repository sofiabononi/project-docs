\documentclass{article}

\usepackage[italian]{babel}
\usepackage[margin=20mm, footskip = 20pt]{geometry}
\usepackage{graphicx}
\usepackage{subfiles}
\usepackage{hyperref}
\usepackage{nameref}
\usepackage{titlesec}
\usepackage{longtable}
\usepackage[table]{xcolor}
\usepackage{titling}
\usepackage{lastpage}
\usepackage{ifthen}
\usepackage{calc}
\usepackage{soulutf8}
\usepackage{contour}
\usepackage{float}
\usepackage{fancyhdr}
\usepackage{multirow}
\usepackage{pgfgantt}

% definizione dei percorsi in cui cercare immagini
\graphicspath{ {./}
    {./img/}
}

% setup della sottolineatura
\setuldepth{Flat}
\contourlength{0.8pt}

\newcommand{\uline}[1]{%
  \ul{{\phantom{#1}}}%
  \llap{\contour{white}{#1}}%
}

% setup dei link
\hypersetup{
  % set true if you want colored links (instead of boxes)
  colorlinks=true,
  % set to all if you want both sections and subsections linked
  linktoc=all,
  % set color for file links
  filecolor=blue,
  % set color for internal links
  linkcolor=black,
  % set url color
  urlcolor=blue,
  % set characters encoding in the bookmarks tab
  pdfencoding=unicode,
}

% setup forma \paragraph e \subparagraph
\titleformat{\paragraph}[hang]{\normalfont\normalsize\bfseries}{\theparagraph}{1em}{}
\titleformat{\subparagraph}[hang]{\normalfont\normalsize\bfseries}{\thesubparagraph}{1em}{}

% setup profondità indice di default
\setcounter{secnumdepth}{5}
\setcounter{tocdepth}{5}

\makeatletter %% non togliere, i comandi che definiscono i placeholder vanno qui
% esempio di utilizzo: \appendToGraphicspath{./img/} (un comando diverso per ogni path da includere)
% N.B.: ci DEVE essere un forward slash alla fine del path, a indicare che è una cartella.
\newcommand\appendToGraphicspath[1]{%
  \g@addto@macro\Ginput@path{{#1}}%
}

\newcommand{\setTitle}[1]{%
  \newcommand{\@placeholderTitle}{#1}%
}
\newcommand{\placeholderTitle}{\@placeholderTitle}

\newcommand{\setUso}[1]{%
  \newcommand{\@uso}{#1}%
}
\newcommand{\uso}{\@uso}

\newcommand{\setVersione}[1]{%
  \newcommand{\@versione}{#1}%
}
\newcommand{\versione}{\@versione}

\newcommand{\disabilitaVersione}{%
  \renewcommand{\setVersione}[1]{}%
  \renewcommand{\versione}{DISABILITATA}
}

\newcommand{\setResponsabile}[1]{%
  \newcommand{\@responsabile}{#1}%
}
\newcommand{\responsabile}{\@responsabile}

\newcommand{\setRedattori}[1]{%
  \newcommand{\@redattori}{#1}%
}
\newcommand{\redattori}{\@redattori}

\newcommand{\setVerificatori}[1]{%
  \newcommand{\@verificatori}{#1}%
}
\newcommand{\verificatori}{\@verificatori}

\newcommand{\setDescrizione}[1]{%
  \newcommand{\@descrizione}{#1}%
}
\newcommand{\descrizione}{\@descrizione}

\newcommand{\setModifiche}[1]{%
  \newcommand{\@modifiche}{#1}%
}

\newcommand{\modifiche}{\@modifiche}
\makeatother %% non togliere, i comandi che definiscono i placeholder vanno qui

% hook per lo script che genera il glossario
\newcommand{\glossario}[1]{\underline{#1}\textsubscript{g}}

% comandi per rendere opzionali gli elenchi di figure
\newcommand{\elencoFigure}{%
  \renewcommand{\listfigurename}{Elenco delle figure}%
  \listoffigures%
}

\newcommand{\disabilitaElencoFigure}{%
  \renewcommand{\elencoFigure}{}%
}

% comandi per rendere opzionali le tabelle
\newcommand{\elencoTabelle}{%
  \renewcommand{\listtablename}{Elenco delle tabelle}%
  \listoftables%
}

\newcommand{\disabilitaElencoTabelle}{%
  \renewcommand{\elencoTabelle}{}%
}

%Qui ci andrà il percorso delle immagini da includere in analisi dei requisiti
\appendToGraphicspath{../../commons/img/}

%Tutti questi set permettono di modificare in modo adatto i placeholder nel template
\setTitle{Piano di Qualifica}

\setVersione{}

\setResponsabile{}

\setRedattori{}

\setVerificatori{}

\setUso{Esterno}

\setDescrizione{Documento finalizzato a descrivere gli obiettivi che \emph{CoffeeCode} prefigge di raggiungere per ottenere processi e prodotti di \glossario{QUALITÀ}.}

\setModifiche{}

\begin{document}

\pagenumbering{gobble}

\begin{titlepage}% per non stampare il numero della pagina

  \raggedleft% allinea a destra la pagina
  \rule{1pt}{\textheight}% linea verticale
  \hspace{0.05\textwidth}% spazio tra linea e testo
  % lasciare questa riga per il corretto funziomento di \parbox
  \parbox[b]{0.75\textwidth}{% paragrafo che tiene il testo a destra della riga cambiando la larghezza il testo si muove a destra o a sinistra
  {\hspace{0.15\textwidth}\includegraphics[width=3cm,height=3cm]{logo.jpg}}\\[2\baselineskip] % logo
  {\Huge\bfseries CoffeeCode \\[0.5\baselineskip] Predire in Grafana}\\[5\baselineskip] % titolo
  {\Large\textsc{\placeholderTitle{}}}\\[6\baselineskip] % nome del documento
  {\begin{tabular}{r l}
    % testo in grassetto
    \textbf{Versione}     & \versione{}               \\
    \textbf{Approvazione} & \responsabile{}           \\
    \textbf{Redazione}    & \redattori{}              \\
    \textbf{Verifica}     & \verificatori{}           \\
    \textbf{Uso}          & \uso{}                    \\
    \textbf{Destinato a}  & CoffeeCode                \\
                          & prof.\ Vardanega Tullio   \\
                          & prof.\ Cardin Riccardo    \\
    \ifthenelse{\equal{\uso}{Esterno}}{
                          & Zucchetti Group SPA       \\
    }{}
  \end{tabular}}\\[5\baselineskip]

  {\bfseries Descrizione}\\
  {\descrizione{}}\\[2\baselineskip]
  {\texttt{coffeecodeswe@gmail.com}}\\[\baselineskip] % email
  }

\end{titlepage}


\section{Introduzione}%
\label{sec:introduzione}

\subsection{Premessa}%
\label{sub:premessa}
Il piano di qualifica sarà un documento soggetto a probabili modifiche nel corso dell'intera durata del progetto.
Tutto ciò è comprensibile dato che le metriche e i processi, identificati inizialmente, possono rivelarsi insufficienti o non adatti al fine di mantenere un'ottima qualità del progetto e all'interno del team di lavoro.
Per queste ragioni il documento sarà prodotto incrementalmente nel corso dello svolgimento del progetto.

\subsection{Scopo del documento}%
\label{sub:scopo_del_documento}
Lo scopo del documento è quello di illustrare formalmente tutte le modalità di verifica, validazione e le norme adottate, con il fine ultimo quello di preservare la qualità di prodotto e di processo.
Per raggiungere l'obiettivo prefissato verranno svolte continuamente verifiche sulle attività e sui processi in corso in modo da identificare nel minor tempo possibili errori e/o anomalie in modo da rendere più facile gli interventi di manutenzione e riducendo al minimo lo spreco di risorse.

\subsection{Scopo del prodotto}%
\label{sub:scopo_del_prodotto}
Lo scopo del prodotto è la realizzazione di due plug-in, in \glossario{Javascript}, per ottenere delle elaborazioni su il flusso di dati raccolti dai sistemi dell’azienda,
per poter poi prevedere i loro stati futuri così da inviare possibile segnalazioni o allarmi agli operatori del servizio cloud e alla linea di produzione del software.
In particolare le previsioni che si desiderano ottenere sono di due tipi:
\begin{itemize}
    \item ”Classificazioni”: per valutare il gruppo di appartenenza degli eventi dei dati ”\glossario{predittori}”;
    \item ”Regressioni”: nel caso in cui il valore cercato sia numerico e con campo continuo. 
\end{itemize}
I due plug-in devono estendere lo strumento di monitoraggio dei dati in \glossario{Grafana} applicando il Machine Learning attraverso le tecniche di 
Support Vector Machine e di Regressione Lineare utilizzate per la predizione su dati.

\subsection{Glossario}%
\label{sub:glossario}
All'interno del documento sono presenti termini che possono avere dei significati ambigui a seconda del contesto. Per evitare questa ambiguità è stato creato un documento di nome \textit{Glossario} che conterrà tali termini con il loro significato specifico. Per segnalare che il termine del testo è presente all'interno del glossario verrà segnalato con una G a pedice a fianco del termine.

\subsection{Riferimenti}%
\label{sub:riferimenti}

\subsubsection{Normativi}%
\label{subs:normativi}
\begin{itemize}
    \item \textbf{Norme di progetto}: \textit{Norme di progetto};
    \item \textbf{Capitolato d'appalto C4}: \href{https://www.math.unipd.it/~tullio/IS-1/2019/Progetto/C4.pdf}{Predire in Grafana}.
\end{itemize}

\subsubsection{Informativi}%
\label{subs:informativi}
\begin{itemize}
    \item \textbf{Presentazione Capitolato C4}: \href{https://www.math.unipd.it/~tullio/IS-1/2019/Dispense/C4a.pdf}{Predire in Grafana};
    \item \textbf{Slide del corso di Ingegneria del Software (SWE)}:
    \begin{itemize}
        \item Qualità di prodotto: \href{https://www.math.unipd.it/~tullio/IS-1/2019/Dispense/L12.pdf}{https://www.math.unipd.it/~tullio/IS-1/2019/Dispense/L12.pdf};
        \item Qualità di processo: \href{https://www.math.unipd.it/~tullio/IS-1/2019/Dispense/L13.pdf}{https://www.math.unipd.it/~tullio/IS-1/2019/Dispense/L13.pdf};
        \item Verifica e validazione: \href{https://www.math.unipd.it/~tullio/IS-1/2019/Dispense/L14.pdf}{https://www.math.unipd.it/~tullio/IS-1/2019/Dispense/L14.pdf}.
    \end{itemize}
    \item \textbf{Standard ISO/IEC 15504 (SPYCE)}: \href{https://www.cs.helsinki.fi/u/paakki/Pyhajarvi.pdf}{https://www.cs.helsinki.fi/u/paakki/Pyhajarvi.pdf};
\end{itemize}

\newpage
\section{Qualità di processo}%
\label{sec:qualita_di_processo}
\subfile{components/qualita-di-processo.tex}

\newpage
\section{Qualità di prodotto}%
\label{sec:qualita_di_prodotto}
\subfile{components/qualita-di-prodotto.tex}

\newpage
\section{Standard di qualità}%come fare appendice?
\label{sec:standard_di_qualita}
\subfile{components/standard-di-qualita.tex}

\newpage
\section{Resoconto attività di verifica}%come fare appendice?
\label{sec:resoconto_attivita_di_verifica}
\subfile{components/attivita-di-verifica.tex}

\newpage
\section{Valutazioni per il miglioramento}%come fare appendice?
\label{sec:valutazioni_per_il_miglioramento}
\subfile{components/valutazioni-per-il-miglioramento.tex}


\end{document}

