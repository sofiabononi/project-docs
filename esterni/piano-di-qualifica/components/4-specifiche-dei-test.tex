\documentclass[../piano-di-qualifica.tex]{subfiles}

\begin{document}

\subsection{Test di accettazione}
\label{sub:test_di_accettazione}
Dimostrano che il prodotto è in grado di soddisfare i requisiti minimi concordati con il proponente.
I test di accettazione si relazionano con i test di sistema e vengono eseguiti durante il collaudo finale del prodotto, sia dai membri del team che dal proponente sotto la supervisione del team.

\subsection{Test di sistema}
\label{sub:test_di_sistema}
Garantiscono che i requisiti richiesti identificati nel documento \textsc{Analisi dei Requisiti v1.0.0} vengano soddisfatti.

%inizio tabella test di sistema
\rowcolors{2}{white!80!lightgray!90}{white}
\renewcommand{\arraystretch}{2} % 
\begin{longtable}[H]{>{\centering\bfseries}m{2.5cm} >{\centering}m{7.5cm} >{\centering}m{2.5cm} >{\centering\arraybackslash}m{3.5cm}}
  \caption{Tabella test di sistema}%
  \label{tab:tabella_test_di_sistema}                                                    \\
  \rowcolor{lightgray}
  {\textbf{Requisito}} & {\textbf{Descrizione}} & {\textbf{Priorità}} & {\textbf{Implementazione}}  \\
  \endfirsthead%
  \rowcolor{lightgray}
  {\textbf{Requisito}} & {\textbf{Descrizione}} & {\textbf{Priorità}} & {\textbf{Implementazione}}  \\
  \endhead%
  \rowcolor{white}
  \multicolumn{4}{c}{\textit{Continua alla pagina successiva}}
  \endfoot%
  \endlastfoot%
  \textbf{TSRFO1} & L'utente deve poter addestrare il sistema inserendo i dati tramite degli appositi pulsanti. & Obbligatorio & Non Implementato \\

  \textbf{TSRFD1.1.1} & Si verifica che si venga notificati con un messaggio d’errore nel caso si carichino i dati di addestramento in un formato non valido. & Desiderabile & Non Implementato \\

  \textbf{TSRFD1.2.1} & Si verifica che l’utente venga notificato con un messaggio d’errore nel caso si carichi un predittore in un formato non valido. & Desiderabile & Non Implementato \\
  
  \textbf{TSRFO1.3} & L’utente deve poter scegliere il modello di machine learning per l'addestramento. & Obbligatorio & Non Implementato \\

  \textbf{TSRFO1.4} & L'utente deve poter avviare l'addestramento & Obbligatorio & Non Implementato \\
  \textbf{TSRFO1.5} & L’utente deve aver a disposizione un pulsante per il caricamento di un file \glossario{JSON} contenente il \glossario{predittore} allenato in precedenza. & Obbligatorio & Non Implementato \\
  
  \textbf{TSRFF2} & L’utente deve potere addestrare il sistema direttamente da Grafana & Facoltativo & Non Implementato \\

  \textbf{TSRFO3} & L'utente deve poter configurare il plug-in. & Obbligatorio & Non Implementato \\
  
  \textbf{TSRFO3.1} & L’utente deve poter avere a disposizione un metodo per caricare un modello addestrato. & Obbligatorio & Non Implementato \\
  \textbf{TSRFD3.1.1} & Si verifica che si visualizzi un messaggio di errore nel caso in cui venga caricato un file contenente il modello addestrato non valido o non venga caricato nessun file. & Desiderabile & Non Implementato \\

  \textbf{TSRFO3.2} & L’utente deve poter selezionare i nodi su cui desidera effettuare la predizione. & Obbligatorio & Non Implementato \\
  \textbf{TSRFD3.2.1} & Si verifica che si visualizzi un messaggio di errore nel caso in cui venga selezionato un nodo non valido o non venga selezionato nessun nodo. & Desiderabile & Non Implementato \\

  \textbf{TSRFO3.3} & L’utente deve poter selezionare il tipo della visualizzazione della predizione. & Obbligatorio & Non Implementato \\
  \textbf{TSRFD3.3.1} & Si verifica che si venga notificati con un messaggio d’errore nel caso in cui non venga selezionato alcun tipo di visualizzazione. & Desiderabile & Non Implementato \\

  \textbf{TSRFO4} & Si verifica che il sistema metta a disposizione la possibilità di visualizzare la previsione utilizzando un grafico. & Obbligatorio & Non Implementato \\

  \textbf{TSRFO5} & Si verifica che il sistema metta a disposizione la possibilità di visualizzare la previsione utilizzando un indicatore. & Obbligatorio & Non Implementato \\
  
  \textbf{TSRFO6} & L'utente deve poter selezionare il modello SVM per l’allenamento. & Obbligatorio & Non Implementato \\

  \textbf{TSRFO7} & L'utente deve poter selezionare il modello RL per l’allenamento. & Obbligatorio & Non Implementato \\

  \textbf{TSRFF8} & L'utente deve poter selezionare la regressione esponenziale per l’allenamento. & Facoltativo & Non Implementato \\

  \textbf{TSRFF9} & L'utente deve poter selezionare la regressione logaritmica per l’allenamento. & Facoltativo & Non Implementato \\
  
  \textbf{TSRFF10} & L'utente deve poter selezionare il modello rete neurale per l’allenamento. & Facoltativo & Non Implementato \\

  \textbf{TSRFF11} & L'utente deve poter selezionare il modello SVM adattata alla regressione per l’allenamento. & Facoltativo & Non Implementato \\

  \textbf{TSRFO12} & L'utente deve poter avere a disposizione un pulsante per l’avvio della predizione. & Obbligatorio & Non Implementato \\

  \textbf{TSRFO13} & L'utente deve poter avere a disposizione un pulsante per interrompere la predizione. & Obbligatorio & Non Implementato \\

  \textbf{TSRFF14} & L'utente deve avere a disposizione un metodo per l’impostazione degli alert. & Facoltativo & Non Implementato \\

  \textbf{TSRFD14.2.1} & Si verifica che il sistema, utilizzando un alert, notifichi un messaggio di errore nel caso in cui venga superata la soglia impostata. & Desiderabile & Non Implementato \\

  \textbf{TSRFD15} & L'utente deve poter eliminare il pannello di monitoraggio. & Desiderabile & Non Implementato \\

  \textbf{TSRFD15.1} & SI verifica che il sistema visualizzi un messaggio di errore nel caso in cui si cerchi di eliminare il pannello senza interrompere la previsione. & Desiderabile & Non Implementato \\

  \textbf{TSRFO16} & L'utente deve avere a disposizione un metodo per la visualizzazione delle previsioni. & Obbligatorio & Non Implementato \\

\end{longtable}

\subsection{Test di integrazione}
\label{sub:test_di_integrazione}
Garantiscono il corretto funzionamento di ogni componente del sistema quando viene integrata con le altre.
Questi test verranno stabiliti durante il periodo di Progettazione Logica in quanto sarebbe improbabile riuscire a stabilire con certezza dei test adeguati in questo momento.

\subsection{Test di unità}
\label{sub:test_di_unita}
Garantiscono il corretto funzionamento di ogni minimo componente autonomo del sistema.
I test in questione verranno stabiliti durante il periodo di Progettazione di dettaglio e codifica.

\end{document}