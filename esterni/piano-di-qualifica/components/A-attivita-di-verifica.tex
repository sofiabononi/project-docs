\documentclass[../piano-di-qualifica.tex]{subfiles}

\begin{document}
In questa sezione vengono resi noti i risultati dei test relativi al periodo di revisione dei documenti e requisiti, mediante l'utilizzo delle metriche descritte nelle sezioni precedenti.
I risultati possono coincidere o meno con i valori desiderati dal gruppo, nel caso non coincidessero verrà fatta un ulteriore analisi per il miglioramento visibile nella sezione appendice §B.

\subsection{Primo periodo (RR)}
\label{sub:primo_periodo}
Durante questo periodo è stata sottoposta ad una precisa attività di verifica tutta la documentazione da presentare in ingresso in sede di Revisione dei Requisiti.
I verificatori inizialmente hanno effettuato l'analisi utilizzando la tecnica di \glossario{Walkthrough} che ha portato ad una serie di errori frequenti consultabili nella sezione appendice §C del documento.
Successivamente dopo questa prima fase, i verificatori mediante la tecnica dell'\glossario{Inspection} hanno analizzato i documenti in modo più mirato ricavando errori di diverso genere che non verranno trattati in quanto si trattano di piccolezze già facilmente risolte.
\subfile{attivita/attivita-rr.tex}

\subsection{Secondo periodo (RP)}
\label{sub:secondo_periodo}
Nella seconda parte del progetto si è attuata una verifica più approfondita sui documenti, correggendo eventuali errori concettuali individuati nella prima fase di Revisione dei Requisiti.
Sono quindi state attuate verifiche e modifiche più specifiche ai documenti, e verifiche sul software, in particolare sul codice scritto per il \glossario{Proof of Concept}.
Come per il primo periodo, gli errori rilevati verranno trattati nella sezione appendice §B del documento.
\label{sub:secondo_periodo}
\subfile{attivita/attivita-rp.tex}

\subsection{Terzo periodo (RQ)}
\label{sub:terzo_periodo}
Nella terza parte del progetto sono state attuate verifiche per la correzione e il miglioramento dei documenti da presentare in questa fase di Revisione di qualifica.
Sono state quindi effettuate verifiche più approfondite sui prodotti software in seguito alla realizzazione di un'architettura adatta per l'applicazione da noi creata.
Come per i precedenti periodi, gli errori rilevati verranno trattati nella sezione appendice §B del documento. 
\subfile{attivita/attivita-rq.tex}

\end{document}
