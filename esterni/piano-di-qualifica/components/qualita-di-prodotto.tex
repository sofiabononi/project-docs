\documentclass[../piano-di-qualifica.tex]{subfiles}

\begin{document}

\subsection{Scopo}%
\label{sec:scopo}
Al fine di garantire la qualità del prodotto, il team CoffeeCode ha deciso di prendere come riferimento lo standard ISO/IEC 9126.
Lo standard definisce un modello, per la qualità esterna ed interna, delle caratteristiche da tenere in considerazione per ottenere un prodotto di buona qualità.

\subsection{Nomenclatura metriche e obiettivi di qualità}%
\label{sub:nomenclatura_metriche_e_obiettivi_di_qualita}
Vengono di seguito spiegati gli obiettivi, le metriche e come essi verranno identificati, in modo sintetico:
\begin{itemize}
    \item \textbf{Obiettivi}: QoPD[ID][Nome]-[Caratteristica] %da rivedere
        \begin{itemize}
            \item QoPD: sta per "Quality of Product" ossia qualità del prodotto;
            \item ID: identificatore numerico a 3 cifre;
            \item Nome: riassunto della descrizione del processo;
            \item Caratteristica: indica in quale caratteristica appartiene l'obiettivo;
        \end{itemize}
    \item \textbf{Metriche}: MoPD[ID][Nome]-[Caratteristica] %da rivedere
        \begin{itemize}
            \item MoPD: sta per "Metric of Product" ossia metrica del processo;
            \item ID: identificatore numerico a 3 cifre;
            \item Nome: riassunto della descrizione della metrica;
            \item Caratteristica: indica in quale caratteristica appartiene la metrica;
        \end{itemize}
\end{itemize}  
Ulteriori informazioni sulle metriche e sul loro funzionamento sono reperibili nel documento Norme di progetto.

\subsection{Caratteristiche}%
\label{sub:caratteristiche}
Vengono qui descritte le caratteristiche prese in considerazione dal gruppo di lavoro.

\subsubsection{Funzionalità}%
\label{sub:funzionalita}
Indica la capacità del prodotto di fornire funzioni che soddisfano le esigenze dichiarate e implicite, ricavate dall'analisi dei requisiti, quando il software viene utilizzato in un determinato contesto.

\paragraph{Metriche}
\label{sub:metriche}
\begin{itemize}
    \item MoPD001 Completezza di implementazione - [Adeguatezza] ;
    \item MoPD002 Precisione dei dati - [Precisione];
    \item MoPD003 Coerenza di interfaccia - [Interoperabilità].
\end{itemize}

\paragraph{Obiettivi}
\label{sub:obiettivi}
\begin{itemize}
    \item \textbf{QoPD001 Rispetto dell'implementazione funzionale - [Adeguatezza]}: realizzare le stesse funzioni ricavate dall'analisi iniziale;
    \item \textbf{QoPD002 Rispetto della precisione dei dati - [Precisione]}: ottenere la stessa precisione sui dati di quella prevista nell'analisi iniziale;
    \item \textbf{QoPD003 Rispetto delle interfacce - [Interoperabilità]}: realizzare le interfacce del progetto come preventivato nell'analisi iniziale.
\end{itemize}

\subsubsection{Affidabilità}%
\label{sub:affidabilita}
È La capacità del prodotto software di mantenere un determinato livello di prestazioni quando utilizzato sotto specifiche condizioni.

\paragraph{Metriche}
\label{sub:metriche}
\begin{itemize}
    \item MoPD004 Adeguatezza dei test - [Scadenza];
    \item MoPD005 Numero di errori - [Affidabilità].
\end{itemize}

\paragraph{Obiettivi}
\label{sub:obiettivi}
\begin{itemize}
    \item \textbf{QoPD004 Test completi sul codice - [Scadenza]}: i test devono coprire il tutto il codice sviluppato;
    \item \textbf{QoPD005 Classificare gli errori - [Affidabilità]}: trovare il numero di errori nei test e classificarli.
\end{itemize}

\subsubsection{Usabilità}%
\label{sub:usabilita}
L'usabilità indica come il software in questione, può essere compreso, appreso, utilizzato, attraente e conforme alle normative e linee guida sull'usabilità.

\paragraph{Metriche}
\label{sub:metriche}
\begin{itemize}
    \item MoPD004 Documentazione delle funzioni - [Comprensibilità].
\end{itemize}

\paragraph{Obiettivi}
\label{sub:obiettivi}
\begin{itemize}
    \item \textbf{QoPD004 Chiarezza del comportamento - [Comprensibilità]}: descrivere le funzioni implementate per informare l'utente su come esse lavorano.
\end{itemize}

\subsubsection{Efficienza}%
\label{sub:efficienza}
L'efficienza è la capacità del prodotto software di fornire prestazioni adeguate, in relazione alla quantità di
risorse utilizzate, in base alle condizioni indicate.

\paragraph{Metriche}
\label{sub:metriche}
\begin{itemize}
    \item MoPD005 Tempo di risposta - [Comportamento temporale].
\end{itemize}

\paragraph{Obiettivi}
\label{sub:obiettivi}
\begin{itemize}
    \item \textbf{QoPD005 Velocità di esecuzione - [Comportamento temporale]}: riuscire ad ottenere il minor tempo di esecuzione possibile.
\end{itemize}

\subsubsection{Manutenibilità}%
\label{sub:manutenibilita}
La manutenibilità è la capacità del prodotto di essere modificato in modo da migliorarlo, correggerlo o adattarlo al sistema, nei requisiti o nelle funzioni.

\paragraph{Metriche}
\label{sub:metriche}
\begin{itemize}
    \item MoPD006 Commenti sul codice - [Modifica].
\end{itemize}

\paragraph{Obiettivi}
\label{sub:obiettivi}
\begin{itemize}
    \item \textbf{QoPD006 Comprensione del codice - [Modifica]}: capire facilmente cosa fa il codice implementato.
\end{itemize}


\subsubsection{Portabilità}%
\label{sub:portabilita}
La portabilità è capacità del prodotto software di essere trasferito e implementato da un ambiente a un altro.
In questa sezione non sono stati presi in considerazione obietti e relative metriche da adottare, in quanto il prodotto viene richiesto solo per uno specifico ambiente di esecuzione (Grafana).
La portabilità sarà quindi una possibile funzione ausiliaria da implementare nelle modifiche future.

\subsection{Documentazione}%
\label{sub:documentazione}
I documenti redatti e pubblicati devono essere leggibili e comprensibili già dopo una prima lettura, riuscendo comunque a contenere parole di carattere tecnico sull'argomento.

\subsubsection{Metriche}
\label{sub:metriche}
\begin{itemize}
    \item MoPD007 \glossario{INDICE DI GULPEASE} - [Documentazione];
    \item MoPD008 Correttezza lessicale/ortografica - [Documentazione].
\end{itemize}

\subsubsection{Obiettivi}
\label{sub:obiettivi}
Le principali caratteristiche che verranno analizzate in ogni documento sono:
\begin{itemize}
    \item \textbf{QoPD007 Leggibilità del testo - [Documentazione]}: i documenti devono essere leggibili in modo fluido evitando quindi periodi troppo lunghi;
    \item \textbf{QoPD008 Correttezza ortografica - [Documentazione]}: in ogni documento non dovranno esserci errori ortografici.
\end{itemize}

\subsection{Tabelle di qualità di prodotto}
\label{sub:tabelle_di_qualita_di_prodotto}
Gli obiettivi di qualità, discussi nelle precedenti sezioni, che devono essere parte integrante di ogni processo, verranno indicati in tabelle in questa sezione.
Per ogni obiettivo viene indicato:

\begin{itemize}
   \item \textbf{Obiettivo}: indica il codice identificativo dell'obiettivo come descritto nella sezione §3.2;
   \item \textbf{Metrica}: indica, se presente, la metrica adottata per la valutazione dell'obiettivo di qualità come descritto nella sezione §3.2;
   \item \textbf{Valore accettabile}: rappresenta il valore minimo di qualità dell'obiettivo che CoffeCode intende ottenere. Non è presente in caso di mancanza della metrica associata all'obiettivo;
   \item \textbf{Valore desiderato}: rappresenta il valore di qualità dell'obiettivo che CoffeCode intende ottenere una maggiore qualità rispetto a quella minima. Non è presente in caso di mancanza della metrica associata all'obiettivo;
   \item \textbf{Descrizione}: descrizione generale dell'obiettivo.
\end{itemize}
%Tabelle



\end{document}