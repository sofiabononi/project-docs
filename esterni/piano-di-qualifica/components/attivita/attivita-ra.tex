\subsubsection{Esiti verifiche sui documenti}
\label{sub:esiti_verifiche_sui_documenti}
Sui documenti sono state utilizzate le metriche per:
    \begin{itemize}
        \item \textbf{Indice di Gulpease};
        \item \textbf{Correttezza lessicale/ortografica}.
    \end{itemize}
Per evitare risultati errati nel calcole dell'Indice di Gulpease si è deciso di non tenere in considerazione:
    \begin{itemize}
        \item Il frontespizio di ogni documento;
        \item Le tabelle presenti nei documenti;
        \item Il diario delle modifiche all'interno di ogni documento.
    \end{itemize}
Per quanto riguarda la correttezza lessicale/ortografica, gli errori trovati non verranno classificati ma semplicemente conteggiati uniformemente includendo anche errori derivati dal mancato rispetto delle convenzioni scritte nelle \textsc{Norme di Progetto v3.11-2.4.1}.
\\ Per ogni metrica di ogni documento verrà inoltre inserito l'esito dell'analisi che può avere risultato soddisfacente o non.

\paragraph{Norme di Progetto}
\label{sub:norme_di_progetto}
Le \textsc{Norme di Progetto v3.11-2.4.1} hanno subito poche modifiche che sono state revisionate in sole 2 fasi di approvazione del documento.
Il suo indice di Gulpease si è assestato sul valore 71.

\rowcolors{2}{white!80!lightgray!90}{white}
\renewcommand{\arraystretch}{2} % allarga le righe con dello spazio sotto e sopra
\begin{longtable}[H]{>{\centering\bfseries}m{6cm} >{\centering}m{2cm} >{\centering}m{2.5cm} >{\centering}m{2.5cm} >{\centering\arraybackslash}m{2.5cm}}  
  \rowcolor{lightgray}
  {\textbf{Documento}} & {\textbf{Risultato indice}} & {\textbf{Errori presenti}} & {\textbf{Esito indice}} & {\textbf{Esito errori}}  \\
  \endfirsthead%
  \rowcolor{lightgray}
  {\textbf{Documento}} & {\textbf{Risultato indice}} & {\textbf{Errori presenti}} & {\textbf{Esito indice}} & {\textbf{Esito errori}}  \\
  \endhead%
  \textbf{Norme di Progetto v3.11-2.4.1} & 71               & 0               & Soddisfatto & Soddisfatto \\
  \caption{Risultati metriche per le Norme di Progetto v3.11-2.4.1}
  \label{tab:my-table}
\end{longtable}

\begin{figure}[H]
  \centering
  \includegraphics[width=10cm]{img/erroriNdPv3.11-2.4.1.png}
  \label{fig:errori_ndp}
  \caption{Grafico errori per \textsc{Norme di Progetto v3.11-2.4.1}}
\end{figure}

\begin{figure}[H]
  \centering
  \includegraphics[width=10cm]{img/gulpeaseNdPv3.11-2.4.1.png}
  \label{fig:gulpease_ndp}
  \caption{Grafico indice di Gulpease per \textsc{Norme di Progetto v3.11-2.4.1}}
\end{figure}


\paragraph{Piano di Qualifica}
\label{sub:piano_di_qualifica}
Il \textsc{Piano di Qualifica v3.11-4.0.0} ha subito modifiche principalmente riguardanti i test e i valori delle metriche per la seguente fase.
Il suo indice di Gulpease si è assestato sul valore 72.

\rowcolors{2}{white!80!lightgray!90}{white}
\renewcommand{\arraystretch}{2} % allarga le righe con dello spazio sotto e sopra
\begin{longtable}[H]{>{\centering\bfseries}m{6cm} >{\centering}m{2cm} >{\centering}m{2.5cm} >{\centering}m{2.5cm} >{\centering\arraybackslash}m{2.5cm}}  
  \rowcolor{lightgray}
  {\textbf{Documento}} & {\textbf{Risultato indice}} & {\textbf{Errori presenti}} & {\textbf{Esito indice}} & {\textbf{Esito errori}}  \\
  \endfirsthead%
  \rowcolor{lightgray}
  {\textbf{Documento}} & {\textbf{Risultato indice}} & {\textbf{Errori presenti}} & {\textbf{Esito indice}} & {\textbf{Esito errori}}  \\
  \endhead%
  \textbf{Piano di Qualifica v3.11-4.0.0} & 72              & 0               & Soddisfatto & Soddisfatto \\
  \caption{Risultati metriche per il Piano di Qualifica v3.11-4.0.0}
  \label{tab:my-table}
\end{longtable}

\begin{figure}[H]
  \centering
  \includegraphics[width=10cm]{img/erroriPdQv3.11-4.0.0.png}
  \label{fig:errori_pdq}
  \caption{Grafico errori per \textsc{Piano di Qualifica v3.11-4.0.0}}
\end{figure}

\begin{figure}[H]
  \centering
  \includegraphics[width=10cm]{img/GulpeasePdQv3.11-4.0.0.png}
  \label{fig:gulpease_pdq}
  \caption{Grafico indice di Gulpease per \textsc{Piano di Qualifica v3.11-4.0.0}}
\end{figure}


\paragraph{Piano di Progetto}
\label{sub:piano_di_progetto}
Il \textsc{Piano di Progetto v3.11-4.4.0} ha subito leggere modifiche come tutti i documenti in questa ultima fase di sviluppo del progetto, il che ha portato ad attuare due fasi di approvazione del documento.
Il suo indice di Gulpease è aumentato di 1 rispetto alla fase di sviluppo precedente.

\rowcolors{2}{white!80!lightgray!90}{white}
\renewcommand{\arraystretch}{2} % allarga le righe con dello spazio sotto e sopra
\begin{longtable}[H]{>{\centering\bfseries}m{6cm} >{\centering}m{2cm} >{\centering}m{2.5cm} >{\centering}m{2.5cm} >{\centering\arraybackslash}m{2.5cm}}  
  \rowcolor{lightgray}
  {\textbf{Documento}} & {\textbf{Risultato indice}} & {\textbf{Errori presenti}} & {\textbf{Esito indice}} & {\textbf{Esito errori}}  \\
  \endfirsthead%
  \rowcolor{lightgray}
  {\textbf{Documento}} & {\textbf{Risultato indice}} & {\textbf{Errori presenti}} & {\textbf{Esito indice}} & {\textbf{Esito errori}}  \\
  \endhead%
  \textbf{Piano di Progetto v3.11-4.4.0} & 73               & 0               & Soddisfatto & Soddisfatto \\
  \caption{Risultati metriche per il Piano di Progetto v3.11-4.4.0}
  \label{tab:my-table}
\end{longtable}

\begin{figure}[H]
  \centering
  \includegraphics[width=10cm]{img/erroriPdPv3.11-4.4.0.png}
  \label{fig:errori_pdq}
  \caption{Grafico errori per \textsc{Piano di Progetto v3.11-4.4.0}}
\end{figure}

\begin{figure}[H]
  \centering
  \includegraphics[width=10cm]{img/gulpeasePdPv3.11-4.4.0.png}
  \label{fig:gulpease_pdq}
  \caption{Grafico indice di Gulpease per \textsc{Piano di Progetto v3.11-4.4.0}}
\end{figure}


\paragraph{Analisi dei Requisiti}
\label{sub:analisi_dei_requisiti}
Nell'\textsc{Analisi dei Requisiti v3.11-1.4.0} sono state apportate modifiche solo a pochi casi d'uso il che non ha richiesto grandi fasi di verifica e approvazione.
L'indice di Gulpease del documento è comunque sceso di un punto restando a 72.

\rowcolors{2}{white!80!lightgray!90}{white}
\renewcommand{\arraystretch}{2} % allarga le righe con dello spazio sotto e sopra
\begin{longtable}[H]{>{\centering\bfseries}m{6cm} >{\centering}m{2cm} >{\centering}m{2.5cm} >{\centering}m{2.5cm} >{\centering\arraybackslash}m{2.5cm}}  
  \rowcolor{lightgray}
  {\textbf{Documento}} & {\textbf{Risultato indice}} & {\textbf{Errori presenti}} & {\textbf{Esito indice}} & {\textbf{Esito errori}}  \\
  \endfirsthead%
  \rowcolor{lightgray}
  {\textbf{Documento}} & {\textbf{Risultato indice}} & {\textbf{Errori presenti}} & {\textbf{Esito indice}} & {\textbf{Esito errori}}  \\
  \endhead%
  \textbf{Analisi dei Requisiti v3.11-1.4.0} &  72              & 0               & Soddisfatto & Soddisfatto \\
  \caption{Risultati metriche per l'Analisi dei Requisiti v3.11-1.4.0}
  \label{tab:my-table}
\end{longtable}

L'\textsc{Analisi dei Requisiti v3.11-1.4.0} non presenta grafici in quanto è stata effettuata una sola approvazione del documento finale.


\paragraph{Glossario}
\label{sub:glossario}
Nel \textsc{Glossario v3.11-1.3.1} sono stati aggiornati solo alcuni termini della sezione F, il che non ha richiesto grandi fasi di verifica e approvazione del documento.
Il suo indice di Gulpease è rimasto invariato rispetto alla fase di sviluppo precedente.

\rowcolors{2}{white!80!lightgray!90}{white}
\renewcommand{\arraystretch}{2} % allarga le righe con dello spazio sotto e sopra
\begin{longtable}[H]{>{\centering\bfseries}m{6cm} >{\centering}m{2cm} >{\centering}m{2.5cm} >{\centering}m{2.5cm} >{\centering\arraybackslash}m{2.5cm}}  
  \rowcolor{lightgray}
  {\textbf{Documento}} & {\textbf{Risultato indice}} & {\textbf{Errori presenti}} & {\textbf{Esito indice}} & {\textbf{Esito errori}}  \\
  \endfirsthead%
  \rowcolor{lightgray}
  {\textbf{Documento}} & {\textbf{Risultato indice}} & {\textbf{Errori presenti}} & {\textbf{Esito indice}} & {\textbf{Esito errori}}  \\
  \endhead%
  \textbf{Glossario v3.11-1.3.1} & 65               & 0               & Soddisfatto & Soddisfatto \\
  \caption{Risultati metriche per il Glossario v3.11-1.3.1}
  \label{tab:my-table}
\end{longtable}

A causa delle poche modifiche e cambiamenti significativi non sono stati inseriti i grafici che non comunicavano alcuna informazione aggiuntiva.

\paragraph{Manuale Utente}
\label{sub:manuale_utente}
Il \textsc{Manuale Utente v3.11-1.0.0} ha ricevuto modifiche e ampliamenti in quasi tutte le sue sezioni ma non così significative da richiedere molte fasi di verifica e approvazione.
Il suo indice di Gulpease in questa fase di sviluppo è salito a 74.

\rowcolors{2}{white!80!lightgray!90}{white}
\renewcommand{\arraystretch}{2} % allarga le righe con dello spazio sotto e sopra
\begin{longtable}[H]{>{\centering\bfseries}m{6cm} >{\centering}m{2cm} >{\centering}m{2.5cm} >{\centering}m{2.5cm} >{\centering\arraybackslash}m{2.5cm}}  
  \rowcolor{lightgray}
  {\textbf{Documento}} & {\textbf{Risultato indice}} & {\textbf{Errori presenti}} & {\textbf{Esito indice}} & {\textbf{Esito errori}}  \\
  \endfirsthead%
  \rowcolor{lightgray}
  {\textbf{Documento}} & {\textbf{Risultato indice}} & {\textbf{Errori presenti}} & {\textbf{Esito indice}} & {\textbf{Esito errori}}  \\
  \endhead%
  \textbf{Manuale Utente v3.11-1.0.0} & 74               & 0               & Soddisfatto & Soddisfatto \\
  \caption{Risultati metriche per il Manuale Utente v3.11-1.0.0}
  \label{tab:my-table}
\end{longtable}

A causa delle poche fasi di verifica richieste non sono stati inseriti i grafici che non comunicavano alcuna informazione aggiuntiva.

\paragraph{Manuale Sviluppatore}
\label{sub:manuale_sviluppatore}
Il \textsc{Manuale Sviluppatore v3.11-2.0.0} ha ricevuto modifiche principalmente ai diagrammi UML sia della training app che del plug-in, oltre ad altre modifiche relative alla parte relativa al plug-in.
Il suo indice di Gulpease in questa fase di sviluppo è rimasto invariato a 71.

\rowcolors{2}{white!80!lightgray!90}{white}
\renewcommand{\arraystretch}{2} % allarga le righe con dello spazio sotto e sopra
\begin{longtable}[H]{>{\centering\bfseries}m{6cm} >{\centering}m{2cm} >{\centering}m{2.5cm} >{\centering}m{2.5cm} >{\centering\arraybackslash}m{2.5cm}}  
  \rowcolor{lightgray}
  {\textbf{Documento}} & {\textbf{Risultato indice}} & {\textbf{Errori presenti}} & {\textbf{Esito indice}} & {\textbf{Esito errori}}  \\
  \endfirsthead%
  \rowcolor{lightgray}
  {\textbf{Documento}} & {\textbf{Risultato indice}} & {\textbf{Errori presenti}} & {\textbf{Esito indice}} & {\textbf{Esito errori}}  \\
  \endhead%
  \textbf{Manuale Sviluppatore v3.11-2.0.0} & 71               & 0               & Soddisfatto & Soddisfatto \\
  \caption{Risultati metriche per il Manuale Sviluppatore v3.11-2.0.0}
  \label{tab:my-table}
\end{longtable}

A causa delle poche fasi di verifica richieste non sono stati inseriti i grafici che non comunicavano alcuna informazione aggiuntiva.

\subsubsection{Esiti verifiche sui processi}
\label{sub:esiti_verifiche_sui_processi}
In questa sezione vengono visualizzati gli esiti delle metriche prese in considerazione per quanto riguarda i processi produttivi.
Come per i documenti, anche per queste metriche verrà fornito un esito che può essere soddisfacente o meno.

\paragraph{Processo PRC001}
\label{sub:processo_PRC001}

\rowcolors{2}{white!80!lightgray!90}{white}
\renewcommand{\arraystretch}{2} % allarga le righe con dello spazio sotto e sopra
\begin{longtable}[H]{>{\centering\bfseries}m{5cm} >{\centering}m{5cm} >{\centering}m{2.5cm} >{\centering\arraybackslash}m{2.5cm}}  
  \rowcolor{lightgray}
  {\textbf{Obiettivo}} & {\textbf{Metrica}} & {\textbf{Risultato}} & {\textbf{Esito}}  \\
  \endfirsthead%
  \rowcolor{lightgray}
  {\textbf{Obiettivo}} & {\textbf{Metrica}} & {\textbf{Risultato}} & {\textbf{Esito}}  \\
  \endhead%
  \textbf{QoPR001 Rispetto delle scadenze della pianificazione} & MoPR001 Varianza dei tempi &  1.02 & Soddisfatto  \\
  \caption{Risultati metrica MoPR001}
  \label{tab:my-table}
\end{longtable}
\textbf{Nota}: I ritardi nelle scadenze pianificate sono diminuiti nel corso di quest'ultima fase di progetto.

\rowcolors{2}{white!80!lightgray!90}{white}
\renewcommand{\arraystretch}{2} % allarga le righe con dello spazio sotto e sopra
\begin{longtable}[H]{>{\centering\bfseries}m{5cm} >{\centering}m{5cm} >{\centering}m{2.5cm} >{\centering\arraybackslash}m{2.5cm}}  
  \rowcolor{lightgray}
  {\textbf{Obiettivo}} & {\textbf{Metrica}} & {\textbf{Risultato}} & {\textbf{Esito}}  \\
  \endfirsthead%
  \rowcolor{lightgray}
  {\textbf{Obiettivo}} & {\textbf{Metrica}} & {\textbf{Risultato}} & {\textbf{Esito}}  \\
  \endhead%
  \textbf{QoPR002 Rispetto del budget istanziato} & MoPR002 Varianza dei costi & -2 & Soddisfatto  \\
  \caption{Risultati metrica MoPR002}
  \label{tab:my-table}
\end{longtable}
\textbf{Nota}: La variazione rispetto al preventivo iniziale rientra nel range deciso per la metrica, per cui l'obiettivo è stato soddisfatto. Il discostamento dal preventivo totale rilevato è di: -2 euro.

\rowcolors{2}{white!80!lightgray!90}{white}
\renewcommand{\arraystretch}{2} % allarga le righe con dello spazio sotto e sopra
\begin{longtable}[H]{>{\centering\bfseries}m{5cm} >{\centering}m{5cm} >{\centering}m{2.5cm} >{\centering\arraybackslash}m{2.5cm}}  
  \rowcolor{lightgray}
  {\textbf{Obiettivo}} & {\textbf{Metrica}} & {\textbf{Risultato}} & {\textbf{Esito}}  \\
  \endfirsthead%
  \rowcolor{lightgray}
  {\textbf{Obiettivo}} & {\textbf{Metrica}} & {\textbf{Risultato}} & {\textbf{Esito}}  \\
  \endhead%
  \textbf{QoPR003 Rispetto del ciclo di vita scelto} & MoPR003 Aderenza agli standard & Livello di maturità: 3 \\ Valutazione attributi: L  &  Soddisfatto \\
  \caption{Risultati metrica MoPR003}
  \label{tab:my-table}
\end{longtable}
\textbf{Nota}: IL miglioramento per quanto riguarda l'aderenza agli standard è rimasto costante anche in questa fase.

\rowcolors{2}{white!80!lightgray!90}{white}
\renewcommand{\arraystretch}{2} % allarga le righe con dello spazio sotto e sopra
\begin{longtable}[H]{>{\centering\bfseries}m{5cm} >{\centering}m{5cm} >{\centering}m{2.5cm} >{\centering\arraybackslash}m{2.5cm}}  
  \rowcolor{lightgray}
  {\textbf{Obiettivo}} & {\textbf{Metrica}} & {\textbf{Risultato}} & {\textbf{Esito}}  \\
  \endfirsthead%
  \rowcolor{lightgray}
  {\textbf{Obiettivo}} & {\textbf{Metrica}} & {\textbf{Risultato}} & {\textbf{Esito}}  \\
  \endhead%
  \textbf{QoPR004 Rispetto dei ruoli e identificazione nei prodotti} & MoPR004 Aderenza ai ruoli & 0 & Soddisfatto \\
  \caption{Risultati metrica MoPR004}
  \label{tab:my-table}
\end{longtable}
\textbf{Nota}: In ogni documento prodotto i ruoli sono stati identificati e rispettati.

\rowcolors{2}{white!80!lightgray!90}{white}
\renewcommand{\arraystretch}{2} % allarga le righe con dello spazio sotto e sopra
\begin{longtable}[H]{>{\centering\bfseries}m{5cm} >{\centering}m{5cm} >{\centering}m{2.5cm} >{\centering\arraybackslash}m{2.5cm}}  
  \rowcolor{lightgray}
  {\textbf{Obiettivo}} & {\textbf{Metrica}} & {\textbf{Risultato}} & {\textbf{Esito}}  \\
  \endfirsthead%
  \rowcolor{lightgray}
  {\textbf{Obiettivo}} & {\textbf{Metrica}} & {\textbf{Risultato}} & {\textbf{Esito}}  \\
  \endhead%
  \textbf{QoPR005 Rispetto del versionamento dei prodotti} & MoPR005 Controllo prodotti & 25.8 & Soddisfatto \\
  \caption{Risultati metrica MoPR005}
  \label{tab:my-table}
\end{longtable}
\textbf{Nota}: Il numero di commit in questa fase è stato parecchio elevato, il che soddisfa appieno l'obiettivo prefissato.

\paragraph{Processo PRC002}
\label{sub:processo_PRC002}

\rowcolors{2}{white!80!lightgray!90}{white}
\renewcommand{\arraystretch}{2} % allarga le righe con dello spazio sotto e sopra
\begin{longtable}[H]{>{\centering\bfseries}m{5cm} >{\centering}m{5cm} >{\centering}m{2.5cm} >{\centering\arraybackslash}m{2.5cm}}  
  \rowcolor{lightgray}
  {\textbf{Obiettivo}} & {\textbf{Metrica}} & {\textbf{Risultato}} & {\textbf{Esito}}  \\
  \endfirsthead%
  \rowcolor{lightgray}
  {\textbf{Obiettivo}} & {\textbf{Metrica}} & {\textbf{Risultato}} & {\textbf{Esito}}  \\
  \endhead%
  \textbf{QoPR006 Soddisfazione dei requisiti obbligatori} & MoPR006 Verifica requisiti obbligatori & 100\% & Soddisfatto \\
  \caption{Risultati metrica MoPR006}
  \label{tab:my-table}
\end{longtable}
\textbf{Nota}: Tutti i requisiti obbligatori sono stati soddisfatti per questa revisione finale.

\rowcolors{2}{white!80!lightgray!90}{white}
\renewcommand{\arraystretch}{2} % allarga le righe con dello spazio sotto e sopra
\begin{longtable}[H]{>{\centering\bfseries}m{5cm} >{\centering}m{5cm} >{\centering}m{2.5cm} >{\centering\arraybackslash}m{2.5cm}}  
  \rowcolor{lightgray}
  {\textbf{Obiettivo}} & {\textbf{Metrica}} & {\textbf{Risultato}} & {\textbf{Esito}}  \\
  \endfirsthead%
  \rowcolor{lightgray}
  {\textbf{Obiettivo}} & {\textbf{Metrica}} & {\textbf{Risultato}} & {\textbf{Esito}}  \\
  \endhead%
  \textbf{QoPR007 Soddisfazione dei requisiti opzionali e desiderabili} & MoPR007 Verifica requisiti opzionali \\ MoPR008 Verifica requisiti desiderabili & 40\% & Soddisfatto \\
  \caption{Risultati metrica MoPR007 e metrica MoPR008}
  \label{tab:my-table}
\end{longtable}
\textbf{Nota}: Il valore della metrica rientra nel range di valori per l'accettazione, anche se non raggiunge un livello di totale soddisfacimento.

\rowcolors{2}{white!80!lightgray!90}{white}
\renewcommand{\arraystretch}{2} % allarga le righe con dello spazio sotto e sopra
\begin{longtable}[H]{>{\centering\bfseries}m{5cm} >{\centering}m{5cm} >{\centering}m{2.5cm} >{\centering\arraybackslash}m{2.5cm}}  
  \rowcolor{lightgray}
  {\textbf{Obiettivo}} & {\textbf{Metrica}} & {\textbf{Risultato}} & {\textbf{Esito}}  \\
  \endfirsthead%
  \rowcolor{lightgray}
  {\textbf{Obiettivo}} & {\textbf{Metrica}} & {\textbf{Risultato}} & {\textbf{Esito}}  \\
  \endhead%
  \textbf{QoPR008 Verifica dei rischi previsti} & MoPR009 Verifica rischi non pervenuti & 0 & Soddisfatto \\
  \caption{Risultati metrica MoPR009}
  \label{tab:my-table}
\end{longtable}
\textbf{Nota}: Non sono ancora stati rivelati rischi importanti che non siano stati previsti precedentemente dal gruppo.

\paragraph{Processo PRC003}
\label{sub:processo_PRC003}

\rowcolors{2}{white!80!lightgray!90}{white}
\renewcommand{\arraystretch}{2} % allarga le righe con dello spazio sotto e sopra
\begin{longtable}[H]{>{\centering\bfseries}m{5cm} >{\centering}m{5cm} >{\centering}m{2.5cm} >{\centering\arraybackslash}m{2.5cm}}  
  \rowcolor{lightgray}
  {\textbf{Obiettivo}} & {\textbf{Metrica}} & {\textbf{Risultato}} & {\textbf{Esito}}  \\
  \endfirsthead%
  \rowcolor{lightgray}
  {\textbf{Obiettivo}} & {\textbf{Metrica}} & {\textbf{Risultato}} & {\textbf{Esito}}  \\
  \endhead%
  \textbf{QoPR09 Rispetto delle fasi del ciclo di vita} & MoPR010 Analisi Way of Working & / & Soddisfatto \\
  \caption{Risultati metrica MoPR010}
  \label{tab:my-table}
\end{longtable}
\textbf{Nota}: I componenti del gruppo svolgono costantemente un aggiornamento in base alle modifiche fatte alle \textsc{Norme di Progetto v3.11-2.4.1}.

\rowcolors{2}{white!80!lightgray!90}{white}
\renewcommand{\arraystretch}{2} % allarga le righe con dello spazio sotto e sopra
\begin{longtable}[H]{>{\centering\bfseries}m{5cm} >{\centering}m{5cm} >{\centering}m{2.5cm} >{\centering\arraybackslash}m{2.5cm}}  
  \rowcolor{lightgray}
  {\textbf{Obiettivo}} & {\textbf{Metrica}} & {\textbf{Risultato}} & {\textbf{Esito}}  \\
  \endfirsthead%
  \rowcolor{lightgray}
  {\textbf{Obiettivo}} & {\textbf{Metrica}} & {\textbf{Risultato}} & {\textbf{Esito}}  \\
  \endhead%
  \textbf{QoPR010 Rispetto nella redazione dei documenti} & MoPR011 Analisi documenti & 2 & Soddisfatto \\
  \caption{Risultati metrica MoPR011}
  \label{tab:my-table}
\end{longtable}
\textbf{Nota}: Anche se nei documenti sono state apportate poche modifiche mediamente ogni documento è stato revisionato e approvato 2 volte che rimane un valore accettabile per quest'ultima parte di sviluppo.

\paragraph{Processo PRC004}
\label{sub:processo_PRC004}

\rowcolors{2}{white!80!lightgray!90}{white}
\renewcommand{\arraystretch}{2} % allarga le righe con dello spazio sotto e sopra
\begin{longtable}[H]{>{\centering\bfseries}m{5cm} >{\centering}m{5cm} >{\centering}m{2.5cm} >{\centering\arraybackslash}m{2.5cm}}  
  \rowcolor{lightgray}
  {\textbf{Obiettivo}} & {\textbf{Metrica}} & {\textbf{Risultato}} & {\textbf{Esito}}  \\
  \endfirsthead%
  \rowcolor{lightgray}
  {\textbf{Obiettivo}} & {\textbf{Metrica}} & {\textbf{Risultato}} & {\textbf{Esito}}  \\
  \endhead%
  \textbf{QoPR011 Attuare una verifica costante} & MoPR012 Frequenza di controlli & / & Soddisfatto \\
  \caption{Risultati metrica MoPR012}
  \label{tab:my-table}
\end{longtable}
\textbf{Nota}: I verificatori stanno continuando ad effettuare delle verifiche costanti ai prodotti.


\rowcolors{2}{white!80!lightgray!90}{white}
\renewcommand{\arraystretch}{2} % allarga le righe con dello spazio sotto e sopra
\begin{longtable}[H]{>{\centering\bfseries}m{5cm} >{\centering}m{5cm} >{\centering}m{2.5cm} >{\centering\arraybackslash}m{2.5cm}}  
  \rowcolor{lightgray}
  {\textbf{Obiettivo}} & {\textbf{Metrica}} & {\textbf{Risultato}} & {\textbf{Esito}}  \\
  \endfirsthead%
  \rowcolor{lightgray}
  {\textbf{Obiettivo}} & {\textbf{Metrica}} & {\textbf{Risultato}} & {\textbf{Esito}}  \\
  \endhead%
  \textbf{QoPR013 Rispettare le fasi di verifica} & MoPR010 Analisi Way of Working & / & Soddisfatto \\
  \caption{Risultati QoPR13}
  \label{tab:my-table}
\end{longtable}
\textbf{Nota}: Le fasi di verifica dei prodotti stanno vedendo rispettate secondo le indicazioni interne al gruppo.

\rowcolors{2}{white!80!lightgray!90}{white}
\renewcommand{\arraystretch}{2} % allarga le righe con dello spazio sotto e sopra
\begin{longtable}[H]{>{\centering\bfseries}m{5cm} >{\centering}m{5cm} >{\centering}m{2.5cm} >{\centering\arraybackslash}m{2.5cm}}  
  \rowcolor{lightgray}
  {\textbf{Obiettivo}} & {\textbf{Metrica}} & {\textbf{Risultato}} & {\textbf{Esito}}  \\
  \endfirsthead%
  \rowcolor{lightgray}
  {\textbf{Obiettivo}} & {\textbf{Metrica}} & {\textbf{Risultato}} & {\textbf{Esito}}  \\
  \endhead%
  \textbf{QoPR014 Soddisfare i test richiesti} & MoPR013 Percentuale di test soddisfatti & 88.35\% & Soddisfatto \\
  \caption{Risultati MoPR013}
  \label{tab:my-table}
\end{longtable}
\textbf{Nota}: Come preventivato, in questa ultima fase di progetto siamo riusciti ad ottenere un valore accettabile per quanto riguarda il numero di test implementati.

\begin{figure}[H]
  \centering
  \includegraphics[width=12cm]{img/testImplRa.png}
  \label{fig:test_implementati}
  \caption{Grafico con il numero test implementati}
\end{figure}

\begin{figure}[H]
  \centering
  \includegraphics[width=12cm]{img/percentTestImplRa.png}
  \label{fig:test_implementati}
  \caption{Grafico con la Percentuale di test implementati}
\end{figure}

\subsubsection{Esiti verifiche sui prodotti}

\label{sub:esiti_verifiche_sui_prodotti}
In questa sezione vengono visualizzati gli esiti delle metriche prese in considerazione per quanto riguarda i prodotti realizzati. Come per i documenti, anche per queste metriche verrà fornito un esito che può essere soddisfacente o meno.

\paragraph{Funzionalità}
\label{sub:funzionalita}

\rowcolors{2}{white!80!lightgray!90}{white}
\renewcommand{\arraystretch}{2} % allarga le righe con dello spazio sotto e sopra
\begin{longtable}[H]{>{\centering\bfseries}m{5cm} >{\centering}m{5cm} >{\centering}m{2.5cm} >{\centering\arraybackslash}m{2.5cm}}  
  \rowcolor{lightgray}
  {\textbf{Obiettivo}} & {\textbf{Metrica}} & {\textbf{Risultato}} & {\textbf{Esito}}  \\
  \endfirsthead%
  \rowcolor{lightgray}
  {\textbf{Obiettivo}} & {\textbf{Metrica}} & {\textbf{Risultato}} & {\textbf{Esito}}  \\
  \endhead%
  \textbf{QoPD001 Rispetto dell’implementazione funzionale - [Adeguatezza]} & MoPD001 Completezza di implementazione - [Adeguatezza] & 0 & Soddisfatto \\
  \caption{Risultati metrica MoPD001}
  \label{tab:my-table}
\end{longtable}
\textbf{Nota}: Tutte le funzioni da implementare in questa fase sono state implementate.

\rowcolors{2}{white!80!lightgray!90}{white}
\renewcommand{\arraystretch}{2} % allarga le righe con dello spazio sotto e sopra
\begin{longtable}[H]{>{\centering\bfseries}m{5cm} >{\centering}m{5cm} >{\centering}m{2.5cm} >{\centering\arraybackslash}m{2.5cm}}  
  \rowcolor{lightgray}
  {\textbf{Obiettivo}} & {\textbf{Metrica}} & {\textbf{Risultato}} & {\textbf{Esito}}  \\
  \endfirsthead%
  \rowcolor{lightgray}
  {\textbf{Obiettivo}} & {\textbf{Metrica}} & {\textbf{Risultato}} & {\textbf{Esito}}  \\
  \endhead%
  \textbf{QoPD002 Rispetto delle interfacce - [Interoperabilità]} & MoPD002 Coerenza di interfaccia - [Interoperabilità] & 92\% & Soddisfatto \\
  \caption{Risultati metrica MoPD002}
  \label{tab:my-table}
\end{longtable}
\textbf{Nota}: il 92\% delle interfacce realizzate è fedele a quanto preventivamente definito.

\paragraph{Affidabilità}
\label{sub:affidabilita}

\rowcolors{2}{white!80!lightgray!90}{white}
\renewcommand{\arraystretch}{2} % allarga le righe con dello spazio sotto e sopra
\begin{longtable}[H]{>{\centering\bfseries}m{5cm} >{\centering}m{5cm} >{\centering}m{2.5cm} >{\centering\arraybackslash}m{2.5cm}}  
  \rowcolor{lightgray}
  {\textbf{Obiettivo}} & {\textbf{Metrica}} & {\textbf{Risultato}} & {\textbf{Esito}}  \\
  \endfirsthead%
  \rowcolor{lightgray}
  {\textbf{Obiettivo}} & {\textbf{Metrica}} & {\textbf{Risultato}} & {\textbf{Esito}}  \\
  \endhead%
  \textbf{QoPD003 Test completi sul codice - [Maturità]} & MoPD003 Copertura dei test- [Maturità] & 90.55\% & Soddisfatto \\
  \caption{Risultati metrica MoPD003}
  \label{tab:my-table}
\end{longtable}
\textbf{Nota}: 

\begin{figure}[H]
  \centering
  \includegraphics[width=10cm]{img/coperturaTest.png}
  \label{fig:test_implementati}
  \caption{Grafico con la percentuale di copertura del codice}
\end{figure}

\rowcolors{2}{white!80!lightgray!90}{white}
\renewcommand{\arraystretch}{2} % allarga le righe con dello spazio sotto e sopra
\begin{longtable}[H]{>{\centering\bfseries}m{5cm} >{\centering}m{5cm} >{\centering}m{2.5cm} >{\centering\arraybackslash}m{2.5cm}}  
  \rowcolor{lightgray}
  {\textbf{Obiettivo}} & {\textbf{Metrica}} & {\textbf{Risultato}} & {\textbf{Esito}}  \\
  \endfirsthead%
  \rowcolor{lightgray}
  {\textbf{Obiettivo}} & {\textbf{Metrica}} & {\textbf{Risultato}} & {\textbf{Esito}}  \\
  \endhead%
  \textbf{QoPD004 Individuazione test falliti - [Affidabilità]} & MoPD004 Densità degli errori - [Affidabilità] & 5\% & Soddisfatto \\
  \caption{Risultati metrica MoPD004}
  \label{tab:my-table}
\end{longtable}
\textbf{Nota}: Al momento il numero dei test ha prodotto errori solo nel 5\% dei casi il che è un valore accettabile dal gruppo.

\paragraph{Usabilità}
\label{sub:usabilita}

\rowcolors{2}{white!80!lightgray!90}{white}
\renewcommand{\arraystretch}{2} % allarga le righe con dello spazio sotto e sopra
\begin{longtable}[H]{>{\centering\bfseries}m{5cm} >{\centering}m{5cm} >{\centering}m{2.5cm} >{\centering\arraybackslash}m{2.5cm}}  
  \rowcolor{lightgray}
  {\textbf{Obiettivo}} & {\textbf{Metrica}} & {\textbf{Risultato}} & {\textbf{Esito}}  \\
  \endfirsthead%
  \rowcolor{lightgray}
  {\textbf{Obiettivo}} & {\textbf{Metrica}} & {\textbf{Risultato}} & {\textbf{Esito}}  \\
  \endhead%
  \textbf{QoPD005 Chiarezza del comportamento - [Comprensibilità]} & MoPD005 Documentazione delle funzioni - [Comprensibilità] & 90\%  & Soddisfatto \\
  \caption{Risultati metrica MoPD005}
  \label{tab:my-table}
\end{longtable}
\textbf{Nota}: Quasi tutte le funzioni implementate sono state descritte con un commento, tranne le funzioni che non richiedono particolare spiegazione.

\rowcolors{2}{white!80!lightgray!90}{white}
\renewcommand{\arraystretch}{2} % allarga le righe con dello spazio sotto e sopra
\begin{longtable}[H]{>{\centering\bfseries}m{5cm} >{\centering}m{5cm} >{\centering}m{2.5cm} >{\centering\arraybackslash}m{2.5cm}}  
  \rowcolor{lightgray}
  {\textbf{Obiettivo}} & {\textbf{Metrica}} & {\textbf{Risultato}} & {\textbf{Esito}}  \\
  \endfirsthead%
  \rowcolor{lightgray}
  {\textbf{Obiettivo}} & {\textbf{Metrica}} & {\textbf{Risultato}} & {\textbf{Esito}}  \\
  \endhead%
  \textbf{QoPD006 Chiarimento degli errori - [Comprensibilità]} & MoPD006 Messaggi di errore - [Comprensibilità] & 6\% & Soddisfatto \\
  \caption{Risultati metrica MoPD006}
  \label{tab:my-table}
\end{longtable}
\textbf{Nota}: Il valore è migliorate in quest'ultima fase e rimane soddisfacente per il gruppo.

\paragraph{Efficienza}
\label{sub:efficienza}

\rowcolors{2}{white!80!lightgray!90}{white}
\renewcommand{\arraystretch}{2} % allarga le righe con dello spazio sotto e sopra
\begin{longtable}[H]{>{\centering\bfseries}m{5cm} >{\centering}m{5cm} >{\centering}m{2.5cm} >{\centering\arraybackslash}m{2.5cm}}  
  \rowcolor{lightgray}
  {\textbf{Obiettivo}} & {\textbf{Metrica}} & {\textbf{Risultato}} & {\textbf{Esito}}  \\
  \endfirsthead%
  \rowcolor{lightgray}
  {\textbf{Obiettivo}} & {\textbf{Metrica}} & {\textbf{Risultato}} & {\textbf{Esito}}  \\
  \endhead%
  \textbf{QoPD007 Velocità di esecuzione - [Comportamento temporale]} & MoPD007 Tempo medio di risposta - [Comportamento temporale] & 1.66 & Soddisfatto \\
  \caption{Risultati metrica MoPD007}
  \label{tab:my-table}
\end{longtable}
\textbf{Nota}: Il tempo di risposta soddisfa le aspettative di qualità per la metrica per essendo leggermente aumentato.
\paragraph{Manutenibilità}
\label{sub:manutenibilita}

\rowcolors{2}{white!80!lightgray!90}{white}
\renewcommand{\arraystretch}{2} % allarga le righe con dello spazio sotto e sopra
\begin{longtable}[H]{>{\centering\bfseries}m{5cm} >{\centering}m{5cm} >{\centering}m{2.5cm} >{\centering\arraybackslash}m{2.5cm}}  
  \rowcolor{lightgray}
  {\textbf{Obiettivo}} & {\textbf{Metrica}} & {\textbf{Risultato}} & {\textbf{Esito}}  \\
  \endfirsthead%
  \rowcolor{lightgray}
  {\textbf{Obiettivo}} & {\textbf{Metrica}} & {\textbf{Risultato}} & {\textbf{Esito}}  \\
  \endhead%
  \textbf{QoPD08 Comprensione del codice - [Modifica]} & MoPD008 Commenti sul codice - [Modifica] & 19.3\% & Soddisfatto \\
  \caption{Risultati metrica MoPD006}
  \label{tab:my-table}
\end{longtable}
\textbf{Nota}: Il valore della metrica è leggermente aumentato rispetto alla revisione precedente rimanendo quindi soddisfacente.

\subsubsection{Conclusioni}%
\label{sub:conclusioni}
In questa fase il team ha provveduto principalmente a sanare le lacune riguardanti l'implementazione dei test che risultavano essere carenti nella fase di sviluppo precedente.
Sono stati raggiunti tutti i valori accettabili per ciascuna metrica definita, il che rende il gruppo di lavoro molto soddisfatto, pur avendo comunque capito di aver commesso diversi errori durante ogni fase di sviluppo del processo, tali considerazioni possono essere consultate nella sezione appendice §B.4.
