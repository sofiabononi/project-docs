
\subsubsection{Esiti verifiche sui documenti}
\label{sub:esiti_verifiche_sui_documenti}
Sui documenti sono state utilizzate le metriche per:
    \begin{itemize}
        \item \textbf{Indice di Gulpease};
        \item \textbf{Correttezza lessicale/ortografica}.
    \end{itemize}
Per evitare risultati errati nel calcole dell'Indice di Gulpease si è deciso di non tenere in considerazione:
    \begin{itemize}
        \item Il frontespizio di ogni documento;
        \item Le tabelle presenti nei documenti;
        \item Il diario delle modifiche all'interno di ogni documento.
    \end{itemize}
Per quanto riguarda la correttezza lessicale/ortografica, gli errori trovati non verranno classificati ma semplicemente conteggiati uniformemente includendo anche errori derivati dal mancato rispetto delle convenzioni scritte nelle \textsc{Norme di Progetto v1.9-2.3.3}.
\\ Per ogni metrica di ogni documento verrà inoltre inserito l'esito dell'analisi che può avere risultato soddisfacente o non.

\paragraph{Norme di Progetto}
\label{sub:norme_di_progetto}
Le \textsc{Norme di Progetto v1.9-2.3.3} ha subito sostanziali cambiamenti specialmente per quanto riguarda la parte della definizione degli strumenti utilizzati per la rilevazione delle metriche il che ha portato ad un leggero aumento degli errori nelle ultime fasi di approvazione, il che ha comunque portato ad avere 0 errori nel documento.
Il suo indice di Gulpease finale è salito di 1 punto rispetto alla fase di sviluppo precedente.

\rowcolors{2}{white!80!lightgray!90}{white}
\renewcommand{\arraystretch}{2} % allarga le righe con dello spazio sotto e sopra
\begin{longtable}[H]{>{\centering\bfseries}m{6cm} >{\centering}m{2cm} >{\centering}m{2.5cm} >{\centering}m{2.5cm} >{\centering\arraybackslash}m{2.5cm}}  
  \rowcolor{lightgray}
  {\textbf{Documento}} & {\textbf{Risultato indice}} & {\textbf{Errori presenti}} & {\textbf{Esito indice}} & {\textbf{Esito errori}}  \\
  \endfirsthead%
  \rowcolor{lightgray}
  {\textbf{Documento}} & {\textbf{Risultato indice}} & {\textbf{Errori presenti}} & {\textbf{Esito indice}} & {\textbf{Esito errori}}  \\
  \endhead%
  \textbf{Norme di Progetto v1.9-2.3.3} & 71               & 0               & Soddisfatto & Soddisfatto \\
  \caption{Risultati metriche per le Norme di Progetto v1.9-2.3.3}
  \label{tab:my-table}
\end{longtable}

\begin{figure}[H]
  \centering
  \includegraphics[width=10cm]{img/erroriNdPv1.9-2.3.3.png}
  \label{fig:errori_ndp}
  \caption{Grafico errori per \textsc{Norme di Progetto v1.9-2.3.3}}
\end{figure}

\begin{figure}[H]
  \centering
  \includegraphics[width=10cm]{img/gulpeaseNdPv1.9-2.3.3.png}
  \label{fig:gulpease_ndp}
  \caption{Grafico indice di Gulpease per \textsc{Norme di Progetto v1.9-2.3.3}}
\end{figure}


\paragraph{Piano di Qualifica}
\label{sub:piano_di_qualifica}
Il \textsc{Piano di Qualifica v1.9-3.1.0} è stato il documento con più fasi di verifica il che ha portato ad avere una ricerca degli errori più approfondita.
Il suo indice di Gulpease è rimasto invariato rispetto alla fase di sviluppo precedente.

\rowcolors{2}{white!80!lightgray!90}{white}
\renewcommand{\arraystretch}{2} % allarga le righe con dello spazio sotto e sopra
\begin{longtable}[H]{>{\centering\bfseries}m{6cm} >{\centering}m{2cm} >{\centering}m{2.5cm} >{\centering}m{2.5cm} >{\centering\arraybackslash}m{2.5cm}}  
  \rowcolor{lightgray}
  {\textbf{Documento}} & {\textbf{Risultato indice}} & {\textbf{Errori presenti}} & {\textbf{Esito indice}} & {\textbf{Esito errori}}  \\
  \endfirsthead%
  \rowcolor{lightgray}
  {\textbf{Documento}} & {\textbf{Risultato indice}} & {\textbf{Errori presenti}} & {\textbf{Esito indice}} & {\textbf{Esito errori}}  \\
  \endhead%
  \textbf{Piano di Qualifica v1.9-3.1.0} & 72              & 0               & Soddisfatto & Soddisfatto \\
  \caption{Risultati metriche per il Piano di Qualifica v1.9-3.1.0}
  \label{tab:my-table}
\end{longtable}

\begin{figure}[H]
  \centering
  \includegraphics[width=10cm]{img/erroriPdQv1.9-3.1.0.png}
  \label{fig:errori_pdq}
  \caption{Grafico errori per \textsc{Piano di Qualifica v1.9-3.1.0}}
\end{figure}

\begin{figure}[H]
  \centering
  \includegraphics[width=10cm]{img/GulpeasePdQv1.9-3.1.0.png}
  \label{fig:gulpease_pdq}
  \caption{Grafico indice di Gulpease per \textsc{Piano di Qualifica v1.9-3.1.0}}
\end{figure}


\paragraph{Piano di Progetto}
\label{sub:piano_di_progetto}
Il \textsc{Piano di Progetto v1.9-4.2.0} ha subito parecchie modifiche che hanno richiesto molte fasi di verifica e approvazione che hanno sanato tutti gli errori.
Il suo indice di Gulpease è aumentato di 1 rispetto alla fase di sviluppo precedente.

\rowcolors{2}{white!80!lightgray!90}{white}
\renewcommand{\arraystretch}{2} % allarga le righe con dello spazio sotto e sopra
\begin{longtable}[H]{>{\centering\bfseries}m{6cm} >{\centering}m{2cm} >{\centering}m{2.5cm} >{\centering}m{2.5cm} >{\centering\arraybackslash}m{2.5cm}}  
  \rowcolor{lightgray}
  {\textbf{Documento}} & {\textbf{Risultato indice}} & {\textbf{Errori presenti}} & {\textbf{Esito indice}} & {\textbf{Esito errori}}  \\
  \endfirsthead%
  \rowcolor{lightgray}
  {\textbf{Documento}} & {\textbf{Risultato indice}} & {\textbf{Errori presenti}} & {\textbf{Esito indice}} & {\textbf{Esito errori}}  \\
  \endhead%
  \textbf{Piano di Progetto v1.9-4.2.0} & 73               & 0               & Soddisfatto & Soddisfatto \\
  \caption{Risultati metriche per il Piano di Progetto v1.9-4.2.0}
  \label{tab:my-table}
\end{longtable}

\begin{figure}[H]
  \centering
  \includegraphics[width=10cm]{img/erroriPdPv1.9-4.2.0.png}
  \label{fig:errori_pdq}
  \caption{Grafico errori per \textsc{Piano di Progetto v1.9-4.2.0}}
\end{figure}

\begin{figure}[H]
  \centering
  \includegraphics[width=10cm]{img/gulpeasePdPv1.9-4.2.0.png}
  \label{fig:gulpease_pdq}
  \caption{Grafico indice di Gulpease per \textsc{Piano di Progetto v1.9-4.2.0}}
\end{figure}


\paragraph{Analisi dei Requisiti}
\label{sub:analisi_dei_requisiti}
Le principali modifiche all'\textsc{Analisi dei Requisiti v1.9-1.3.0} riguardano l'aggiunta di nuovi casi d'uso e la modifica alla numerazione di certi requisiti, sono quindi state svolte solo 2 verifiche e approvazioni del documento.
Il suo indice di Gulpease è rimasto invariato rispetto alla fase di sviluppo precedente.

\rowcolors{2}{white!80!lightgray!90}{white}
\renewcommand{\arraystretch}{2} % allarga le righe con dello spazio sotto e sopra
\begin{longtable}[H]{>{\centering\bfseries}m{6cm} >{\centering}m{2cm} >{\centering}m{2.5cm} >{\centering}m{2.5cm} >{\centering\arraybackslash}m{2.5cm}}  
  \rowcolor{lightgray}
  {\textbf{Documento}} & {\textbf{Risultato indice}} & {\textbf{Errori presenti}} & {\textbf{Esito indice}} & {\textbf{Esito errori}}  \\
  \endfirsthead%
  \rowcolor{lightgray}
  {\textbf{Documento}} & {\textbf{Risultato indice}} & {\textbf{Errori presenti}} & {\textbf{Esito indice}} & {\textbf{Esito errori}}  \\
  \endhead%
  \textbf{Analisi dei Requisiti v1.9-1.3.0} &  73              & 0               & Soddisfatto & Soddisfatto \\
  \caption{Risultati metriche per l'Analisi dei Requisiti v1.9-1.3.0}
  \label{tab:my-table}
\end{longtable}

\begin{figure}[H]
  \centering
  \includegraphics[width=10cm]{img/erroriAdRv1.9-1.3.0.png}
  \label{fig:errori_adr}
  \caption{Grafico errori per \textsc{Analisi dei Requisiti v1.9-1.3.0}}
\end{figure}

\begin{figure}[H]
  \centering
  \includegraphics[width=10cm]{img/gulpeaseAdrv1.9-1.3.0.png}
  \label{fig:gulpease_adr}
  \caption{Grafico indice di Gulpease per \textsc{Analisi dei Requisiti v1.9-1.3.0}}
\end{figure}


\paragraph{Glossario}
\label{sub:glossario}
Il \textsc{Glossario v1.9-1.3.0} ha subito modifiche poco rilevanti per cui è bastata una sola revisione e approvazione generale al documento.
Il suo indice di Gulpease è rimasto invariato rispetto alla fase di sviluppo precedente.

\rowcolors{2}{white!80!lightgray!90}{white}
\renewcommand{\arraystretch}{2} % allarga le righe con dello spazio sotto e sopra
\begin{longtable}[H]{>{\centering\bfseries}m{6cm} >{\centering}m{2cm} >{\centering}m{2.5cm} >{\centering}m{2.5cm} >{\centering\arraybackslash}m{2.5cm}}  
  \rowcolor{lightgray}
  {\textbf{Documento}} & {\textbf{Risultato indice}} & {\textbf{Errori presenti}} & {\textbf{Esito indice}} & {\textbf{Esito errori}}  \\
  \endfirsthead%
  \rowcolor{lightgray}
  {\textbf{Documento}} & {\textbf{Risultato indice}} & {\textbf{Errori presenti}} & {\textbf{Esito indice}} & {\textbf{Esito errori}}  \\
  \endhead%
  \textbf{Glossario v1.9-1.3.0} & 65               & 0               & Soddisfatto & Soddisfatto \\
  \caption{Risultati metriche per il Glossario v1.9-1.3.0}
  \label{tab:my-table}
\end{longtable}

A causa delle poche modifiche e cambiamenti significativi non sono stati inseriti i grafici che non comunicavano alcuna informazione aggiuntiva.

\paragraph{Manuale Utente}
\label{sub:glossario}
Il \textsc{Manuale Utente v1.9-0.4.0} è stato redatto in questa fase di sviluppo e ha riportato moderati errori errori durante le sua verifiche che sono stati sanati.
Il suo indice di Gulpease in questa fase di sviluppo è 73.

\rowcolors{2}{white!80!lightgray!90}{white}
\renewcommand{\arraystretch}{2} % allarga le righe con dello spazio sotto e sopra
\begin{longtable}[H]{>{\centering\bfseries}m{6cm} >{\centering}m{2cm} >{\centering}m{2.5cm} >{\centering}m{2.5cm} >{\centering\arraybackslash}m{2.5cm}}  
  \rowcolor{lightgray}
  {\textbf{Documento}} & {\textbf{Risultato indice}} & {\textbf{Errori presenti}} & {\textbf{Esito indice}} & {\textbf{Esito errori}}  \\
  \endfirsthead%
  \rowcolor{lightgray}
  {\textbf{Documento}} & {\textbf{Risultato indice}} & {\textbf{Errori presenti}} & {\textbf{Esito indice}} & {\textbf{Esito errori}}  \\
  \endhead%
  \textbf{Manuale Utente v1.9-0.4.0} & 73               & 0               & Soddisfatto & Soddisfatto \\
  \caption{Risultati metriche per il Manuale Utente v1.9-0.4.0}
  \label{tab:my-table}
\end{longtable}

\begin{figure}[H]
  \centering
  \includegraphics[width=10cm]{img/erroriMUv1.9-0.4.0.png}
  \label{fig:errori_adr}
  \caption{Grafico errori per \textsc{Manuale Utente v1.9-0.4.0}}
\end{figure}

\begin{figure}[H]
  \centering
  \includegraphics[width=10cm]{img/gulpeaseMUv1.9-0.4.0.png}
  \label{fig:gulpease_adr}
  \caption{Grafico indice di Gulpease per \textsc{Manuale Utente v1.9-0.4.0}}
\end{figure}

\paragraph{Manuale Sviluppatore}
\label{sub:glossario}
Il \textsc{Manuale Sviluppatore v1.9-1.0.0}, come per il \textsc{Manuale Utente v1.9-0.4.0}, è stato redatto in questa fase e ha avuto una curva di errori più altalenante che sono stati comunque sanati.
Il suo indice di Gulpease in questa fase di sviluppo è 71.

\rowcolors{2}{white!80!lightgray!90}{white}
\renewcommand{\arraystretch}{2} % allarga le righe con dello spazio sotto e sopra
\begin{longtable}[H]{>{\centering\bfseries}m{6cm} >{\centering}m{2cm} >{\centering}m{2.5cm} >{\centering}m{2.5cm} >{\centering\arraybackslash}m{2.5cm}}  
  \rowcolor{lightgray}
  {\textbf{Documento}} & {\textbf{Risultato indice}} & {\textbf{Errori presenti}} & {\textbf{Esito indice}} & {\textbf{Esito errori}}  \\
  \endfirsthead%
  \rowcolor{lightgray}
  {\textbf{Documento}} & {\textbf{Risultato indice}} & {\textbf{Errori presenti}} & {\textbf{Esito indice}} & {\textbf{Esito errori}}  \\
  \endhead%
  \textbf{Manuale Sviluppatore v1.9-1.0.0} & 71               & 0               & Soddisfatto & Soddisfatto \\
  \caption{Risultati metriche per il Manuale Sviluppatore v1.9-1.0.0}
  \label{tab:my-table}
\end{longtable}

\begin{figure}[H]
  \centering
  \includegraphics[width=10cm]{img/erroriMSv1.9-1.0.0.png}
  \label{fig:errori_adr}
  \caption{Grafico errori per \textsc{Manuale Sviluppatore v1.9-1.0.0}}
\end{figure}

\begin{figure}[H]
  \centering
  \includegraphics[width=10cm]{img/gulpeaseMSv1.9-1.0.0.png}
  \label{fig:gulpease_adr}
  \caption{Grafico indice di Gulpease per \textsc{Manuale Sviluppatore v1.9-1.0.0}}
\end{figure}

\subsubsection{Esiti verifiche sui processi}
\label{sub:esiti_verifiche_sui_processi}
In questa sezione vengono visualizzati gli esiti delle metriche prese in considerazione per quanto riguarda i processi produttivi.
Come per i documenti, anche per queste metriche verrà fornito un esito che può essere soddisfacente o meno.

\paragraph{Processo PRC001}
\label{sub:processo_PRC001}

\rowcolors{2}{white!80!lightgray!90}{white}
\renewcommand{\arraystretch}{2} % allarga le righe con dello spazio sotto e sopra
\begin{longtable}[H]{>{\centering\bfseries}m{5cm} >{\centering}m{5cm} >{\centering}m{2.5cm} >{\centering\arraybackslash}m{2.5cm}}  
  \rowcolor{lightgray}
  {\textbf{Obiettivo}} & {\textbf{Metrica}} & {\textbf{Risultato}} & {\textbf{Esito}}  \\
  \endfirsthead%
  \rowcolor{lightgray}
  {\textbf{Obiettivo}} & {\textbf{Metrica}} & {\textbf{Risultato}} & {\textbf{Esito}}  \\
  \endhead%
  \textbf{QoPR001 Rispetto delle scadenze della pianificazione} & MoPR001 Varianza dei tempi & 1.54  & Soddisfatto  \\
  \caption{Risultati metrica MoPR001}
  \label{tab:my-table}
\end{longtable}
\textbf{Nota}: Il valore metrica di riferimento è aumentato rispetto alle prime 2 fasi di progetto, dovuto anche a problemi descritti nella sezione appendice §B, tale valore rientra comunque nel range di soddisfacimento del gruppo.

\rowcolors{2}{white!80!lightgray!90}{white}
\renewcommand{\arraystretch}{2} % allarga le righe con dello spazio sotto e sopra
\begin{longtable}[H]{>{\centering\bfseries}m{5cm} >{\centering}m{5cm} >{\centering}m{2.5cm} >{\centering\arraybackslash}m{2.5cm}}  
  \rowcolor{lightgray}
  {\textbf{Obiettivo}} & {\textbf{Metrica}} & {\textbf{Risultato}} & {\textbf{Esito}}  \\
  \endfirsthead%
  \rowcolor{lightgray}
  {\textbf{Obiettivo}} & {\textbf{Metrica}} & {\textbf{Risultato}} & {\textbf{Esito}}  \\
  \endhead%
  \textbf{QoPR002 Rispetto del budget istanziato} & MoPR002 Varianza dei costi & -2 & Soddisfatto  \\
  \caption{Risultati metrica MoPR002}
  \label{tab:my-table}
\end{longtable}
\textbf{Nota}: La variazione rispetto al preventivo iniziale rientra nel range deciso per la metrica, per cui l'obiettivo è stato soddisfatto. Il discostamento dal preventivo totale rilevato è di: -2 euro.

\rowcolors{2}{white!80!lightgray!90}{white}
\renewcommand{\arraystretch}{2} % allarga le righe con dello spazio sotto e sopra
\begin{longtable}[H]{>{\centering\bfseries}m{5cm} >{\centering}m{5cm} >{\centering}m{2.5cm} >{\centering\arraybackslash}m{2.5cm}}  
  \rowcolor{lightgray}
  {\textbf{Obiettivo}} & {\textbf{Metrica}} & {\textbf{Risultato}} & {\textbf{Esito}}  \\
  \endfirsthead%
  \rowcolor{lightgray}
  {\textbf{Obiettivo}} & {\textbf{Metrica}} & {\textbf{Risultato}} & {\textbf{Esito}}  \\
  \endhead%
  \textbf{QoPR003 Rispetto del ciclo di vita scelto} & MoPR003 Aderenza agli standard & Livello di maturità: 3 \\ Valutazione attributi: L  &  Soddisfatto \\
  \caption{Risultati metrica MoPR003}
  \label{tab:my-table}
\end{longtable}
\textbf{Nota}: IL miglioramento per quanto riguarda l'aderenza agli standard è presente ed è, già adesso, soddisfacente per il gruppo.

\rowcolors{2}{white!80!lightgray!90}{white}
\renewcommand{\arraystretch}{2} % allarga le righe con dello spazio sotto e sopra
\begin{longtable}[H]{>{\centering\bfseries}m{5cm} >{\centering}m{5cm} >{\centering}m{2.5cm} >{\centering\arraybackslash}m{2.5cm}}  
  \rowcolor{lightgray}
  {\textbf{Obiettivo}} & {\textbf{Metrica}} & {\textbf{Risultato}} & {\textbf{Esito}}  \\
  \endfirsthead%
  \rowcolor{lightgray}
  {\textbf{Obiettivo}} & {\textbf{Metrica}} & {\textbf{Risultato}} & {\textbf{Esito}}  \\
  \endhead%
  \textbf{QoPR004 Rispetto dei ruoli e identificazione nei prodotti} & MoPR004 Aderenza ai ruoli & 0 & Soddisfatto \\
  \caption{Risultati metrica MoPR004}
  \label{tab:my-table}
\end{longtable}
\textbf{Nota}: In ogni documento prodotto i ruoli sono stati identificati e rispettati.

\rowcolors{2}{white!80!lightgray!90}{white}
\renewcommand{\arraystretch}{2} % allarga le righe con dello spazio sotto e sopra
\begin{longtable}[H]{>{\centering\bfseries}m{5cm} >{\centering}m{5cm} >{\centering}m{2.5cm} >{\centering\arraybackslash}m{2.5cm}}  
  \rowcolor{lightgray}
  {\textbf{Obiettivo}} & {\textbf{Metrica}} & {\textbf{Risultato}} & {\textbf{Esito}}  \\
  \endfirsthead%
  \rowcolor{lightgray}
  {\textbf{Obiettivo}} & {\textbf{Metrica}} & {\textbf{Risultato}} & {\textbf{Esito}}  \\
  \endhead%
  \textbf{QoPR005 Rispetto del versionamento dei prodotti} & MoPR005 Controllo prodotti & 26.4 & Soddisfatto \\
  \caption{Risultati metrica MoPR005}
  \label{tab:my-table}
\end{longtable}
\textbf{Nota}: Il numero di commit in questa fase è stato parecchio elevato, il che soddisfa appieno l'obiettivo prefissato.

\paragraph{Processo PRC002}
\label{sub:processo_PRC002}

\rowcolors{2}{white!80!lightgray!90}{white}
\renewcommand{\arraystretch}{2} % allarga le righe con dello spazio sotto e sopra
\begin{longtable}[H]{>{\centering\bfseries}m{5cm} >{\centering}m{5cm} >{\centering}m{2.5cm} >{\centering\arraybackslash}m{2.5cm}}  
  \rowcolor{lightgray}
  {\textbf{Obiettivo}} & {\textbf{Metrica}} & {\textbf{Risultato}} & {\textbf{Esito}}  \\
  \endfirsthead%
  \rowcolor{lightgray}
  {\textbf{Obiettivo}} & {\textbf{Metrica}} & {\textbf{Risultato}} & {\textbf{Esito}}  \\
  \endhead%
  \textbf{QoPR006 Soddisfazione dei requisiti obbligatori} & MoPR006 Verifica requisiti obbligatori & 94\% & Non soddisfatto \\
  \caption{Risultati metrica MoPR006}
  \label{tab:my-table}
\end{longtable}
\textbf{Nota}: Il valore della metrica non è ancora soddisfacente ma buono in quanto mancano solo da sanare 2 requisiti obbligatori.

\rowcolors{2}{white!80!lightgray!90}{white}
\renewcommand{\arraystretch}{2} % allarga le righe con dello spazio sotto e sopra
\begin{longtable}[H]{>{\centering\bfseries}m{5cm} >{\centering}m{5cm} >{\centering}m{2.5cm} >{\centering\arraybackslash}m{2.5cm}}  
  \rowcolor{lightgray}
  {\textbf{Obiettivo}} & {\textbf{Metrica}} & {\textbf{Risultato}} & {\textbf{Esito}}  \\
  \endfirsthead%
  \rowcolor{lightgray}
  {\textbf{Obiettivo}} & {\textbf{Metrica}} & {\textbf{Risultato}} & {\textbf{Esito}}  \\
  \endhead%
  \textbf{QoPR007 Soddisfazione dei requisiti opzionali e desiderabili} & MoPR007 Verifica requisiti opzionali \\ MoPR008 Verifica requisiti desiderabili & / & Non soddisfatto \\
  \caption{Risultati metrica MoPR007 e metrica MoPR008}
  \label{tab:my-table}
\end{longtable}
\textbf{Nota}: Non sono ancora stati soddisfatti requisiti facoltativi/desiderabili importanti.

\rowcolors{2}{white!80!lightgray!90}{white}
\renewcommand{\arraystretch}{2} % allarga le righe con dello spazio sotto e sopra
\begin{longtable}[H]{>{\centering\bfseries}m{5cm} >{\centering}m{5cm} >{\centering}m{2.5cm} >{\centering\arraybackslash}m{2.5cm}}  
  \rowcolor{lightgray}
  {\textbf{Obiettivo}} & {\textbf{Metrica}} & {\textbf{Risultato}} & {\textbf{Esito}}  \\
  \endfirsthead%
  \rowcolor{lightgray}
  {\textbf{Obiettivo}} & {\textbf{Metrica}} & {\textbf{Risultato}} & {\textbf{Esito}}  \\
  \endhead%
  \textbf{QoPR008 Verifica dei rischi previsti} & MoPR009 Verifica rischi non pervenuti & 0 & Soddisfatto \\
  \caption{Risultati metrica MoPR009}
  \label{tab:my-table}
\end{longtable}
\textbf{Nota}: Non sono ancora stati rivelati rischi importanti che non siano stati previsti precedentemente dal gruppo.

\paragraph{Processo PRC003}
\label{sub:processo_PRC003}

\rowcolors{2}{white!80!lightgray!90}{white}
\renewcommand{\arraystretch}{2} % allarga le righe con dello spazio sotto e sopra
\begin{longtable}[H]{>{\centering\bfseries}m{5cm} >{\centering}m{5cm} >{\centering}m{2.5cm} >{\centering\arraybackslash}m{2.5cm}}  
  \rowcolor{lightgray}
  {\textbf{Obiettivo}} & {\textbf{Metrica}} & {\textbf{Risultato}} & {\textbf{Esito}}  \\
  \endfirsthead%
  \rowcolor{lightgray}
  {\textbf{Obiettivo}} & {\textbf{Metrica}} & {\textbf{Risultato}} & {\textbf{Esito}}  \\
  \endhead%
  \textbf{QoPR09 Rispetto delle fasi del ciclo di vita} & MoPR010 Analisi Way of Working & / & Soddisfatto \\
  \caption{Risultati metrica MoPR010}
  \label{tab:my-table}
\end{longtable}
\textbf{Nota}: I componenti del gruppo svolgono costantemente un aggiornamento in base alle modifiche fatte alle \textsc{Norme di Progetto}.

\rowcolors{2}{white!80!lightgray!90}{white}
\renewcommand{\arraystretch}{2} % allarga le righe con dello spazio sotto e sopra
\begin{longtable}[H]{>{\centering\bfseries}m{5cm} >{\centering}m{5cm} >{\centering}m{2.5cm} >{\centering\arraybackslash}m{2.5cm}}  
  \rowcolor{lightgray}
  {\textbf{Obiettivo}} & {\textbf{Metrica}} & {\textbf{Risultato}} & {\textbf{Esito}}  \\
  \endfirsthead%
  \rowcolor{lightgray}
  {\textbf{Obiettivo}} & {\textbf{Metrica}} & {\textbf{Risultato}} & {\textbf{Esito}}  \\
  \endhead%
  \textbf{QoPR010 Rispetto nella redazione dei documenti} & MoPR011 Analisi documenti & 3+ & Soddisfatto \\
  \caption{Risultati metrica MoPR011}
  \label{tab:my-table}
\end{longtable}
\textbf{Nota}: Tutti i documenti, ad eccezione del glossario,	sono stati verificati e approvati almeno 3 volte il che soddisfa pienamente le aspettative del gruppo.

\paragraph{Processo PRC004}
\label{sub:processo_PRC004}

\rowcolors{2}{white!80!lightgray!90}{white}
\renewcommand{\arraystretch}{2} % allarga le righe con dello spazio sotto e sopra
\begin{longtable}[H]{>{\centering\bfseries}m{5cm} >{\centering}m{5cm} >{\centering}m{2.5cm} >{\centering\arraybackslash}m{2.5cm}}  
  \rowcolor{lightgray}
  {\textbf{Obiettivo}} & {\textbf{Metrica}} & {\textbf{Risultato}} & {\textbf{Esito}}  \\
  \endfirsthead%
  \rowcolor{lightgray}
  {\textbf{Obiettivo}} & {\textbf{Metrica}} & {\textbf{Risultato}} & {\textbf{Esito}}  \\
  \endhead%
  \textbf{QoPR011 Attuare una verifica costante} & MoPR012 Frequenza di controlli & / & Soddisfatto \\
  \caption{Risultati metrica MoPR012}
  \label{tab:my-table}
\end{longtable}
\textbf{Nota}: I verificatori stanno continuando ad effettuare delle verifiche costanti ai prodotti.


\rowcolors{2}{white!80!lightgray!90}{white}
\renewcommand{\arraystretch}{2} % allarga le righe con dello spazio sotto e sopra
\begin{longtable}[H]{>{\centering\bfseries}m{5cm} >{\centering}m{5cm} >{\centering}m{2.5cm} >{\centering\arraybackslash}m{2.5cm}}  
  \rowcolor{lightgray}
  {\textbf{Obiettivo}} & {\textbf{Metrica}} & {\textbf{Risultato}} & {\textbf{Esito}}  \\
  \endfirsthead%
  \rowcolor{lightgray}
  {\textbf{Obiettivo}} & {\textbf{Metrica}} & {\textbf{Risultato}} & {\textbf{Esito}}  \\
  \endhead%
  \textbf{QoPR013 Rispettare le fasi di verifica} & MoPR010 Analisi Way of Working & / & Soddisfatto \\
  \caption{Risultati QoPR13}
  \label{tab:my-table}
\end{longtable}
\textbf{Nota}: Le fasi di verifica dei prodotti stanno vedendo rispettate secondo le indicazioni interne al gruppo.

\rowcolors{2}{white!80!lightgray!90}{white}
\renewcommand{\arraystretch}{2} % allarga le righe con dello spazio sotto e sopra
\begin{longtable}[H]{>{\centering\bfseries}m{5cm} >{\centering}m{5cm} >{\centering}m{2.5cm} >{\centering\arraybackslash}m{2.5cm}}  
  \rowcolor{lightgray}
  {\textbf{Obiettivo}} & {\textbf{Metrica}} & {\textbf{Risultato}} & {\textbf{Esito}}  \\
  \endfirsthead%
  \rowcolor{lightgray}
  {\textbf{Obiettivo}} & {\textbf{Metrica}} & {\textbf{Risultato}} & {\textbf{Esito}}  \\
  \endhead%
  \textbf{QoPR014 Soddisfare i test richiesti} & MoPR013 Percentuale di test soddisfatti & 7\% & Non soddisfatto \\
  \caption{Risultati MoPR013}
  \label{tab:my-table}
\end{longtable}
\textbf{Nota}: La percentuale di test soddisfatti e quindi realizzati è ancora bassa dati i pochi test implementati in questa fase che saranno oggetto di implementazione nella fase successiva.

\subsubsection{Esiti verifiche sui prodotti}
\label{sub:esiti_verifiche_sui_prodotti}
In questa sezione vengono visualizzati gli esiti delle metriche prese in considerazione per quanto riguarda i prodotti realizzati. Come per i documenti, anche per queste metriche verrà fornito un esito che può essere soddisfacente o meno.

\paragraph{Funzionalità}
\label{sub:funzionalita}

\rowcolors{2}{white!80!lightgray!90}{white}
\renewcommand{\arraystretch}{2} % allarga le righe con dello spazio sotto e sopra
\begin{longtable}[H]{>{\centering\bfseries}m{5cm} >{\centering}m{5cm} >{\centering}m{2.5cm} >{\centering\arraybackslash}m{2.5cm}}  
  \rowcolor{lightgray}
  {\textbf{Obiettivo}} & {\textbf{Metrica}} & {\textbf{Risultato}} & {\textbf{Esito}}  \\
  \endfirsthead%
  \rowcolor{lightgray}
  {\textbf{Obiettivo}} & {\textbf{Metrica}} & {\textbf{Risultato}} & {\textbf{Esito}}  \\
  \endhead%
  \textbf{QoPD001 Rispetto dell’implementazione funzionale - [Adeguatezza]} & MoPD001 Completezza di implementazione - [Adeguatezza] & 0 & Soddisfatto \\
  \caption{Risultati metrica MoPD001}
  \label{tab:my-table}
\end{longtable}
\textbf{Nota}: Tutte le funzioni da implementare in questa fase sono state implementate.

\rowcolors{2}{white!80!lightgray!90}{white}
\renewcommand{\arraystretch}{2} % allarga le righe con dello spazio sotto e sopra
\begin{longtable}[H]{>{\centering\bfseries}m{5cm} >{\centering}m{5cm} >{\centering}m{2.5cm} >{\centering\arraybackslash}m{2.5cm}}  
  \rowcolor{lightgray}
  {\textbf{Obiettivo}} & {\textbf{Metrica}} & {\textbf{Risultato}} & {\textbf{Esito}}  \\
  \endfirsthead%
  \rowcolor{lightgray}
  {\textbf{Obiettivo}} & {\textbf{Metrica}} & {\textbf{Risultato}} & {\textbf{Esito}}  \\
  \endhead%
  \textbf{QoPD002 Rispetto delle interfacce - [Interoperabilità]} & MoPD002 Coerenza di interfaccia - [Interoperabilità] & 90\% & Soddisfatto \\
  \caption{Risultati metrica MoPD002}
  \label{tab:my-table}
\end{longtable}
\textbf{Nota}: il 90\% delle interfacce realizzate è fedele a quanto preventivamente definito.

\paragraph{Affidabilità}
\label{sub:affidabilita}

\rowcolors{2}{white!80!lightgray!90}{white}
\renewcommand{\arraystretch}{2} % allarga le righe con dello spazio sotto e sopra
\begin{longtable}[H]{>{\centering\bfseries}m{5cm} >{\centering}m{5cm} >{\centering}m{2.5cm} >{\centering\arraybackslash}m{2.5cm}}  
  \rowcolor{lightgray}
  {\textbf{Obiettivo}} & {\textbf{Metrica}} & {\textbf{Risultato}} & {\textbf{Esito}}  \\
  \endfirsthead%
  \rowcolor{lightgray}
  {\textbf{Obiettivo}} & {\textbf{Metrica}} & {\textbf{Risultato}} & {\textbf{Esito}}  \\
  \endhead%
  \textbf{QoPD003 Test completi sul codice - [Maturità]} & MoPD003 Copertura dei test- [Maturità] & / & Non soddisfatto \\
  \caption{Risultati metrica MoPD003}
  \label{tab:my-table}
\end{longtable}
\textbf{Nota}: Nonostante la realizzazione di alcuni test, non è stato ancora verificato il coverage totale dei test in quanto risulterebbe molto basso al momento.

\rowcolors{2}{white!80!lightgray!90}{white}
\renewcommand{\arraystretch}{2} % allarga le righe con dello spazio sotto e sopra
\begin{longtable}[H]{>{\centering\bfseries}m{5cm} >{\centering}m{5cm} >{\centering}m{2.5cm} >{\centering\arraybackslash}m{2.5cm}}  
  \rowcolor{lightgray}
  {\textbf{Obiettivo}} & {\textbf{Metrica}} & {\textbf{Risultato}} & {\textbf{Esito}}  \\
  \endfirsthead%
  \rowcolor{lightgray}
  {\textbf{Obiettivo}} & {\textbf{Metrica}} & {\textbf{Risultato}} & {\textbf{Esito}}  \\
  \endhead%
  \textbf{QoPD004 Individuazione test falliti - [Affidabilità]} & MoPD004 Densità degli errori - [Affidabilità] & 6\% & Soddisfatto \\
  \caption{Risultati metrica MoPD004}
  \label{tab:my-table}
\end{longtable}
\textbf{Nota}: Al momento il numero dei test ha prodotto errori solo nel 6\% dei casi il che è un valore accettabile dal gruppo.

\paragraph{Usabilità}
\label{sub:usabilita}

\rowcolors{2}{white!80!lightgray!90}{white}
\renewcommand{\arraystretch}{2} % allarga le righe con dello spazio sotto e sopra
\begin{longtable}[H]{>{\centering\bfseries}m{5cm} >{\centering}m{5cm} >{\centering}m{2.5cm} >{\centering\arraybackslash}m{2.5cm}}  
  \rowcolor{lightgray}
  {\textbf{Obiettivo}} & {\textbf{Metrica}} & {\textbf{Risultato}} & {\textbf{Esito}}  \\
  \endfirsthead%
  \rowcolor{lightgray}
  {\textbf{Obiettivo}} & {\textbf{Metrica}} & {\textbf{Risultato}} & {\textbf{Esito}}  \\
  \endhead%
  \textbf{QoPD005 Chiarezza del comportamento - [Comprensibilità]} & MoPD005 Documentazione delle funzioni - [Comprensibilità] & 90\%  & Soddisfatto \\
  \caption{Risultati metrica MoPD005}
  \label{tab:my-table}
\end{longtable}
\textbf{Nota}: Quasi tutte le funzioni implementate sono state descritte con un commento, tranne le funzioni che non richiedono particolare spiegazione

\rowcolors{2}{white!80!lightgray!90}{white}
\renewcommand{\arraystretch}{2} % allarga le righe con dello spazio sotto e sopra
\begin{longtable}[H]{>{\centering\bfseries}m{5cm} >{\centering}m{5cm} >{\centering}m{2.5cm} >{\centering\arraybackslash}m{2.5cm}}  
  \rowcolor{lightgray}
  {\textbf{Obiettivo}} & {\textbf{Metrica}} & {\textbf{Risultato}} & {\textbf{Esito}}  \\
  \endfirsthead%
  \rowcolor{lightgray}
  {\textbf{Obiettivo}} & {\textbf{Metrica}} & {\textbf{Risultato}} & {\textbf{Esito}}  \\
  \endhead%
  \textbf{QoPD006 Chiarimento degli errori - [Comprensibilità]} & MoPD006 Messaggi di errore - [Comprensibilità] & 8\% & Soddisfatto \\
  \caption{Risultati metrica MoPD006}
  \label{tab:my-table}
\end{longtable}
\textbf{Nota}: Il valore della metrica si sta riducendo e risulta essere comunque accettabile.

\paragraph{Efficienza}
\label{sub:efficienza}

\rowcolors{2}{white!80!lightgray!90}{white}
\renewcommand{\arraystretch}{2} % allarga le righe con dello spazio sotto e sopra
\begin{longtable}[H]{>{\centering\bfseries}m{5cm} >{\centering}m{5cm} >{\centering}m{2.5cm} >{\centering\arraybackslash}m{2.5cm}}  
  \rowcolor{lightgray}
  {\textbf{Obiettivo}} & {\textbf{Metrica}} & {\textbf{Risultato}} & {\textbf{Esito}}  \\
  \endfirsthead%
  \rowcolor{lightgray}
  {\textbf{Obiettivo}} & {\textbf{Metrica}} & {\textbf{Risultato}} & {\textbf{Esito}}  \\
  \endhead%
  \textbf{QoPD007 Velocità di esecuzione - [Comportamento temporale]} & MoPD007 Tempo medio di risposta - [Comportamento temporale] & 1.52 sec. & Soddisfatto \\
  \caption{Risultati metrica MoPD007}
  \label{tab:my-table}
\end{longtable}
\textbf{Nota}: Il tempo di risposta soddisfa le aspettative di qualità per la metrica.
\paragraph{Manutenibilità}
\label{sub:manutenibilita}

\rowcolors{2}{white!80!lightgray!90}{white}
\renewcommand{\arraystretch}{2} % allarga le righe con dello spazio sotto e sopra
\begin{longtable}[H]{>{\centering\bfseries}m{5cm} >{\centering}m{5cm} >{\centering}m{2.5cm} >{\centering\arraybackslash}m{2.5cm}}  
  \rowcolor{lightgray}
  {\textbf{Obiettivo}} & {\textbf{Metrica}} & {\textbf{Risultato}} & {\textbf{Esito}}  \\
  \endfirsthead%
  \rowcolor{lightgray}
  {\textbf{Obiettivo}} & {\textbf{Metrica}} & {\textbf{Risultato}} & {\textbf{Esito}}  \\
  \endhead%
  \textbf{QoPD08 Comprensione del codice - [Modifica]} & MoPD008 Commenti sul codice - [Modifica] & 17.2\% & Soddisfatto \\
  \caption{Risultati metrica MoPD006}
  \label{tab:my-table}
\end{longtable}
\textbf{Nota}: Il numero di righe di commento inserite è aumentato di molto rispetto alla fase precedente.

\subsubsection{Conclusioni}%
\label{sub:conclusioni}
L'andamento del lavoro all'interno del gruppo continua ad essere molto buono, è stato rilevato un piacevole miglioramento in tutte le metriche specialmente quelle
riguardanti la qualità di prodotto. Le metriche sulla copertura dei test e sui test realizzati continuano a non essere soddisfacenti anche al causa del poco tempo rimasto a
disposizione del gruppo per concentrarsi su quelle metriche, accerta però che saranno raggiunti i valori soddisfacenti di quelle metriche nella prossima fase del progetto.
