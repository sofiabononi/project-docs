
\subsubsection{Esiti verifiche sui documenti}
\label{sub:esiti_verifiche_sui_documenti}
Sui documenti sono state utilizzate le metriche per:
    \begin{itemize}
        \item \textbf{Indice di Gulpease};
        \item \textbf{Correttezza lessicale/ortografica}.
    \end{itemize}
Per evitare risultati errati nel calcole dell'Indice di Gulpease si è deciso di non tenere in considerazione:
    \begin{itemize}
        \item Il frontespizio di ogni documento;
        \item Le tabelle presenti nei documenti;
        \item Il diario delle modifiche all'interno di ogni documento.
    \end{itemize}
Per quanto riguarda la correttezza lessicale/ortografica, gli errori trovati non verranno classificati ma semplicemente conteggiati uniformemente includendo anche errori derivati dal mancato rispetto delle convenzioni scritte nelle \textsc{Norme di Progetto v2.0.0}.
\\ Per ogni metrica di ogni documento verrà inoltre inserito l'esito dell'analisi che può avere risultato soddisfacente o non.

\paragraph{Norme di Progetto}
\label{sub:norme_di_progetto}
Le \textsc{Norme di Progetto v1.4-2.2.2} hanno subito parecchie modifiche e miglioramenti in base a diverse problematiche trattate anche nella sezione appendice B, il che ha portato ad un alternarsi di errori nelle diverse fasi di verifica e approvazione.
Il suo indice di Gulpease finale si è assestato intorno a 70.

\rowcolors{2}{white!80!lightgray!90}{white}
\renewcommand{\arraystretch}{2} % allarga le righe con dello spazio sotto e sopra
\begin{longtable}[H]{>{\centering\bfseries}m{6cm} >{\centering}m{2cm} >{\centering}m{2.5cm} >{\centering}m{2.5cm} >{\centering\arraybackslash}m{2.5cm}}  
  \rowcolor{lightgray}
  {\textbf{Documento}} & {\textbf{Risultato indice}} & {\textbf{Errori presenti}} & {\textbf{Esito indice}} & {\textbf{Esito errori}}  \\
  \endfirsthead%
  \rowcolor{lightgray}
  {\textbf{Documento}} & {\textbf{Risultato indice}} & {\textbf{Errori presenti}} & {\textbf{Esito indice}} & {\textbf{Esito errori}}  \\
  \endhead%
  \textbf{Norme di Progetto v1.4-2.2.2} & 70                 & 0               & Soddisfatto & Soddisfatto \\
  \caption{Risultati metriche per le Norme di Progetto v1.4-2.2.2}
  \label{tab:my-table}
\end{longtable}

\begin{figure}[H]
  \centering
  \includegraphics[width=10cm]{img/erroriNdPv1.4-2.2.2.png}
  \label{fig:errori_ndp}
  \caption{Grafico errori per \textsc{Norme di Progetto v1.4-2.2.2}}
\end{figure}

\begin{figure}[H]
  \centering
  \includegraphics[width=10cm]{img/gulpeaseNdPv1.4-2.2.2.png}
  \label{fig:gulpease_ndp}
  \caption{Grafico indice di Gulpease per \textsc{Norme di Progetto v1.4-2.2.2}}
\end{figure}


\paragraph{Piano di Qualifica}
\label{sub:piano_di_qualifica}
Il \textsc{Piano di Qualifica v1.4-2.0.0} è stato verificato e approvato parecchie volte, nelle quali non sono stati individuati grandi quantità di errori nonostante le molte modifiche apportate ad esso.
Il suo indice di Gulpease è mutato leggermente nel tempo settandosi su un valore di 72 nell'ultima approvazione effettuata.

\rowcolors{2}{white!80!lightgray!90}{white}
\renewcommand{\arraystretch}{2} % allarga le righe con dello spazio sotto e sopra
\begin{longtable}[H]{>{\centering\bfseries}m{6cm} >{\centering}m{2cm} >{\centering}m{2.5cm} >{\centering}m{2.5cm} >{\centering\arraybackslash}m{2.5cm}}  
  \rowcolor{lightgray}
  {\textbf{Documento}} & {\textbf{Risultato indice}} & {\textbf{Errori presenti}} & {\textbf{Esito indice}} & {\textbf{Esito errori}}  \\
  \endfirsthead%
  \rowcolor{lightgray}
  {\textbf{Documento}} & {\textbf{Risultato indice}} & {\textbf{Errori presenti}} & {\textbf{Esito indice}} & {\textbf{Esito errori}}  \\
  \endhead%
  \textbf{Piano di Qualifica v1.4-2.0.0} & 72                & 0               & Soddisfatto & Soddisfatto \\
  \caption{Risultati metriche per il Piano di Qualifica v1.4-2.0.0}
  \label{tab:my-table}
\end{longtable}

\begin{figure}[H]
  \centering
  \includegraphics[width=10cm]{img/erroriPdQV1.4-2.0.0.png}
  \label{fig:errori_pdq}
  \caption{Grafico errori per \textsc{Piano di Qualifica v1.4-2.0.0}}
\end{figure}

\begin{figure}[H]
  \centering
  \includegraphics[width=10cm]{img/GulpeasePdQV1.4-2.0.0.png}
  \label{fig:gulpease_pdq}
  \caption{Grafico indice di Gulpease per \textsc{Piano di Qualifica v1.4-2.0.0}}
\end{figure}


\paragraph{Piano di Progetto}
\label{sub:piano_di_progetto}
Il \textsc{Piano di Progetto v1.4-3.1.0} ha subito molte modifiche durante questa fase di sviluppo, il che ha portato ad attuare molte fasi di verifica durante questo periodo di sviluppo del progetto.
Il suo indice di Gulpease è mutato leggermente nel tempo settandosi su un valore di 74 già nella v1.3-3.0.0.

\rowcolors{2}{white!80!lightgray!90}{white}
\renewcommand{\arraystretch}{2} % allarga le righe con dello spazio sotto e sopra
\begin{longtable}[H]{>{\centering\bfseries}m{6cm} >{\centering}m{2cm} >{\centering}m{2.5cm} >{\centering}m{2.5cm} >{\centering\arraybackslash}m{2.5cm}}  
  \rowcolor{lightgray}
  {\textbf{Documento}} & {\textbf{Risultato indice}} & {\textbf{Errori presenti}} & {\textbf{Esito indice}} & {\textbf{Esito errori}}  \\
  \endfirsthead%
  \rowcolor{lightgray}
  {\textbf{Documento}} & {\textbf{Risultato indice}} & {\textbf{Errori presenti}} & {\textbf{Esito indice}} & {\textbf{Esito errori}}  \\
  \endhead%
  \textbf{Piano di Progetto v1.4-3.1.0} & 74                 & 0               & Soddisfatto & Soddisfatto \\
  \caption{Risultati metriche per il Piano di Progetto v1.4-3.1.0}
  \label{tab:my-table}
\end{longtable}

\begin{figure}[H]
  \centering
  \includegraphics[width=10cm]{img/erroriPdPv1.4-3.1.0.png}
  \label{fig:errori_pdq}
  \caption{Grafico errori per \textsc{Piano di Progetto v1.4-3.1.0}}
\end{figure}

\begin{figure}[H]
  \centering
  \includegraphics[width=10cm]{img/gulpeasePdPv1.4-3.1.0.png}
  \label{fig:gulpease_pdq}
  \caption{Grafico indice di Gulpease per \textsc{Piano di Progetto v1.4-3.1.0}}
\end{figure}


\paragraph{Analisi dei Requisiti}
\label{sub:analisi_dei_requisiti}
Nell' \textsc{Analisi dei Requisiti v1.4-1.2.1} le modifiche apportate fanno riferimento solamente ai casi d'uso al suo interno, il numero di errori rilevati è quindi contenuto e anche il numero di verifiche e approvazioni del documento è più basso rispetto agli altri documenti.
Il suo indice di Gulpease finale è rimasto a 73.

\rowcolors{2}{white!80!lightgray!90}{white}
\renewcommand{\arraystretch}{2} % allarga le righe con dello spazio sotto e sopra
\begin{longtable}[H]{>{\centering\bfseries}m{6cm} >{\centering}m{2cm} >{\centering}m{2.5cm} >{\centering}m{2.5cm} >{\centering\arraybackslash}m{2.5cm}}  
  \rowcolor{lightgray}
  {\textbf{Documento}} & {\textbf{Risultato indice}} & {\textbf{Errori presenti}} & {\textbf{Esito indice}} & {\textbf{Esito errori}}  \\
  \endfirsthead%
  \rowcolor{lightgray}
  {\textbf{Documento}} & {\textbf{Risultato indice}} & {\textbf{Errori presenti}} & {\textbf{Esito indice}} & {\textbf{Esito errori}}  \\
  \endhead%
  \textbf{Analisi dei Requisiti v1.4-1.2.1} & 73                 & 0               & Soddisfatto & Soddisfatto \\
  \caption{Risultati metriche per l'Analisi dei Requisiti v1.4-1.2.1}
  \label{tab:my-table}
\end{longtable}

\begin{figure}[H]
  \centering
  \includegraphics[width=10cm]{img/erroriAdRv1.4-1.2.1.png}
  \label{fig:errori_adr}
  \caption{Grafico errori per \textsc{Analisi dei Requisiti v1.4-1.2.1}}
\end{figure}

\begin{figure}[H]
  \centering
  \includegraphics[width=10cm]{img/gulpeaseAdrv1.4-1.2.1.png}
  \label{fig:gulpease_adr}
  \caption{Grafico indice di Gulpease per \textsc{Analisi dei Requisiti v1.4-1.2.1}}
\end{figure}


\paragraph{Glossario}
\label{sub:glossario}
Il \textsc{Glossario v1.4-1.2.0} essendo un documento in cui sono state apportate poche modifiche significative, ha richiesto poche approvazioni e verifiche, dati i pochi errori riscontrati.
Il suo indice di Gulpease è rimasto immutato in ogni fase di approvazione rimanendo a 65.

\rowcolors{2}{white!80!lightgray!90}{white}
\renewcommand{\arraystretch}{2} % allarga le righe con dello spazio sotto e sopra
\begin{longtable}[H]{>{\centering\bfseries}m{6cm} >{\centering}m{2cm} >{\centering}m{2.5cm} >{\centering}m{2.5cm} >{\centering\arraybackslash}m{2.5cm}}  
  \rowcolor{lightgray}
  {\textbf{Documento}} & {\textbf{Risultato indice}} & {\textbf{Errori presenti}} & {\textbf{Esito indice}} & {\textbf{Esito errori}}  \\
  \endfirsthead%
  \rowcolor{lightgray}
  {\textbf{Documento}} & {\textbf{Risultato indice}} & {\textbf{Errori presenti}} & {\textbf{Esito indice}} & {\textbf{Esito errori}}  \\
  \endhead%
  \textbf{Glossario v1.4-1.2.0} & 65                 & 0               & Soddisfatto & Soddisfatto \\
  \caption{Risultati metriche per il Glossario v1.4-1.2.0}
  \label{tab:my-table}
\end{longtable}

\begin{figure}[H]
  \centering
  \includegraphics[width=10cm]{img/erroriGlosv1.4-1.2.0.png}
  \label{fig:errori_glos}
  \caption{Grafico errori per \textsc{Glossario v1.4-1.2.0}}
\end{figure}

\begin{figure}[H]
  \centering
  \includegraphics[width=10cm]{img/gulpeaseGlosv1.4-1.2.0.png}
  \label{fig:gulpease_glos}
  \caption{Grafico indice di Gulpease per \textsc{Glossario v1.4-1.2.0}}
\end{figure}

\subsubsection{Esiti verifiche sui processi}
\label{sub:esiti_verifiche_sui_processi}
In questa sezione vengono visualizzati gli esiti delle metriche prese in considerazione per quanto riguarda i processi produttivi.
Come per i documenti, anche per queste metriche verrà fornito un esito che può essere soddisfacente o meno.

\paragraph{Processo PRC001}
\label{sub:processo_PRC001}

\rowcolors{2}{white!80!lightgray!90}{white}
\renewcommand{\arraystretch}{2} % allarga le righe con dello spazio sotto e sopra
\begin{longtable}[H]{>{\centering\bfseries}m{5cm} >{\centering}m{5cm} >{\centering}m{2.5cm} >{\centering\arraybackslash}m{2.5cm}}  
  \rowcolor{lightgray}
  {\textbf{Obiettivo}} & {\textbf{Metrica}} & {\textbf{Risultato}} & {\textbf{Esito}}  \\
  \endfirsthead%
  \rowcolor{lightgray}
  {\textbf{Obiettivo}} & {\textbf{Metrica}} & {\textbf{Risultato}} & {\textbf{Esito}}  \\
  \endhead%
  \textbf{QoPR001 Rispetto delle scadenze della pianificazione} & MoPR001 Varianza dei tempi & 1.10  & Soddisfatto  \\
  \caption{Risultati metrica MoPR001}
  \label{tab:my-table}
\end{longtable}
\textbf{Nota}: Il valore della varianza dei tempi è migliorato rispetto alla prima fase e rientra ancora nel valore accettabile definito dal gruppo.

\rowcolors{2}{white!80!lightgray!90}{white}
\renewcommand{\arraystretch}{2} % allarga le righe con dello spazio sotto e sopra
\begin{longtable}[H]{>{\centering\bfseries}m{5cm} >{\centering}m{5cm} >{\centering}m{2.5cm} >{\centering\arraybackslash}m{2.5cm}}  
  \rowcolor{lightgray}
  {\textbf{Obiettivo}} & {\textbf{Metrica}} & {\textbf{Risultato}} & {\textbf{Esito}}  \\
  \endfirsthead%
  \rowcolor{lightgray}
  {\textbf{Obiettivo}} & {\textbf{Metrica}} & {\textbf{Risultato}} & {\textbf{Esito}}  \\
  \endhead%
  \textbf{QoPR002 Rispetto del budget istanziato} & MoPR002 Varianza dei costi & -2 & Soddisfatto  \\
  \caption{Risultati metrica MoPR002}
  \label{tab:my-table}
\end{longtable}
\textbf{Nota}: La variazione rispetto al preventivo iniziale rientra nel range deciso per la metrica, per cui l’obiettivo è stato soddisfatto. Il discostamento dal preventivo totale rilevato è di: -2 euro;

\rowcolors{2}{white!80!lightgray!90}{white}
\renewcommand{\arraystretch}{2} % allarga le righe con dello spazio sotto e sopra
\begin{longtable}[H]{>{\centering\bfseries}m{5cm} >{\centering}m{5cm} >{\centering}m{2.5cm} >{\centering\arraybackslash}m{2.5cm}}  
  \rowcolor{lightgray}
  {\textbf{Obiettivo}} & {\textbf{Metrica}} & {\textbf{Risultato}} & {\textbf{Esito}}  \\
  \endfirsthead%
  \rowcolor{lightgray}
  {\textbf{Obiettivo}} & {\textbf{Metrica}} & {\textbf{Risultato}} & {\textbf{Esito}}  \\
  \endhead%
  \textbf{QoPR003 Rispetto del ciclo di vita scelto} & MoPR003 Aderenza agli standard & Livello di maturità:  2 \\ Valutazione attributi:  L &  Non soddisfatto \\
  \caption{Risultati metrica MoPR003}
  \label{tab:my-table}
\end{longtable}
\textbf{Nota}: Si è intravisto un miglioramento all'interno del gruppo per quanto riguarda l'aderenza agli standard, ma non ancora sufficiente per gli obiettivi prefissati.

\rowcolors{2}{white!80!lightgray!90}{white}
\renewcommand{\arraystretch}{2} % allarga le righe con dello spazio sotto e sopra
\begin{longtable}[H]{>{\centering\bfseries}m{5cm} >{\centering}m{5cm} >{\centering}m{2.5cm} >{\centering\arraybackslash}m{2.5cm}}  
  \rowcolor{lightgray}
  {\textbf{Obiettivo}} & {\textbf{Metrica}} & {\textbf{Risultato}} & {\textbf{Esito}}  \\
  \endfirsthead%
  \rowcolor{lightgray}
  {\textbf{Obiettivo}} & {\textbf{Metrica}} & {\textbf{Risultato}} & {\textbf{Esito}}  \\
  \endhead%
  \textbf{QoPR004 Rispetto dei ruoli e identificazione nei prodotti} & MoPR004 Aderenza ai ruoli & 0 & Soddisfatto \\
  \caption{Risultati metrica MoPR004}
  \label{tab:my-table}
\end{longtable}
\textbf{Nota}: In ogni documento prodotto i ruoli sono stati identificati e rispettati.

\rowcolors{2}{white!80!lightgray!90}{white}
\renewcommand{\arraystretch}{2} % allarga le righe con dello spazio sotto e sopra
\begin{longtable}[H]{>{\centering\bfseries}m{5cm} >{\centering}m{5cm} >{\centering}m{2.5cm} >{\centering\arraybackslash}m{2.5cm}}  
  \rowcolor{lightgray}
  {\textbf{Obiettivo}} & {\textbf{Metrica}} & {\textbf{Risultato}} & {\textbf{Esito}}  \\
  \endfirsthead%
  \rowcolor{lightgray}
  {\textbf{Obiettivo}} & {\textbf{Metrica}} & {\textbf{Risultato}} & {\textbf{Esito}}  \\
  \endhead%
  \textbf{QoPR005 Rispetto del versionamento dei prodotti} & MoPR005 Controllo prodotti & 29.8 & Soddisfatto \\
  \caption{Risultati metrica MoPR005}
  \label{tab:my-table}
\end{longtable}
\textbf{Nota}: Il numero di commit in questa fase è stato parecchio elevato, il che soddisfa appieno l'obiettivo prefissato per questa norma.

\paragraph{Processo PRC002}
\label{sub:processo_PRC002}

\rowcolors{2}{white!80!lightgray!90}{white}
\renewcommand{\arraystretch}{2} % allarga le righe con dello spazio sotto e sopra
\begin{longtable}[H]{>{\centering\bfseries}m{5cm} >{\centering}m{5cm} >{\centering}m{2.5cm} >{\centering\arraybackslash}m{2.5cm}}  
  \rowcolor{lightgray}
  {\textbf{Obiettivo}} & {\textbf{Metrica}} & {\textbf{Risultato}} & {\textbf{Esito}}  \\
  \endfirsthead%
  \rowcolor{lightgray}
  {\textbf{Obiettivo}} & {\textbf{Metrica}} & {\textbf{Risultato}} & {\textbf{Esito}}  \\
  \endhead%
  \textbf{QoPR006 Soddisfazione dei requisiti obbligatori} & MoPR006 Verifica requisiti obbligatori & / & Non soddisfatto \\
  \caption{Risultati metrica MoPR006}
  \label{tab:my-table}
\end{longtable}
\textbf{Nota}: La revisione dei requisiti obbligatori soddisfatti non è ancora stata eseguita.

\rowcolors{2}{white!80!lightgray!90}{white}
\renewcommand{\arraystretch}{2} % allarga le righe con dello spazio sotto e sopra
\begin{longtable}[H]{>{\centering\bfseries}m{5cm} >{\centering}m{5cm} >{\centering}m{2.5cm} >{\centering\arraybackslash}m{2.5cm}}  
  \rowcolor{lightgray}
  {\textbf{Obiettivo}} & {\textbf{Metrica}} & {\textbf{Risultato}} & {\textbf{Esito}}  \\
  \endfirsthead%
  \rowcolor{lightgray}
  {\textbf{Obiettivo}} & {\textbf{Metrica}} & {\textbf{Risultato}} & {\textbf{Esito}}  \\
  \endhead%
  \textbf{QoPR007 Soddisfazione dei requisiti opzionali e desiderabili} & MoPR007 Verifica requisiti opzionali \\ MoPR008 Verifica requisiti desiderabili & / & Non soddisfatto \\
  \caption{Risultati metrica MoPR007 e metrica MoPR008}
  \label{tab:my-table}
\end{longtable}
\textbf{Nota}: La revisione dei requisiti opzionali e desiderabili soddisfatti non è ancora stata eseguita.

\rowcolors{2}{white!80!lightgray!90}{white}
\renewcommand{\arraystretch}{2} % allarga le righe con dello spazio sotto e sopra
\begin{longtable}[H]{>{\centering\bfseries}m{5cm} >{\centering}m{5cm} >{\centering}m{2.5cm} >{\centering\arraybackslash}m{2.5cm}}  
  \rowcolor{lightgray}
  {\textbf{Obiettivo}} & {\textbf{Metrica}} & {\textbf{Risultato}} & {\textbf{Esito}}  \\
  \endfirsthead%
  \rowcolor{lightgray}
  {\textbf{Obiettivo}} & {\textbf{Metrica}} & {\textbf{Risultato}} & {\textbf{Esito}}  \\
  \endhead%
  \textbf{QoPR008 Verifica dei rischi previsti} & MoPR009 Verifica rischi non pervenuti & 0 & Soddisfatto \\
  \caption{Risultati metrica MoPR009}
  \label{tab:my-table}
\end{longtable}
\textbf{Nota}: Non sono ancora stati rivelati rischi importanti che non siano stati previsti precedentemente dal gruppo.

\paragraph{Processo PRC003}
\label{sub:processo_PRC003}

\rowcolors{2}{white!80!lightgray!90}{white}
\renewcommand{\arraystretch}{2} % allarga le righe con dello spazio sotto e sopra
\begin{longtable}[H]{>{\centering\bfseries}m{5cm} >{\centering}m{5cm} >{\centering}m{2.5cm} >{\centering\arraybackslash}m{2.5cm}}  
  \rowcolor{lightgray}
  {\textbf{Obiettivo}} & {\textbf{Metrica}} & {\textbf{Risultato}} & {\textbf{Esito}}  \\
  \endfirsthead%
  \rowcolor{lightgray}
  {\textbf{Obiettivo}} & {\textbf{Metrica}} & {\textbf{Risultato}} & {\textbf{Esito}}  \\
  \endhead%
  \textbf{QoPR09 Rispetto delle fasi del ciclo di vita} & MoPR010 Analisi Way of Working & / & Soddisfatto \\
  \caption{Risultati metrica MoPR010}
  \label{tab:my-table}
\end{longtable}
\textbf{Nota}: I componenti del gruppo svolgono costantemente un aggiornamento in base alle modifiche fatte alle \textsc{Norme di Progetto v1.4-2.2.2}.

\rowcolors{2}{white!80!lightgray!90}{white}
\renewcommand{\arraystretch}{2} % allarga le righe con dello spazio sotto e sopra
\begin{longtable}[H]{>{\centering\bfseries}m{5cm} >{\centering}m{5cm} >{\centering}m{2.5cm} >{\centering\arraybackslash}m{2.5cm}}  
  \rowcolor{lightgray}
  {\textbf{Obiettivo}} & {\textbf{Metrica}} & {\textbf{Risultato}} & {\textbf{Esito}}  \\
  \endfirsthead%
  \rowcolor{lightgray}
  {\textbf{Obiettivo}} & {\textbf{Metrica}} & {\textbf{Risultato}} & {\textbf{Esito}}  \\
  \endhead%
  \textbf{QoPR010 Rispetto nella redazione dei documenti} & MoPR011 Analisi documenti & 4+ & Soddisfatto \\
  \caption{Risultati metrica MoPR011}
  \label{tab:my-table}
\end{longtable}
\textbf{Nota}: Tutti i documenti più corposi sono stati verificati almeno 4 volte, il che soddisfa pienamente le aspettative del gruppo.

\paragraph{Processo PRC004}
\label{sub:processo_PRC004}

\rowcolors{2}{white!80!lightgray!90}{white}
\renewcommand{\arraystretch}{2} % allarga le righe con dello spazio sotto e sopra
\begin{longtable}[H]{>{\centering\bfseries}m{5cm} >{\centering}m{5cm} >{\centering}m{2.5cm} >{\centering\arraybackslash}m{2.5cm}}  
  \rowcolor{lightgray}
  {\textbf{Obiettivo}} & {\textbf{Metrica}} & {\textbf{Risultato}} & {\textbf{Esito}}  \\
  \endfirsthead%
  \rowcolor{lightgray}
  {\textbf{Obiettivo}} & {\textbf{Metrica}} & {\textbf{Risultato}} & {\textbf{Esito}}  \\
  \endhead%
  \textbf{QoPR011 Attuare una verifica costante} & MoPR012 Frequenza di controlli & / & Soddisfatto \\
  \caption{Risultati metrica MoPR012}
  \label{tab:my-table}
\end{longtable}
\textbf{Nota}: I verificatori stanno continuando ad effettuare delle verifiche costanti ai prodotti.


\rowcolors{2}{white!80!lightgray!90}{white}
\renewcommand{\arraystretch}{2} % allarga le righe con dello spazio sotto e sopra
\begin{longtable}[H]{>{\centering\bfseries}m{5cm} >{\centering}m{5cm} >{\centering}m{2.5cm} >{\centering\arraybackslash}m{2.5cm}}  
  \rowcolor{lightgray}
  {\textbf{Obiettivo}} & {\textbf{Metrica}} & {\textbf{Risultato}} & {\textbf{Esito}}  \\
  \endfirsthead%
  \rowcolor{lightgray}
  {\textbf{Obiettivo}} & {\textbf{Metrica}} & {\textbf{Risultato}} & {\textbf{Esito}}  \\
  \endhead%
  \textbf{QoPR013 Rispettare le fasi di verifica} & MoPR010 Analisi Way of Working & / & Soddisfatto \\
  \caption{Risultati QoPR13}
  \label{tab:my-table}
\end{longtable}
\textbf{Nota}: Le fasi di verifica dei prodotti stanno vedendo rispettate secondo le indicazioni interne al gruppo.

\rowcolors{2}{white!80!lightgray!90}{white}
\renewcommand{\arraystretch}{2} % allarga le righe con dello spazio sotto e sopra
\begin{longtable}[H]{>{\centering\bfseries}m{5cm} >{\centering}m{5cm} >{\centering}m{2.5cm} >{\centering\arraybackslash}m{2.5cm}}  
  \rowcolor{lightgray}
  {\textbf{Obiettivo}} & {\textbf{Metrica}} & {\textbf{Risultato}} & {\textbf{Esito}}  \\
  \endfirsthead%
  \rowcolor{lightgray}
  {\textbf{Obiettivo}} & {\textbf{Metrica}} & {\textbf{Risultato}} & {\textbf{Esito}}  \\
  \endhead%
  \textbf{QoPR014 Soddisfare i test richiesti} & MoPR013 Percentuale di test soddisfatti & 0\% & Non soddisfatto \\
  \caption{Risultati MoPR013}
  \label{tab:my-table}
\end{longtable}
\textbf{Nota}: L'obiettivo per questa metrica non è ancora stato soddisfatto in quanto l'implementazione dei test avverrà nel prossimo periodo di sviluppo.

\subsubsection{Esiti verifiche sui prodotti}
\label{sub:esiti_verifiche_sui_prodotti}
In questa sezione vengono visualizzati gli esiti delle metriche prese in considerazione per quanto riguarda i prodotti realizzati. Come per i documenti, anche per queste metriche verrà fornito un esito che può essere soddisfacente o meno.

\paragraph{Funzionalità}
\label{sub:funzionalita}

\rowcolors{2}{white!80!lightgray!90}{white}
\renewcommand{\arraystretch}{2} % allarga le righe con dello spazio sotto e sopra
\begin{longtable}[H]{>{\centering\bfseries}m{5cm} >{\centering}m{5cm} >{\centering}m{2.5cm} >{\centering\arraybackslash}m{2.5cm}}  
  \rowcolor{lightgray}
  {\textbf{Obiettivo}} & {\textbf{Metrica}} & {\textbf{Risultato}} & {\textbf{Esito}}  \\
  \endfirsthead%
  \rowcolor{lightgray}
  {\textbf{Obiettivo}} & {\textbf{Metrica}} & {\textbf{Risultato}} & {\textbf{Esito}}  \\
  \endhead%
  \textbf{QoPD001 Rispetto dell’implementazione funzionale - [Adeguatezza]} & MoPD001 Completezza di implementazione - [Adeguatezza] & / & Non Soddisfatto \\
  \caption{Risultati metrica MoPD001}
  \label{tab:my-table}
\end{longtable}
\textbf{Nota}: Questa metrica risulta essere 0 in quanto in questa fase di sviluppo non sono ancora state definite chiaramente le funzioni totali da realizzare.

\rowcolors{2}{white!80!lightgray!90}{white}
\renewcommand{\arraystretch}{2} % allarga le righe con dello spazio sotto e sopra
\begin{longtable}[H]{>{\centering\bfseries}m{5cm} >{\centering}m{5cm} >{\centering}m{2.5cm} >{\centering\arraybackslash}m{2.5cm}}  
  \rowcolor{lightgray}
  {\textbf{Obiettivo}} & {\textbf{Metrica}} & {\textbf{Risultato}} & {\textbf{Esito}}  \\
  \endfirsthead%
  \rowcolor{lightgray}
  {\textbf{Obiettivo}} & {\textbf{Metrica}} & {\textbf{Risultato}} & {\textbf{Esito}}  \\
  \endhead%
  \textbf{QoPD002 Rispetto delle interfacce - [Interoperabilità]} & MoPD002 Coerenza di interfaccia - [Interoperabilità] & / & Non soddisfatto \\
  \caption{Risultati metrica MoPD002}
  \label{tab:my-table}
\end{longtable}
\textbf{Nota}: Il totale delle interfacce delle funzioni non è ancora stato definito.

\paragraph{Affidabilità}
\label{sub:affidabilita}

\rowcolors{2}{white!80!lightgray!90}{white}
\renewcommand{\arraystretch}{2} % allarga le righe con dello spazio sotto e sopra
\begin{longtable}[H]{>{\centering\bfseries}m{5cm} >{\centering}m{5cm} >{\centering}m{2.5cm} >{\centering\arraybackslash}m{2.5cm}}  
  \rowcolor{lightgray}
  {\textbf{Obiettivo}} & {\textbf{Metrica}} & {\textbf{Risultato}} & {\textbf{Esito}}  \\
  \endfirsthead%
  \rowcolor{lightgray}
  {\textbf{Obiettivo}} & {\textbf{Metrica}} & {\textbf{Risultato}} & {\textbf{Esito}}  \\
  \endhead%
  \textbf{QoPD003 Test completi sul codice - [Maturità]} & MoPD003 Copertura dei test- [Maturità] & / & Non soddisfatto \\
  \caption{Risultati metrica MoPD003}
  \label{tab:my-table}
\end{longtable}
\textbf{Nota}: Per il Proof of concept non è stato implementato un sistema di copertura dei test automatico.

\rowcolors{2}{white!80!lightgray!90}{white}
\renewcommand{\arraystretch}{2} % allarga le righe con dello spazio sotto e sopra
\begin{longtable}[H]{>{\centering\bfseries}m{5cm} >{\centering}m{5cm} >{\centering}m{2.5cm} >{\centering\arraybackslash}m{2.5cm}}  
  \rowcolor{lightgray}
  {\textbf{Obiettivo}} & {\textbf{Metrica}} & {\textbf{Risultato}} & {\textbf{Esito}}  \\
  \endfirsthead%
  \rowcolor{lightgray}
  {\textbf{Obiettivo}} & {\textbf{Metrica}} & {\textbf{Risultato}} & {\textbf{Esito}}  \\
  \endhead%
  \textbf{QoPD004 Individuazione test falliti - [Affidabilità]} & MoPD004 Densità degli errori - [Affidabilità] & / & Non Soddisfatto \\
  \caption{Risultati metrica MoPD004}
  \label{tab:my-table}
\end{longtable}
\textbf{Nota}: Visto il non implemento di un sistema di copertura dei test automatico per il Proof of concept, non è stato possibile verificare il numero di test falliti.

\paragraph{Usabilità}
\label{sub:usabilita}

\rowcolors{2}{white!80!lightgray!90}{white}
\renewcommand{\arraystretch}{2} % allarga le righe con dello spazio sotto e sopra
\begin{longtable}[H]{>{\centering\bfseries}m{5cm} >{\centering}m{5cm} >{\centering}m{2.5cm} >{\centering\arraybackslash}m{2.5cm}}  
  \rowcolor{lightgray}
  {\textbf{Obiettivo}} & {\textbf{Metrica}} & {\textbf{Risultato}} & {\textbf{Esito}}  \\
  \endfirsthead%
  \rowcolor{lightgray}
  {\textbf{Obiettivo}} & {\textbf{Metrica}} & {\textbf{Risultato}} & {\textbf{Esito}}  \\
  \endhead%
  \textbf{QoPD005 Chiarezza del comportamento - [Comprensibilità]} & MoPD005 Documentazione delle funzioni - [Comprensibilità] & 30\%  & Non soddisfatto \\
  \caption{Risultati metrica MoPD005}
  \label{tab:my-table}
\end{longtable}
\textbf{Nota}: La percentuale di funzioni documentate con un commento è inferiore rispetto a quanto prefissate, il gruppo si impegnerà però a risolvere questa mancanza nella prossima fase di sviluppo.

\rowcolors{2}{white!80!lightgray!90}{white}
\renewcommand{\arraystretch}{2} % allarga le righe con dello spazio sotto e sopra
\begin{longtable}[H]{>{\centering\bfseries}m{5cm} >{\centering}m{5cm} >{\centering}m{2.5cm} >{\centering\arraybackslash}m{2.5cm}}  
  \rowcolor{lightgray}
  {\textbf{Obiettivo}} & {\textbf{Metrica}} & {\textbf{Risultato}} & {\textbf{Esito}}  \\
  \endfirsthead%
  \rowcolor{lightgray}
  {\textbf{Obiettivo}} & {\textbf{Metrica}} & {\textbf{Risultato}} & {\textbf{Esito}}  \\
  \endhead%
  \textbf{QoPD006 Chiarimento degli errori - [Comprensibilità]} & MoPD006 Messaggi di errore - [Comprensibilità] & 10\% & Soddisfatto \\
  \caption{Risultati metrica MoPD006}
  \label{tab:my-table}
\end{longtable}
\textbf{Nota}: Il valore della metrica è alto ma soddisfacente, anche perché il numero di messaggi di errore da gestire è ancora ristretto.

\paragraph{Efficienza}
\label{sub:efficienza}

\rowcolors{2}{white!80!lightgray!90}{white}
\renewcommand{\arraystretch}{2} % allarga le righe con dello spazio sotto e sopra
\begin{longtable}[H]{>{\centering\bfseries}m{5cm} >{\centering}m{5cm} >{\centering}m{2.5cm} >{\centering\arraybackslash}m{2.5cm}}  
  \rowcolor{lightgray}
  {\textbf{Obiettivo}} & {\textbf{Metrica}} & {\textbf{Risultato}} & {\textbf{Esito}}  \\
  \endfirsthead%
  \rowcolor{lightgray}
  {\textbf{Obiettivo}} & {\textbf{Metrica}} & {\textbf{Risultato}} & {\textbf{Esito}}  \\
  \endhead%
  \textbf{QoPD007 Velocità di esecuzione - [Comportamento temporale]} & MoPD007 Tempo medio di risposta - [Comportamento temporale] & minore di 1 sec. & Soddisfatto \\
  \caption{Risultati metrica MoPD007}
  \label{tab:my-table}
\end{longtable}
\textbf{Nota}: Il tempo di risposta soddisfa le aspettative di qualità anche se le funzionalità totali del programma realizzato per il Proof of concept rimangono ancora in numero ristretto.

\paragraph{Manutenibilità}
\label{sub:manutenibilita}

\rowcolors{2}{white!80!lightgray!90}{white}
\renewcommand{\arraystretch}{2} % allarga le righe con dello spazio sotto e sopra
\begin{longtable}[H]{>{\centering\bfseries}m{5cm} >{\centering}m{5cm} >{\centering}m{2.5cm} >{\centering\arraybackslash}m{2.5cm}}  
  \rowcolor{lightgray}
  {\textbf{Obiettivo}} & {\textbf{Metrica}} & {\textbf{Risultato}} & {\textbf{Esito}}  \\
  \endfirsthead%
  \rowcolor{lightgray}
  {\textbf{Obiettivo}} & {\textbf{Metrica}} & {\textbf{Risultato}} & {\textbf{Esito}}  \\
  \endhead%
  \textbf{QoPD08 Comprensione del codice - [Modifica]} & MoPD008 Commenti sul codice - [Modifica] & 5.36\% & Non soddisfatto \\
  \caption{Risultati metrica MoPD006}
  \label{tab:my-table}
\end{longtable}
\textbf{Nota}: Il risultato della metrica non soddisfa le aspettative del gruppo che si prefissa come obiettivo principale quello di aumentare il numero di commenti significativi all'interno del codice.

\subsubsection{Conclusioni}%
\label{sub:conclusioni}
Il gruppo si ritiene soddisfatto dell'andamento dello sviluppo soprattutto analizzando i dati degli esiti delle metriche più significative sui processi, nota però una grande opportunità di miglioramento specialmente nelle metriche riguardanti i prodotti, soprattutto di natura software.
Da quest'ultima riflessione il gruppo si impegna a mantenere un buon andamento dello sviluppo, cercando di migliorare significativamente le metriche riguardanti gli obiettivi dei prodotti.