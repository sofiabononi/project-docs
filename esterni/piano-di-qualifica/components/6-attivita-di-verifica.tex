\documentclass[../piano-di-qualifica.tex]{subfiles}

\begin{document}
In questa sezione vengono resi noti i risultati dei test relativi al periodo di revisione dei documenti e requisiti, mediante l'utilizzo delle metriche descritte nelle sezioni precedenti.
I risultati possono coincidere o meno con i valori desiderati dal gruppo, nel caso non coincidessero verrà fatta un ulteriore analisi per il miglioramento visibile nel punto §7.

\subsection{Periodo di Analisi}
\label{sub:periodo_di_analisi}
Durante questo periodo è stata sottoposta ad una precisa attività di verifica tutta la documentazione da presentare in ingresso in sede di Revisione dei Requisiti.
I verificatori inizialmente hanno effettuato l'analisi utilizzando la tecnica di \glossario{Walkthrough} che ha portato ad una serie di errori frequenti consultabili nella sezione §7 del documento.
Successivamente dopo questa prima fase, i verificatori mediante la tecnica dell'\glossario{Inspection} hanno analizzato i documenti in modo più mirato ricavando errori di diverso genere che non verranno trattati in quanto si trattano di piccolezze già facilmente risolte.

\subsubsection{Esiti verifiche sui documenti}
\label{sub:esiti_verifiche_sui_documenti}
Sui documenti sono state utilizzate le metriche per:
    \begin{itemize}
        \item \textbf{Indice di Gulpease};
        \item \textbf{Correttezza lessicale/ortografica}.
    \end{itemize}
Per evitare risultati errati nel calcole dell'Indice di Gulpease si è deciso di non tenere in considerazione:
    \begin{itemize}
        \item Il frontespizio di ogni documento;
        \item Le tabelle presenti nei documenti;
        \item Il diario delle modifiche all'interno di ogni documento.
    \end{itemize}
Per quanto riguarda la correttezza lessicale/ortografica, gli errori trovati non verranno classificati ma semplicemente conteggiati uniformemente includendo anche errori derivati dal mancato rispetto delle convenzioni scritte nelle \textsc{Norme di Progetto}.
\\ Per ogni metrica di ogni documento verrà inoltre inserito l'esito dell'analisi che può avere risultato soddisfacente o non.

\paragraph{Norme di Progetto}
\label{sub:norme_di_progetto}
Pur essendo le \textsc{Norme di Progetto} un documento abbastanza corposo e ricco di pagine di spiegazione, il numero di errori trovati non è molto alto ed è progressivamente calato in base alle versioni del documento dopo ogni fase di verifica delle modifiche, grazie anche a una buona verifica da parte dei Verificatori e di tutto il team.

\rowcolors{2}{white!80!lightgray!90}{white}
\renewcommand{\arraystretch}{2} % allarga le righe con dello spazio sotto e sopra
\begin{longtable}[H]{>{\centering\bfseries}m{6cm} >{\centering}m{2cm} >{\centering}m{2.5cm} >{\centering}m{2.5cm} >{\centering\arraybackslash}m{2.5cm}}  
  \rowcolor{lightgray}
  {\textbf{Documento}} & {\textbf{Risultato indice}} & {\textbf{Errori presenti}} & {\textbf{Esito indice}} & {\textbf{Esito errori}}  \\
  \endfirsthead%
  \rowcolor{lightgray}
  {\textbf{Documento}} & {\textbf{Risultato indice}} & {\textbf{Errori presenti}} & {\textbf{Esito indice}} & {\textbf{Esito errori}}  \\
  \endhead%
  \textbf{Norme di Progetto v1.0.0} &                  & 0               & Non soddisfatto & Soddisfatto \\
  \caption{Risultati metriche per le Norme di Progetto}
  \label{tab:my-table}
\end{longtable}

    \begin{figure}[H]
        \centering
        \includegraphics[width=10cm]{img/erroriNorme.png}
        \label{fig:scice_documenti}
        \caption{Grafico errori per \textsc{Norme Di Progetto}}
    \end{figure}

\paragraph{Studio di Fattibilità}
\label{sub:studio_di_fattibilita}
Essendo lo \textsc{Studio di Fattibilità} uno dei primi documenti redatti dal gruppo, il numero di errori è risultato alto rispetto al contenuto soprattutto per errori di dimenticanza nel rispettare quanto scritto nelle \textsc{Norme di Progetto}.
In questo documento sono state fatte 2 sole fasi di verifica in quanto il documento aveva contenuti più ristretti rispetto agli altri documenti e dopo la seconda verifica non vi sono stati aggiunti ulteriori contenuti.

\rowcolors{2}{white!80!lightgray!90}{white}
\renewcommand{\arraystretch}{2} % allarga le righe con dello spazio sotto e sopra
\begin{longtable}[H]{>{\centering\bfseries}m{6cm} >{\centering}m{2cm} >{\centering}m{2.5cm} >{\centering}m{2.5cm} >{\centering\arraybackslash}m{2.5cm}}  
  \rowcolor{lightgray}
  {\textbf{Documento}} & {\textbf{Risultato indice}} & {\textbf{Errori presenti}} & {\textbf{Esito indice}} & {\textbf{Esito errori}}  \\
  \endfirsthead%
  \rowcolor{lightgray}
  {\textbf{Documento}} & {\textbf{Risultato indice}} & {\textbf{Errori presenti}} & {\textbf{Esito indice}} & {\textbf{Esito errori}}  \\
  \endhead%
  \textbf{Studio di Fattibilità v1.0.0} &                  & 0               & Non soddisfatto & Soddisfatto \\
  \caption{Risultati metriche per lo Studio di Fattibilità}
  \label{tab:my-table}
\end{longtable}

    \begin{figure}[H]
        \centering
        \includegraphics[width=10cm]{img/erroriStudio.png}
        \label{fig:scice_documenti}
        \caption{Grafico errori per \textsc{Studio di Fattibilità}}
    \end{figure}

\paragraph{Piano di Qualifica}
\label{sub:piano_di_qualifica}
Il \textsc{Piano di Qualifica} inizialmente presentava molti errori che però sono scesi drasticamente ad ogni fase di verifica effettuata sul documento.
Essendo un documento ricco di sezioni e contenuti si è reso necessario effettuare 4 fasi di verifica per il documento.
L'indice di Gulpease rientra nel range definito tra 55 e 65.

\rowcolors{2}{white!80!lightgray!90}{white}
\renewcommand{\arraystretch}{2} % allarga le righe con dello spazio sotto e sopra
\begin{longtable}[H]{>{\centering\bfseries}m{6cm} >{\centering}m{2cm} >{\centering}m{2.5cm} >{\centering}m{2.5cm} >{\centering\arraybackslash}m{2.5cm}}  
  \rowcolor{lightgray}
  {\textbf{Documento}} & {\textbf{Risultato indice}} & {\textbf{Errori presenti}} & {\textbf{Esito indice}} & {\textbf{Esito errori}}  \\
  \endfirsthead%
  \rowcolor{lightgray}
  {\textbf{Documento}} & {\textbf{Risultato indice}} & {\textbf{Errori presenti}} & {\textbf{Esito indice}} & {\textbf{Esito errori}}  \\
  \endhead%
  \textbf{Piano di Qualifica v1.0.0} &  59                & 0               & Soddisfatto & Soddisfatto \\
  \caption{Risultati metriche per il Piano di Qualifica}
  \label{tab:my-table}
\end{longtable}

\begin{figure}[H]
    \centering
    \includegraphics[width=10cm]{img/erroriPdQ.png}
    \label{fig:scice_documenti}
    \caption{Grafico errori per \textsc{Piano di Qualifica}}
\end{figure}

\paragraph{Piano di Progetto}
\label{sub:piano_di_progetto}
Il \textsc{Piano di Progetto} ha presentato parecchi errori inizialmente così come il \textsc{Piano di Qualifica} ma dopo ogni verifica il numero di errori è sceso in modo quasi lineare fino a raggiungere l'obiettivo di 0 errori prefissato.

\rowcolors{2}{white!80!lightgray!90}{white}
\renewcommand{\arraystretch}{2} % allarga le righe con dello spazio sotto e sopra
\begin{longtable}[H]{>{\centering\bfseries}m{6cm} >{\centering}m{2cm} >{\centering}m{2.5cm} >{\centering}m{2.5cm} >{\centering\arraybackslash}m{2.5cm}}  
  \rowcolor{lightgray}
  {\textbf{Documento}} & {\textbf{Risultato indice}} & {\textbf{Errori presenti}} & {\textbf{Esito indice}} & {\textbf{Esito errori}}  \\
  \endfirsthead%
  \rowcolor{lightgray}
  {\textbf{Documento}} & {\textbf{Risultato indice}} & {\textbf{Errori presenti}} & {\textbf{Esito indice}} & {\textbf{Esito errori}}  \\
  \endhead%
  \textbf{Piano di Progetto v1.0.0} &                  & 0               & Non soddisfatto & Soddisfatto \\
  \caption{Risultati metriche per il Piano di Progetto}
  \label{tab:my-table}
\end{longtable}

\begin{figure}[H]
    \centering
    \includegraphics[width=10cm]{img/erroriPdP.png}
    \label{fig:scice_documenti}
    \caption{Grafico errori per \textsc{Piano di Progetto}}
\end{figure}

\paragraph{Analisi dei Requisiti}
\label{sub:analisi_dei_requisiti}
Come per le \textsc{Norme di Progetto} essendo anche l'\textsc{Analisi dei Requisiti} un documento abbastanza corposo il numero di errori rilevati durante le verifiche è stato elevato ma non eccessivo, inoltre si nota che nella fase di verifica che ha portato alla creazione della versione 0.3.0 il numero di errori è drasticamente calato. 

\rowcolors{2}{white!80!lightgray!90}{white}
\renewcommand{\arraystretch}{2} % allarga le righe con dello spazio sotto e sopra
\begin{longtable}[H]{>{\centering\bfseries}m{6cm} >{\centering}m{2cm} >{\centering}m{2.5cm} >{\centering}m{2.5cm} >{\centering\arraybackslash}m{2.5cm}}  
  \rowcolor{lightgray}
  {\textbf{Documento}} & {\textbf{Risultato indice}} & {\textbf{Errori presenti}} & {\textbf{Esito indice}} & {\textbf{Esito errori}}  \\
  \endfirsthead%
  \rowcolor{lightgray}
  {\textbf{Documento}} & {\textbf{Risultato indice}} & {\textbf{Errori presenti}} & {\textbf{Esito indice}} & {\textbf{Esito errori}}  \\
  \endhead%
  \textbf{Analisi dei Requisiti v1.0.0} &                  & 0               & Non soddisfatto & Soddisfatto \\
  \caption{Risultati metriche per l'Analisi dei Requisiti}
  \label{tab:my-table}
\end{longtable}

    \begin{figure}[H]
        \centering
        \includegraphics[width=10cm]{img/erroriAnalisi.png}
        \label{fig:scice_documenti}
        \caption{Grafico errori per \textsc{Analisi dei Requisiti}}
    \end{figure}

\paragraph{Glossario}
\label{sub:glossario}
Essendo il \textsc{Glossario} un documento abbastanza semplice e con lo scopo di informare il lettore sul significato di termini particolari, si è deciso di effettuare una sola fase di verifica finale.     

\rowcolors{2}{white!80!lightgray!90}{white}
\renewcommand{\arraystretch}{2} % allarga le righe con dello spazio sotto e sopra
\begin{longtable}[H]{>{\centering\bfseries}m{6cm} >{\centering}m{2cm} >{\centering}m{2.5cm} >{\centering}m{2.5cm} >{\centering\arraybackslash}m{2.5cm}}  
  \rowcolor{lightgray}
  {\textbf{Documento}} & {\textbf{Risultato indice}} & {\textbf{Errori presenti}} & {\textbf{Esito indice}} & {\textbf{Esito errori}}  \\
  \endfirsthead%
  \rowcolor{lightgray}
  {\textbf{Documento}} & {\textbf{Risultato indice}} & {\textbf{Errori presenti}} & {\textbf{Esito indice}} & {\textbf{Esito errori}}  \\
  \endhead%
  \textbf{Glossario v1.0.0} &                  & 0               & Non soddisfatto & Soddisfatto \\
  \caption{Risultati metriche per il Glossario}
  \label{tab:my-table}
\end{longtable}

\begin{figure}[H]
  \centering
  \includegraphics[width=10cm]{img/erroriGlossario.png}
  \label{fig:scice_documenti}
  \caption{Grafico errori per \textsc{Glossario}}
\end{figure}

\subsubsection{Esiti verifiche sui processi}
\label{sub:esiti_verifiche_sui_processi}
In questa sezione vengono visualizzati gli esiti delle metriche prese in considerazione per quanto riguarda i processi produttivi.
Come per i documenti, anche per queste metriche verrà fornito un esito che può essere soddisfacente o meno.

\paragraph{Processo PRC001}
\label{sub:processo_PRC001}

\rowcolors{2}{white!80!lightgray!90}{white}
\renewcommand{\arraystretch}{2} % allarga le righe con dello spazio sotto e sopra
\begin{longtable}[H]{>{\centering\bfseries}m{5cm} >{\centering}m{5cm} >{\centering}m{2.5cm} >{\centering\arraybackslash}m{2.5cm}}  
  \rowcolor{lightgray}
  {\textbf{Obiettivo}} & {\textbf{Metrica}} & {\textbf{Risultato}} & {\textbf{Esito}}  \\
  \endfirsthead%
  \rowcolor{lightgray}
  {\textbf{Obiettivo}} & {\textbf{Metrica}} & {\textbf{Risultato}} & {\textbf{Esito}}  \\
  \endhead%
  \textbf{QoPR001 Rispetto delle scadenze della pianificazione} & MoPR001 Varianza dei tempi & 1.31 & Soddisfatto \\
  \caption{Risultati metrica MoPR001}
  \label{tab:my-table}
\end{longtable}
\textbf{Nota}: il ritardo medio, nel rispetto delle scadenze interne al gruppo, è stato di 1.31 giorni che soddisfa la metrica in quanto non sono stati superati i 2 giorni di media.

\rowcolors{2}{white!80!lightgray!90}{white}
\renewcommand{\arraystretch}{2} % allarga le righe con dello spazio sotto e sopra
\begin{longtable}[H]{>{\centering\bfseries}m{5cm} >{\centering}m{5cm} >{\centering}m{2.5cm} >{\centering\arraybackslash}m{2.5cm}}  
  \rowcolor{lightgray}
  {\textbf{Obiettivo}} & {\textbf{Metrica}} & {\textbf{Risultato}} & {\textbf{Esito}}  \\
  \endfirsthead%
  \rowcolor{lightgray}
  {\textbf{Obiettivo}} & {\textbf{Metrica}} & {\textbf{Risultato}} & {\textbf{Esito}}  \\
  \endhead%
  \textbf{QoPR002 Rispetto del budget istanziato} & MoPR002 Varianza dei costi &  & Soddisfatto \\
  \caption{Risultati metrica MoPR002}
  \label{tab:my-table}
\end{longtable}
\textbf{Nota}:

\rowcolors{2}{white!80!lightgray!90}{white}
\renewcommand{\arraystretch}{2} % allarga le righe con dello spazio sotto e sopra
\begin{longtable}[H]{>{\centering\bfseries}m{5cm} >{\centering}m{5cm} >{\centering}m{2.5cm} >{\centering\arraybackslash}m{2.5cm}}  
  \rowcolor{lightgray}
  {\textbf{Obiettivo}} & {\textbf{Metrica}} & {\textbf{Risultato}} & {\textbf{Esito}}  \\
  \endfirsthead%
  \rowcolor{lightgray}
  {\textbf{Obiettivo}} & {\textbf{Metrica}} & {\textbf{Risultato}} & {\textbf{Esito}}  \\
  \endhead%
  \textbf{QoPR003 Rispetto del ciclo di vita scelto} & MoPR003 Aderenza agli standard & Livello di maturità: 2 Valutazione attributi: L & Non soddisfatto \\
  \caption{Risultati metrica MoPR003}
  \label{tab:my-table}
\end{longtable}
\textbf{Nota}: essendo il progetto ancora nelle fasi iniziali è prevedibile che il livello di maturità desiderato non sia ancora stato soddisfatto in quanto il raggiungimento dell'obiettivo è fissato per la fine del progetto.

\rowcolors{2}{white!80!lightgray!90}{white}
\renewcommand{\arraystretch}{2} % allarga le righe con dello spazio sotto e sopra
\begin{longtable}[H]{>{\centering\bfseries}m{5cm} >{\centering}m{5cm} >{\centering}m{2.5cm} >{\centering\arraybackslash}m{2.5cm}}  
  \rowcolor{lightgray}
  {\textbf{Obiettivo}} & {\textbf{Metrica}} & {\textbf{Risultato}} & {\textbf{Esito}}  \\
  \endfirsthead%
  \rowcolor{lightgray}
  {\textbf{Obiettivo}} & {\textbf{Metrica}} & {\textbf{Risultato}} & {\textbf{Esito}}  \\
  \endhead%
  \textbf{QoPR004 Rispetto dei ruoli e identificazione nei prodotti} & MoPR004 Aderenza ai ruoli & 0 & Soddisfatto \\
  \caption{Risultati metrica MoPR004}
  \label{tab:my-table}
\end{longtable}
\textbf{Nota}: in ogni documento prodotto i ruoli sono stati identificati e rispettati.

\rowcolors{2}{white!80!lightgray!90}{white}
\renewcommand{\arraystretch}{2} % allarga le righe con dello spazio sotto e sopra
\begin{longtable}[H]{>{\centering\bfseries}m{5cm} >{\centering}m{5cm} >{\centering}m{2.5cm} >{\centering\arraybackslash}m{2.5cm}}  
  \rowcolor{lightgray}
  {\textbf{Obiettivo}} & {\textbf{Metrica}} & {\textbf{Risultato}} & {\textbf{Esito}}  \\
  \endfirsthead%
  \rowcolor{lightgray}
  {\textbf{Obiettivo}} & {\textbf{Metrica}} & {\textbf{Risultato}} & {\textbf{Esito}}  \\
  \endhead%
  \textbf{QoPR005 Rispetto del versionamento dei prodotti} & MoPR005 Controllo prodotti & 21.4 & Soddisfatto \\
  \caption{Risultati metrica MoPR005}
  \label{tab:my-table}
\end{longtable}
\textbf{Nota}: essendo la media dei commit nella repository oltre il valore desiderabile (che era 20), l'obiettivo è pienamente soddisfatto.

\paragraph{Processo PRC003}
\label{sub:processo_PRC003}

\rowcolors{2}{white!80!lightgray!90}{white}
\renewcommand{\arraystretch}{2} % allarga le righe con dello spazio sotto e sopra
\begin{longtable}[H]{>{\centering\bfseries}m{5cm} >{\centering}m{5cm} >{\centering}m{2.5cm} >{\centering\arraybackslash}m{2.5cm}}  
  \rowcolor{lightgray}
  {\textbf{Obiettivo}} & {\textbf{Metrica}} & {\textbf{Risultato}} & {\textbf{Esito}}  \\
  \endfirsthead%
  \rowcolor{lightgray}
  {\textbf{Obiettivo}} & {\textbf{Metrica}} & {\textbf{Risultato}} & {\textbf{Esito}}  \\
  \endhead%
  \textbf{QoPR010 Rispetto nella redazione dei documenti} & MoPR011 Analisi documenti & 3 & Soddisfatto \\
  \caption{Risultati metrica MoPR011}
  \label{tab:my-table}
\end{longtable}
\textbf{Nota}: tutti i documenti più corposi e discorsivi sono stati verificati almeno 3 volte, mentre per i documenti più esili di contenuti sono bastate 1 o 2 fasi di verifica. Nonostante tutto il team ritiene l'obiettivo comunque soddisfatto soprattutto per il fatto che non sono stati rilasciati documenti con errori.

\subsection{Conclusioni}%
\label{sub:conclusioni}
I dati analizzati esprimono un buon andamento del lavoro svolto dal team che si prefissa come obiettivo quello di continuare a migliorare sotto ogni aspetto relativo alle metriche e agli obiettivi prefissati.

\end{document}