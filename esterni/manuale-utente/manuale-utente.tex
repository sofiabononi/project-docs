\documentclass{article}

\input{../../commons/config}
%Qui ci andrà il percorso delle immagini da includere in analisi dei requisiti
\appendToGraphicspath{../../commons/img/}

%Tutti questi set permettono di modificare in modo adatto i placeholder nel template
\setTitle{Manuale Utente}

\setVersione{
}

\setResponsabile{
}

\setRedattori{
}

\setVerificatori{
}

\setUso{Esterno}

\setDescrizione{Documento contenente una guida per il progetto \textit{Predire in Grafana} del gruppo \emph{CoffeeCode}.
}

\setModifiche{
}

\disabilitaElencoTabelle{}

\begin{document}

\pagenumbering{gobble}

\begin{titlepage}% per non stampare il numero della pagina

  \raggedleft% allinea a destra la pagina
  \rule{1pt}{\textheight}% linea verticale
  \hspace{0.05\textwidth}% spazio tra linea e testo
  % lasciare questa riga per il corretto funziomento di \parbox
  \parbox[b]{0.75\textwidth}{% paragrafo che tiene il testo a destra della riga cambiando la larghezza il testo si muove a destra o a sinistra
  {\hspace{0.15\textwidth}\includegraphics[width=3.5cm,height=3.5cm]{logo.jpg}}\\[3\baselineskip] % logo
  {\Huge\bfseries CoffeeCode \\[1\baselineskip] Predire in Grafana}\\[4\baselineskip] % titolo
  {\Large\textsc{\placeholderTitle{}}}\\[4\baselineskip] % nome del documento
  {\begin{tabular}{r l}
    % testo in grassetto
    \textbf{Versione}     & \versione{}               \\
    \textbf{Approvazione} & \responsabile{}           \\
    \textbf{Redazione}    & \redattori{}              \\
    \textbf{Verifica}     & \verificatori{}           \\
    \textbf{Uso}          & \uso{}                    \\
    \textbf{Destinato a}  & CoffeeCode                \\
                          & prof.\ Vardanega Tullio   \\
                          & prof.\ Cardin Riccardo    \\
    \ifthenelse{\equal{\uso}{Esterno}}{
                          & Zucchetti S.p.A.       \\
    }{}
  \end{tabular}}\\[1\baselineskip]

  {\bfseries Descrizione}\\
  {\descrizione{}}\\[1\baselineskip]
  {\texttt{coffeecodeswe@gmail.com}}\\[\baselineskip] % email
  }

\end{titlepage}

\newgeometry{textheight=660pt, lmargin=2cm, tmargin=2cm, rmargin=2cm}

% setup di header e footer nelle pagine senza numero
\fancypagestyle{nopage}{%
  \fancyhf{}%
  \fancyhead[L]{\includegraphics[width=1.3cm]{logo.jpg}}%
  \fancyhead[R]{\emph{CoffeeCode}\\\placeholderTitle{}}%
}
% setup di header e footer nelle pagine col numero
\fancypagestyle{usual}{%
  \fancyhf{}%
  \fancyhead[L]{\includegraphics[width=1.3cm]{logo.jpg}}%
  \fancyhead[R]{\emph{CoffeeCode}\\\placeholderTitle{}}%
  \fancyfoot[R]{\thepage\ di~\pageref{LastPage}}%
}
\setlength{\headheight}{1.8cm}

\newpage
\pagestyle{nopage}

\setcounter{table}{-1}

\section*{Registro delle modifiche}%
\label{sec:registro_delle_modifiche}

\rowcolors{2}{white!80!lightgray!90}{white}
\renewcommand{\arraystretch}{2} % allarga le righe con dello spazio sotto e sopra
\begin{longtable}[H]{>{\centering\bfseries}m{2cm} >{\centering}m{3.5cm} >{\centering}m{2.5cm} >{\centering}m{3cm} >{\centering\arraybackslash}m{5cm}}
  \rowcolor{lightgray}
  {\textbf{Versione}} & {\textbf{Nominativo}} & {\textbf{Ruolo}} & {\textbf{Data}} & {\textbf{Descrizione}}  \\
  \endfirsthead%
  \rowcolor{lightgray}
  {\textbf{Versione}} & {\textbf{Nominativo}}  & {\textbf{Ruolo}} & {\textbf{Data}} & {\textbf{Descrizione}}  \\
  \endhead%
  \modifiche{}%
\end{longtable}
% section registro_delle_modifiche (end)

\newpage
\thispagestyle{nopage}
\pagenumbering{roman}
\tableofcontents

\elencoFigure{}%

\elencoTabelle{}%

\newpage

\pagestyle{usual}
\pagenumbering{arabic}


\section{Introduzione}%
\label{sec:introduzione}
\subfile{components/1-introduzione.tex}

\newpage

\section{Requisiti di sistema}%
\label{sec:requisiti di sistema}
\subfile{components/2-requisiti.tex}

\newpage

\section{Installazione}%
\label{sec:installazione}
\subfile{components/3-installazione.tex}

\newpage

\section{Utilizzo training app}%
\label{sec:utilizzo-training-app}
\subfile{components/4-utilizzo-training-app.tex}

\newpage
\appendix
\setcounter{secnumdepth}{1} % No section number
\section{Glossario}%
\label{sec:glossario}
\subfile{components/A-glossario}



\end{document}
