\documentclass[../manuale-utente.tex]{subfiles}

\begin{document}

\subsection{Scelta e inserimento di un file}
\label{subs:scelta-e-inserimento}
La prima azione che bisogna fare una volta installata la training app è la scelta di un file \glossario{CSV}, questo file al suo interno deve contenere i dati con cui si vuole compiere l'addestramento. Per caricare il file bisogna premere sul pulsante che si trova sotto la scritta "CSV data file:"\\
\centerline{Scegli file}\\
e selezionare il file che si vuole utilizzare. Una volta fatto questo, vicino al pulsante comparirà il nome del file scelto.



\subsection{Scelta dell'algoritmo}
\label{subs:scelta-algoritmo}
Dopo aver inserito il file, bisogna scegliere con che algoritmo di machine learning compiere l'addestramento. E' possibile scegliere tra
\begin{itemize}
  \item RL (Regressione Lineare);
  \item SVM (Support Vector Machine).
\end{itemize}


\end{document}
