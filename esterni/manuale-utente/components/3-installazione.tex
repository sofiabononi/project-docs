\documentclass[../manuale-utente.tex]{subfiles}

\begin{document}

Per installare il prodotto è necessario seguire i passaggi descritti in seguito. Il prodotto è composto da due parti distinte:
\begin{itemize}
  \item Programma di addestramento;
  \item Plug-in.
\end{itemize}

\subsection{Installazione training app}
\label{subs:installazione-training-app}
Per installare ed avviare il modulo di addestramento è necessario:
\begin{itemize}
  \item Scaricare la cartella contenente la Training App presente nella \glossario{repository}: \\
  \centerline{\url{https://github.com/CoffeeCodeSWE/swe-training-app}}
  \item Aprire il Prompt dei comandi (Windows) o il terminale (Linux/Mac);
  \item Posizionarsi nella cartella appena scaricata tramite: \\
  \centerline{\texttt{cd /}\textit{percorso}}
  \item Solo per il primo utilizzo eseguire: \\
  \centerline{\texttt{yarn install}}
  \item Per avviare è sufficiente eseguire: \\
  \centerline{\texttt{yarn start}}
\end{itemize}

\subsection{Installazione plug-in}%
\label{subs:installazione-plug-in}
Per installare ed avviare il plug-in è neccessario:
\begin{enumerate}
    \item Scaricare o clonare la repository del plug-in dalla seguente risorsa web: \href{https://github.com/CoffeeCodeSWE/swe-grafana-plugin}{https://github.com/CoffeeCodeSWE/swe-grafana-plugin};
    \item Spostare la cartella del plug-in all'interno dell'apposita cartella di Grafana. Questa varia a seconda del tipo di installazione della piattaforma e del sistema operativo. Nello specifico:
    \begin{itemize}
      \item Se la piattaforma viene utilizzata tramite binari \textit{standalone}, la cartella si trova alla posizione \texttt{grafana/data/plugins};
      \item Se la piattaforma è globalmente installata su un sistema Windows, la cartella si trova alla posizione \texttt{FS:\textbackslash Program Files\textbackslash GrafanaLabs\textbackslash grafana\textbackslash data\textbackslash plugins}, dove con \texttt{FS} si intende la lettera del drive contenente il sistema operativo;
      \item Se la piattaforma è globalmente installata su un sistema GNU/Linux, la cartella si trova alla posizione \texttt{/var/lib/grafana/plugins} o \texttt{/usr/local/var/lib/grafana/plugins};
      \item Se la piattaforma è globalmente installata su un sistema macOS, la cartella si trova alla posizione \texttt{/usr/local/var/lib/grafana/plugins}.
    \end{itemize}
    \item Installare le dipendenze con il seguente comando da terminale:
    \\ \centerline{\texttt{yarn install}}
    In alternativa, è possibile utilizzare il software npm. Per l'installazione delle dipendenze tramite npm, eseguire il seguente comando da terminale:
    \\ \centerline{\texttt{npm install}}
    \item Compilare il plug-in con il seguente comando da terminale:
    \\ \centerline{\texttt{yarn build}}
    In alternativa, è possibile utilizzare il software npm. Per la compilazione tramite npm, eseguire il seguente comando da terminale:
    \\ \centerline{\texttt{npm build}}
    È inoltre possibile compilare il plug-in in modalità sviluppatore tramite il comando
    \\ \centerline{\texttt{yarn dev}}
    oppure, alternativamente
    \\ \centerline{\texttt{npm dev}}
    Questo compila il componente software senza ottimizzazioni e senza eseguire i test di unità automatici;
    \item Avviare Grafana;
    \item Recarsi in Configurazione $>$ Plugins $>$ \emph{CoffeeCode Prediction Panel} e abilitare il plug-in;
    \item Selezionare \emph{CoffeCode Prediction Panel} durante l'aggiunta di un pannello alla dashboard.
\end{enumerate}

\end{document}
