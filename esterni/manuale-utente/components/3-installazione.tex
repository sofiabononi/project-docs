\documentclass[../manuale-utente.tex]{subfiles}

\begin{document}

Per installare il prodotto è necessario seguire i passaggi descritti in seguito. Il prodotto è composto da due parti distinte:
\begin{itemize}
  \item Programma di addestramento;
  \item Plug-in.
\end{itemize}

\subsection{Installazione training app}
\label{subs:installazione-training-app}
Per installare ed avviare il modulo di addestramento è necessario:
\begin{itemize}
  \item Scaricare la cartella contenente la Training App presente nella \glossario{Repository}: \\
  \centerline{\url{https://github.com/CoffeeCodeSWE/swe-training-app}}
  \item Aprire il Prompt dei comandi (Windows) o il terminale (Linux/Mac);
  \item Posizionarsi nella cartella appena scaricata tramite: \\
  \centerline{\texttt{cd /}\textit{percorso}}
  \item Solo per il primo utilizzo eseguire: \\
  \centerline{\texttt{yarn install}}
  \item Per avviare è sufficiente eseguire: \\
  \centerline{\texttt{yarn start}}
\end{itemize}

\subsection{Installazione plug-in}%
\label{subs:installazione-plug-in}
Per installare ed avviare il plug-in è neccessario:
\begin{itemize}
  \item Scaricare la cartella contenente il plug-in presente nella Repository: \\
  \centerline{\url{https://github.com/CoffeeCodeSWE/swe-grafana-plugin}}
  \item Estrarre i file nella cartella \texttt{../grafana/plugins} presente nella root di Grafana;
  \item Aprire il Prompt dei comandi (Windows) o il terminale (Linux/Mac);
  \item Posizionarsi nella cartella appena scaricata tramite: \\
  \centerline{\texttt{cd /}\textit{percorso}}
  \item Solo per il primo utilizzo eseguire: \\
  \centerline{\texttt{yarn install}}
  \item Per avviare è sufficiente eseguire: \\
  \centerline{\texttt{yarn build}} \\
  \newline
  Alternativamente è possibile rendere il plug-in operativo, dopo il processo di installazione e sempre nel terminale nella cartella dove è stato installato il plug-in, eseguendo il seguente comando: \\
  \newline
  \centerline{\texttt{yarn dev}}
\end{itemize}

\end{document}
