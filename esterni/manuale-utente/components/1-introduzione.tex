\documentclass[../manuale-utente.tex]{subfiles}

\begin{document}

\subsection{Scopo del documento}%
\label{subs:scopo_del_documento}
Il documento ha la finalità di spiegare le funzionalità e le modalità di utilizzo del software \textit{Predire in Grafana}. Esso rappresenta sia una guida che un riferimento completo per l'utilizzo del prodotto.

\subsection{Scopo del prodotto}%
\label{subs:scopo_del_prodotto}
Il \glossario{capitolato} C4 - "Predire in Grafana" ha come obiettivo la realizzazione di un \glossario{plug-in}, scritto in linguaggio \glossario{JavaScript} per lo strumento \glossario{open source} \glossario{Grafana}, che applichi le tecniche di \glossario{Machine Learning} \glossario{Regressione Lineare} e \glossario{Support Vector Machines} ad un flusso di dati. Questo plug-in provvederà quindi a svolgere un'analisi dei dati e, da questi, fornire una predizione al fine di monitorare la \glossario{liveliness} del sistema e di consigliare interventi alla linea di produzione del software tramite specifici \glossario{alert}. \\
Prima della realizzazione di tale plug-in si dovrà sviluppare un programma comprendente un'interfaccia web con l'utilizzo del \glossario{framework} \glossario{Electron}, attraverso la quale verrà eseguito l'\glossario{addestramento} necessario al corretto funzionamento degli algoritmi di Machine Learning. \\
I risultati ottenuti dovranno poi essere visualizzati, in forma di grafico, in una \glossario{dashboard} sulla piattaforma Grafana.

\subsection{Glossario}
\label{subs:glossario}
All'interno del documento sono presenti termini che possono avere dei significati ambigui a seconda del contesto. Per evitare questa ambiguità è stato creato un glossario, il quale si trova nell'appendice §A, che conterrà tali termini con il loro significato specifico. Per indicare che il termine del testo è presente all'interno del glossario verrà segnalato con una sottolineatura e una G a pedice.

\end{document}
