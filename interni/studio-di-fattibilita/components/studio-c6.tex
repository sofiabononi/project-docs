\documentclass[../studio-di-fattibilita.tex]{subfiles}

\begin{document}

\subsection{Informazioni generali}%
\label{sub:informazioni_generale}
\begin{description}
  \item \textbf{Nome}: ThiReMa - Things Relationship Management.
  \item \textbf{Proponente}: Sanmarco Informatica.
  \item \textbf{Committente}: prof. Tullio Vardanega e prof. Riccardo. Cardin.
\end{description}

\subsection{Descrizione}%
\label{sub:descrizione}
Viene richiesto lo sviluppo di un software in grado di ricevere misurazioni da sensori di diverse tipologie e dislocati in diverse aree geografiche. I dati ricavati dalle misurazioni devono essere accumulati in maniera efficiente e affidabile in un \glossario{database centralizzato}. Questa applicazione viene poi completata da un servizio di \glossario{dispatching}, basato su \glossario{Telegram}, con lo scopo inoltrare informazioni utili alla gestione dei dispositivi.

\subsection{Finalità del progetto}%
\label{sub:finalita_del_progetto}
Creare una applicatione web che permetta di  valutare la correlazione tra dati operativi (misure) e i fattori influenzanti. Il software deve fornire uno o più algoritmi che successivamente permetteranno l'analisi dei dati al fine di offrire vari servizi come, ad esempio, la \glossario{manutenzione predittiva}.
Per ogni tipologia di informazioni rilevate, dovrà anche essere possibile assegnare il monitoraggio ad un particolare ente.

L'applicazione web dovrà essere suddivisa in 3 macro-sezioni:
\begin{itemize}
    \item Censimento dei sensori e dei relativi dati;
    \item Modulo di analisi di correlazione;
    \item Modulo di monitoraggio per ente.
\end{itemize}


\subsection{Tecnologie interessate}%
\label{sub:tecnologie_interessate}
\begin{itemize}
    \item \glossario{Apache Kafka}: componente che consente la gestione di un elevato numero di operazioni in tempo reale da migliaia di client, sia in lettura che in scrittura;
    \item \glossario{API Producer}, \glossario{Consumer}, \glossario{Connect} e \glossario{Stream}: \glossario{librerie} per la realizzazione di \glossario{componenti custom};
    \item \textbf{Java}: linguaggio di programmazione di alto livello \glossario{orientato agli oggetti};
    \item \glossario{PostgreSQL}, \glossario{TimescaleDB}, \glossario{ClickHouse}: tipologie di database suggerite;
    \item \glossario{Bootstrap}: Framework per la realizzazione di applicazioni web;
    \item \glossario{Docker}: tecnologia di \glossario{containerizzazione} che consente la creazione e l'utilizzo dei \glossario{container}.
\end{itemize}

\subsection{Aspetti positivi}%
\label{sub:aspetti_positivi}
\begin{itemize}
    \item Il linguaggio Java è già conosciuto da tutto il gruppo;
    \item Occasione per l'ampliamento delle conoscenze sulle tecnologie proposte.
\end{itemize}

\subsection{Rischi}%
\begin{itemize}
    \item La maggior parte delle tecnologie da utilizzare non sono conosciute dai membri del gruppo, ciò richiederebbe una discreta quantità di tempo da investire per la formazione;
    \item Vengono richieste conoscenze di analisi ed elaborazioni dei dati che il gruppo non possiede.
\end{itemize}


\label{sub:rischi}


\subsection{Conclusione}%
\label{sub:conclusione}
Il capitolato non è stato preso in considerazione perché non era più disponibile per la scelta.

\end{document}
