\documentclass{article}

\input{../../commons/config}
%Qui ci andrà il percorso delle immagini da includere in analisi dei requisiti
\appendToGraphicspath{../../commons/img/}

\setTitle{Studio di Fattibilità}

\setVersione{v1.4-1.1.0}

\setResponsabile{
  Enrico Galdeman
}

\setRedattori{
  Ian Nicolas Di Menna
}

\setVerificatori{
  Enrico Buratto
}

\setUso{Interno}

\setDescrizione{Documento che analizza tutti i capitolati proposti.}

\disabilitaElencoFigure{}
\disabilitaElencoTabelle{}

\setModifiche{
  v1.4-1.1.0 & Enrico Galdeman & Responsabile & 2020-05-09 & Approvazione del documento \\
  v1.1-1.1.0 & Ian Nicolas Di Menna, Enrico Buratto & Analista e Verificatore & 2020-05-01 & Ampliato studio di fattibilità capitolato C4 \\
  v1.0.0-old & Ian Nicolas Di Menna & Responsabile & 2020-03-24 & Approvazione del documento \\
  v0.2.0-old & Nicholas Miazzo & Verificatore & 2020-03-23 & Verifica dello studio di fattibilità del capitolato C4 \\
  v0.1.1-old & Alessandro Franchin & Analista & 2020-03-21 & Approfondimento dello studio di fattibilità del capitolato C4 \\
  v0.1.0-old & Marco Nardelotto & Verificatore & 2020-03-16 & Verifica dello studio di fattibilità dei capitolati C1, C2, C3, C4, C5 e C6 \\
  v0.0.5-old & Enrico Buratto & Analista & 2020-03-14 & Stesura dello studio di fattibilità dei capitolati C4 e C6 \\
  v0.0.4-old & Ian Nicolas Di Menna & Analista & 2020-03-13 & Stesura dello studio di fattibilità dei capitolati C3 e C5 \\
  v0.0.3-old & Ian Nicolas Di Menna & Analista & 2020-03-12 & Stesura dello studio di fattibilità del capitolato C2 \\
  v0.0.2-old & Enrico Buratto & Analista & 2020-03-12 & Stesura introduzione e studio di fattibilità del capitolato C1 \\
  v0.0.1-old & Alessandro Franchin & Responsabile & 2020-03-11 & Creazione del documento \\
}

\begin{document}

\pagenumbering{gobble}

\begin{titlepage}% per non stampare il numero della pagina

  \raggedleft% allinea a destra la pagina
  \rule{1pt}{\textheight}% linea verticale
  \hspace{0.05\textwidth}% spazio tra linea e testo
  % lasciare questa riga per il corretto funziomento di \parbox
  \parbox[b]{0.75\textwidth}{% paragrafo che tiene il testo a destra della riga cambiando la larghezza il testo si muove a destra o a sinistra
  {\hspace{0.15\textwidth}\includegraphics[width=3.5cm,height=3.5cm]{logo.jpg}}\\[3\baselineskip] % logo
  {\Huge\bfseries CoffeeCode \\[1\baselineskip] Predire in Grafana}\\[4\baselineskip] % titolo
  {\Large\textsc{\placeholderTitle{}}}\\[4\baselineskip] % nome del documento
  {\begin{tabular}{r l}
    % testo in grassetto
    \textbf{Versione}     & \versione{}               \\
    \textbf{Approvazione} & \responsabile{}           \\
    \textbf{Redazione}    & \redattori{}              \\
    \textbf{Verifica}     & \verificatori{}           \\
    \textbf{Uso}          & \uso{}                    \\
    \textbf{Destinato a}  & CoffeeCode                \\
                          & prof.\ Vardanega Tullio   \\
                          & prof.\ Cardin Riccardo    \\
    \ifthenelse{\equal{\uso}{Esterno}}{
                          & Zucchetti S.p.A.       \\
    }{}
  \end{tabular}}\\[1\baselineskip]

  {\bfseries Descrizione}\\
  {\descrizione{}}\\[1\baselineskip]
  {\texttt{coffeecodeswe@gmail.com}}\\[\baselineskip] % email
  }

\end{titlepage}

\newgeometry{textheight=660pt, lmargin=2cm, tmargin=2cm, rmargin=2cm}

% setup di header e footer nelle pagine senza numero
\fancypagestyle{nopage}{%
  \fancyhf{}%
  \fancyhead[L]{\includegraphics[width=1.3cm]{logo.jpg}}%
  \fancyhead[R]{\emph{CoffeeCode}\\\placeholderTitle{}}%
}
% setup di header e footer nelle pagine col numero
\fancypagestyle{usual}{%
  \fancyhf{}%
  \fancyhead[L]{\includegraphics[width=1.3cm]{logo.jpg}}%
  \fancyhead[R]{\emph{CoffeeCode}\\\placeholderTitle{}}%
  \fancyfoot[R]{\thepage\ di~\pageref{LastPage}}%
}
\setlength{\headheight}{1.8cm}

\newpage
\pagestyle{nopage}

\setcounter{table}{-1}

\section*{Registro delle modifiche}%
\label{sec:registro_delle_modifiche}

\rowcolors{2}{white!80!lightgray!90}{white}
\renewcommand{\arraystretch}{2} % allarga le righe con dello spazio sotto e sopra
\begin{longtable}[H]{>{\centering\bfseries}m{2cm} >{\centering}m{3.5cm} >{\centering}m{2.5cm} >{\centering}m{3cm} >{\centering\arraybackslash}m{5cm}}
  \rowcolor{lightgray}
  {\textbf{Versione}} & {\textbf{Nominativo}} & {\textbf{Ruolo}} & {\textbf{Data}} & {\textbf{Descrizione}}  \\
  \endfirsthead%
  \rowcolor{lightgray}
  {\textbf{Versione}} & {\textbf{Nominativo}}  & {\textbf{Ruolo}} & {\textbf{Data}} & {\textbf{Descrizione}}  \\
  \endhead%
  \modifiche{}%
\end{longtable}
% section registro_delle_modifiche (end)

\newpage
\thispagestyle{nopage}
\pagenumbering{roman}
\tableofcontents

\elencoFigure{}%

\elencoTabelle{}%

\newpage

\pagestyle{usual}
\pagenumbering{arabic}


\section{Introduzione}%
\label{sec:introduzione}

\subsection{Scopo del documento}%
\label{sub:scopo_del_documento}
L'obiettivo dello \textsc{Studio di fattibilità} è quello di fornire le motivazioni che hanno portato alla scelta del \glossario{capitolato} C4 \emph{Predire in Grafana}, con la conseguente esclusione degli altri capitolati proposti.

\subsection{Glossario}%
\label{sub:glossario}
All'interno del documento sono presenti termini che possono avere dei significati ambigui a seconda del contesto. Per evitare questa ambiguità è stato creato un documento di nome \textsc{Glossario v3.11-1.3.1} che conterrà tali termini con il loro significato specifico. Per segnalare che il termine del testo è presente all'interno del glossario verrà segnalato con una G a pedice a fianco del termine.

\subsection{Riferimenti}%
\label{sub:riferimenti}

\subsubsection{Normativi}%
\label{subs:normativi}

\begin{itemize}
  \item \textbf{Norme di progetto}: \textsc{Norme di progetto v1.0.0}
\end{itemize}

\subsubsection{Informativi}%
\label{subs:informativi}

\begin{itemize}
  \item \textbf{Capitolato d'appalto C1 - Autonomous Highlights Platform}: \url{https://www.math.unipd.it/~tullio/IS-1/2019/Progetto/C1.pdf};
  \item \textbf{Capitolato d'appalto C2 - Etherless}: \url{https://www.math.unipd.it/~tullio/IS-1/2019/Progetto/C2.pdf};
  \item \textbf{Capitolato d'appalto C3 - NaturalAPI}: \url{https://www.math.unipd.it/~tullio/IS-1/2019/Progetto/C3.pdf};
  \item \textbf{Capitolato d'appalto C4 - Predire in Grafana}: \url{https://www.math.unipd.it/~tullio/IS-1/2019/Progetto/C4.pdf};
  \item \textbf{Capitolato d'appalto C5 - Stalker}: \url{https://www.math.unipd.it/~tullio/IS-1/2019/Progetto/C5.pdf};
  \item \textbf{Capitolato d'appalto C6 - ThiReMa - Things Relationship Management}: \url{https://www.math.unipd.it/~tullio/IS-1/2019/Progetto/C6.pdf}.
\end{itemize}

\newpage
\section{Capitolato C4 --- Predire in Grafana}%
\label{sec:c4}
\subfile{components/studio-c4.tex}

\newpage
\section{Capitolato C1 --- Autonomous Highlights Platform}%
\label{sec:c1}
\subfile{components/studio-c1.tex}

\newpage
\section{Capitolato C2 --- Etherless}%
\label{sec:c2}
\subfile{components/studio-c2.tex}

\newpage
\section{Capitolato C3 ---  NaturalAPI}%
\label{sec:c3}
\subfile{components/studio-c3.tex}

\newpage
\section{Capitolato C5 --- Stalker}%
\label{sec:c5}
\subfile{components/studio-c5.tex}

\newpage
\section{Capitolato C6 --- ThiReMa - Things Relationship Management}%
\label{sec:c6}
\subfile{components/studio-c6.tex}

\end{document}
