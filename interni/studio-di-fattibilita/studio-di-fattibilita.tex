\documentclass{article}

\usepackage[italian]{babel}
\usepackage[margin=20mm, footskip = 20pt]{geometry}
\usepackage{graphicx}
\usepackage{subfiles}
\usepackage{hyperref}
\usepackage{nameref}
\usepackage{titlesec}
\usepackage{longtable}
\usepackage[table]{xcolor}
\usepackage{titling}
\usepackage{lastpage}
\usepackage{ifthen}
\usepackage{calc}
\usepackage{soulutf8}
\usepackage{contour}
\usepackage{float}
\usepackage{fancyhdr}
\usepackage{multirow}
\usepackage{pgfgantt}

% definizione dei percorsi in cui cercare immagini
\graphicspath{ {./}
    {./img/}
}

% setup della sottolineatura
\setuldepth{Flat}
\contourlength{0.8pt}

\newcommand{\uline}[1]{%
  \ul{{\phantom{#1}}}%
  \llap{\contour{white}{#1}}%
}

% setup dei link
\hypersetup{
  % set true if you want colored links (instead of boxes)
  colorlinks=true,
  % set to all if you want both sections and subsections linked
  linktoc=all,
  % set color for file links
  filecolor=blue,
  % set color for internal links
  linkcolor=black,
  % set url color
  urlcolor=blue,
  % set characters encoding in the bookmarks tab
  pdfencoding=unicode,
}

% setup forma \paragraph e \subparagraph
\titleformat{\paragraph}[hang]{\normalfont\normalsize\bfseries}{\theparagraph}{1em}{}
\titleformat{\subparagraph}[hang]{\normalfont\normalsize\bfseries}{\thesubparagraph}{1em}{}

% setup profondità indice di default
\setcounter{secnumdepth}{5}
\setcounter{tocdepth}{5}

\makeatletter %% non togliere, i comandi che definiscono i placeholder vanno qui
% esempio di utilizzo: \appendToGraphicspath{./img/} (un comando diverso per ogni path da includere)
% N.B.: ci DEVE essere un forward slash alla fine del path, a indicare che è una cartella.
\newcommand\appendToGraphicspath[1]{%
  \g@addto@macro\Ginput@path{{#1}}%
}

\newcommand{\setTitle}[1]{%
  \newcommand{\@placeholderTitle}{#1}%
}
\newcommand{\placeholderTitle}{\@placeholderTitle}

\newcommand{\setUso}[1]{%
  \newcommand{\@uso}{#1}%
}
\newcommand{\uso}{\@uso}

\newcommand{\setVersione}[1]{%
  \newcommand{\@versione}{#1}%
}
\newcommand{\versione}{\@versione}

\newcommand{\disabilitaVersione}{%
  \renewcommand{\setVersione}[1]{}%
  \renewcommand{\versione}{DISABILITATA}
}

\newcommand{\setResponsabile}[1]{%
  \newcommand{\@responsabile}{#1}%
}
\newcommand{\responsabile}{\@responsabile}

\newcommand{\setRedattori}[1]{%
  \newcommand{\@redattori}{#1}%
}
\newcommand{\redattori}{\@redattori}

\newcommand{\setVerificatori}[1]{%
  \newcommand{\@verificatori}{#1}%
}
\newcommand{\verificatori}{\@verificatori}

\newcommand{\setDescrizione}[1]{%
  \newcommand{\@descrizione}{#1}%
}
\newcommand{\descrizione}{\@descrizione}

\newcommand{\setModifiche}[1]{%
  \newcommand{\@modifiche}{#1}%
}

\newcommand{\modifiche}{\@modifiche}
\makeatother %% non togliere, i comandi che definiscono i placeholder vanno qui

% hook per lo script che genera il glossario
\newcommand{\glossario}[1]{\underline{#1}\textsubscript{g}}

% comandi per rendere opzionali gli elenchi di figure
\newcommand{\elencoFigure}{%
  \renewcommand{\listfigurename}{Elenco delle figure}%
  \listoffigures%
}

\newcommand{\disabilitaElencoFigure}{%
  \renewcommand{\elencoFigure}{}%
}

% comandi per rendere opzionali le tabelle
\newcommand{\elencoTabelle}{%
  \renewcommand{\listtablename}{Elenco delle tabelle}%
  \listoftables%
}

\newcommand{\disabilitaElencoTabelle}{%
  \renewcommand{\elencoTabelle}{}%
}

%Qui ci andrà il percorso delle immagini da includere in analisi dei requisiti
\appendToGraphicspath{../../commons/img/}

\setTitle{Studio di Fattibilità}

\setVersione{v1.0.0}

\setResponsabile{
  Ian Nicolas Di Menna \\ &
  Alessandro Franchin
}

\setRedattori{
  Enrico Buratto \\ &
  Ian Nicolas Di Menna \\ &
  Alessandro Franchin
}

\setVerificatori{
  Nicholas Miazzo \\ &
  Marco Nardelotto
}

\setUso{Interno}

\setDescrizione{Documento che analizza tutti i capitolati proposti.}

\setModifiche{
  v1.0.0 & Ian Nicolas Di Menna & Responsabile & 2020-03-24 & Approvazione del documento \\
  v0.2.0 & Nicholas Miazzo & Verificatore & 2020-03-23 & Verifica dello studio di fattibilità del capitolato C4 \\
  v0.1.1 & Alessandro Franchin & Analista & 2020-03-21 & Approfondimento dello studio di fattibilità del capitolato C4 \\
  v0.1.0 & Marco Nardelotto & Verificatore & 2020-03-16 & Verifica dello studio di fattibilità dei capitolati C1, C2, C3, C4, C5 e C6 \\
  v0.0.5 & Enrico Buratto & Analista & 2020-03-14 & Stesura dello studio di fattibilità dei capitolati C4 e C6 \\
  v0.0.4 & Ian Nicolas Di Menna & Analista & 2020-03-13 & Stesura dello studio di fattibilità dei capitolati C3 e C5 \\
  v0.0.3 & Ian Nicolas Di Menna & Analista & 2020-03-12 & Stesura dello studio di fattibilità del capitolato C2 \\
  v0.0.2 & Enrico Buratto & Analista & 2020-03-12 & Stesura introduzione e studio di fattibilità del capitolato C1 \\
  v0.0.1 & Alessandro Franchin & Responsabile & 2020-03-11 & Creazione del documento \\
}

\begin{document}

\pagenumbering{gobble}

\begin{titlepage}% per non stampare il numero della pagina

  \raggedleft% allinea a destra la pagina
  \rule{1pt}{\textheight}% linea verticale
  \hspace{0.05\textwidth}% spazio tra linea e testo
  % lasciare questa riga per il corretto funziomento di \parbox
  \parbox[b]{0.75\textwidth}{% paragrafo che tiene il testo a destra della riga cambiando la larghezza il testo si muove a destra o a sinistra
  {\hspace{0.15\textwidth}\includegraphics[width=3cm,height=3cm]{logo.jpg}}\\[2\baselineskip] % logo
  {\Huge\bfseries CoffeeCode \\[0.5\baselineskip] Predire in Grafana}\\[5\baselineskip] % titolo
  {\Large\textsc{\placeholderTitle{}}}\\[6\baselineskip] % nome del documento
  {\begin{tabular}{r l}
    % testo in grassetto
    \textbf{Versione}     & \versione{}               \\
    \textbf{Approvazione} & \responsabile{}           \\
    \textbf{Redazione}    & \redattori{}              \\
    \textbf{Verifica}     & \verificatori{}           \\
    \textbf{Uso}          & \uso{}                    \\
    \textbf{Destinato a}  & CoffeeCode                \\
                          & prof.\ Vardanega Tullio   \\
                          & prof.\ Cardin Riccardo    \\
    \ifthenelse{\equal{\uso}{Esterno}}{
                          & Zucchetti Group SPA       \\
    }{}
  \end{tabular}}\\[5\baselineskip]

  {\bfseries Descrizione}\\
  {\descrizione{}}\\[2\baselineskip]
  {\texttt{coffeecodeswe@gmail.com}}\\[\baselineskip] % email
  }

\end{titlepage}


\section{Introduzione}%
\label{sec:introduzione}

\subsection{Scopo del documento}%
\label{sub:scopo_del_documento}
L'obiettivo dello \textsc{Studio di fattibilità} è quello di fornire le motivazioni che hanno portato alla scelta del \glossario{capitolato} C4 \emph{Predire in Grafana}, con la conseguente esclusione degli altri capitolati proposti.

\subsection{Glossario}%
\label{sub:glossario}
All'interno del documento sono presenti termini che possono avere dei significati ambigui a seconda del contesto. Per evitare questa ambiguità è stato creato un documento di nome \textsc{Glossario} che conterrà tali termini con il loro significato specifico. Per segnalare che il termine del testo è presente all'interno del glossario verrà segnalato con una G a pedice a fianco del termine.

\subsection{Riferimenti}%
\label{sub:riferimenti}

\subsubsection{Normativi}%
\label{subs:normativi}

\begin{itemize}
  \item \textbf{Norme di progetto}: \textsc{Norme di progetto}
\end{itemize}

\subsubsection{Informativi}%
\label{subs:informativi}

\begin{itemize}
  \item \textbf{Capitolato d'appalto C1 - Autonomous Highlights Platform}: \url{https://www.math.unipd.it/~tullio/IS-1/2019/Progetto/C1.pdf}
  \item \textbf{Capitolato d'appalto C2 - Etherless}: \url{https://www.math.unipd.it/~tullio/IS-1/2019/Progetto/C2.pdf}
  \item \textbf{Capitolato d'appalto C3 - NaturalAPI}: \url{https://www.math.unipd.it/~tullio/IS-1/2019/Progetto/C3.pdf}
  \item \textbf{Capitolato d'appalto C4 - Predire in Grafana}: \url{https://www.math.unipd.it/~tullio/IS-1/2019/Progetto/C4.pdf}
  \item \textbf{Capitolato d'appalto C5 - Stalker}: \url{https://www.math.unipd.it/~tullio/IS-1/2019/Progetto/C5.pdf}
  \item \textbf{Capitolato d'appalto C6 - ThiReMa - Things Relationship Management}: \url{https://www.math.unipd.it/~tullio/IS-1/2019/Progetto/C6.pdf}
\end{itemize}

\newpage
\section{Capitolato C4 --- Predire in Grafana}%
\label{sec:c4}
\subfile{components/studio-c4.tex}

\newpage
\section{Capitolato C1 --- Autonomous Highlights Platform}%
\label{sec:c1}
\subfile{components/studio-c1.tex}

\newpage
\section{Capitolato C2 --- Etherless}%
\label{sec:c2}
\subfile{components/studio-c2.tex}

\newpage
\section{Capitolato C3 ---  NaturalAPI}%
\label{sec:c3}
\subfile{components/studio-c3.tex}

\newpage
\section{Capitolato C5 --- Stalker}%
\label{sec:c5}
\subfile{components/studio-c5.tex}

\newpage
\section{Capitolato C6 --- ThiReMa - Things Relationship Management}%
\label{sec:c6}
\subfile{components/studio-c6.tex}

\end{document}
