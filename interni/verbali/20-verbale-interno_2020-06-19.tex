\documentclass{article}

\input{../../commons/config}
%Qui ci andrà il percorso delle immagini da includere in analisi dei requisiti
\appendToGraphicspath{../../commons/img/}

\setTitle{Verbale interno --- 2020-06-19}

\setVersione{v3.10-1.0.0}

\setResponsabile{Sofia Bononi}

\setRedattori{Enrico Galdeman}

\setVerificatori{Ian Nicolas Di Menna}

\setUso{Interno}

\setDescrizione{Verbale della chiamata di \emph{CoffeeCode} del 2020-06-19.}

\setModifiche{
  v3.10-1.0.0 & Sofia Bononi & Responsabile & 2020-06-19 & Approvazione del documento \\
  v3.10-0.0.1 & Ian Nicolas di Menna & Verificatore & 2020-06-19 & Verifica del documento \\
v3.10-0.0.1 & Enrico Galdeman & Redattore & 2020-06-19 & Redazione del documento \\
v3.10-0.0.1 & Enrico Galdeman & Redattore & 2020-06-19 & Creazione del documento
}


%ADDMELATER
\disabilitaElencoFigure{}
\disabilitaElencoTabelle{}

\begin{document}

\pagenumbering{gobble}

\begin{titlepage}% per non stampare il numero della pagina

  \raggedleft% allinea a destra la pagina
  \rule{1pt}{\textheight}% linea verticale
  \hspace{0.05\textwidth}% spazio tra linea e testo
  % lasciare questa riga per il corretto funziomento di \parbox
  \parbox[b]{0.75\textwidth}{% paragrafo che tiene il testo a destra della riga cambiando la larghezza il testo si muove a destra o a sinistra
  {\hspace{0.15\textwidth}\includegraphics[width=3.5cm,height=3.5cm]{logo.jpg}}\\[3\baselineskip] % logo
  {\Huge\bfseries CoffeeCode \\[1\baselineskip] Predire in Grafana}\\[4\baselineskip] % titolo
  {\Large\textsc{\placeholderTitle{}}}\\[4\baselineskip] % nome del documento
  {\begin{tabular}{r l}
    % testo in grassetto
    \textbf{Versione}     & \versione{}               \\
    \textbf{Approvazione} & \responsabile{}           \\
    \textbf{Redazione}    & \redattori{}              \\
    \textbf{Verifica}     & \verificatori{}           \\
    \textbf{Uso}          & \uso{}                    \\
    \textbf{Destinato a}  & CoffeeCode                \\
                          & prof.\ Vardanega Tullio   \\
                          & prof.\ Cardin Riccardo    \\
    \ifthenelse{\equal{\uso}{Esterno}}{
                          & Zucchetti S.p.A.       \\
    }{}
  \end{tabular}}\\[1\baselineskip]

  {\bfseries Descrizione}\\
  {\descrizione{}}\\[1\baselineskip]
  {\texttt{coffeecodeswe@gmail.com}}\\[\baselineskip] % email
  }

\end{titlepage}

\newgeometry{textheight=660pt, lmargin=2cm, tmargin=2cm, rmargin=2cm}

% setup di header e footer nelle pagine senza numero
\fancypagestyle{nopage}{%
  \fancyhf{}%
  \fancyhead[L]{\includegraphics[width=1.3cm]{logo.jpg}}%
  \fancyhead[R]{\emph{CoffeeCode}\\\placeholderTitle{}}%
}
% setup di header e footer nelle pagine col numero
\fancypagestyle{usual}{%
  \fancyhf{}%
  \fancyhead[L]{\includegraphics[width=1.3cm]{logo.jpg}}%
  \fancyhead[R]{\emph{CoffeeCode}\\\placeholderTitle{}}%
  \fancyfoot[R]{\thepage\ di~\pageref{LastPage}}%
}
\setlength{\headheight}{1.8cm}

\newpage
\pagestyle{nopage}

\setcounter{table}{-1}

\section*{Registro delle modifiche}%
\label{sec:registro_delle_modifiche}

\rowcolors{2}{white!80!lightgray!90}{white}
\renewcommand{\arraystretch}{2} % allarga le righe con dello spazio sotto e sopra
\begin{longtable}[H]{>{\centering\bfseries}m{2cm} >{\centering}m{3.5cm} >{\centering}m{2.5cm} >{\centering}m{3cm} >{\centering\arraybackslash}m{5cm}}
  \rowcolor{lightgray}
  {\textbf{Versione}} & {\textbf{Nominativo}} & {\textbf{Ruolo}} & {\textbf{Data}} & {\textbf{Descrizione}}  \\
  \endfirsthead%
  \rowcolor{lightgray}
  {\textbf{Versione}} & {\textbf{Nominativo}}  & {\textbf{Ruolo}} & {\textbf{Data}} & {\textbf{Descrizione}}  \\
  \endhead%
  \modifiche{}%
\end{longtable}
% section registro_delle_modifiche (end)

\newpage
\thispagestyle{nopage}
\pagenumbering{roman}
\tableofcontents

\elencoFigure{}%

\elencoTabelle{}%

\newpage

\pagestyle{usual}
\pagenumbering{arabic}


\section{Informazioni generali}%
\label{sec:informazioni_generali}

\subsection{Informazioni incontro}%
\label{sub:informazioni_incontro}

\begin{description}
  \item[Modalità] Applicazione mobile \glossario{Discord}
  \item[Data] 2020-06-19
  \item[Ora inizio] 11:00
  \item[Ora fine] 12:00
\end{description}

\subsection{Partecipanti}%
\label{sub:partecipanti}

\begin{enumerate}
  \item Sofia Bononi
  \item Enrico Buratto
  \item Ian Nicolas Di Menna
  \item Alessandro Franchin
  \item Enrico Galdeman
  \item Nicholas Miazzo
  \item Marco Nardelotto
\end{enumerate}

\section{Ordine del Giorno}%
\label{ordine_del_giorno}
Di seguito sono riportati i punti dell'ordine del giorno che sono stati discussi durante la riunione.
\begin{enumerate}
  \item Discussione sull'andamento della presentazione per la RQ;
  \item Discussioni sui prossimi impegni per completare il prodotto
\end{enumerate}

\section{Resoconto}%
\label{resoconto}
\paragraph*{Discussione sull'andamento della presentazione per la RQ}
Il gruppo si dice soddisfatto dell'andamento della presentazione per la RQ. L'unica pecca evidenziata è stata la breve durata della presentazione; per quanto 
completa, infatti, era già stato segnalato in sede di RP che una breve durata della presentazione può indicare scarsità di argomenti.

\paragraph*{Discussioni sui prossimi impegni per completare il prodotto}
Il gruppo si è confrontato sul da farsi per completare il prodotto e migliorarne la qualità. I componenti hanno deciso di dare particolare importanza alla correzione 
dei diagrammi contenuti nel Manuale dello sviluppatore, poiché opinione comune che una buona progettazione semplfichi molto l'attività di codifica. \\
Il gruppo ha inoltre constatato la presenza di \textit{bug} nel plug-in, e si impegna a correggerli al più presto.

\section{Tracciamento delle decisioni}
\begin{table}[H]
  \centering
  \begin{tabular}{p{4cm}|p{12cm}}
    \rowcolor{lightgray}
    \textbf{Codice}  & \textbf{Descrizione}      \\
    VI-2020-06-19\_20.1 & Il gruppo si impegna a provare maggiormente la presentazione per la RA, al fine di segnalare buona preparazione e discreta quantità di argomenti \\
    VI-2020-06-19\_20.2 & Il gruppo si impegna a correggere quanto prima i diagrammi contenuti nel Manuale dello sviluppatore \\
    VI-2020-06-19\_20.3 & Il gruppo ha diviso i compiti per la risoluzione dei bug presenti nel plug-in di Grafana.
  \end{tabular}
\end{table}

\end{document}
