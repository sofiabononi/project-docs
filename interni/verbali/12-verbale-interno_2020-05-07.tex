\documentclass{article}

\usepackage[italian]{babel}
\usepackage[margin=20mm, footskip = 20pt]{geometry}
\usepackage{graphicx}
\usepackage{subfiles}
\usepackage{hyperref}
\usepackage{nameref}
\usepackage{titlesec}
\usepackage{longtable}
\usepackage[table]{xcolor}
\usepackage{titling}
\usepackage{lastpage}
\usepackage{ifthen}
\usepackage{calc}
\usepackage{soulutf8}
\usepackage{contour}
\usepackage{float}
\usepackage{fancyhdr}
\usepackage{multirow}
\usepackage{pgfgantt}

% definizione dei percorsi in cui cercare immagini
\graphicspath{ {./}
    {./img/}
}

% setup della sottolineatura
\setuldepth{Flat}
\contourlength{0.8pt}

\newcommand{\uline}[1]{%
  \ul{{\phantom{#1}}}%
  \llap{\contour{white}{#1}}%
}

% setup dei link
\hypersetup{
  % set true if you want colored links (instead of boxes)
  colorlinks=true,
  % set to all if you want both sections and subsections linked
  linktoc=all,
  % set color for file links
  filecolor=blue,
  % set color for internal links
  linkcolor=black,
  % set url color
  urlcolor=blue,
  % set characters encoding in the bookmarks tab
  pdfencoding=unicode,
}

% setup forma \paragraph e \subparagraph
\titleformat{\paragraph}[hang]{\normalfont\normalsize\bfseries}{\theparagraph}{1em}{}
\titleformat{\subparagraph}[hang]{\normalfont\normalsize\bfseries}{\thesubparagraph}{1em}{}

% setup profondità indice di default
\setcounter{secnumdepth}{5}
\setcounter{tocdepth}{5}

\makeatletter %% non togliere, i comandi che definiscono i placeholder vanno qui
% esempio di utilizzo: \appendToGraphicspath{./img/} (un comando diverso per ogni path da includere)
% N.B.: ci DEVE essere un forward slash alla fine del path, a indicare che è una cartella.
\newcommand\appendToGraphicspath[1]{%
  \g@addto@macro\Ginput@path{{#1}}%
}

\newcommand{\setTitle}[1]{%
  \newcommand{\@placeholderTitle}{#1}%
}
\newcommand{\placeholderTitle}{\@placeholderTitle}

\newcommand{\setUso}[1]{%
  \newcommand{\@uso}{#1}%
}
\newcommand{\uso}{\@uso}

\newcommand{\setVersione}[1]{%
  \newcommand{\@versione}{#1}%
}
\newcommand{\versione}{\@versione}

\newcommand{\disabilitaVersione}{%
  \renewcommand{\setVersione}[1]{}%
  \renewcommand{\versione}{DISABILITATA}
}

\newcommand{\setResponsabile}[1]{%
  \newcommand{\@responsabile}{#1}%
}
\newcommand{\responsabile}{\@responsabile}

\newcommand{\setRedattori}[1]{%
  \newcommand{\@redattori}{#1}%
}
\newcommand{\redattori}{\@redattori}

\newcommand{\setVerificatori}[1]{%
  \newcommand{\@verificatori}{#1}%
}
\newcommand{\verificatori}{\@verificatori}

\newcommand{\setDescrizione}[1]{%
  \newcommand{\@descrizione}{#1}%
}
\newcommand{\descrizione}{\@descrizione}

\newcommand{\setModifiche}[1]{%
  \newcommand{\@modifiche}{#1}%
}

\newcommand{\modifiche}{\@modifiche}
\makeatother %% non togliere, i comandi che definiscono i placeholder vanno qui

% hook per lo script che genera il glossario
\newcommand{\glossario}[1]{\underline{#1}\textsubscript{g}}

% comandi per rendere opzionali gli elenchi di figure
\newcommand{\elencoFigure}{%
  \renewcommand{\listfigurename}{Elenco delle figure}%
  \listoffigures%
}

\newcommand{\disabilitaElencoFigure}{%
  \renewcommand{\elencoFigure}{}%
}

% comandi per rendere opzionali le tabelle
\newcommand{\elencoTabelle}{%
  \renewcommand{\listtablename}{Elenco delle tabelle}%
  \listoftables%
}

\newcommand{\disabilitaElencoTabelle}{%
  \renewcommand{\elencoTabelle}{}%
}

%Qui ci andrà il percorso delle immagini da includere in analisi dei requisiti
\appendToGraphicspath{../../commons/img/}

\setTitle{Verbale interno --- 2020-05-07}

\setVersione{v1.4-1.0.0}

\setResponsabile{Enrico Galdeman}

\setRedattori{Sofia Bononi}

\setVerificatori{Enrico Buratto}

\setUso{Interno}

\setDescrizione{Verbale della chiamata di \emph{CoffeeCode} del 2020-05-07.}

\setModifiche{v1.4-1.0.0 & Enrico Buratto e Enrico Galdeman & Verificatore e responsabile & 2020-05-07 & Verifica e approvazione del documento \\
v1.4-0.0.1 & Sofia Bononi & Redattore & 2020-05-07 & Redazione del documento \\
v1.4-0.0.1 & Sofia Bononi & Redattore & 2020-05-07 & Creazione del documento}


%ADDMELATER
\disabilitaElencoFigure{}
\disabilitaElencoTabelle{}

\begin{document}

\pagenumbering{gobble}

\begin{titlepage}% per non stampare il numero della pagina

  \raggedleft% allinea a destra la pagina
  \rule{1pt}{\textheight}% linea verticale
  \hspace{0.05\textwidth}% spazio tra linea e testo
  % lasciare questa riga per il corretto funziomento di \parbox
  \parbox[b]{0.75\textwidth}{% paragrafo che tiene il testo a destra della riga cambiando la larghezza il testo si muove a destra o a sinistra
  {\hspace{0.15\textwidth}\includegraphics[width=3cm,height=3cm]{logo.jpg}}\\[2\baselineskip] % logo
  {\Huge\bfseries CoffeeCode \\[0.5\baselineskip] Predire in Grafana}\\[5\baselineskip] % titolo
  {\Large\textsc{\placeholderTitle{}}}\\[6\baselineskip] % nome del documento
  {\begin{tabular}{r l}
    % testo in grassetto
    \textbf{Versione}     & \versione{}               \\
    \textbf{Approvazione} & \responsabile{}           \\
    \textbf{Redazione}    & \redattori{}              \\
    \textbf{Verifica}     & \verificatori{}           \\
    \textbf{Uso}          & \uso{}                    \\
    \textbf{Destinato a}  & CoffeeCode                \\
                          & prof.\ Vardanega Tullio   \\
                          & prof.\ Cardin Riccardo    \\
    \ifthenelse{\equal{\uso}{Esterno}}{
                          & Zucchetti Group SPA       \\
    }{}
  \end{tabular}}\\[5\baselineskip]

  {\bfseries Descrizione}\\
  {\descrizione{}}\\[2\baselineskip]
  {\texttt{coffeecodeswe@gmail.com}}\\[\baselineskip] % email
  }

\end{titlepage}


\section{Informazioni generali}%
\label{sec:informazioni_generali}

\subsection{Informazioni incontro}%
\label{sub:informazioni_incontro}

\begin{description}
  \item[Modalità] Applicazione mobile \glossario{Discord}
  \item[Data] 2020-05-07
  \item[Ora inizio] 10:00
  \item[Ora fine] 11:30
\end{description}

\subsection{Partecipanti}%
\label{sub:partecipanti}

\begin{enumerate}
  \item Sofia Bononi
  \item Enrico Buratto
  \item Ian Nicolas Di Menna
  \item Alessandro Franchin
  \item Enrico Galdeman
  \item Nicholas Miazzo
  \item Marco Nardelotto
\end{enumerate}

\section{Ordine del Giorno}%
\label{ordine_del_giorno}
Di seguito sono riportati i punti dell'ordine del giorno che sono stati discussi durante la riunione.
\begin{enumerate}
  \item Feedback su quanto sviluppato per il programma di addestramento
  \item Preparazione della presentazione per la Technology Baseline
  \item Discussioni sul versionamento
\end{enumerate}

\section{Resoconto}%
\label{resoconto}
\paragraph*{Feedback su quanto sviluppato per il programma di addestramento}
Il gruppo si dice contento di quanto finora sviluppato per il Proof of Concept; il programma è ora completo per quanto riguarda quello che il gruppo si era prefissato di preparare per la Technology Baseline.

\paragraph*{Preparazione della presentazione per la Technology Baseline}
Il gruppo ha creato un nuovo file su Google Drive contenente la presentazione per la Technology Baseline. È stata abbozzata la struttura e il gruppo si è diviso le slides da preparare.

\paragraph*{Discussioni sul versionamento}
In seguito alle segnalazioni del Prof. Tullio Vardanega riguardanti l'errato utilizzo del versionamento, il gruppo si è confrontato sul vero significato di questo. È stato quindi deciso un nuovo metodo di versionamento, ed è stata preparata un'email da inviare al professore per chiedere conferma della correttezza di questo.

\section{Tracciamento delle decisioni}
\begin{table}[H]
  \centering
  \begin{tabular}{p{4cm}|p{12cm}}
    \rowcolor{lightgray}
    \textbf{Codice}  & \textbf{Descrizione}      \\
     VI-2020-05-07\_12.1 & Il gruppo ha deciso di utilizzare un diverso metodo di versionamento e di scrivere al professore per chiedere conferma della correttezza. Il nuovo metodo di versionamento è consultabile tra le mail inviate al professore. \\
  \end{tabular}
\end{table}

\end{document}
