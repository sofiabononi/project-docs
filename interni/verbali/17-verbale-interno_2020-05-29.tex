\documentclass{article}

\input{../../commons/config}
%Qui ci andrà il percorso delle immagini da includere in analisi dei requisiti
\appendToGraphicspath{../../commons/img/}

\setTitle{Verbale interno --- 2020-05-29}

\setVersione{v1.7-1.0.0}

\setResponsabile{Nicholas Miazzo}

\setRedattori{Enrico Buratto}

\setVerificatori{Alessandro Franchin}

\setUso{Interno}

\setDescrizione{Verbale della chiamata di \emph{CoffeeCode} del 2020-05-29.}

\setModifiche{v1.7-1.0.0 & Alessandro Franchin e Nicholas Miazzo & Verificatore e responsabile & 2020-05-29 & Verifica e approvazione del documento \\
v1.7-0.0.1 & Enrico Buratto & Redattore & 2020-05-29 & Redazione del documento \\
v1.7-0.0.1 & Enrico Buratto & Redattore & 2020-05-29 & Creazione del documento}


%ADDMELATER
\disabilitaElencoFigure{}
\disabilitaElencoTabelle{}

\begin{document}

\pagenumbering{gobble}

\begin{titlepage}% per non stampare il numero della pagina

  \raggedleft% allinea a destra la pagina
  \rule{1pt}{\textheight}% linea verticale
  \hspace{0.05\textwidth}% spazio tra linea e testo
  % lasciare questa riga per il corretto funziomento di \parbox
  \parbox[b]{0.75\textwidth}{% paragrafo che tiene il testo a destra della riga cambiando la larghezza il testo si muove a destra o a sinistra
  {\hspace{0.15\textwidth}\includegraphics[width=3.5cm,height=3.5cm]{logo.jpg}}\\[3\baselineskip] % logo
  {\Huge\bfseries CoffeeCode \\[1\baselineskip] Predire in Grafana}\\[4\baselineskip] % titolo
  {\Large\textsc{\placeholderTitle{}}}\\[4\baselineskip] % nome del documento
  {\begin{tabular}{r l}
    % testo in grassetto
    \textbf{Versione}     & \versione{}               \\
    \textbf{Approvazione} & \responsabile{}           \\
    \textbf{Redazione}    & \redattori{}              \\
    \textbf{Verifica}     & \verificatori{}           \\
    \textbf{Uso}          & \uso{}                    \\
    \textbf{Destinato a}  & CoffeeCode                \\
                          & prof.\ Vardanega Tullio   \\
                          & prof.\ Cardin Riccardo    \\
    \ifthenelse{\equal{\uso}{Esterno}}{
                          & Zucchetti S.p.A.       \\
    }{}
  \end{tabular}}\\[1\baselineskip]

  {\bfseries Descrizione}\\
  {\descrizione{}}\\[1\baselineskip]
  {\texttt{coffeecodeswe@gmail.com}}\\[\baselineskip] % email
  }

\end{titlepage}

\newgeometry{textheight=660pt, lmargin=2cm, tmargin=2cm, rmargin=2cm}

% setup di header e footer nelle pagine senza numero
\fancypagestyle{nopage}{%
  \fancyhf{}%
  \fancyhead[L]{\includegraphics[width=1.3cm]{logo.jpg}}%
  \fancyhead[R]{\emph{CoffeeCode}\\\placeholderTitle{}}%
}
% setup di header e footer nelle pagine col numero
\fancypagestyle{usual}{%
  \fancyhf{}%
  \fancyhead[L]{\includegraphics[width=1.3cm]{logo.jpg}}%
  \fancyhead[R]{\emph{CoffeeCode}\\\placeholderTitle{}}%
  \fancyfoot[R]{\thepage\ di~\pageref{LastPage}}%
}
\setlength{\headheight}{1.8cm}

\newpage
\pagestyle{nopage}

\setcounter{table}{-1}

\section*{Registro delle modifiche}%
\label{sec:registro_delle_modifiche}

\rowcolors{2}{white!80!lightgray!90}{white}
\renewcommand{\arraystretch}{2} % allarga le righe con dello spazio sotto e sopra
\begin{longtable}[H]{>{\centering\bfseries}m{2cm} >{\centering}m{3.5cm} >{\centering}m{2.5cm} >{\centering}m{3cm} >{\centering\arraybackslash}m{5cm}}
  \rowcolor{lightgray}
  {\textbf{Versione}} & {\textbf{Nominativo}} & {\textbf{Ruolo}} & {\textbf{Data}} & {\textbf{Descrizione}}  \\
  \endfirsthead%
  \rowcolor{lightgray}
  {\textbf{Versione}} & {\textbf{Nominativo}}  & {\textbf{Ruolo}} & {\textbf{Data}} & {\textbf{Descrizione}}  \\
  \endhead%
  \modifiche{}%
\end{longtable}
% section registro_delle_modifiche (end)

\newpage
\thispagestyle{nopage}
\pagenumbering{roman}
\tableofcontents

\elencoFigure{}%

\elencoTabelle{}%

\newpage

\pagestyle{usual}
\pagenumbering{arabic}


\section{Informazioni generali}%
\label{sec:informazioni_generali}

\subsection{Informazioni incontro}%
\label{sub:informazioni_incontro}

\begin{description}
  \item[Modalità] Applicazione mobile \glossario{Discord}
  \item[Data] 2020-05-29
  \item[Ora inizio] 10:00
  \item[Ora fine] 11:00
\end{description}

\subsection{Partecipanti}%
\label{sub:partecipanti}

\begin{enumerate}
  \item Sofia Bononi
  \item Enrico Buratto
  \item Ian Nicolas Di Menna
  \item Alessandro Franchin
  \item Enrico Galdeman
  \item Nicholas Miazzo
  \item Marco Nardelotto
\end{enumerate}

\section{Ordine del Giorno}%
\label{ordine_del_giorno}
Di seguito sono riportati i punti dell'ordine del giorno che sono stati discussi durante la riunione.
\begin{enumerate}
  \item Decisione giorno colloquio product baseline;
  \item Discussione risultati RP;
  \item Discussione implementazione gestione errori.
\end{enumerate}

\section{Resoconto}%
\label{resoconto}
\paragraph*{Decisione giorno colloquio Product Baseline}
Il gruppo ha discusso quale fosse il giorno migliore per fissare il colloquio per la product baseline.
\paragraph*{Risultati RP}
Il gruppo ha discusso le segnalazioni sui documenti presentati in RP e le eventuali modifiche e aggiunte da apportare.
\paragraph*{Implementazione gestione errori}
Il gruppo ha discusso quale fosse il modo migliore per gestire eventuali errori che potrebbero sorgere durante l'esecuzione della training app o del plug-in.

\section{Tracciamento delle decisioni}
\begin{table}[H]
  \centering
  \begin{tabular}{p{4cm}|p{12cm}}
    \rowcolor{lightgray}
    \textbf{Codice}  & \textbf{Descrizione}      \\
    VI-2020-05-29\_17.1 & Il gruppo ha deciso di inviare una mail al Prof. Cardin per fissare il colloquio il giorno 2020-06-09 \\
  \end{tabular}
\end{table}

\end{document}
