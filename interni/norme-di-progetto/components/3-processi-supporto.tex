\documentclass[../norme-di-progetto.tex]{subfiles}
\begin{document}
\subsection{Documentazione}
\subsubsection{Scopo}
Lo scopo di questo processo è quello di fornire i dettagli e le regole su cui deve essere basata la redazione e la manutenzione di tutta la documentazione durante l'intero ciclo di vita del software.
\subsubsection{Aspettative}
Le aspettative della corretta implementazione di questo processo sono:
\begin{itemize}
  \item L'identificazione di regole su cui deve essere basata la corretta stesura dei documenti;
  \item L'individuazione di una struttura rigorosa sulla quale basare la redazione dei documenti durante tutti i processi di sviluppo.
\end{itemize}
\subsubsection{Descrizione}
Questa sezione descrive i dettagli su come deve essere redatta e verificata la documentazione. Al fine di ottenere l'uniformità e il rigore espositivo di tutti i documenti prodotti durante il ciclo di vita del software, ogni documento deve basarsi sulle regole descritte in questa sede.
\subsubsection{Procedure}
\paragraph{Ciclo di vita dei documenti}
Il documento viene innanzitutto creato; a tale scopo, il gruppo ha deciso di creare un template \LaTeX sul quale basare la creazione e la stesura di ogni documento. Questo assicura l'uniformità della stesura dei documenti, facilitando il versionamento. \\
Dopo la creazione, il ciclo di vita dei documenti è suddiviso in tre differenti attività:
\begin{itemize}
  \item \textbf{Stesura del documento}: la stesura del documento consiste nella scrittura del documento stesso. Questa attività viene svolta da un redattore, assegnato dal responsabile di progetto; dopo la stesura, il responsabile autorizzerà all'avanzamento della stesura, rimanendo in attesa dell'esito della revisione;
  \item \textbf{Revisione}: dopo la stesura, il documento verrà revisionato dai verificatori, che ne controlleranno l'aderenza alle norme di progetto e la correttezza sintattica e semantica. Dopo questo controllo, a seconda che l'esito sia positivo o negativo, lo comunicheranno al responsabile, che farà avanzare il documento all'approvazione finale, o ai  redattori, che correggeranno eventuali segnalazioni e lo riproporranno ai verificatori;
  \item \textbf{Approvazione finale}: con l'approvazione ricevuta da parte dei verificatori, il responsabile di progetto confermerà quindi il documento, eseguendone il rilascio.
\end{itemize}



\paragraph{Struttura dei documenti}
Al fine di organizzare coerentemente e uniformamente la struttura dei documenti, si è deciso di utilizzare il \textit{package} \LaTeX \textbf{subfiles}. La repository contiene una \textit{directory} denominata \texttt{commons/} contenente:
\begin{itemize}
  \item Il file \texttt{config.tex}, contenente le definizioni dei comandi \LaTeX;
  \item Il file \texttt{template.tex}, contenente il \textit{template} su cui si basano tutti i documenti;
  \item La cartella \texttt{img/}, contenente le immagini comuni a tutti i documenti.
\end{itemize}
Per ogni documento è quindi presente una \textit{directory} contenente:
\begin{itemize}
  \item  Il file principale, chiamato con il nome del documento, in formato \texttt{.tex};
  \item La cartella \texttt{components/}, contenente gli altri file in formato \texttt{.tex} che vengono inclusi nel file principale.
\end{itemize}
Basandosi su un \textit{template}, ogni documento ha una struttura fissa.

\subparagraph{Frontespizio}
Il frontespizio fornisce tutti i dati principali del documento. Esso presenta il logo del gruppo, centrato, seguito dal nome del gruppo e dal titolo del progetto; sotto di questo è presente il titolo del documento. \\
Sono fornite anche altre informazioni essenziali, le quali sono:
\begin{itemize}
  \item \textbf{Versione}: versione attuale del documento;
  \item \textbf{Approvazione}: stato di approvazione del documento, che può essere "In redazione" o recare il nome del responsabile che ha approvato il documento;
  \item \textbf{Redazione}: redattori del documento;
  \item \textbf{Verifica}: verificatori del documento;
  \item \textbf{Uso}: destinazione del documento, che può essere "Interno" o "Esterno";
  \item \textbf{Destinato a}: destinazione specifica del documento.
\end{itemize}
\subparagraph{Registro delle modifiche}
Nella seconda pagina di ogni documento è presente il registro delle modifiche del documento, consistente in una tabella contentente, per ogni modifica:
\begin{itemize}
  \item \textbf{Versione}: versione del documento aggiornata alla modifica effettuata;
  \item \textbf{Nominativo}: nome dell'autore della modifica;
  \item \textbf{Ruolo}: ruolo dell'autore della modifica;
  \item \textbf{Data}: data in cui è stata effettuata la modifica;
  \item \textbf{Descrizione}: descrizione della modifica effettuata.
\end{itemize}
\subparagraph{Indice}
L'indice riepiloga la struttura del documento. Ha una struttura standard: la numerazione ha struttura \\ \centerline{\textbf{a.b.c.d.e}} \\ dove:
\begin{itemize}
  \item \textbf{a} è il capitolo principale;
  \item \textbf{b} è la sezione del capitolo;
  \item \textbf{c} è la sottosezione della sezione;
  \item \textbf{d} è il paragrafo della sezione
  \item \textbf{e} è il sottoparagrafo del paragrafo.
\end{itemize}
La massima profondità della numerazione è quindi 5; nel caso in cui si rendesse necessario aggiungere un livello, questo verrà indicato come sottoparagrafo del sottoparagrafo senza ulteriore numerazione.

\subparagraph{Elenco delle figure}
L'elenco delle figure consiste nell'indice di tutte le figure presenti nel documento, eccezion fatta per il logo del gruppo che non viene riportato.

\subparagraph{Elenco delle tabelle}
L'elenco delle tabelle indicizza tutte le tabelle presenti nel documento, compresa quella del Registro delle modifiche.

\subparagraph{Contenuto principale}
Ogni pagina del contenuto principale del documento è composta da:
\begin{itemize}
  \item Il logo del gruppo, posizionato in alto a sinistra;
  \item Il nome del gruppo, posizionato in alto a destra;
  \item Il nome del documento, posizionato in alto a destra sotto al nome del gruppo;
  \item Una riga orizzontale, posizionata in alto e che attraversa tutta la pagina, che divide logo, nome del gruppo e nome del documento dal contenuto della pagina;
  \item Il numero di pagina, posizionato in basso a destra e nel formato \\ \centerline{\textbf{N di T}} \\ dove
\begin{itemize}
  \item \textbf{N} è il numero della pagina corrente;
  \item \textbf{T} è il numero di pagine totali.
\end{itemize}
Il conteggio delle pagine inizia dalla prima pagina dopo l'Elenco delle tabelle.
\end{itemize}
Nello spazio compreso tra la riga orizzontale e il numero di pagina si articola il contenuto proprio della pagina.

\paragraph{Classi di documenti}
I documenti prodotti appartengono a diverse categorie, secondo cui sono classificati.

\subparagraph{Documenti ufficiali}
I documenti ufficiali sono quelli che seguono la struttura elencata in questo documento, e che sono stati verificati e approvati dal responsabile di progetto. Essi possono essere Interni o Esterni, a seconda della destinazione che hanno.

\subparagraph{Glossario}
In accompagnamento ai documenti ufficiali viene fornito un Glossario, contenente tutti i termini caratterizzati da sottolineatura e una G maiuscola come pedice. Il Glossario ha la funzione di chiarire possibili fraintendimenti provenienti dall'utilizzo di determinati termini. \\
Perché un termine venga incluso nel Glossario deve soddisfare almeno uno tra i seguenti requisiti:
\begin{itemize}
  \item Essere un termine tecnico. Rientrano in questo criterio i nomi propri delle tecnologie utilizzate, i termini propri dell'Ingegneria del Software,
  \item Essere presente in tutti i documenti;
  \item Avere più di un significato plausibile nel contesto in cui viene utilizzato;
  \item Essere un acronimo specifico e raramente usato.
\end{itemize}
Il Glossario possiede un'impostazione grafica identica agli altri documenti, con l'eccezione di non possedere numerazione nell'indice. I termini sono ordinati in ordine alfabetico, e divisi in base alla prima lettera; ogni lettera deve avere la sua sezione, anche nel caso in cui nessun termine vi appartenga. \\
I termini appartenenti al Glossario sono contrassegnati come tali solo una volta per ogni documento, nello specifico la prima volta che il termine compare in esso.

\subparagraph{Verbali}
I verbali documentano gli argomenti discussi in una riunione; essi possono essere interni, e quindi riferirsi a una riunione tra soli membri del gruppo, o eserni, e quindi riferirsi a una riunione a cui partecipa anche l'azienda. La struttura del documento è identica a quella dei documenti ufficiali; ogni verbale è composto dai seguenti componenti:
\begin{itemize}
  \item \textbf{Informazioni generali}: informazioni dell'incontro, consistenti nella modalità, nella data e negli orari di inizio e fine, e i partecipanti all'incontro;
  \item \textbf{Ordine del giorno}: lista degli argomenti e dei temi di cui il gruppo ha deciso di discutere durante l'incontro;
  \item \textbf{Resoconto}: riassunto di quanto detto durante l'incontro punto per punto.
\end{itemize}

\subparagraph{Lettere}
Eventuali lettere, tra cui quella di presentazione pubblicata, seguono il formato classico di una lettera: esse includono quindi uno o più mittenti e uno o più destinatari, il logo e il nome del gruppo. \\
La lettera di presentazione contiene anche l'elenco dei documenti rilasciati e il preventivo per lo svolgimento del progetto.

\paragraph{Norme tipografiche}
\subparagraph{Nomi dei file}
La nomenclatura di tutti i file segue la convenzione \textit{kebab case}, nota anche come \textit{lisp-case}. Le regole di questa convenzione sono le seguenti:
\begin{itemize}
  \item Ogni file inizia con una lettera minuscola;
  \item Tra ogni parola viene inserito come separatore il tratto d'unione \textbf{-};
  \item In caso di presenza di apostrofo, esso verrà sostituito dal tratto d'unione;
  \item Le lettere accentate sono abolite, sostituite dalla semplice lettera senza accento.
\end{itemize}
L'unica eccezione che è stata adottata è l'utilizzo del segno di interpunzione "trattino basso" \textbf{\_} tra il nome del documento e la data di redazione nei nomi dei verbali e tra la data e la versione. \\
Tutti i nomi di file avranno quindi il seguente formato: \\ \centerline{\textbf{a-b-c[\_YYYY-MM-DD]\_vX.Y.Z.ext}} \\
dove:
\begin{itemize}
  \item \textbf{[ ]} si applicano solo al caso dei verbali;
  \item \textbf{a}, \textbf{b} e \textbf{c} sono parole singole. Il numero di parole singole non è limitato a tre ma si può aumentare o diminuire. Nel caso specifico dei verbali, la forma di questa porzione di nome è \\ \centerline{\textbf{verbale-[interno/esterno]}};
  \item \textbf{vX.Y.Z} indicano la versione attuale del documento;
  \item \textbf{ext} è l'estensione del file.
\end{itemize}

\subparagraph{Stile del testo}
Ogni diversa formattazione usata all'interno dei documenti ha un significato preciso, e ogni documento deve obbligatoriamente seguire questo stile. Gli stili sono:
\begin{itemize}
  \item Utilizzo di testo in corsivo per i termini in lingua diversa da quella italiana;
  \item Utilizzo di testo in caratteri dattilografici per i nomi di file e delle \textit{directories}; queste ultime devono finire con il carattere barra \textbf{/};
  \item Utilizzo di testo in grassetto per i termini conseguentemente definiti e per i titoli dei sottoparagrafi di sottoparagrafi;
  \item Utilizzo di testo sottolineato, in aggiunta al pedice G maiusola, per i termini appartenenti al Glossario;
  \item Utilizzo di testo maiuscoletto per i nomi dei documenti all'interno dei documenti stessi.
\end{itemize}

\subparagraph{Figure}
L'utilizzo di figure all'interno dei documenti è permesso; le immagini devono essere centrate nella pagina, avere larghezza o lunghezza fissa di 10cm ed essere seguite dalla seguente dicitura: \\ \centerline{\textbf{Figura N: D}} \\ dove:
\begin{itemize}
  \item \textbf{N}: numero progressivo della figura all'interno del documento in cui è inserita;
  \item \textbf{D}: breve descrizione dell'immagine.
\end{itemize}
Il logo del gruppo non segue queste regole, poiché non possiede né numero progressivo né descrizione.

\subparagraph{Tabelle}
Le tabelle devono essere centrate nella pagina, ed essere seguite dalla seguente dicitura: \\ \centerline{\textbf{Tabella N: T}} \\ dove:
\begin{itemize}
  \item \textbf{N}: numero progressivo della tabella all'interno del documento in cui è inserita;
  \item \textbf{T}: titolo della tabella.
\end{itemize}
La tabella del Registro delle modifiche non segue queste regole, poiché non possiede né numero progressivo né titolo.

\subparagraph{Formati comuni}
I formati di data, ora e versione sono determinati e univoci per ogni documento. Nello specifico:
\begin{itemize}
  \item La data segue lo standard ISO 8601, e quindi è nel formato \\ \centerline{\textbf{YYYY-MM-DD}} \\ dove:
  \begin{itemize}
    \item \textbf{YYYY}: l'anno, scritto per intero;
    \item \textbf{MM}: il mese, scritto per intero;
    \item \textbf{DD}: il giorno, scritto per intero
  \end{itemize};
  \item L'ora è nel formato \\ \centerline{\textbf{HH:MM}} \\ dove:
  \begin{itemize}
    \item \textbf{HH}: l'ora in formato 24 ore;
    \item \textbf{MM}: i minuti
  \end{itemize};
  \item La versione è nel formato DA COMPLETARE - SEMVER? DIPENDE DA PDP
\end{itemize}

\subparagraph{Sigle}
All'interno della documentazione possono essere utilizzate delle sigle. Esse sono:
\begin{itemize}
  \item \textbf{Sigle dei documenti}:
  \begin{itemize}
    \item \textbf{NP}: Norme di Progetto;
    \item \textbf{SF}: Studio di Fattibilità;
    \item \textbf{AR}: Analisi dei Requisiti;
    \item \textbf{PP}: Piano di Progetto;
    \item \textbf{PQ}: Piano di Qualifica;
    \item \textbf{Gl}: Glossario;
    EVENTUALMENTE ALTRE? MANUALI?
  \end{itemize}
  \item \textbf{Fasi del progetto}:
  \begin{itemize}
  \item \textbf{RR}: Revisione dei Requisiti;
  \item \textbf{RP}: Revisione di Progettazione;
  \item \textbf{RQ}: Revisione di Qualifica;
  \item \textbf{RA}: Revisione di Accettazione.
  \end{itemize}
  \item \textbf{Ruoli}:
  \begin{itemize}
  \item \textbf{RdP}: Responsabile di Progetto;
  \item \textbf{AdP}: Amministratore di Progetto;
  \item \textbf{An}: Analista;
  \item \textbf{Pr}: Progettista;
  \item \textbf{Pg}: Programmatore;
  \item \textbf{Ve}: Verificatore.
  \end{itemize}
  \item \textbf{Altre sigle:}
  \begin{itemize}
    \item \textbf{CLI}: \glossario{Command Line};
    \item \textbf{IDE}: \glossario{Integrated Developement Environment};
    \item EVENTUALI ALTRE SIGLE
  \end{itemize}
\end{itemize}

\subparagraph{Altre norme tipografiche}
\subparagraph*{Elenchi}
Gli elenchi presenti nella documentazione sono unicamente di tipo puntato, indicati cioè con un cerchietto pieno, o con un trattino (lo stesso carattere del tratto d'unione) nel caso in cui si tratti di un sotto-elenco di un elenco puntato. \\
Ogni parola appartenente a un elenco puntato deve iniziare con la lettera maiuscola; ogni elemento di un elenco puntato deve finire con un punto e virgola, ad eccezione dell'ultimo che deve finire con un punto.

ALTRE NORME TIPOGRAFICHE EVENTUALI



\subsection{Gestione della configurazione}
\subsubsection{Scopo}
Lo scopo di questo processo è delineare le norme riguardanti la configurazione del progetto;

\subsubsection{Versionamento}
Ogni documento deve essere versionato, al fine di avere una cronologia delle modifiche e dell'importanza di queste sul contenuto del documento; fanno eccezione a questa regola i verbali, i quali per definizione non hanno bisogno di essere modificati dopo la stesura e la verifica, ma vanno direttamente approvati. Ad ogni versione del documento corrisponde una riga nel Registro delle modifiche. \\
Ogni versione viene espressa nel seguente modo: \\ \centerline{\textbf{vX.Y.Z}} \\ dove:
\begin{itemize}
  \item \textbf{X}:
  \item \textbf{Y}:
  \item \textbf{Z}:
\end{itemize}

\subsubsection{Repository}
Una \textit{repository} è un ambiente in cui vengono memorizzati, mantenuti e versionati i file di un progetto software, durante il suo intero ciclo di vita. Il sistema di controllo che il gruppo utilizza è \glossario{Git}
PARLARE DI GIT, GITHUB E GIT COMMITIZEN

\paragraph{Struttura della repository e workflow}
La struttura adottata sfrutta il meccanismo di \textit{branching} offerto dallo strumento Git; questo permette lo sviluppo parallelo di più funzionalità e il lavoro in parallelo sulla stessa \textit{repository}. \\
La \textit{repository} utilizzata si suddivide nei seguenti \textit{branch}:
\begin{itemize}
  \item \texttt{master}: branch principale, sul quale vengono fatti le \glossario{major releases};
  \item \texttt{initial-configuration}: ???
  \item \texttt{other-configurations}: ???
  \item \texttt{norme-di-progetto}: branch secondario nel quale viene sviluppato l'intero documento riguardante le Norme di Progetto;
  \item \texttt{analisi-dei-requisiti}: branch secondario nel quale viene sviluppato l'intero documento riguardante l'Analisi dei Requisiti;
  \item \texttt{glossario}: branch secondario nel quale viene sviluppato l'intero documento Glossario;
  \item \texttt{piano-di-progetto}: branch secondario nel quale viene sviluppato l'intero documento riguardante il Piano di Progetto;
  \item \texttt{piano-di-qualifica}: branch secondario nel quale viene sviluppato l'intero documento riguardante il Piano di Qualifica;
  \item \texttt{studio-di-fattibilita}: branch secondario nel quale viene sviluppato l'intero documento riguardante lo Studio di Fattibilità;
  \item \texttt{verbali}:: branch secondario nel quale vengono sviluppati i verbali interni ed esterni.
\end{itemize}
Il workflow adottato è basato quindi su un principio della separazione dei documenti: questo, unito ai sistemi anti-collisione di git, permette che durante il periodo in cui un membro del gruppo è assegnato a un ruolo di redazione specifico, egli sia l'unico a lavorare su tale documento.

\subparagraph{Struttura dei branch}
Ogni branch è costituito da file, cartelle e sotto-cartelle.
\subparagraph*{File}
I file presenti in ogni branch hanno tipo fisso e predefinito; essi sono:
\begin{itemize}
  \item File di configurazione di Github:
  \begin{itemize}
    \item \texttt{.gitignore}: contiene la lista di file e directory spuri che possono essere presenti nelle \textit{repository} locali ma non nella \textit{repository} remota;
    \item File \texttt{.yml}: contengono le istruzioni per l'esecuzione delle Github Actions;
    \item \texttt{README.md}: file markdown contenente una breve descrizione della \textit{repository}.
  \end{itemize}
  \item File di configurazione di Git Commitizen:
  \begin{itemize}
    \item \texttt{.czcr}: contiene le impostazioni, definite dal gruppo, di Git Commitizen.
  \end{itemize}
  \item File di configurazione di glossario{Visual Studio Code} (cfr. PUNTO IN CUI SI PARLA DI VS CODE)
  \begin{itemize}
    \item \texttt{.editorconfig}: contiene le impostazioni, definite dal gruppo, del Workspace di Visual Studio Code;
    \item \texttt{settings.json}: contiene le impostazioni specifiche, definite dal gruppo, dell'editor Visual Studio Code;
    \item \texttt{extensions.json}: contiene le estenzioni, consigliate per lo sviluppo del progetto, da includere in Visual Studio Code.
  \end{itemize}
  \item File di documentazione:
  \begin{itemize}
    \item File \texttt{.tex}: file \LaTeX contenenti il codice sorgente per la generazione dei documenti;
    \item File \texttt{.png}: immagini che vengono incluse nei documenti.
  \end{itemize}
\end{itemize}

\subparagraph*{Cartelle}
Ogni branch presenta un insieme di cartelle predefinito:
\begin{itemize}
\item \texttt{.github/}: contiene i file propri di Github, tra cui i file \texttt{.yml} delle Github Actions;
  \item \texttt{.vscode/}: contiene i file di configurazione di Visual Studio Code;
  \item \texttt{commons/}: contiene i file template di \LaTeX;
  \item \texttt{interni/}: contiene i documenti ad uso interno. In ogni branch questa cartella contiene una sottocartella con lo stesso nome del branch, contenente i file sorgente \LaTeX dello specifico documento;
  \item \texttt{esterni/}: contiene i documenti ad uso esterno. In ogni branch questa cartella contiene una sottocartella con lo stesso nome del branch, contenente i file sorgente \LaTeX dello specifico documento;
\end{itemize}

\paragraph{Strumenti}
Le attività di gestione della configurazione sono coadiuvate dall'utilizzo di strumenti software, che permettono il perseguimento del giusto workflow e l'applicazione delle norme.
\subparagraph{Git}
La repository è ospitata dal servizio di \glossario{hosting} Github; per operare sulla repository vengono utilizzati tre strumenti:
\begin{itemize}
  \item Portale Github.com;
  \item Git Commitizen;
  \item Visual Studio Code.
\end{itemize}
\subparagraph*{Portale Github.com}
Il portale Github.com permette la consultazione immediata di quanto fatto e pubblicato, e la gestione delle Issues e delle Milestones.
\begin{figure}[H]
  \centering
  \includegraphics[width=10cm]{img/github.png}
  \label{fig:github}
  \caption{Portale in Github della repository di progetto.}
\end{figure}

\subparagraph*{Git Commitizen}
Per il \glossario{commit} delle modifiche ai file della repository viene utilizzato Git Commitizen attraverso \glossario{Command Line}. Questo assicura l'uniformità dei commit e quindi una maggiore chiarezza di quanto viene versionato.

\subparagraph*{Visual Studio Code}
Visual Studio Code è l'IDE che viene uniformemente usato da ogni componente del gruppo. Questo strumento software permette l'estensione delle funzionalità attraverso l'installazione di componenti aggiuntivi. Tra questi, due in particolare permettono di operare sulla repository:
\begin{itemize}
  \item \textbf{Git Graph}: visualizzazione del grafo della repository e possibilità di svolgere azioni Git direttamente su di esso;
  \item \textbf{GitHub Pull Requests}: strumento di revisione delle Pull Requests di Github.
\end{itemize}
Per una trattazione più esaustiva in merito all'utilizzo di Visual Studio Code si rimanda a CAPITOLO DEI PROCESSI ORGRANIZZATIVI DOVE SI PARLA DI VSCODE

\subparagraph{DevOps} DA DISCUTERNE \\
Github mette a disposizione lo strumento delle Github Actions per lo sviluppo delle operazioni di \glossario{DevOps}. Questo strumento è quindi utilizzato per:
\begin{itemize}
  \item CI
  \item CD
  \item ...
\end{itemize}


\end{document}
