\documentclass[../norme-di-progetto.tex]{subfiles}
\begin{document}
\subsection{Scopo del documento}
Questo documento ha lo scopo di definire il \glossario{way of working} del gruppo \emph{CoffeeCode} per lo svolgimento del \glossario{progetto}.
Le attività descritte in questo documento istanziano i \glossario{processi} descritti nello standard \textbf{ISO/IEC 12207:1995}. Al fine di ottenere l'omogeneità di quanto presentato in questo progetto, ogni membro del gruppo è tenuto a visionare questo documento; ogni cambiamento, aggiunta o rimozione al presente documento deve essere inoltre notificato ad ogni membro del gruppo.

\subsection{Scopo del prodotto}
Il \glossario{capitolato} C4 - "Predire in Grafana" ha come obiettivo la realizzazione di un \glossario{plug-in}, scritto in linguaggio \glossario{Typescript} per lo strumento \glossario{open source} \glossario{Grafana}, che applichi le tecniche di \glossario{Machine Learning} \glossario{Regressione Lineare} e \glossario{Support Vector Machines} a un flusso di dati. Questo plug-in provvederà quindi a svolgere un'analisi dei dati e, da questi, fornire una predizione al fine di monitorare la \glossario{liveliness} del sistema e di consigliare interventi alla linea di produzione del software tramite specifici \glossario{alert}. \\
Prima della realizzazione di tale plug-in si dovrà sviluppare un programma comprendente un'interfaccia web con l'utilizzo del \glossario{framework} \glossario{Electron}, attraverso la quale verrà eseguito l'\glossario{addestramento} necessario al corretto funzionamento degli algoritmi di Machine Learning. \\
I risultati ottenuti dovranno poi essere visualizzati, in forma di grafico, in una \glossario{dashboard} sulla piattaforma Grafana.

\subsection{Glossario}
Al fine di evitare ambiguità relative alle terminologie usate, viene fornito il \textsc{Glossario v1.9-1.3.0}, nel quale sono definiti i termini sottolineati e con una G maiuscola come pedice presenti negli altri documenti.

\subsection{Riferimenti}
\subsubsection{Riferimenti normativi}
\begin{itemize}
  \item \textbf{Capitolato d'appalto C4 - "Predire in Grafana"}: \\ \\ \href{https://www.math.unipd.it/~tullio/IS-1/2019/Progetto/C4.pdf}{https://www.math.unipd.it/\textasciitilde tullio/IS-1/2019/Progetto/C4.pdf}.
\end{itemize}

\subsubsection{Riferimenti informativi}
\begin{itemize}
  \item \textbf{Standard ISO/IEC 12207:1995}: \\ \\ \href{https://www.math.unipd.it/~tullio/IS-1/2009/Approfondimenti/ISO_12207-1995.pdf}{https://www.math.unipd.it/\textasciitilde tullio/IS-1/2009/Approfondimenti/ISO\_12207-1995.pdf};
  \item \textbf{Standard ISO 8601}: \\ \\ \href{https://it.wikipedia.org/wiki/ISO_8601}{https://it.wikipedia.org/wiki/ISO\_8601};
\item \textbf{Documentazione \LaTeX}: \\ \\ \href{https://www.latex-project.org/help/documentation/}{https://www.latex-project.org/help/documentation/};
\item \textbf{Documentazione Git}: \\ \\ \href{https://git-scm.com/doc/}{https://git-scm.com/doc/};
\item \textbf{Documentazione Git Commitizen}: \\ \\ \href{https://commitizen.github.io/cz-cli/}{https://commitizen.github.io/cz-cli/};
\item \textbf{Documentazione Github}: \\ \\ \href{https://help.github.com/en/github/}{https://help.github.com/en/github/};
\item \textbf{Documentazione Visual Studio Code}: \\ \\ \href{https://code.visualstudio.com/docs}{https://code.visualstudio.com/docs};
\item \textbf{Documentazione kebab-case}: \\ \\ \href{https://it.wikipedia.org/wiki/Kebab_case}{https://it.wikipedia.org/wiki/Kebab\_case};
\item \textbf{Semantic versioning 2.0.0}: \\ \\ \href{https://semver.org/}{https://semver.org/}.
\end{itemize}

\end{document}
