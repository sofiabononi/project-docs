\documentclass[../norme-di-progetto.tex]{subfiles}
\begin{document}

%%%%%%%%%%%%%%%%%%%%%%%%%%%%%%%%%%%%
%%% 4.1 - GESTIONE ORGANIZZATIVA %%%
%%%%%%%%%%%%%%%%%%%%%%%%%%%%%%%%%%%%
\subsection{Gestione dei processi}
\subsubsection{Scopo}
Il processo di gestione dei processi ha lo scopo di:
\begin{itemize}
  \item Creare un modello organizzativo per identificare i possibili rischi;
  \item Definire un modello di sviluppo da seguire;
  \item Pianificare il lavoro per poter rientrare nelle scadenze prefissate;
  \item Calcolare un preventivo orario per i vari ruoli e un prospetto economico per lo sviluppo del progetto;
  \item Effettuare un bilancio finale delle spese.
\end{itemize}
Queste attività sono documentate dal responsabile nel \textsc{Piano di Progetto v1.9-4.2.0}.

\subsubsection{Aspettative}
Le aspettative di questo processo sono:
\begin{itemize}
  \item La corretta pianificazione delle attività da seguire;
  \item Il giusto coordinamento dei ruoli e dei compiti dei vari componenti del gruppo;
  \item La corretta gestione delle comunicazioni tra i componenti del gruppo e con il proponente;
  \item Il mantenimento del controllo sul progetto, monitorando quanto prodotto dai componenti del gruppo.
\end{itemize}

\subsubsection{Descrizione}
Le attività previste da questo processo sono contenute nel \textsc{Piano di Progetto v1.9-4.2.0}; queste sono:
\begin{itemize}
  \item Analisi dei rischi e loro classificazione;
  \item Istanziazione dei processi che realizzano il modello di sviluppo scelto;
  \item Pianificazione delle fasi del progetto e assegnazione dei ruoli;
  \item Stima temporale ed economica dei costi.
\end{itemize}

\subsubsection{Attività}
\paragraph{Pianificazione}
Il responsabile decide e documenta i piani per l'esecuzione dei processi; inoltre, suddivide i ruoli e le relative durate temporali tra i vari componenti del gruppo. \\
I diversi ruoli sono cambiati a rotazione, in modo che ogni membro del gruppo possa ricoprire ogni ruolo almeno una volta durante il progetto. Nel \textsc{Piano di Progetto v1.9-4.2.0} sono pianificate e organizzate le attività assegnate ad ogni ruolo. \\
I ruoli che ogni componente del gruppo deve rappresentare sono descritti di seguito.
\subparagraph{Responsabile}
Il ruolo del responsabile è la figura cardine per quanto concerne le responsabilità di pianificazione, controllo, gestione e coordinamento di risorse all'interno del progetto. Altro compito del responsabile è quello di fare da intermediario tra il gruppo e le persone esterne, cioè committente e proponente del capitolato. Egli, infine, è il responsabile ultimo dei risultati del progetto. \\
Nello specifico, il responsabile:
\begin{itemize}
  \item Gestisce, coordina e controlla le risorse e le attività del gruppo;
  \item Elabora e gestisce i piani e le scadenze dei componenti del gruppo;
  \item Gestisce le criticità che emergono durante lo sviluppo del progetto;
  \item Redige il \textsc{Piano di Progetto v1.9-4.2.0};
  \item Approva i documenti e l'offerta proposta al committente.
\end{itemize}

\subparagraph{Amministratore}
L'amministratore è la figura che gestisce e controlla l'ambiente di lavoro. Nello specifico, l'amministratore:
\begin{itemize}
  \item Organizza e dirige le infrastrutture di supporto;
  \item Gestisce i problemi legati alla gestione dei processi;
  \item Gestisce la documentazione del progetto;
  \item Collabora alla redazione del \textsc{Piano di Progetto v1.9-4.2.0};
  \item Redige le \textsc{Norme di Progetto v1.9-2.3.3}, in particolar modo è responsabile dei piani e delle procedure di Gestione per la Qualità.
\end{itemize}

\subparagraph{Analista}
L'analista è colui che si occupa di tutte le attività di analisi. Questa figura, per sua natura, non è presente all'interno del gruppo per tutta la durata del progetto, bensì il suo ruolo si ferma all'esaurimento delle attività da lui svolte, le quali sono concentrate nelle prime fasi del ciclo di sviluppo. \\
Nello specifico, l'analista:
\begin{itemize}
  \item Studia il dominio del progetto;
  \item Definisce e fissa i requisiti del progetto;
  \item Redige i documenti \textsc{Analisi dei requisiti v1.9-1.3.0} e \textsc{Studio di Fattibilità v1.0.0-old}.
\end{itemize}

\subparagraph{Progettista}
Il progettista è la figura responsabile delle attività di progettazione. Nello specifico, il progettista:
\begin{itemize}
  \item Effettua scelte concernenti gli aspetti tecnici del progetto, basandosi sui principi di \glossario{efficacia} ed \glossario{efficienza};
  \item Definisce e sviluppa l'architettura del prodotto, sfruttando tecnologie note ed ottimizzate;
  \item Redige la \textsc{Specifica Tecnica} e la \textsc{Definizione di Prodotto}.
\end{itemize}

\subparagraph{Programmatore}
Il programmatore è colui che si occupa della codifica del progetto. Nello specifico, il programmatore:
\begin{itemize}
  \item Implementa le scelte prese dal progettista e documentate nella \textsc{Specifica Tecnica};
  \item Codifica e gestisce i componenti di supporto per la verifica e la validazione del codice.
\end{itemize}

\subparagraph{Verificatore}
Il verificatore è la figura che si occupa delle attività di controllo di quanto svolto dagli altri componenti del gruppo. Per fare ciò, il verificatore deve rifarsi alle \textsc{Norme di Progetto v1.9-2.3.3} per controllare che quanto prodotto, sia esso documentazione o codice, segua correttamente tutte le norme documentate. \\
Nel caso in cui trovi un errore di qualunque natura, il verificatore non è autorizzato alla correzione di esso; deve bensì contattare colui che ha commesso il possibile errore e comunicarglielo. Sarà poi compito del redattore correggere eventualmente quanto svolto. \\
Riassumendo, il verificatore:
\begin{itemize}
  \item Ispeziona i prodotti in fase di revisione, facendo capo a quanto scritto nelle \textsc{Norme di Progetto v1.9-2.3.3};
  \item Evidenzia eventuali errori del prodotto ispezionato e li comunica all'autore del prodotto stesso;
  \item Redige il \textsc{Piano di Qualifica v1.9-3.1.0}.
\end{itemize}

\paragraph{Gestione delle comunicazioni}
Le comunicazioni possono essere di due tipi:
\begin{itemize}
  \item \textbf{Interne}: tra i soli componenti del gruppo;
  \item \textbf{Esterne}: tra i membri del gruppo, con l'aggiunta del committente e/o dell'azienda proponente.
\end{itemize}
\subparagraph{Comunicazioni interne}
Le comunicazioni interne avvengono su tre diverse piattaforme:
\begin{itemize}
  \item \glossario{Slack};
  \item \glossario{Discord};
  \item \glossario{Telegram}.
\end{itemize}
\subparagraph*{Slack}
La piattaforma Slack viene utilizzata per le comunicazioni scritte ufficiali; è stato creato un workspace, chiamato \textbf{CoffeeCode}, consistente di diversi canali:
\begin{itemize}
  \item \textbf{\#general}: canale principale, utilizzato per comunicazioni generali di natura tecnica;
  \item \textbf{\#analisi-dei-requisiti}: canale per le comunicazioni riguardanti il documento \textsc{Analisi dei requisiti v1.9-1.3.0};
  \item \textbf{\#norme-di-progetto}: canale per le comunicazioni riguardanti il documento \textsc{Norme di Progetto v1.9-2.3.3};
  \item \textbf{\#piano-di-progetto}: canale per le comunicazioni riguardanti il documento \textsc{Piano di Progetto v1.9-4.2.0};
  \item \textbf{\#piano-di-qualifica}: canale per le comunicazioni riguardanti il documento \textsc{Piano di Qualifica v1.9-3.1.0};
  \item \textbf{\#studio-di-fattibilità}: canale per le comunicazioni riguardanti il documento \textsc{Studio di Fattibilità v1.0.0-old};
  \item \textbf{\#technology-and-tool-faq}: canale che raccoglie tutte le domande che i componenti del gruppo pongono in merito alle tecnologie utilizzate;
  \item \textbf{\#predire-in-grafana}: canale collegato alla repository Github; ogni operazione svolta sulla repository viene qui riportata dal sistema automatico \textbf{Github for Slack};
  \item \textbf{\#backend}: canale per le comunicazioni riguardanti lo sviluppo delle componenti di back-end degli applicativi appartenenti al progetto;
  \item \textbf{\#frontend}: canale per le comunicazioni riguardanti lo sviluppo delle componenti di front-end degli applicativi appartenenti al progetto;
  \item \textbf{\#random}: canale riguardante le comunicazioni che non rientrano in nessuno dei precedenti argomenti.
\end{itemize}

\begin{figure}[H]
  \centering
  \includegraphics[width=15cm]{img/slack.png}
  \label{fig:slack}
  \caption{Slack per GNU/Linux.}
\end{figure}

\subparagraph*{Discord}
A causa dell'impossibilità di svolgere incontri \textit{vis-à-vis}, è stato deciso di utilizzare una piattaforma consona di comunicazione vocale. È quindi stato creato un server Discord, chiamato \textbf{CoffeeCode}, contenente un canale vocale \textbf{\#general} per gli incontri in tempo reale tra i componenti del gruppo.

\begin{figure}[H]
  \centering
  \includegraphics[width=15cm]{img/discord.png}
  \label{fig:discord}
  \caption{Discord per browser.}
\end{figure}

\subparagraph*{Telegram}
Oltre alle precedenti piattaforme, è stato creato anche un gruppo Telegram, chiamato \textbf{CoffeeCode}, che il gruppo usa per le comunicazioni informali tra i componenti.

\subparagraph{Comunicazioni esterne}
Le comunicazioni esterne sono di competenza del Responsabile. Lo strumento predefinito per le comunicazioni ufficiali è la posta elettronica, tramite l'indirizzo mail \href{coffecodeswe@gmail.com}{coffecodeswe@gmail.com}. \\
Viene inoltre utilizzata la piattaforma \glossario{Skype} per le comunicazioni del gruppo con l'azienda proponente, nella figura del Dr. Gregorio Piccoli.

\paragraph{Incontri}
\subparagraph{Incontri interni}
In caso se ne presenti il bisogno, il gruppo può riunirsi in via telematica attraverso la piattaforma Discord. Gli incontri interni del gruppo sono indetti dal responsabile, in accordo con tutto il gruppo. Per fissare un appuntamento il responsabile chiede, attraverso i già citati strumenti per le comunicazioni, la disponibilità di una data e di un orario; il gruppo conferma la disponibilità e l'incontro viene calendarizzato sulla piattaforma Google Calendar. Ogni componente del gruppo è quindi poi tenuto a partecipare. \\
Ogni incontro interno viene verbalizzato; a turno viene quindi individuato un segretario da parte del responsabile, che sarà tenuto a redigere un verbale contenente le informazioni generali dell'incontro, l'ordine del giorno, le discussioni svolte tra i componenti del gruppo e il tracciamento delle decisioni prese durante la riunione.

\subparagraph{Incontri esterni}
Il responsabile, tra i cui compiti è presente anche quello di interfacciare il gruppo con i soggetti esterni, provvede a raccogliere le disponibilità dei componenti del gruppo e, successivamente, a mediare la data e l'orario con i terzi. Una volta fissato l'appuntamento, il responsabile lo calendarizza sulla piattaforma Google Calendar e tutti i componenti del gruppo che hanno dato la disponibilità sono tenuti a presentarsi all'orario stabilito. \\
Anche in questo caso, ogni incontro viene verbalizzato; le modalità di verbalizzazione sono le stesse di quelle adottate per gli incontri interni.

\paragraph{Gestione degli strumenti di coordinamento}
L'andamento dello sviluppo viene tracciato da un ITS. Questo permette ai membri del gruppo di avere ben chiaro, in ogni momento, quali attività sono in corso, quali sono state completate e quali saranno le prossime da svolgere. \\
L'ITS che viene utilizzato è Github Issues, il quale è persistente sulla repository stessa del progetto; da questa interfaccia il responsabile può aprire nuove \textit{issues}, includendo una descrizione di ciò che deve essere fatto e assegnandole a specifici membri del gruppo.

\begin{figure}[H]
  \centering
  \includegraphics[width=15cm]{img/issues.png}
  \label{fig:sheets}
  \caption{Github Issues.}
\end{figure}

\paragraph{Gestione dei rischi}
I rischi vengono rilevati dal responsabile e da lui riportati nel \textsc{Piano di Progetto v1.9-4.2.0}; è una procedura dinamica: nel caso in cui vengano rilevati nuovi rischi, questi saranno aggiunti nell'analisi dei rischi. \\
La procedura che il responsabile dovrà seguire per la gestione dei rischi è la seguente:
\begin{itemize}
  \item Individuare i nuovi rischi e monitorare quelli già previsti;
  \item Registrare ogni riscontro previsto dei rischi nel \textsc{Piano di Progetto v1.9-4.2.0};
  \item Aggiungere i nuovi rischi che vengono man mano individuati nel \textsc{Piano di Progetto v1.9-4.2.0};
  \item Ridefinire, nel caso in cui ciò si rendesse necessario, le strategie per la gestione dei rischi.
\end{itemize}
\subparagraph*{Codifica}
La codifica dei rischi è così definita: \\
\begin{center}
\centering
\textbf{RK-Categoria-ID-Gravità}
\end{center} dove:
\begin{itemize}
  \item \textbf{Categoria} può essere:
  \begin{itemize}
    \item \textbf{O}: organizzativo;
    \item \textbf{P}: personale;
    \item \textbf{T}: tecnologico;
  \end{itemize}
  \item \textbf{ID} identifica un numero intero incrementale all'interno della categoria;
  \item \textbf{Gravità} può essere:
  \begin{itemize}
    \item \textbf{1}: indice di gravità bassa;
    \item \textbf{2}: indice di gravità media;
    \item \textbf{3}: indice di gravità alta;
    \item \textbf{4}: indice di gravità critica.
  \end{itemize}
\end{itemize}

\subsubsection{Strumenti}
Per tutta la durata del progetto, il gruppo si servirà di svariati strumenti per organizzare, gestire e svolgere le varie attività.
\paragraph{Visual Studio Code}
Lo sviluppo dell'intero prodotto è eseguito nell'IDE Visual Studio Code, con l'aggiunta delle seguenti estensioni:
\begin{itemize}
  \item \textbf{Git Graph}: permette di utilizzare azioni Git a partire dal grafo della repository;
  \item \textbf{LaTeX Workshop}: fornisce tutte le feature necessarie a scrittura, anteprima e compilazione di documenti \LaTeX;
  \item \textbf{GitHub Pull Requests}: permette di revisionare e gestire le Pull Requests di Github;
  \item \textbf{GitLens}: permette di visualizzare la paternità di ogni porzione di codice all'interno dell'editor;
  \item \textbf{Spell Right}: \glossario{spellchecker} multilingue e offline, permette il controllo grammaticale di quanto scritto;
  \item \textbf{Todo Tree}: permette la redazione di \glossario{TODO lists};
  \item \textbf{Prettier - Code formatter}: permette di definire ulteriori regole di \glossario{linting} per una formattazione uniforme del codice;
  \item \textbf{PowerShell}: permette di sviluppare codice in linguaggio PowerShell;
  \item \textbf{PlantUML}: permette di sviluppare codice per la creazione di diagrammi UML all'interno di Visual Studio Code.
\end{itemize}

\paragraph{Google Calendar}
Viene utilizzata la piattaforma Google Calendar per fissare e ricordare gli appuntamenti del gruppo, in particolar modo gli incontri interni ed esterni. Il calendario utilizzato risiede sull'account Google del gruppo.

\paragraph{Altri strumenti}
Gli altri strumenti di cui il gruppo fa uso, già documentati nelle \textsc{Norme di Progetto v1.9-2.3.3}, sono:
\begin{itemize}
  \item Telegram;
  \item Git;
  \item Git Commitizen;
  \item Github;
  \item Google Drive;
  \item Skype;
  \item GanttProject.
\end{itemize}

\paragraph{Sistemi operativi}
Ogni strumento utilizzato dal gruppo è compatibile con ogni piattaforma; i sistemi operativi usati dai componenti del gruppo possono essere perciò diversi. I sistemi utilizzati sono:
\begin{itemize}
  \item Windows 10;
  \item macOS Catalina;
  \item Ubuntu 19.10;
  \item Debian 10;
  \item Manjaro Linux.
\end{itemize}

%METRICHEPDQ1
\subsubsection{Metriche}
\paragraph{PRC001 Organizzazione e pianificazione del progetto e della sua struttura}
Si vuole misurare la copertura delle risorse disponibili per il progetto, in particorale si vuole assicurare che l'utilizzo di queste sia coerente con quanto annunciato nel \textsc{Piano di Progetto v1.9-4.2.0}. \\
Si fa quindi uso delle seguenti metriche:
\begin{itemize}
  \item MoPR001 Varianza dei tempi;
  \item MoPR002 Varianza dei costi;
  \item MoPR003 Aderenza agli standard;
  \item MoPR004 Aderenza ai ruoli;
  \item MoPR005 Controllo prodotti.
\end{itemize}

\subparagraph{MoPR001 Varianza dei tempi}
\begin{itemize}
  \item \textbf{Descrizione}: la varianza dei tempi indica il ritardo nel rispetto delle scadenze previste internamente al gruppo;
  \item \textbf{Unità di misura}: numero di giorni;
  \item \textbf{Formula}: la formula della metrica è la seguente:
    \begin{displaymath}
    \sum \frac{(Y_i - X_i)}{n}
    \end{displaymath}
    dove:
    \begin{itemize}
      \item $ Y_i $: giorno di completamento della scadenza;
      \item $ X_i $: giorni di scadenza prefissati.
    \end{itemize}
    \item \textbf{Procedura per il calcolo}: questa metrica viene calcolata manualmente, riferendosi a quanto riportato nel \textsc{Piano di Progetto v1.9-4.2.0} per quanto riguarda i giorni di scadenza prefissati.
    \item \textbf{Risultato}: il risultato della rilevazione assume i seguenti significati:
    \begin{itemize}
      \item Se il risultato è uguale a 0, le scadenze previste internamente al gruppo sono state rispettate;
      \item Se il risultato è minore di 0, le scadenze previste internamente al gruppo sono state rispettate con un numero di giorni di anticipo diverso da zero;
      \item Se il risultato è maggiore di 0, le scadenze previste internamente al gruppo non sono state rispettate.
    \end{itemize}
\end{itemize}

\subparagraph{MoPR002 Varianza dei costi}
\begin{itemize}
  \item \textbf{Descrizione}: la varianza dei costi indica il discostamento del totale finale rispeto al totale del preventivo iniziale;
  \item \textbf{Unità di misura}: percentuale;
  \item \textbf{Formula}: la formula della metrica è la seguente:
  \begin{displaymath}
    tot + d
  \end{displaymath}
  dove:
  \begin{itemize}
    \item $ tot $: totale preventivo iniziale;
    \item $ d $: discostamento calcolato.
  \end{itemize}
  \item \textbf{Procedura per il calcolo}: questa metrica viene calcolata manualmente, riferendosi a quanto riportato nel \textsc{Piano di Progetto v1.9-4.2.0} per quanto riguarda il totale del preventivo iniziale e il discostamento calcolato. Tali valori vengono confrontati per ricavare il valore della metrica in oggetto.
  \item \textbf{Risultato}: il risultato della rilevazione assume i seguenti significati:
  \begin{itemize}
    \item Se il risultato è uguale a $tot$, il totale finale corrisponde al totale del preventivo iniziale;
    \item Se il risultato è minore di $tot$, il totale finale è inferiore al totale del preventivo iniziale;
    \item Se il risultato è maggiore di $tot$, il totale finale è superiore al totale del preventivo iniziale.
  \end{itemize}
\end{itemize}

\subparagraph{MoPR003 Aderenza agli standard}
\begin{itemize}
  \item \textbf{Descrizione}: l'aderenza agli standard indica quanto il progetto stia proseguendo nel rispetto degli standard. Questa metrica è verificabile attraverso le valutazioni del processo di SPICE;
  \item \textbf{Unità di misura}: valutazione degli attributi;
  \item \textbf{Formula}: il valore viene controllato sulla base di quanto descritto negli standard di qualità definiti.
\end{itemize}

\subparagraph{MoPR004 Aderenza ai ruoli}
\begin{itemize}
  \item \textbf{Descrizione}: l'aderenza ai ruoli indica il numero di documenti nei quali non sono stati definiti correttamente i ruoli;
  \item \textbf{Unità di misura}: numero puro (numero di prodotti);
  \item \textbf{Formula}: la formula della metrica è la seguente:
  \begin{displaymath}
    N - X
  \end{displaymath}
  dove:
  \begin{itemize}
    \item $ N $: numero totale di prodotti realizzati;
    \item $ X $: numero di prodotti nei quali i ruoli non sono stati definiti adeguatamente.
  \end{itemize}
  \item \textbf{Procedura per il calcolo}: questa metrica viene calcolata manualmente, esaminando i documenti.
  \item \textbf{Risultato}: il risultato della rilevazione assume i seguenti significati:
  \begin{itemize}
    \item Se il risultato è uguale a $N$, il numero di documenti nei quali non sono stati definiti correttamente i ruoli è nullo;
    \item Se il risultato è compreso tra $N$ e 0, in alcuni documenti non sono stati definiti correttamente i ruoli;
    \item Se il risultato è uguale a 0, in tutti i documenti non sono stati definiti correttamente i ruoli.
  \end{itemize}
\end{itemize}

\subparagraph{MoPR005 Controllo prodotti}
\begin{itemize}
  \item \textbf{Descrizione}: il controllo prodotti verifica il numero medio di commit totali eseguiti ogni settimana durante la realizzazione dei prodotti;
  \item \textbf{Unità di misura}: numero puro (numero di commit);
  \item \textbf{Formula}: la formula della metrica è la seguente:
  \begin{displaymath}
    \sum \frac{C_i}{n}
  \end{displaymath}
  dove:
  \begin{itemize}
    \item $ C_i $: numero di commit nella settimana $ i $;
    \item $ n $: numero di settimane di lavoro.
  \end{itemize}
  \item \textbf{Procedura per il calcolo}: questa metrica viene calcolata con l'ausilio degli strumenti di versionamento, i quali permettono di estrapolare facilmente il numero di commit effettuati ogni settimana all'interno delle diverse repository. Questi valori vengono confrontati con il numero di settimane di lavoro, le quali possono essere ricavate dal \textsc{Piano di Progetto v1.9-4.2.0}.
  \item \textbf{Risultato}: il risultato della rilevazione assume i seguenti significati:
  \begin{itemize}
    \item Se il risultato è uguale a 0, nella settimana in questione non sono stati eseguiti commit;
    \item Se il risultato è maggiore di 0, nella settimana in questione sono stati eseguiti commit.
  \end{itemize}
\end{itemize}

%%%%%%%%%%%%%%%%%%%%%%%%
%%% 4.2 - FORMAZIONE %%%
%%%%%%%%%%%%%%%%%%%%%%%%
\subsection{Formazione}
\subsubsection{Scopo}
I membri del gruppo sono tenuti ad essere formati sui diversi ambiti su cui verte il progetto, dalle tecnologie necessarie allo sviluppo del software ai diversi strumenti organizzativi e di supporto normati da questo documento.

\subsubsection{Descrizione}
Ogni componente del gruppo dovrà essere a conoscenza e documentarsi individualmente sui linguaggi e gli strumenti che verranno usati durante il periodo di sviluppo del progetto.

\subsubsection{Attività}
La formazione avviene in due diverse modalità:
\begin{itemize}
  \item Tramite erogazione di brevi seminari sulle tecnologie da parte dell'azienda;
  \item Tramite auto-apprendimento, studiando cioè la documentazione necessaria individualmente.
\end{itemize}
I brevi seminari erogati dall'azienda coprono buona parte delle conoscenze di Machine Learning necessarie allo sviluppo del software.
\paragraph{Materiale per la formazione}
Vengono forniti di seguito i link presso cui trovare la documentazione necessaria alla formazione individuale dei componenti del gruppo.

\subparagraph*{Linguaggi di programmazione}
\begin{itemize}
  \item \textbf{\LaTeX}: \href{https://www.latex-project.org/help/documentation/}{https://www.latex-project.org/help/documentation/};
  \item \textbf{JavaScript}: \href{https://devdocs.io/javascript/}{https://devdocs.io/javascript/};
  \item \textbf{JavaScript ES6}: \href{http://es6-features.org/}{http://es6-features.org/};
  \item \textbf{TypeScript}: \href{https://www.typescriptlang.org/docs/home.html}{https://www.typescriptlang.org/docs/home.html};
  \item \textbf{HTML e CSS}: \href{https://www.w3schools.com/html/}{https://www.w3schools.com/html/}.
\end{itemize}

\subparagraph*{Tecnologie}
\begin{itemize}
  \item \textbf{Framework Electron}: \href{https://www.electronjs.org/docs}{https://www.electronjs.org/docs};
  \item \textbf{Node.js}: \href{https://nodejs.org/it/docs/}{https://nodejs.org/it/docs/};
  \item \textbf{Framework ReactJS}: \href{https://it.reactjs.org/docs/getting-started.html}{https://it.reactjs.org/docs/getting-started.html};
  \item \textbf{Plot.ly}: \href{https://www.npmjs.com/package/plotly.js/v/1.47.4}{https://www.npmjs.com/package/plotly.js/v/1.47.4};
  \item \textbf{Grafana}: \href{https://grafana.com/docs/grafana/latest/}{https://grafana.com/docs/grafana/latest/};
  \item \textbf{Git}: \href{https://git-scm.com/doc}{https://git-scm.com/doc};
  \item \textbf{Github}: \href{https://help.github.com/en}{https://help.github.com/en};
  \item \textbf{GitFlow}: \href{https://www.atlassian.com/git/tutorials/comparing-workflows/gitflow-workflow}{https://www.atlassian.com/git/tutorials/comparing-workflows/gitflow-workflow};
  \item \textbf{Git Commitizen}: \href{http://commitizen.github.io/cz-cli/}{http://commitizen.github.io/cz-cli/};
  \item \textbf{Git submodules}: \href{https://git-scm.com/book/it/v2/Git-Tools-Submodules}{https://git-scm.com/book/it/v2/Git-Tools-Submodules};
  \item \textbf{SonarCloud}: \href{https://sonarcloud.io/documentation/branches/overview/}{https://sonarcloud.io/documentation/branches/overview/};
  \item \textbf{Ganttproject}: \href{http://docs.ganttproject.biz/}{http://docs.ganttproject.biz/}.
\end{itemize}

\subparagraph*{Argomenti teorici}
\begin{itemize}
  \item \textbf{Regressione Lineare}: \href{https://it.wikipedia.org/wiki/Regressione_lineare}{https://it.wikipedia.org/wiki/Regressione\_lineare};
  \item \textbf{Support Vector Machines}: \href{https://lorenzogovoni.com/support-vector-machine/}{https://lorenzogovoni.com/support-vector-machine/}.
\end{itemize}

\subsubsection{Metriche}
Il processo di formazione non fa uso di metriche di qualità.

\end{document}
