\documentclass{article}

\usepackage[italian]{babel}
\usepackage[margin=20mm, footskip = 20pt]{geometry}
\usepackage{graphicx}
\usepackage{subfiles}
\usepackage{hyperref}
\usepackage{nameref}
\usepackage{titlesec}
\usepackage{longtable}
\usepackage[table]{xcolor}
\usepackage{titling}
\usepackage{lastpage}
\usepackage{ifthen}
\usepackage{calc}
\usepackage{soulutf8}
\usepackage{contour}
\usepackage{float}
\usepackage{fancyhdr}
\usepackage{multirow}
\usepackage{pgfgantt}

% definizione dei percorsi in cui cercare immagini
\graphicspath{ {./}
    {./img/}
}

% setup della sottolineatura
\setuldepth{Flat}
\contourlength{0.8pt}

\newcommand{\uline}[1]{%
  \ul{{\phantom{#1}}}%
  \llap{\contour{white}{#1}}%
}

% setup dei link
\hypersetup{
  % set true if you want colored links (instead of boxes)
  colorlinks=true,
  % set to all if you want both sections and subsections linked
  linktoc=all,
  % set color for file links
  filecolor=blue,
  % set color for internal links
  linkcolor=black,
  % set url color
  urlcolor=blue,
  % set characters encoding in the bookmarks tab
  pdfencoding=unicode,
}

% setup forma \paragraph e \subparagraph
\titleformat{\paragraph}[hang]{\normalfont\normalsize\bfseries}{\theparagraph}{1em}{}
\titleformat{\subparagraph}[hang]{\normalfont\normalsize\bfseries}{\thesubparagraph}{1em}{}

% setup profondità indice di default
\setcounter{secnumdepth}{5}
\setcounter{tocdepth}{5}

\makeatletter %% non togliere, i comandi che definiscono i placeholder vanno qui
% esempio di utilizzo: \appendToGraphicspath{./img/} (un comando diverso per ogni path da includere)
% N.B.: ci DEVE essere un forward slash alla fine del path, a indicare che è una cartella.
\newcommand\appendToGraphicspath[1]{%
  \g@addto@macro\Ginput@path{{#1}}%
}

\newcommand{\setTitle}[1]{%
  \newcommand{\@placeholderTitle}{#1}%
}
\newcommand{\placeholderTitle}{\@placeholderTitle}

\newcommand{\setUso}[1]{%
  \newcommand{\@uso}{#1}%
}
\newcommand{\uso}{\@uso}

\newcommand{\setVersione}[1]{%
  \newcommand{\@versione}{#1}%
}
\newcommand{\versione}{\@versione}

\newcommand{\disabilitaVersione}{%
  \renewcommand{\setVersione}[1]{}%
  \renewcommand{\versione}{DISABILITATA}
}

\newcommand{\setResponsabile}[1]{%
  \newcommand{\@responsabile}{#1}%
}
\newcommand{\responsabile}{\@responsabile}

\newcommand{\setRedattori}[1]{%
  \newcommand{\@redattori}{#1}%
}
\newcommand{\redattori}{\@redattori}

\newcommand{\setVerificatori}[1]{%
  \newcommand{\@verificatori}{#1}%
}
\newcommand{\verificatori}{\@verificatori}

\newcommand{\setDescrizione}[1]{%
  \newcommand{\@descrizione}{#1}%
}
\newcommand{\descrizione}{\@descrizione}

\newcommand{\setModifiche}[1]{%
  \newcommand{\@modifiche}{#1}%
}

\newcommand{\modifiche}{\@modifiche}
\makeatother %% non togliere, i comandi che definiscono i placeholder vanno qui

% hook per lo script che genera il glossario
\newcommand{\glossario}[1]{\underline{#1}\textsubscript{g}}

% comandi per rendere opzionali gli elenchi di figure
\newcommand{\elencoFigure}{%
  \renewcommand{\listfigurename}{Elenco delle figure}%
  \listoffigures%
}

\newcommand{\disabilitaElencoFigure}{%
  \renewcommand{\elencoFigure}{}%
}

% comandi per rendere opzionali le tabelle
\newcommand{\elencoTabelle}{%
  \renewcommand{\listtablename}{Elenco delle tabelle}%
  \listoftables%
}

\newcommand{\disabilitaElencoTabelle}{%
  \renewcommand{\elencoTabelle}{}%
}

%Qui ci andrà il percorso delle immagini da includere in analisi dei requisiti
\appendToGraphicspath{../../commons/img/}

%Tutti questi set permettono di modificare in modo adatto i placeholder nel template
\setTitle{Norme di Progetto}

\setVersione{}

\setResponsabile{}

\setRedattori{}

\setVerificatori{}

\setUso{Interno}

\setDescrizione{Documento che racchiude le regole, gli strumenti e le convenzioni adottate dal gruppo \emph{CoffeeCode} durante la realizzazione del progetto \emph{Predire in Grafana}.}

\setModifiche{}

\begin{document}

\pagenumbering{gobble}

\begin{titlepage}% per non stampare il numero della pagina

  \raggedleft% allinea a destra la pagina
  \rule{1pt}{\textheight}% linea verticale
  \hspace{0.05\textwidth}% spazio tra linea e testo
  % lasciare questa riga per il corretto funziomento di \parbox
  \parbox[b]{0.75\textwidth}{% paragrafo che tiene il testo a destra della riga cambiando la larghezza il testo si muove a destra o a sinistra
  {\hspace{0.15\textwidth}\includegraphics[width=3cm,height=3cm]{logo.jpg}}\\[2\baselineskip] % logo
  {\Huge\bfseries CoffeeCode \\[0.5\baselineskip] Predire in Grafana}\\[5\baselineskip] % titolo
  {\Large\textsc{\placeholderTitle{}}}\\[6\baselineskip] % nome del documento
  {\begin{tabular}{r l}
    % testo in grassetto
    \textbf{Versione}     & \versione{}               \\
    \textbf{Approvazione} & \responsabile{}           \\
    \textbf{Redazione}    & \redattori{}              \\
    \textbf{Verifica}     & \verificatori{}           \\
    \textbf{Uso}          & \uso{}                    \\
    \textbf{Destinato a}  & CoffeeCode                \\
                          & prof.\ Vardanega Tullio   \\
                          & prof.\ Cardin Riccardo    \\
    \ifthenelse{\equal{\uso}{Esterno}}{
                          & Zucchetti Group SPA       \\
    }{}
  \end{tabular}}\\[5\baselineskip]

  {\bfseries Descrizione}\\
  {\descrizione{}}\\[2\baselineskip]
  {\texttt{coffeecodeswe@gmail.com}}\\[\baselineskip] % email
  }

\end{titlepage}


%INTRODUZIONE

\section{Introduzione}%
\label{sec:introduzione}
\subsection{Scopo del documento}
\label{scopo_del_documento}
Questo documento ha come scopo quello di definire il \emph{way of working} del gruppo CoffeeCode per lo svolgimento del progetto.
Le attività descritte in questo documento ricalcano i processi descriti nello standard ISO/IEC 12207:1995. Al fine di ottenere l'omogeneità di quanto presentato in questo progetto, ogni membro del gruppo è tenuto a visionare questo documento; ogni cambiamento, aggiunta o rimozione al presente documento deve essere inoltre notificato ad ogni membro del gruppo.

\subsection{Scopo del prodotto}
\label{scopo_del_prodotto}
Il capitolato \textbf{C4 - "Predire in Grafana"} ha come obiettivo la realizzazione di un \glossario{plug-in}, scritto in linguaggio \glossario{Javascript} per lo strumento \glossario{Grafana}, che applichi le tecniche di \glossario{Machine Learning} \glossario{Regressione Lineare} e \glossario{Support Vector Machines} ai dati ricevuti, al fine di monitorare la \glossario{liveliness} del sistema e di consigliare interventi alla linea di produzione del software tramite specifici \glossario{alert}.
%DA CONTINUARE%

\subsection{Glossario}
\label{glossario}
Al fine di evitare ambiguità relative alle terminologie usate, viene fornito il \textit{Glossario v1.0.0}, nel quale sono definiti i termini sottolineati e con una G maiuscola come pedice presenti negli altri documenti.

\subsection{Riferimenti}
\label{riferimenti}
\subsubsection{Riferimenti normativi}
\begin{itemize}
  \item \textbf{Capitolato d'appalto C4 - "Predire in Grafana"}: \\ \href{https://www.math.unipd.it/~tullio/IS-1/2019/Progetto/C4.pdf}{https://www.math.unipd.it/\textasciitilde tullio/IS-1/2019/Progetto/C4.pdf}.
\end{itemize}

\subsubsection{Riferimenti informativi}
\begin{itemize}
  \item \textbf{Standard ISO/IEC 12207:1995}: \\ \href{https://www.math.unipd.it/~tullio/IS-1/2009/Approfondimenti/ISO_12207-1995.pdf}{https://www.math.unipd.it/\textasciitilde tullio/IS-1/2009/Approfondimenti/ISO\_12207-1995.pdf}.
\item \textbf{Documentazione \LaTeX}: \\ \href{https://www.latex-project.org/help/documentation/}{https://www.latex-project.org/help/documentation/}.
\item \textbf{Documentazione Git}: \\ \href{https://git-scm.com/doc/}{https://git-scm.com/doc/}.
\item \textbf{Documentazione Git Commitizen}: \\ \href{https://commitizen.github.io/cz-cli/}{https://commitizen.github.io/cz-cli/}.
\item \textbf{Documentazione Github}: \\ \href{https://help.github.com/en/github/}{https://help.github.com/en/github/}.
\end{itemize}

\newpage

\section{Processi primari}%
\label{sec:processi-primari}
\subfile{components/2-processi-primari.tex}

\newpage

\section{Processi di Supporto}%
\label{sec:processi-primari}
\subfile{components/3-processi-supporto.tex}

\newpage

\section{Processi organizzativi}%
\label{sec:processi-primari}
\subfile{components/4-processi-organizzativi.tex}


\end{document}
